%%%%%%%%%%%%%%%%%%%%%%%%%%%%%%%%%%%%%%%%%%%%%%%%%%%%%%%%%%%%%%%%%%
%%%                      Homework 1                            %%%
%%%%%%%%%%%%%%%%%%%%%%%%%%%%%%%%%%%%%%%%%%%%%%%%%%%%%%%%%%%%%%%%%%

\documentclass[letter]{article}

\usepackage{lipsum}
\usepackage[pdftex]{graphicx}
\usepackage[margin=1.5in]{geometry}
\usepackage[english]{babel}
\usepackage{listings}
\usepackage{amsthm}
\usepackage{amssymb}
\usepackage{framed} 
\usepackage{amsmath}
\usepackage{titling}
\usepackage{fancyhdr}

\pagestyle{fancy}


\newtheorem{theorem}{Theorem}
\newtheorem{definition}{Definition}

\newenvironment{menumerate}{%
  \edef\backupindent{\the\parindent}%
  \enumerate%
  \setlength{\parindent}{\backupindent}%
}{\endenumerate}







%%%%%%%%%%%%%%%
%% DOC INFO %%%
%%%%%%%%%%%%%%%
\newcommand{\bHWN}{1}
\newcommand{\bCLASS}{MATH 105}

\title{\bCLASS: Homework \bHWN}
\author{William Guss\\26793499\\wguss@berkeley.edu}

\fancyhead[L]{\bCLASS}
\fancyhead[CO]{Homework \bHWN}
\fancyhead[CE]{GUSS}
\fancyhead[R]{\thepage}
\fancyfoot[LR]{}
\fancyfoot[C]{}
\usepackage{csquotes}

%%%%%%%%%%%%%%

\begin{document}
\maketitle
\thispagestyle{empty}


%%%%%%% Be sure to set the counter and use menumerate
\setcounter{section}{4}
\section{Multivariable Calculus}
\begin{menumerate}
	\setcounter{enumi}{2}
	\item %3
	Prove the following.
	\begin{theorem}
		Let $T: V\to W$ be a linear transformation between normed spaces. Then, 
		\begin{equation}
			\begin{aligned}
				\|T\| &= \sup \{|Tv|\;:\;|v| < 1 \} \\
				 &= \sup \{|Tv|\;:\;|v| \leq 1 \} \\
				 &= \sup \{|Tv|\;:\;|v| = 1 \} \\
				 &= \inf \{M\;:\;v \in V \implies |Tv| \leq M|v| \}
			\end{aligned}
		\end{equation}
	\end{theorem}
	\begin{proof}
		Let the following defenitions stand,
		\begin{equation}
			\begin{aligned}
				A &= \sup \{|Tv|\;:\;|v| < 1 \} \\
				B &= \sup \{|Tv|\;:\;|v| \leq 1 \} \\
				C &= \sup \{|Tv|\;:\;|v| = 1 \} \\
				D &= \inf \{M\;:\;v \in V \implies |Tv| \leq M|v| \}
			\end{aligned}
		\end{equation}
		Observe that $A \leq B$ and $C \leq B$ since the family considting of the underlying sets
		is respectiverley ordered by size.
		By definition we have that, 
		$$\|T\| = \sup\left\{|Tv|/|v|\right\},$$ and nameley $|Tv|/|v| = |T(v/|v|)|$. Therefore
		$\|T\| \leq C.$
		If $|v| \leq 1$ then $|Tv| \leq |Tv|/|v|$ and so $B \leq \|T\|.$
		We yield that $\|T\| = B= C.$ 

		By the same logic $A \leq \|T\|$ and therefore is equivalent. 
		Lastly $|Tv| \leq \|T\||v|$ and so by the epsilon property $D= A$.
	\end{proof}

	\setcounter{enumi}{3}
	\item %4

	\setcounter{enumi}{5}
	\item %6
	x

	\setcounter{enumi}{11}
	\item Prove the following.
	\begin{theorem}
		If $V$ is a normed finite dimensional vector space, then the unit ball, $B = \{v:|v| =1\}$ is compact.
	\end{theorem}
	\begin{proof}
	$\dim V = n \in \mathbb{N} \implies V \cong \mathbb{R}^n \implies B \cong S^{n-1 } \implies B$ compact.	 
	\end{proof} 

	\item Prove the following.
	\begin{theorem}
		The set of invertible $n\times n$ matrices is not dense in $\mathcal{M}.$
	\end{theorem}
	\begin{proof}
		Consider the set of matrix all of whose entries are the same. They create a linear subspace
		which is a disjoint open subset of $\mathcal{M}.$ Therefore the set of invertible matrices could
		not possibly be dense in $\mathcal{M}.$	 
	\end{proof}
\end{menumerate}

\end{document}