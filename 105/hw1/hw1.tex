%%%%%%%%%%%%%%%%%%%%%%%%%%%%%%%%%%%%%%%%%%%%%%%%%%%%%%%%%%%%%%%%%%
%%%                      Homework 1                            %%%
%%%%%%%%%%%%%%%%%%%%%%%%%%%%%%%%%%%%%%%%%%%%%%%%%%%%%%%%%%%%%%%%%%

\documentclass[letter]{article}

\usepackage{lipsum}
\usepackage[pdftex]{graphicx}
\usepackage[margin=1.5in]{geometry}
\usepackage[english]{babel}
\usepackage{listings}
\usepackage{amsthm}
\usepackage{amssymb}
\usepackage{framed} 
\usepackage{amsmath}
\usepackage{titling}
\usepackage{fancyhdr}

\pagestyle{fancy}


\newtheorem{theorem}{Theorem}
\newtheorem{definition}{Definition}

\newenvironment{menumerate}{%
  \edef\backupindent{\the\parindent}%
  \enumerate%
  \setlength{\parindent}{\backupindent}%
}{\endenumerate}







%%%%%%%%%%%%%%%
%% DOC INFO %%%
%%%%%%%%%%%%%%%
\newcommand{\bHWN}{1}
\newcommand{\bCLASS}{MATH 105}

\title{\bCLASS: Homework \bHWN}
\author{William Guss\\26793499\\wguss@berkeley.edu}

\fancyhead[L]{\bCLASS}
\fancyhead[CO]{Homework \bHWN}
\fancyhead[CE]{GUSS}
\fancyhead[R]{\thepage}
\fancyfoot[LR]{}
\fancyfoot[C]{}
\usepackage{csquotes}

%%%%%%%%%%%%%%

\begin{document}
\maketitle
\thispagestyle{empty}


%%%%%%% Be sure to set the counter and use menumerate
\setcounter{section}{4}
\section{Multivariable Calculus}
\begin{menumerate}
	\setcounter{enumi}{2}
	\item %3
	Prove the following.
	\begin{theorem}
		Let $T: V\to W$ be a linear transformation between normed spaces. Then, 
		\begin{equation}
			\begin{aligned}
				\|T\| &= \sup \{|Tv|\;:\;|v| < 1 \} \\
				 &= \sup \{|Tv|\;:\;|v| \leq 1 \} \\
				 &= \sup \{|Tv|\;:\;|v| = 1 \} \\
				 &= \inf \{M\;:\;v \in V \implies |Tv| \leq M|v| \}
			\end{aligned}
		\end{equation}
	\end{theorem}
	\begin{proof}
		Let the following defenitions stand,
		\begin{equation}
			\begin{aligned}
				A &= \sup \{|Tv|\;:\;|v| < 1 \} \\
				B &= \sup \{|Tv|\;:\;|v| \leq 1 \} \\
				C &= \sup \{|Tv|\;:\;|v| = 1 \} \\
				D &= \inf \{M\;:\;v \in V \implies |Tv| \leq M|v| \}
			\end{aligned}
		\end{equation}
		Observe that $A \leq B$ and $C \leq B$ since the family considting of the underlying sets
		is respectiverley ordered by size.
		By definition we have that, 
		$$\|T\| = \sup\left\{|Tv|/|v|\right\},$$ and nameley $|Tv|/|v| = |T(v/|v|)|$. Therefore
		$\|T\| \leq C.$
		If $|v| \leq 1$ then $|Tv| \leq |Tv|/|v|$ and so $B \leq \|T\|.$
		We yield that $\|T\| = B= C.$ 

		By the same logic $A \leq \|T\|$ and therefore is equivalent. 
		Lastly $|Tv| \leq \|T\||v|$ and so by the epsilon property $D= A$.
	\end{proof}

	\setcounter{enumi}{3}
	\item 

	Consider the following theorem.
	\begin{theorem}
		If $T:V\to V$ is a linear transformation on the normed vector space $V$. Let $A = \sup\{r : B_r(0) \supset TU\}$
		and $B = \inf \{r : B_r(0) \subset TU\}.$ Then, $A = \|T\|$ and $B = m(T).$
	\end{theorem}
	\begin{proof}
		Observe $U \subset V$ is the unit ball induced by $| . |$ and therefore $U$ is compact.
		$T$ is linear so by its continuity we have that $TU$ is compact and
		thereby contains all its limit points.

		Then there is a sequence in $TU$ so that $v_n \to v \in \partial TU \cap B_r(0).$
		In particular $|v| = A.$ Likewise there is a sequence $w_n \to w$ in $TU$ so that $|w| = B.$

		Suppose that $\|T\| < A.$ Then $\|T\| < |v|.$ There exists a $u$
		so that $Tu = v$ and $v \in \partial TU$ implies that $u \in \partial U$ by continuity and linearity.
		Thus $\|T\| < |Tu|/|u|$ which is a contradiction. 

		Suppose that $\|T\| > A$ or equivalently $\|T\| -A = \epsilon > 0.$ By the linearity of $T$
		we have that for all $z \in V$  $\|T\| - |z| \leq \epsilon$ since $ z = \alpha q$ for $\alpha \in \mathbb{R}$
		and $q \in TU.$ So $\|T\| = \sup\{|Tu|/|u| : u \in U\} + \epsilon$ which is a contradiction.

		So $\|T\| = A.$ 


		Suppose that $m(T)  < B.$ Then $m(T) < |w|.$ There exists a $u$
		so that $Tu = w$ and $w \in \partial TU$ implies that $u \in \partial U$ by continuity and linearity.
		Thus $m(T) > |Tu|/|u|$ which is a contradiction. 

		Suppose that $m(T) > B$ or equivalently $m(T) -B = \epsilon > 0.$ By the linearity of $T$
		we hawe that for all $z \in V$  $m(T) - |z| \leq \epsilon$ since $ z = \alpha q$ for $\alpha \in \mathbb{R}$
		and $q \in TU.$ So $m(T) = \inf\{|Tu|/|u| : u \in U\} + \epsilon$ which is a contradiction.

		So $m(T) = B.$ 
	\end{proof}

	\begin{theorem}
		If $T:V\to V$ is a linear isomorphism then, $m(T) > 0.$
	\end{theorem}
	\begin{proof}
		In the contrapositive, $m(T) = 0$ implies that the largest ball which is contained
		in $TU$ is the $0$ ball and so the kernel of $T$ is non-triial. Therefore $T$ is note an isomorphism.

	\end{proof}

	\begin{theorem}
		If $T: V \to V$ has positive conorm and is linear, then it is an isomorphism.
	\end{theorem}
	\begin{proof}
		Positive conorm impliesn that $T$ has a trivial kernel and so by the invertible matrix theorem,
		$T \equiv A$ where $A$ is invertible and so $T$ is invertible.
	\end{proof}

	\begin{theorem}
	If $T:V \to V$ and $T$ is linear, then $T$ is the identity.
	\end{theorem}
	\begin{proof}
		The conorm is equal to the norm if and only if $U \mapsto U.$ Then by linearity $v/|v| \mapsto v/|v|$ implies 
		$v \mapsto v.$
	\end{proof}

	\setcounter{enumi}{5}
	\item %6
	Consider the following theorem.
	\begin{theorem}
		$\mathcal{L}_n$ and $\mathcal{M}_n$ are rings where the abelian operator is
		pointwise and componentwise respectveley, and where the monoid
		law of composition is multiplication and functional composition 
		respectiveley.
	\end{theorem}
	\begin{proof}
		The set of linear transformations $\mathcal{L}_n$  is Abelian with respect to
		addition since it occurs over the field $\mathbb{R}$; that is,
		 $$+_\mathcal{L} : \mathcal{L}_n\times \mathcal{L}_n \to \mathbb{R} \times \mathbb{R} \to \mathbb{R} \to \mathcal{L}_n.$$


		As for monoid laws of composition, we show the distributive properties. First,
		$f, id \in \mathcal{L}_n$ imples that $f \circ id : V \to W$ with the mapping
		$x \mapsto x \mapsto f(x) \equiv x \mapsto f(x)$. So, $f \circ id \equiv f.$
		Now consider $g,h \in \mathcal{L}_n.$ The composition $f \circ (g +_\mathcal{L} h): V \to W$
		has the mapping
		$$x \mapsto h(x) + g(x) \mapsto f(h(x) + g(x)).$$
		By linearity, we equivelently have $x \mapsto f(h(x)) + f(g(x)).$ So in total
		$f \circ(g +_\mathcal{L} h) \equiv f \circ g +_\mathcal{L} f \circ h.$ Lastly,
		$(f \circ  g) \circ h \equiv f \circ (g \circ h)$ by the same logic. Therefore, $\mathcal{L}_n$ is a ring.
		
		Matrices have the following result. Take $M,N,L \in \mathcal{M}_n$.
		gain the addition operator is Abelian since it maps to $\mathbb{R}^n$; that is
		$$+_M: \mathcal{M}_n \times \mathcal{M}_n \to \mathbb{R}^{n\times n} \times \mathbb{R}^{n\times n} \to \mathbb{R}^{n\times n} \to \mathcal{M}_n.$$ 
		Then it follows that, $MI = M$ by the rules matrix multiplication. 
		Furthermore matrix multiplication is associative and distributive. Therefore 
		$\mathcal{M}_n$ is a ring.

	\end{proof}

	\begin{theorem}
	There exists a ring isomorphism between $\mathcal{M}_n$ and $\mathcal{L}_n.$
	\end{theorem}
	\begin{proof}
	Let $\tau : \mathcal{M}_n \to \mathcal{L}_n$ be defined by the mapping $A \mapsto (x \mapsto Ax).$ 
	Clearly this mapping is a surjection since given any $f \in \mathcal{L}_n$ there is at least a corresponding matrix
	in $\mathcal{M}_n$ by the following construction. Take the standard basis of $V$ and produce $$A = [f(e_1) \dots f(e_n)].$$
	Then $f(v) = f(e_1)v_1 + \dots + f(e_n)v_n = Av.$ Suppose there were another matrix $B$ such that 
	$\tau(B) = f = \tau(A)$. Then $\tau(B - A) = \tau(B) - \tau(A) = f - f = 0$ but this contradicts the fact
	that $B \neq A$. Therefore $\tau$ is bijective.

	Finally let $C \in \mathcal{M}_n.$ Then $\tau(A(B+C)) = (x \mapsto A(B+C)x).$ By linearity this is
	 equivalent to $(x \mapsto ABx + ACx) = \tau(A) \circ \tau(B) + \tau(A) \circ(C) = \tau(A) \circ (\tau(B) + \tau(C)).$ So,
	 $\tau$ is a homomorphism.

	 Hence $\tau$ is an isomorphism.  
	\end{proof}


	\setcounter{enumi}{11}
	\item Prove the following.
	\begin{theorem}
		If $V$ is a normed finite dimensional vector space, then the unit ball, $B = \{v:|v| =1\}$ is compact.
	\end{theorem}
	\begin{proof}
	$\dim V = n \in \mathbb{N} \implies V \cong \mathbb{R}^n \implies B \cong S^{n-1 } \implies B$ compact.	 
	\end{proof} 

	\item Prove the following.
	\begin{theorem}
		The set of invertible $n\times n$ matrices is not dense in $\mathcal{M}.$
	\end{theorem}
	\begin{proof}
		Consider the set of matrix all of whose entries are the same ($(a_{ij} = r \forall i \forall j$). They create a linear subspace
		which is a connected open subset of $\mathcal{M}$ disjoint from the set of invertible matrices. Therefore the set of invertible matrices could
		not possibly be dense in $\mathcal{M}.$	 
	\end{proof}
\end{menumerate}

\end{document}