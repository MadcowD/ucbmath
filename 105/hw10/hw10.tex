%%%%%%%%%%%%%%%%%%%%%%%%%%%%%%%%%%%%%%%%%%%%%%%%%%%%%%%%%%%%%%%%%%
%%%                      Homework _                            %%%
%%%%%%%%%%%%%%%%%%%%%%%%%%%%%%%%%%%%%%%%%%%%%%%%%%%%%%%%%%%%%%%%%%

\documentclass[letter]{article}

\usepackage{lipsum}
\usepackage[pdftex]{graphicx}
\usepackage[margin=1.5in]{geometry}
\usepackage[english]{babel}
\usepackage{listings}
\usepackage{amsthm}
\usepackage{amssymb}
\usepackage{framed} 
\usepackage{amsmath}
\usepackage{titling}
\usepackage{fancyhdr}
\usepackage[mathcal]{euscript}
\pagestyle{fancy}


\newtheorem{theorem}{Theorem}
\newtheorem{lemma}{Lemma}
\newtheorem{definition}{Definition}

\newenvironment{menumerate}{%
  \edef\backupindent{\the\parindent}%
  \enumerate%
  \setlength{\parindent}{\backupindent}%
}{\endenumerate}







%%%%%%%%%%%%%%%
%% DOC INFO %%%
%%%%%%%%%%%%%%%
\newcommand{\bHWN}{8}
\newcommand{\bCLASS}{MATH 105}

\title{\bCLASS: Homework \bHWN}
\author{William Guss\\26793499\\wguss@berkeley.edu}

\fancyhead[L]{\bCLASS}
\fancyhead[CO]{Homework \bHWN}
\fancyhead[CE]{GUSS}
\fancyhead[R]{\thepage}
\fancyfoot[LR]{}
\fancyfoot[C]{}
\usepackage{csquotes}

%%%%%%%%%%%%%%

\begin{document}
\maketitle
\thispagestyle{empty}

\begin{menumerate}
    \setcounter{enumi}{42}
    \item \begin{theorem}
    	Let $g(y) = \int_0^\infty e^{-x}\sin(x + y)\ dx$. The function is differentiable with respect to $g(y).$ 
    \end{theorem}
    \begin{proof}
 		Labesgue Dominated Convergance theorem.
 		Consider the following construction.
 		\begin{equation}
 		   	\begin{aligned}
                \lim_{h \to 0} \frac{g(y) + g(h + y)}{h} &= \frac{1}{h} \int \gamma(y) + \int \gamma(h+y) \\
                 &= \int \frac{ e^{-x}sin(x+y)\ + e^{-x}sin(x+y+h)}{h}\ dx \\
                 &=  \int f_h\ d\mu(x).
            \end{aligned}
 		   \end{equation}   
           where $\mu$ is Lebesgue regular measure. We would like to use the dominated convergence theorem to show that this 
           sequence of integrals converges to a limit. First we must show that $f_h$ converges to $f$ almpost everywhere.
           We know from Math $1B$ that the limit of $f_h$ is the derivative  of $\gamma,$ and $\gamma$ is clearly a
           differentiable function. So it is obviouse that $f_h \to f.$

           Furthermore we know that $\gamma(x)$ is dominated by $e^-{x}$. Observe that 
           $e^{-x} \geq f_h(x)$ almost everywhere since $\sin(x+y+h) \leq 1.$

           Therefore by the dominated convergence thereom $f_h\to f(x)$ is integrable with respect to the measure $\mu,$
           and $\int f_h \to \int f = L$ is the derivatice of $g(y)$ at $y.$ Since we did this for abitrary $y$ $g$
           is differentiable everywhere. :)
    \end{proof}
    \setcounter{enumi}{45}
    \item  
    \begin{theorem}
        
    \end{theorem}
    \textbf{Office Hours:} We know that $f$ is rieman integrable since it has one point of fiscontinuity. Therefore we can use calculus.
    Consider the integratipon by parts. We have
    \begin{equation}
    	\int_0^1 \frac{\pi}{x}\sin \frac{\pi}{x}\ dx = x \cos \frac{\pi}{x} |^1_a  - \int_a^1 \cos \frac{\pi}{x}\ dx.
    \end{equation}
    Clearly the right hand side converges to $0$ since it is enveloped by $x.$ The right hand side can be considered as follows.
    Look at the intervals $[1/(k+1), 1/k]$. In this case,  f
    we can bound the integral along this interval by the rectangle of area
    \begin{equation}
    	B_k = \frac{1}{k} - \frac{1}{k+1} = \frac{1}{(k+1)k}.
    \end{equation}
    Accounting for the negative oscilation of the $\cos(\pi/k)$ we get that for $k$ even
    \begin{equation}
    	\int_{\frac{1}{k+2}}^\frac{1}{k} \cos\frac {\pi}{x}\ dx \leq \frac{2}{(k+2)(k+1)k}. 
    \end{equation}
    This is obvious since
    \begin{equation}
    	\int_{\frac{1}{k+2}}^{\frac{1}{k+1}} \cos\frac{\pi}{x}\ dx \leq \sum_{\frac{1}{k+2}}^\frac{1}{k+1} -1\;\;\;\;
    	\int_{\frac{1}{k+1}}^{\frac{1}{k}} \cos\frac{\pi}{x}\ dx \leq \sum_{\frac{1}{k+}}^\frac{1}{k} 1
    \end{equation}
    and we take the sum of the bound and get the same inequality. Essentially we are given a very nice bound
    \begin{equation}
    	\ 0 \leq \int_{\frac{1}{k+2}}^\frac{1}{k} \cos \frac{\pi}{x}\ dx
    	 \leq \frac{2}{(k+2)(k+1)k}.
    \end{equation}
    We then know that the difference of $k=m$ and $k=n$ decreases at least cubically and
    so the series of summing the integrals is cauchy and bounded by the series
    \begin{equation}
    	a = \sum_{k=0}^\infty \frac{2}{(k+2)(k+1)k}.
    \end{equation}
    So the function itself is Riemann integrable (improperly)!

    Now look at the Labesgue integrability condition. It must be that $|f|$ has finite integrable. The absolute value of $f$
    however has area lowerbounded by $\sum 1/100k$ which diverges, therefore it could not be that the area under $|f|$ be finite
    and so the function is not Labesgue integrable. 

    \setcounter{enumi}{47}
    \item The set $S_A$ is where $J'$ is greatert than $a$ we can cover the $S_A$ with intervals $x, x+h$ using a vitali covering. WE can extract
    disjoint guys who effectively do the covering and whoe total length is approximateley the total measure of $S_A.$
    \setcounter{enumi}{49}
    \item Recall Theorem $66$ from the book.
    \begin{theorem}
        The circle or equivalently $[0,1)$, splits into two nonmeasurable disjoint subset, that each has inner measure zero and outer measure one.
    \end{theorem}
    We then set out to prove the following theorem.
    \begin{theorem}
        Every measurable $E \subset \mathbb{R}$ with $mE > 0$ contains a nonmeasurable set $N$ with $m^*N = mE, m_*N = 0$ and for each 
        $E' \subset E$ we have $m(E') = m^*(N\cap E').$
    \end{theorem}
    \begin{proof}
        Since $E$ is measurable it is the union of an open set $F$ and a zeroset $Z.$ Since $F \subset \mathbb{R}$ it is the countable union of
        disjoint open intervals
        \begin{equation}
            F = \bigsqcup_{i=1}^{\eta\in \mathbb{Z} \cup \{\infty\}} I_i.
        \end{equation}
        Let $T_i$ be the rigid transformation which maps $(0,1) \to I_i$ with absolute determinant $m(I_i).$ 
        Then take doppleganger set of $(0,1), N_{(0,1)}$ and consider the new set
        \begin{equation}
        N = \bigsqcup_{i=1}^\eta T_i(N_{(0,1)})
        \end{equation}
        with $\eta$ as before. This set has inner measure $0$ and outer measure $\sum_{i} m(I_i) = m(E)$ and so is not measurable!

        Now take any measurable subset of $E$, say $E'.$ In the same sense that we constructed $N,$ $N \cap E'$
        is also not measurable. It furthermore follows that $m^*(N \cap E') = m^*(E')$ since we construct $E$
        as a disjoint union of open intervals all of which are strict subsets of the respective $I_i$ forming $E.$

        To complete this statement, we must show that the outer measure of these subsets, say $I_i'$ intersect $T_i(N_{(0,1)})$
        have measure $m(I_i').$  Outer measure is importantly additive, therefore 
        \begin{equation}
            \begin{aligned}
                 m^*(I_i) = m^*(I_i') + m^*({I_i'}^c\cap I_i) &= m^*(T_i(N_{(0,1)})) \\
                &= m^*((T(N_{(0,1)}) \cap I_i')^{c}\cap I_i) + m^*(I_i') \\
               \implies m^*(I_i') &= m^*(T(N_{(0,1)}) \cap I_i')
            \end{aligned}
        \end{equation}
        using Lemma $20$.
        This completes the proof.
    \end{proof} 
    \setcounter{enumi}{51}
    \item Show the following theorem.
    \begin{theorem}
        If $f$ is a measurable function then
        \begin{equation}
            (dp(\mathcal{U}f) \cap \mathcal{U}f)^y = dp(\mathcal{U}f^y) \cap \mathcal{U}f^y.
        \end{equation}    
     \end{theorem} 
     \begin{proof}
        W must show that every $x \in (dp(\mathcal{U}f) \cap \mathcal{U}f)^y = dp(\mathcal{U}f^y)$ is also a member of  $dp(\mathcal{U}f^y) \cap \mathcal{U}f^y.$
        If $x \in (dp(\mathcal{U}f) \cap \mathcal{U}f)^y$ then equivalently
        we have
        \begin{equation}
            x \in \{p = (\rho,\gamma) \ |\  \gamma = y \wedge p \in \mathcal{U}f \wedge p\ \text{density point of }\mathcal{U}f \}
          \end{equation}  
     By Theorem $49,$ we have that $\rho$ is a density point of $\mathcal{U}f^y$. Next we must show that if $\gamma = y$ and $p \in \mathcal{U}f$
     then $x \in \mathcal{U}f^y$ This however is the definitions of $\mathcal{U}f^y.$ So it must be that $x \in dp(\mathcal{U}f^y) \cap \mathcal{U}f^y.$.

     In the opposite direction we can again use the same logic. Theorem $49$
     puts $x \in \mathcal{U}f^y $ in $(\mathcal{U}f)^y.$ Applying the same under graph logic as before the proof is complete.
     \end{proof}
43, 46,48,50,52
\end{menumerate}
%%%%%%% Be sure to set the counter and use menumerate

\end{document}