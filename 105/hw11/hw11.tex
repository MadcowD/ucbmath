%%%%%%%%%%%%%%%%%%%%%%%%%%%%%%%%%%%%%%%%%%%%%%%%%%%%%%%%%%%%%%%%%%
%%%                      Homework _                            %%%
%%%%%%%%%%%%%%%%%%%%%%%%%%%%%%%%%%%%%%%%%%%%%%%%%%%%%%%%%%%%%%%%%%

\documentclass[letter]{article}

\usepackage{lipsum}
\usepackage[pdftex]{graphicx}
\usepackage[margin=1.5in]{geometry}
\usepackage[english]{babel}
\usepackage{listings}
\usepackage{amsthm}
\usepackage{amssymb}
\usepackage{framed} 
\usepackage{amsmath}
\usepackage{titling}
\usepackage{fancyhdr}
\usepackage[mathcal]{euscript}
\pagestyle{fancy}


\newtheorem{theorem}{Theorem}
\newtheorem{lemma}{Lemma}
\newtheorem{definition}{Definition}

\newenvironment{menumerate}{%
  \edef\backupindent{\the\parindent}%
  \enumerate%
  \setlength{\parindent}{\backupindent}%
}{\endenumerate}







%%%%%%%%%%%%%%%
%% DOC INFO %%%
%%%%%%%%%%%%%%%
\newcommand{\bHWN}{11}
\newcommand{\bCLASS}{MATH 105}

\title{\bCLASS: Homework \bHWN}
\author{William Guss\\26793499\\wguss@berkeley.edu}

\fancyhead[L]{\bCLASS}
\fancyhead[CO]{Homework \bHWN}
\fancyhead[CE]{GUSS}
\fancyhead[R]{\thepage}
\fancyfoot[LR]{}
\fancyfoot[C]{}
\usepackage{csquotes}

%%%%%%%%%%%%%%

\begin{document}
\maketitle
\thispagestyle{empty}


%%%%%%% Be sure to set the counter and use menumerate
53,55,56,57,60,61,62
\begin{menumerate}
	\setcounter{enumi}{52}
	\item Let the function $f: \mathbb{R}^2 \to \mathbb{R}$ be defined
	\begin{equation}
		f(x,y) = \left\{ \begin{array}{ll}
		\frac{1}{y^2} & \text{if } 0 < x < y < 1 \\
		\frac{-1}{x^2} & \text{if } 0 < y < x < 1 \\
		0 & \text{otherwise}
		\end{array} \right.
	\end{equation}
	\begin{menumerate}
		\item We claim that the iterated integrals exist, but the function is not integrable. \\
		Fix y. Then
		\begin{equation*}
			\begin{aligned}
				\int_\mathbb{R} \int_\mathbb{R} f(x,y)\ d\mu(x)\ d\mu(y) &= \int_\mathbb{R} \int_0^y \frac{1}{y^2}\ d\mu(x) + \int_y^1 \frac{-1}{x^2}\ d\mu(x)\ d\mu(y)\\
				& = \int_\mathbb{R} \chi_{[0,1]}\left(\frac{1}{y} + 1 - \frac{1}{y}\right)\ d\mu(y)= \int_\mathbb{R} \chi_{[0,1]}\ d\mu(y) = 1 
			\end{aligned}
		\end{equation*}
		Taking the other iterated integral, fix $x$. Then
		\begin{equation*}
			\begin{aligned}
				\int_\mathbb{R} \int_\mathbb{R} f(x,y)\ d\mu(y)\ d\mu(x) &= \int_\mathbb{R}\ \int_0^x \frac{-1}{x^2}\ d\mu(y) + \int_x^1 \frac{1}{y^2} \ d\mu(y)\  d\mu(x)\\
				&= \int_\mathbb{R} \chi_{[0,1]}\left(\frac{-x}{x^2} -1 + \frac{1}{x} \right)\ d\mu(x) = -1.
			\end{aligned}
		\end{equation*}
		Since the iterated integrals are not equal, the function is not integrable as it would contradict Calvelerleirleirleri's Principle.
		\item In this case $f: \mathbb{R}^2 \not \to [0, \infty)$.
	\end{menumerate}
	\setcounter{enumi}{54}
	\item We solve for the densities.
	\begin{menumerate}
		\item For the disk, we find that the balanced densities are the interor. This is because on the exterior local linearity of the edge of the disk gives only $1/2$ of
		the volume of the cube contained within the disk. The upper is defined every where since we take the cubes inside the disk one of whose corners is the point. The lower gives us only the interior
		with the border having lower density $0.$

		The general density is defined on the interior.
		\item For the square, the balanced density is $1$ everywhere except for the borders, for the same reason as the disk. In fact all of the densities are the same.
		\item Tripple ***** Inner is not defined and outer is defined on the boundary points.
	\end{menumerate}
	\item Suppose that $P \subset \mathbb{R}$ has the property that for every interval $(a,b) \subset \mathbb{R}$ we have
	\begin{equation}
		\frac{m^*(P \cap (a,b))}{b-a} = \frac{1}{2}.
	\end{equation}
	\begin{menumerate}
		\item Measurability.
		\begin{theorem}
			$P$ is nonmeasurable.
		\end{theorem}
		\begin{proof}
			Suppose that $P$ were measurable. The labesgue measure theory says that every point is a density point. By definition no point of $P$ is a density point. Contradiction!
		\end{proof}
		\item The density always being $1/2$ gives that every point of $P$ is always a boundry.
	\end{menumerate}
	\item Steinhaus!
	\begin{lemma}
		If $F \subset (a,b)_n \subset \mathbb{R}^n$ is measurable and disjoint from its $t$-translate then
		\begin{equation}
			2mF \leq (b_1-a_1)\dots(b_n - a_n) + |t|^n 
		\end{equation}
	\end{lemma}
	\begin{proof}
		$F$ and its $t$-translate have equal measure, so if they do not intersect then their total measure is $2mF$. Any box that contains them must have measure greater than or equal to $2mF$.
		The box 
		\begin{equation}
			B = \times_{i=1}^n (a_i + t_i^+, b_i + t_i^-)
		\end{equation}
		where $t_i^+ = t_i$ when $t_i > 0$ and $0$ otherwise, and $t_i^- = t_i$ when $t_i < 0$ and $0$ otherwise. This box contains $F$ and its $t$-translate.
		The volume of this box is just $(b_1-a_1)\dots(b_n - a_n) + |t|$. 
	\end{proof}
	\begin{lemma}
			If $F \subset (a,b)_n \subset \mathbb{R}^n$ is measurable and disjoint from its $t$-translate along the coordinate axis
		\begin{equation}
			2mF \leq mB + (b-a)^{n-1}|t|
		\end{equation}
	\end{lemma}
	\begin{proof}
		We consider the $t$ translate along axis since in the other cases we get meeting and that's all that matters for the steinhouse proof.
		$F$ and its $t$-translate have equal measure, so if they do not intersect then their total measure is $2mF$. Any box that contains them must have measure greater than or equal to $2mF$.
		The box 
		\begin{equation}
			B = (t \pm a,b \pm+ t)_n
		\end{equation}
		contains the $F$ and its translate along the axis and so 
		The volume of this box is just $mB + (b-a)^{n-1}|t|$.
	\end{proof}
	\begin{theorem}
		If $E \subset \mathbb{R}^n$ is measurable and has positive measure then there exists a $\delta > 0$ such tha for all $t\in (-\delta, \delta)$ the $t$-translate of $E$ meets $E.$ 
	\end{theorem}
	\begin{proof}
		By the Labesgue Density Theorem $E$ has lots of density points so we can find a box $(a_i,b_i)_n$ in which $E$ has concentration more than $1/2$.
		Call $F = E \cap (a_i,b_i)_n$. Then $mF > (b-a)^n/2$. By the previous two lemmas we know that 
		\begin{equation}
			|t| < \min\left\{\frac{2mF - (b_i-a_i)^n}{(b_i - a_i)^{n-1}}, \left(
				2mF - (b_1-a_1)\dots(b_n - a_n)
			\right)^{\frac{1}{n}}\right\}
		\end{equation}
		then the $t$ translate of $F$ meets $F$, 
		so $t$-translate of $E$ meets $E$, which is what the theorem assert. 
	\end{proof}


	\setcounter{enumi}{59}
	\item Starwars.
	 \begin{theorem}
		Satelite densities are equivalent to shrinking cube densities.
	\end{theorem}
	\begin{proof}
		Suppose that $p \in E$ were satelite dense. We wish to show for arbitrary cube sequences $Q\downarrow p$ that the density calculated is equivalent.

		Take one particular cube sequence. Then take the filling satelite trajectory. In particular take for every $Q \in (Q)$ the cube shrinkage sequence, nearly fill $Q$ with finiteley many satelite cubes $(S_i)_{i=n}^{k}$ of side length $diam Q/2 - diam Q/10000000$ which do not touch $p$. Then
		\begin{equation}
		\left|\frac{m(E \cap Q)}{m(Q)} - \frac{m(E \cap \bigcup_{i=n}^k S_i)}{m\left(\bigcup_{i=n}^k S_i\right)}\right| \to 0.
		\end{equation}
		In this case these cubes are within a distance on order $l = 87$ of $p$ and the densities are equivalent in the limit. 

		In the other direction, we wish to show that if $p$ is shrinkage cube dense in $E$ then it is satelite cube dense in $E$.

		If $p$ is dense, then for every $\epsilon >0$ there is an $r$ such that for all cubes $Q$ with $d(p,Q) < r$,
		\begin{equation}
			\frac{m(E \cap Q)}{m(Q)} > 1- \epsilon.
		\end{equation}
		Take any satelite sequence approaching $p$. This satelite sequence will eventually be within radius $r/l$ of $p$. It's density must be more than $1 - \epsilon$
		otherwise we would contradict $p$ dense.
		Therefore, for every satelite sequence we have that the satelite density is $1$ and so $p$ is dense. 

		This completes the proof.
	\end{proof}

	We have shown that satelite density and shrinkag cube density share the same dingles (densities).
	\item Quasiroundness.
	\begin{menumerate}
		\item Nice shapes
		\begin{theorem}
			Squares and equilateral triangles are uniformly $K$-Quasiround.
		\end{theorem}
		\begin{proof}
		In this case take $K = 100000.$ Then
		in the least the smallest ball contained in the equilateral triangle has $diam B = 1/sqrt{3} diam U$. Therefore $K$ need be at least $3\sqrt{3}.$ For squares $K$ need be proportional and higher than $2\sqrt{2}$ so $K=100000$ works.
		\end{proof}
		\item Bad shapes!
		\begin{theorem}
		Isosceles triangles are not uniformly $K$ Quasi-round.
		\end{theorem}
		\begin{proof}
			Consider the family of isoceles triangles with base $b=1$ and height increasing with $n$. The internal ball is maximally bounded by a ball of radius $1/2$ but the external ball is bounded by a radius of $n/2$, and so $K$ is not fixed for the family.
		\end{proof}
		\item Annuli
		\begin{theorem}
			Let $\mathcal{A}_{10}$ denote the set of Annuli with inner radius $r$ and outer radius $R$ such that $R/r \leq 10$. Then, $\mathcal{A}_r$ is uniformly $K$-quasiround.
		\end{theorem}
		\begin{proof}
			Take the any ball $B$ inside of an $a \in \mathcal{A}_{10}$. This ball has diameter $R-r$. Then take the ball $B'$ of radius $R$ containing the $a$.
			We have that $diam B' / diam B  = 1 - r/R.$ This gives us
			\begin{equation}
			 diam\ B' \leq (1 - r/r)diam\ B \leq (1 - 1/10) diam\  B
			 \end{equation} 
			 So take $K = 100$ and $\mathcal{A}_{10}$ is $K$-Quasiround.
		\end{proof}
	\end{menumerate}
	\item Equivalence of K-Quasiroundness. I worked with Lucas and AA Robotis. 
	\begin{theorem}
		K-Quasiroundness implies Measure Theoretic Quasiroundness.
	\end{theorem}
	\begin{proof}
		Suppose that $W \in \mathcal{W} $ is geometricly K-Quasiround. Then we know that there exist $B \subset W \subset B'$
		so that $diam\ B' \leq K^{1/n} diam B.$
		Then we have that 
		\begin{equation}
				 \frac{diam\ B'}{diam\ B} \leq K^{1/n} \implies \frac{\alpha diam(B')^n}{m(W)} \leq \frac{\alpha diam(B')^n}{\alpha diam(B)^n} \leq \alpha K.
		\end{equation}
	\end{proof}
	\begin{theorem}
		Fix a point $p \in \mathbb{R}^n$, and let 
		$\mathcal{W}_K$ be \textbf{the} family of measurable sets containing $p$ which are $K$-quasi-round in the measure theoretic sense. It follows that $p$ is a density point if and only if the concetration of $E$ in $W$ tends to $1$ as $W \in \mathcal{W}_K$ shrinks to $p.$
	\end{theorem}
	\begin{proof}
		Lebesgue premeasure is invariant to the norm so take the $l^\infty$ norm. In this case we know that diameter of the box to the $n$th power 
		is equivalent to its measure. Therefore we will consider all cases of $K.$

		Suppose that $W \in \mathcal{W}_K$ shrinks to $p$ and the concentration of $E$ in $W$ tends to $1$. Every ball is contained in $\mathcal{W}_K$ for $K \geq 1$, so take
		the subset of $l^{\infty}$ cubes $B_i$ shrinking to $P$.
		We know that as $B_i \downarrow p$ 
		\begin{equation}
			\frac{m(B_i \cap E)}{m(B_i)} \to 1
		\end{equation}
		which implies that $p$ is dense, since density is equivalent for ball and cube definitions.

		In the other direction.

		Suppose thjat $p$ is a density point in some set $E.$ Because we have the measure theoretic $K$-quasi-roundness, we know that for any $W \in \mathcal{W}_K$
		\begin{equation}
			\frac{diam(W)^n}{m(W)} = K \implies \frac{mW}{diam(W)^n} = \frac{1}{K} =K'.
		\end{equation}

		If we take $B$ to be a cube with $diam(W)$ containg $W$. Then we can get the inequality
		\begin{equation}
			1 - \frac{mB - m*(B \cap E)}{mW} \leq \frac{m(E \cap W)}{mW} \leq 1.
		\end{equation}
		As $W \downarrow p$ we know that that the diameter of $W$ tends to $0$ and therefore since $m(B) - m(B \cap E) \to 0$ we get
		\begin{equation}
			\frac{m(E \cap W)}{mW} \to 1.
		\end{equation}
		This completes the proof.
	\end{proof}
	Allowable norms as a function of $K$ :)


\end{menumerate}
\end{document}