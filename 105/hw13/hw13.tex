%%%%%%%%%%%%%%%%%%%%%%%%%%%%%%%%%%%%%%%%%%%%%%%%%%%%%%%%%%%%%%%%%%
%%%                      Homework _                            %%%
%%%%%%%%%%%%%%%%%%%%%%%%%%%%%%%%%%%%%%%%%%%%%%%%%%%%%%%%%%%%%%%%%%

\documentclass[letter]{article}

\usepackage{lipsum}
\usepackage[pdftex]{graphicx}
\usepackage[margin=1.5in]{geometry}
\usepackage[english]{babel}
\usepackage{listings}
\usepackage{amsthm}
\usepackage{amssymb}
\usepackage{framed} 
\usepackage{amsmath}
\usepackage{titling}
\usepackage{fancyhdr}
\usepackage[mathcal]{euscript}
\pagestyle{fancy}


\newtheorem{theorem}{Theorem}
\newtheorem{lemma}{Lemma}
\newtheorem{definition}{Definition}

\newenvironment{menumerate}{%
  \edef\backupindent{\the\parindent}%
  \enumerate%
  \setlength{\parindent}{\backupindent}%
}{\endenumerate}


\usepackage{accents}
\newcommand{\ubar}[1]{\underaccent{\bar}{#1}}




%%%%%%%%%%%%%%%
%% DOC INFO %%%
%%%%%%%%%%%%%%%
\newcommand{\bHWN}{13}
\newcommand{\bCLASS}{MATH 105}

\title{\bCLASS: Homework \bHWN}
\author{William Guss\\26793499\\wguss@berkeley.edu}

\fancyhead[L]{\bCLASS}
\fancyhead[CO]{Homework \bHWN}
\fancyhead[CE]{GUSS}
\fancyhead[R]{\thepage}
\fancyfoot[LR]{}
\fancyfoot[C]{}
\usepackage{csquotes}

%%%%%%%%%%%%%%

\begin{document}
\maketitle
\thispagestyle{empty}


%%%%%%% Be sure to set the counter and use menumerate
\begin{menumerate}
	\setcounter{enumi}{80}
	\item \textbf{Stronger Average Value Theorem.}
	\begin{theorem}
		If $f$ is a measurable function then for all most every $p$ in its domain we have that 
		\begin{equation}
			\lim_{Q\downarrow p} \frac{1}{mQ} \int_Q |f- fp|\ d\mu(x) = 0
		\end{equation}
		\begin{proof}
			Get an enumeration of $\mathbb{Q},$ say $\{a_n\}$ there is a sequence $a^{fp}_n \to fp.$ Finally
			consider that for every $n$ the function $|f -a_n|$ is measurable. So we let $f_n^{fp}(x) = |f(x) - a^{fp}_n|.$
			The limit is measurable. By the average value theorem
			\begin{equation}
				\lim_{Q\downarrow p} \frac{1}{mQ} \int_Q |f_n-  a^{fp}_n|\ d\mu(x) = |f_p - a^{fp}_n|.
			\end{equation}
			As $ a^{fp}_n\to fp$ the right hand side  tends towards to $0$ and therefore
			 \begin{equation}
			 	\begin{aligned}
					0 &= \lim_{ n\to \infty}\lim_{Q\downarrow p} \frac{1}{mQ} \int_Q |f- a^{fp}_n|\ d\mu(x)  \\
					&=  \lim_{Q\downarrow p}  \frac{1}{mQ} \int_Q \lim_{ n\to \infty} |f- a^{fp}_n|\ d\mu(x)  \\
					&= \lim_{Q\downarrow p} \frac{1}{mQ} \int_Q |f- fp|\ d\mu(x) .
			 	\end{aligned}
			 \end{equation}
			 We can bring the limit inside by the measurability and uniform convergence of the functions.
			 This completes the proof.

		\end{proof}
	\end{theorem}


	\setcounter{enumi}{83}
	\item \textbf{Almost Absolutely Continuous Functions.}

	Lusin's Lemma extends to absolute continuity for the falling reasons. Take an $f$ satisfying the conditions in Lusin's Lemma. Then $f:[a,b] \to \mathbb{R}$ restricted to $E \subset [a,b]$ is continuous and $E$ is a bounded compact set. Since $f_{|E}$ is continuous on a bounded compact subset, then it is absolutely continuous on that subset. So $f$ satisfying Luzin's lemma is almost absolutely contionuous.
	The lemma used in this reasoning does not require that $f$ be bounded! `'

	\setcounter{enumi}{86}
	\item \textbf{Density Theoretic Boundries}
	\begin{menumerate}
		\item Measure theoretic boarder.
		\begin{theorem}
			If $E$ is a subset of $\mathbb{R}^n$ and $\partial E$ is its boarder then
		    $$\partial_m E \subset \partial E. $$
		\end{theorem}
		\begin{proof}
			If $p \in Ext_m(E)$ then clearly $d(p, E^c) =1$ and therefore $p \in E^c.$
			Conversely $\partial_m E \cup Int_m(E) = E.$ Suppose for the sake of contradiction
			that there exists a $p \in \partial_m E$ such that $p \in E^{o} = Int(E).$
			Then there exists an $r > 0$ such that all $x \in B(p,r)$ are in $E.$ Therefore
			$d(p,E) = 1.$ This a contradiction to $p \in \partial_m E$, so $p \in \partial E.$
			This completes the proof.
		\end{proof}

		\item Consider the following construction. Let $f: [-1,1] \to [0,2]$ such that $x \mapsto x^{2/3} + 1.$
		This function has a cusp at $x = 0$ whose walls get sharper and sharper. Imagine the point on the border of the completed undergraph
		at $x = 0.$ As you shrink the ball it encompasses more of the area on the graph. Untill eventually the limit is one.
		See the picture: \\[3in]
	\end{menumerate}
	\item \textbf{Topological Riemann Integrability}
	\begin{theorem}
		Let $X$ be a compact hypercube in $\mathbb{R}^n$. A function $f: X \to [0, M]$ is Riemann integrable if and only if
		$m(\partial \mathcal{U} f) = 0.$
	\end{theorem}
	\begin{proof}
		Recall that Lemma 69 holds for any arbitrary metric space. Therefore,
		\begin{equation}
			\mathcal{U} \ubar{f} = int(\mathcal{U}f)\;\wedge\;\hat{\mathcal{U}} \bar{f}  = \overline{\mathcal{U} f}
		\end{equation}
		Since open sets and closed set are measurable in $\mathbb{R}^n$, then $\ubar{f}$ and $\overline{f}$ are measurable functions.
		Thus
		\begin{equation}
			m(\partial(\mathcal{U} f)) = m(\overline{\mathcal{U} f} \setminus int(\mathcal{U} f)) 
			= m(\hat{\mathcal{U}} \bar{f}) - m(\mathcal{U} \ubar{f}) = \int_X \bar{f} - \ubar{f}.
		\end{equation}

		Lebesgue theory tells us that the integral is zero if and only if $\bar{f} = \ubar{f}$ almost everywhere, i.e. $f$ is continuous if and only if $f$ is continuous everywhere ($lim_{t\to x} f(t).$), i.e. by the Multivariate Riemann-Lebesgue Theorem if and only if $f$ is Riemann integrable.
	\end{proof}
\end{menumerate}
\end{document}