%%%%%%%%%%%%%%%%%%%%%%%%%%%%%%%%%%%%%%%%%%%%%%%%%%%%%%%%%%%%%%%%%%
%%%                      Homework _                            %%%
%%%%%%%%%%%%%%%%%%%%%%%%%%%%%%%%%%%%%%%%%%%%%%%%%%%%%%%%%%%%%%%%%%

\documentclass[letter]{article}

\usepackage{lipsum}
\usepackage[pdftex]{graphicx}
\usepackage[margin=1.5in]{geometry}
\usepackage[english]{babel}
\usepackage{listings}
\usepackage{amsthm}
\usepackage{amssymb}
\usepackage{framed} 
\usepackage{amsmath}
\usepackage{titling}
\usepackage{fancyhdr}

\pagestyle{fancy}


\newtheorem{theorem}{Theorem}
\newtheorem{definition}{Definition}

\newenvironment{menumerate}{%
  \edef\backupindent{\the\parindent}%
  \enumerate%
  \setlength{\parindent}{\backupindent}%
}{\endenumerate}







%%%%%%%%%%%%%%%
%% DOC INFO %%%
%%%%%%%%%%%%%%%
\newcommand{\bHWN}{2}
\newcommand{\bCLASS}{MATH H105}

\title{\bCLASS: Homework \bHWN}
\author{William Guss\\26793499\\wguss@berkeley.edu}

\fancyhead[L]{\bCLASS}
\fancyhead[CO]{Homework \bHWN}
\fancyhead[CE]{GUSS}
\fancyhead[R]{\thepage}
\fancyfoot[LR]{}
\fancyfoot[C]{}
\usepackage{csquotes}

%%%%%%%%%%%%%%

\begin{document}
\maketitle
\thispagestyle{empty}

\begin{menumerate}
    \setcounter{enumi}{14}
    \item 
    \begin{theorem}
        Let $f: \mathbb{R}^2 \to \mathbb{R}$ be a function so that
        \begin{equation}
            \begin{aligned}
            (x,y) \mapsto \frac{xy}{x^2 + y^2} \\
            (0,0) \mapsto 0.          
            \end{aligned}          
        \end{equation}   
        Then, $f$ has partial derivatives at $(0,0)$ but is not differentiable
        there.
    \end{theorem}
    \begin{proof}
        By definition we take the partial derivative to be the limit
        \begin{equation}
            \begin{aligned}
            \frac{\partial f}{\partial x} &= \lim_{t\to 0} \frac{f(x+h, y) - f(x,y)}{h} \\
            &= \lim_{t\to 0} \frac{{(h)^2}}{h} \\
            &= 0.
            \end{aligned}
         \end{equation} 
         Since the closed form for $f$ is identical, we have the same definition
         for $\partial f / \partial y.$

         However, if $f$ is differentiable, then it is continuous at
         $(0,0).$ But the limik along $y = x$, does not exist unabiguously
         $$\lim_{x\to 0} \frac{x^2}{x^2 + x^2} = \frac12 \neq 0 .$$
         Therefore it could not possibly differentiable at $(0,0).$
    \end{proof}
    \item Yass!!!!!!!!!!!!!!!!!!!!!!!!
        
             We build the matrix of partials accordingly! 
            Using partial differnetiation we get 
            \begin{equation}
                (Df)_p = \left[
                \begin{array}{cc}
                    1 & 0 \\
                    \cos 1 & 0 \\
                    \sin 1 & 0
                \end{array}
                \right].
            \end{equation}
        \begin{equation}
                (Dg)_q = \left[
                \begin{array}{ccc}
                     0 & 0 & 0
                \end{array}
                \right].
            \end{equation}
         \begin{equation}
            (Dg\circ f)_p = (Dg)_q\circ (Df)_p = 0.
            \end{equation}
        We get that 
        \begin{equation}
            g \circ f = w(s,t) = (st)(s\cos t) + (s\cos t) (s \sin t) + (s \sin t)(st)       
        \end{equation}
        and so the pial derivatives at least contain $s$ in every term:
        \begin{equation}
        \begin{aligned}
            D_s w &= 2st(\cos t) + 2s \cos t \sin t + 2s t \sin t \\
            D_t w &= s^2(\cos t - t\sin t) + s^2(\cos t \cos t - \sin t \sin t) + s^2 (\sin t + t \cos t)
         \end{aligned} 
        \end{equation}
        These partials evaluate to 0 and so are 0.

        The statement of multivariable chain rule for functions $g: \mathbb{R} \to \mathbb{R}^m, f:\mathbb{R}^m \to \mathbb{R}$ is that
        $d/dt f \circ g = \sum \partial f/\partial g_i \partial g_i /\partial t$ which is the row vector
        matrix $Df$ with the column vector $Dg.$
    \item
    \item
    \item
    \item
    \item

    \setcounter{enumi}{23}
    \item
    \item
\end{menumerate}

%%%%%%% Be sure to set the counter and use menumerate

\end{document}