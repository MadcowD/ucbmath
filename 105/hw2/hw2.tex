%%%%%%%%%%%%%%%%%%%%%%%%%%%%%%%%%%%%%%%%%%%%%%%%%%%%%%%%%%%%%%%%%%
%%%                      Homework _                            %%%
%%%%%%%%%%%%%%%%%%%%%%%%%%%%%%%%%%%%%%%%%%%%%%%%%%%%%%%%%%%%%%%%%%

\documentclass[letter]{article}

\usepackage{lipsum}
\usepackage[pdftex]{graphicx}
\usepackage[margin=1.5in]{geometry}
\usepackage[english]{babel}
\usepackage{listings}
\usepackage{amsthm}
\usepackage{amssymb}
\usepackage{framed} 
\usepackage{amsmath}
\usepackage{titling}
\usepackage{fancyhdr}

\pagestyle{fancy}


\newtheorem{theorem}{Theorem}
\newtheorem{definition}{Definition}

\newenvironment{menumerate}{%
  \edef\backupindent{\the\parindent}%
  \enumerate%
  \setlength{\parindent}{\backupindent}%
}{\endenumerate}







%%%%%%%%%%%%%%%
%% DOC INFO %%%
%%%%%%%%%%%%%%%
\newcommand{\bHWN}{2}
\newcommand{\bCLASS}{MATH H105}

\title{\bCLASS: Homework \bHWN}
\author{William Guss\\26793499\\wguss@berkeley.edu}

\fancyhead[L]{\bCLASS}
\fancyhead[CO]{Homework \bHWN}
\fancyhead[CE]{GUSS}
\fancyhead[R]{\thepage}
\fancyfoot[LR]{}
\fancyfoot[C]{}
\usepackage{csquotes}

%%%%%%%%%%%%%%

\begin{document}
\maketitle
\thispagestyle{empty}

\begin{menumerate}
    \setcounter{enumi}{14}
    \item 
    \begin{theorem}
        Let $f: \mathbb{R}^2 \to \mathbb{R}$ be a function so that
        \begin{equation}
            \begin{aligned}
            (x,y) \mapsto \frac{xy}{x^2 + y^2} \\
            (0,0) \mapsto 0.          
            \end{aligned}          
        \end{equation}   
        Then, $f$ has partial derivatives at $(0,0)$ but is not differentiable
        there.
    \end{theorem}
    \begin{proof}
        By definition we take the partial derivative to be the limit
        \begin{equation}
            \begin{aligned}
            \frac{\partial f}{\partial x} &= \lim_{t\to 0} \frac{f(x+h, y) - f(x,y)}{h} \\
            &= \lim_{t\to 0} \frac{{(h)^2}}{h} \\
            &= 0.
            \end{aligned}
         \end{equation} 
         Since the closed form for $f$ is identical, we have the same definition
         for $\partial f / \partial y.$

         However, if $f$ is differentiable, then it is continuous at
         $(0,0).$ But the limik along $y = x$, does not exist unabiguously
         $$\lim_{x\to 0} \frac{x^2}{x^2 + x^2} = \frac12 \neq 0 .$$
         Since $(Df)_0(0,1) = 0, (Df)_0(1,0) = 0, $ and the derivative must be linear, it follows that $Df_0(1,1) = 0 \neq \frac12.$ So $f$
         couldn't be differentiable. 
    \end{proof}
    \item Yass!!!!!!!!!!!!!!!!!!!!!!!!
        
             We build the matrix of partials accordingly! 
            Using partial differnetiation we get 
            \begin{equation}
                (Df)_p = \left[
                \begin{array}{cc}
                    1 & 0 \\
                    \cos 1 & 0 \\
                    \sin 1 & 0
                \end{array}
                \right].
            \end{equation}
        \begin{equation}
                (Dg)_q = \left[
                \begin{array}{ccc}
                     0 & 0 & 0
                \end{array}
                \right].
            \end{equation}
         \begin{equation}
            (Dg\circ f)_p = (Dg)_q\circ (Df)_p = 0.
            \end{equation}
        We get that 
        \begin{equation}
            g \circ f = w(s,t) = (st)(s\cos t) + (s\cos t) (s \sin t) + (s \sin t)(st)       
        \end{equation}
        and so the pial derivatives at least contain $s$ in every term:
        \begin{equation}
        \begin{aligned}
            D_s w &= 2st(\cos t) + 2s \cos t \sin t + 2s t \sin t \\
            D_t w &= s^2(\cos t - t\sin t) + s^2(\cos t \cos t - \sin t \sin t) + s^2 (\sin t + t \cos t)
         \end{aligned} 
        \end{equation}
        These partials evaluate to 0 and so are 0.

        The statement of multivariable chain rule for functions $g: \mathbb{R} \to \mathbb{R}^m, f:\mathbb{R}^m \to \mathbb{R}$ is that
        $d/dt f \circ g = \sum \partial f/\partial g_i \partial g_i /\partial t$ which is the row vector
        matrix $Df$ with the column vector $Dg.$
    \item Multidimensional Mean Value Theorem
    \begin{menumerate}
        \item Vector valued functions!
         \begin{theorem}
            Let $n =1, m = 2.$ Then if 
            \begin{equation}
                f(t) = (\cos t, \sin t)
            \end{equation}
            for $\pi \leq 2\pi$ and $p = \pi, q = 2\pi,$ then there is
            no $\theta  \in [p,q]$ which satisfies
            \begin{equation}
                f(p) - f(q) = (Df)_\theta(q-p)= 
                \begin{bmatrix}
                    -\sin \theta \\
                    \cos \theta
                \end{bmatrix}
                (q-p)
            \end{equation}
        \end{theorem}
        \begin{proof}
            Take $f(p) - f(q).$  This value is 
            \begin{equation}
                f(p) - f(q) = 
                \begin{bmatrix}
                    \cos 2\pi\\
                    \sin 2\pi
                \end{bmatrix}
                -
                \begin{bmatrix}
                    \cos \pi \\
                    \sin \pi
                \end{bmatrix}
                = \begin{bmatrix}
                    2 \\
                    0
                \end{bmatrix}.
            \end{equation} 
        \end{proof}
        It could not be that there is a $\theta$ such that $-\sin \theta,$ the first component of the derivative, is $2$. So the theorem holds.
        \item Convex derivative set.
        \begin{theorem}
            If the set of derivatives of $f: U \subset \mathbb{R}^n \to \mathbb{R}^m$,
            \begin{equation}
                S = \{(Df)_x \in \mathcal{L}(\mathbb{R}^n, \mathbb{R}^m) : rx \in [p,q]\}
            \end{equation}
            is convex, then there is a $\theta \in [p,q]$ satisfying 
            \begin{equation}
                f(p) - f(q) = (Df)_\theta(q-p).        
            \end{equation}.
        \end{theorem}
        \begin{proof}
            Since the set of derivatives is convex, given any two points in $S$ every point on the line segment between them is also in $S.$

        \end{proof}

    \end{menumerate}
    \item Directional derivative:
        \begin{menumerate}
            \item By Theorem 5 we know that if $f$ is differentiable, then its derivative is given by 
            \begin{equation}
            (Df)_p = \lim_{t\to 0}\frac{f(p+tu)-f(p)}{t},
            \end{equation}
            and since partial derivatives are given by letting $u$ be the basis to which the partial is tied, it follows that a 'directional' derivative would be given by a projection of each partial contribution of a component of $f$ onto a direction $u$. 

            See the proof of theorem 5, and corollary 7.

            However if $f$ is not differentiable, one must intuit from the formula. The limit observes the 'slope' in the component $f$ directions as $u \to p$ in th $u$ direction. That is if the hyrpersurface, $f(U)$ was sliced along the $u$ direction at $p$,
            the following formula follows
            \begin{equation}
                g(t) = f(p+tu),g(0) = f(p).
            \end{equation}
            So, $g'(t) = \nabla_pf(u)$ since $g' = \lim (g(t)-g(0))/t.$

            \item \begin{theorem}
                Let $f:\mathbb{R}^2 \to\mathbb{R}$ so that,
                \begin{equation}
                    x \mapsto \frac{x^3y}{x^2 + y^2}.
                \end{equation}
                The $f$ has directional derivatives, but is not differentiable.
            \end{theorem}
            \begin{proof}
                Take any $u = \begin{bmatrix}
                    a \\
                    b \\
                \end{bmatrix}$. Then the limit 
                \begin{equation}
                    \lim_{t\to 0} \frac{\frac{(ta)^3(tb)^2}{(ta)^4 +(tb)^2}}{t} = \frac{t^5a^3b^2}{t^3(t^2a^4+b^2)} = 0
                \end{equation}
                when $b \neq 0.$
                In the case that $b = 0,$ we have
                \begin{equation}
                    \lim_{t\to 0} \frac{0}{t^5a^4} = 0.
                \end{equation}

                To show that $f$ is not differentiable, we must show
                that for all suitable $T$
                \begin{equation}
                    f(p+v) = f(p) + T(v) + R(v) \wedge \lim_{|v| \to 0} \frac{R(v)}{|v|} \neq 0.
                \end{equation}
                Suppose that $f$ was differentiable. The only derivative could be the $0$ transformation since it unambiguously determines $\nabla_0 f.$

                So it follows that
                \begin{equation}
                    f(p+v) = f(p) + R(v) \implies \lim_{|v| \to 0} \frac{R(v)}{|v|} =    0 \implies \lim_{|v| \to 0} \frac{f(p+v)}{|v|} = \lim_{|v| \to 0} \frac{f(v) + R(v)}{|v|} = 0.
                \end{equation}
                and $f$ is sublinear! Now consider any approach, say $y=x^2.$ The limit had better be sublinear.
                \begin{equation}
                    \lim_{x\to 0} \frac{\frac{x^5}{2x^4}}{x}= 1,
                \end{equation}
                so $f$ is not sublinear along that curve. A contradiction to$f$ differentiable!

             \end{proof}
        \end{menumerate}
    \item
    \item
    \item

    \setcounter{enumi}{23}
    \item
    \item
\end{menumerate}

%%%%%%% Be sure to set the counter and use menumerate

\end{document}