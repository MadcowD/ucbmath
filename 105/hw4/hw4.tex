98%%%%%%%%%%%%%%%%%%%%%%%%%%%%%%%%%%%%%%%%%%%%%%%%%%%%%%%%%%%%%%%%%%
%%%                      Homework _                            %%%
%%%%%%%%%%%%%%%%%%%%%%%%%%%%%%%%%%%%%%%%%%%%%%%%%%%%%%%%%%%%%%%%%%

\documentclass[letter]{article}

\usepackage{lipsum}
\usepackage[pdftex]{graphicx}
\usepackage[margin=1.5in]{geometry}
\usepackage[english]{babel}
\usepackage{listings}
\usepackage{amsthm}
\usepackage{amssymb}
\usepackage{framed} 
\usepackage{amsmath}
\usepackage{titling}
\usepackage{fancyhdr}

\pagestyle{fancy}


\newtheorem{theorem}{Theorem}
\newtheorem{definition}{Definition}

\newenvironment{menumerate}{%
  \edef\backupindent{\the\parindent}%
  \enumerate%
  \setlength{\parindent}{\backupindent}%
}{\endenumerate}







%%%%%%%%%%%%%%%
%% DOC INFO %%%
%%%%%%%%%%%%%%%
\newcommand{\bHWN}{1}
\newcommand{\bCLASS}{MATH H105}

\title{\bCLASS: Homework \bHWN}
\author{William Guss\\26793499\\wguss@berkeley.edu}

\fancyhead[L]{\bCLASS}
\fancyhead[CO]{Homework \bHWN}
\fancyhead[CE]{GUSS}
\fancyhead[R]{\thepage}
\fancyfoot[LR]{}
\fancyfoot[C]{}
\usepackage{csquotes}

%%%%%%%%%%%%%%

\begin{document}
\maketitle
\thispagestyle{empty}


\begin{menumerate}
    \setcounter{enumi}{54}
    \item Take the following $dx_{321}$, $dx_{546}$. See their wedge product, 
    \begin{equation}
    \begin{aligned}
        dx_{321} &\wedge dx_{546} \\
        &= dx_{3} \wedge dx_{2} \wedge dx_{1} \wedge dx_{5} \wedge dx_{4} \wedge dx_{6} \\
        &= - dx_{2} \wedge dx_{3} \wedge dx_{1} \wedge dx_{5} \wedge dx_{4} \wedge dx_{6} \\
        &=  dx_{1} \wedge dx_{2} \wedge dx_{3} \wedge dx_{5} \wedge dx_{4} \wedge dx_{6} \\
        &= dx_{1} \wedge dx_{2} \wedge dx_{3} \wedge dx_{4} \wedge dx_{5} \wedge dx_{6} \\
        &=  dx_{12} \wedge dx_{3} \wedge dx_{4} \wedge dx_{5} \wedge dx_{6} \\
        &=  dx_{12} \wedge dx_{34} \wedge dx_{5} \wedge dx_{6} \\
        &=  dx_{12} \wedge dx_{34} \wedge dx_{56} \\
        &= dx_{12} \wedge dx_{3456} \\
        &= dx_{123456}.
    \end{aligned}
    \end{equation}
    \item False. Observe that $a = 0$ if and only if $a = -a.$ Take $\omega$ to be a $k$-form. Then observe that
    $\omega \wedge \omega = (-1)^{k^2} \omega \wedge \omega $ if we rearrange the wedge product. So for $k$ odd the square is odd 
    so $\omega \wedge \omega = 0$ except for when $k = 2n,$ $\omega \wedge \omega \neq 0.$
    \item Consider the forms $\alpha + \beta.$ Then, $d(\alpha + \beta) = d(fdx_I + g dx_I) = d((f + g)\wedge dx_I) = d(f+g) \wedge dx_I = (df + dg) \wedge dx_I
    = df\wedge dx_I + dg \wedge dx_I = d(\alpha) + d(\beta).$
    \item
    \item
    \item
    \item See 56.
\end{menumerate}

%%%%%%% Be sure to set the counter and use menumerate

\end{document}