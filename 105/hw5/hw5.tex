%%%%%%%%%%%%%%%%%%%%%%%%%%%%%%%%%%%%%%%%%%%%%%%%%%%%%%%%%%%%%%%%%%
%%%                      Homework _                            %%%
%%%%%%%%%%%%%%%%%%%%%%%%%%%%%%%%%%%%%%%%%%%%%%%%%%%%%%%%%%%%%%%%%%

\documentclass[letter]{article}

\usepackage{lipsum}
\usepackage[pdftex]{graphicx}
\usepackage[margin=1.5in]{geometry}
\usepackage[english]{babel}
\usepackage{listings}
\usepackage{amsthm}
\usepackage{amssymb}
\usepackage{framed} 
\usepackage{amsmath}
\usepackage{titling}
\usepackage{fancyhdr}

\pagestyle{fancy}


\newtheorem{theorem}{Theorem}
\newtheorem{definition}{Definition}

\newenvironment{menumerate}{%
  \edef\backupindent{\the\parindent}%
  \enumerate%
  \setlength{\parindent}{\backupindent}%
}{\endenumerate}







%%%%%%%%%%%%%%%
%% DOC INFO %%%
%%%%%%%%%%%%%%%
\newcommand{\bHWN}{5}
\newcommand{\bCLASS}{MATH 105}

\title{\bCLASS: Homework \bHWN}
\author{William Guss\\26793499\\wguss@berkeley.edu}

\fancyhead[L]{\bCLASS}
\fancyhead[CO]{Homework \bHWN}
\fancyhead[CE]{GUSS}
\fancyhead[R]{\thepage}
\fancyfoot[LR]{}
\fancyfoot[C]{}
\usepackage{csquotes}

%%%%%%%%%%%%%%

\begin{document}
\maketitle
\thispagestyle{empty}


%%%%%%% Be sure to set the counter and use menumerate
\begin{menumerate}
\setcounter{enumi}{64}
\item The winding (w)one form.
\begin{definition}
	We denote the winding one form, $d\theta$, such that
	\begin{equation}
		d\theta = \frac{-y}{r^2}dx + \frac{x}{r^2}dy.
	\end{equation}
\end{definition}
\begin{theorem}
	The winding one form is closed but not exact.
\end{theorem}
\begin{proof}
	We take the exterior derivative of $d\theta$. Using $d(fdx) = df \wedge dx$ we have the following:
	\begin{equation}
	\begin{aligned}
		d(d\theta) &= d\left(\frac{-y}{r^2}dx + \frac{x}{r^2}dy\right)\\
			&= d\left(\frac{-y}{r^2}dx\right) + d\left(\frac{x}{r^2}dy\right)\\
			&= d\left(\frac{-y}{r^2}\right) \wedge dx + d\left(\frac{x}{r^2}\right)
			\wedge dy \\
			&= \left(\frac{\partial}{\partial x}\frac{-y}{x^2+y^2}dx \frac{\partial}{\partial y}\frac{-y}{x^2+y^2}dy + \right) \wedge dx + d\left(\frac{x}{r^2}\right)
			\wedge dy \\
			&= \frac{y^2 -x^2}{r^4}dy \wedge dx + d\left(\frac{x}{r^2}\right) \wedge dy \\
			&= \frac{y^2 -x^2}{r^4}dy \wedge dx + \left(\frac{\partial}{\partial x}\frac{x}{r^2}dx + \frac{\partial}{\partial y} \frac{x}{r^2} dy\right) \wedge dy \\
			&= \frac{y^2 -x^2}{r^4}dy \wedge dx + \frac{\partial}{\partial x}\frac{x}{r^2}dx \wedge dy \\
			&= \frac{y^2 -x^2}{r^4}dy \wedge dx + \frac{y^2 - x^2}{r^4}dx \wedge dy = 0
	\end{aligned}	
	\end{equation}
	Substitution of $r\cos\theta$ for $x$ and similar for $y$ yields that the form is infact $d\theta$,  ($dx = \cos \theta dr - r\sin \theta d \theta).$ If the form were exact then its anti-exterior derivative should be $\theta,$ therefore its evaluation along a $1$ cell should be its net change along its end points.
	
	Consider the curve which takes the unit circle counter clockwise around the origin. 
	\begin{equation}
	 d\theta(c)= \int_0^\tau -\sin t dx + \cos t dy = \int_0^\tau dt = \tau.
	\end{equation}
	And $\tau = 2\pi \neq 0,$ but the net change in $\theta = 0$ So it could not be that this form is exact. 
\end{proof}
	The name $d\theta$ is totally misleading, it implies that $d\theta = d(\theta)$ which is false. 
	\setcounter{enumi}{67}
	\item Closedness of scalar multiplication.
	 \begin{theorem}
		If $\omega$ is closed then $f\omega$ is not necisarrily closed.	
	\end{theorem}
	\begin{proof}
		Take the exterior derivative of the expression and get
		\begin{equation}
		d(f\omega) = df \wedge \omega + f \wedge d\omega = df \wedge \omega.
		\end{equation}
		So if the differential of $f$ is $0$ then $f\omega$ is closed. Otherwise, no.
	\end{proof}
	\begin{theorem}
		
	\end{theorem}
\end{menumerate}

\end{document}