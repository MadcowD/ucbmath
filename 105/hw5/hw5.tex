%%%%%%%%%%%%%%%%%%%%%%%%%%%%%%%%%%%%%%%%%%%%%%%%%%%%%%%%%%%%%%%%%%
%%%                      Homework _                            %%%
%%%%%%%%%%%%%%%%%%%%%%%%%%%%%%%%%%%%%%%%%%%%%%%%%%%%%%%%%%%%%%%%%%

\documentclass[letter]{article}

\usepackage{lipsum}
\usepackage[pdftex]{graphicx}
\usepackage[margin=1.5in]{geometry}
\usepackage[english]{babel}
\usepackage{listings}
\usepackage{amsthm}
\usepackage{amssymb}
\usepackage{framed} 
\usepackage{amsmath}
\usepackage{titling}
\usepackage{fancyhdr}

\pagestyle{fancy}


\newtheorem{theorem}{Theorem}
\newtheorem{hairy}{Hairy Ball theorem}
\newtheorem{definition}{Definition}
\newtheorem{corollary}{Corollary}

\newenvironment{menumerate}{%
  \edef\backupindent{\the\parindent}%
  \enumerate%
  \setlength{\parindent}{\backupindent}%
}{\endenumerate}







%%%%%%%%%%%%%%%
%% DOC INFO %%%
%%%%%%%%%%%%%%%
\newcommand{\bHWN}{5}
\newcommand{\bCLASS}{MATH 105}

\title{\bCLASS: Homework \bHWN}
\author{William Guss\\26793499\\wguss@berkeley.edu}

\fancyhead[L]{\bCLASS}
\fancyhead[CO]{Homework \bHWN}
\fancyhead[CE]{GUSS}
\fancyhead[R]{\thepage}
\fancyfoot[LR]{}
\fancyfoot[C]{}
\usepackage{csquotes}

%%%%%%%%%%%%%%

\begin{document}
\maketitle
\thispagestyle{empty}


%%%%%%% Be sure to set the counter and use menumerate
\begin{menumerate}
\setcounter{enumi}{64}
\item The winding (w)one form.
\begin{definition}
	We denote the winding one form, $d\theta$, such that
	\begin{equation}
		d\theta = \frac{-y}{r^2}dx + \frac{x}{r^2}dy.
	\end{equation}
\end{definition}
\begin{theorem}
	The winding one form is closed but not exact.
\end{theorem}
\begin{proof}
	We take the exterior derivative of $d\theta$. Using $d(fdx) = df \wedge dx$ we have the following:
	\begin{equation}
	\begin{aligned}
		d(d\theta) &= d\left(\frac{-y}{r^2}dx + \frac{x}{r^2}dy\right)\\
			&= d\left(\frac{-y}{r^2}dx\right) + d\left(\frac{x}{r^2}dy\right)\\
			&= d\left(\frac{-y}{r^2}\right) \wedge dx + d\left(\frac{x}{r^2}\right)
			\wedge dy \\
			&= \left(\frac{\partial}{\partial x}\frac{-y}{x^2+y^2}dx \frac{\partial}{\partial y}\frac{-y}{x^2+y^2}dy + \right) \wedge dx + d\left(\frac{x}{r^2}\right)
			\wedge dy \\
			&= \frac{y^2 -x^2}{r^4}dy \wedge dx + d\left(\frac{x}{r^2}\right) \wedge dy \\
			&= \frac{y^2 -x^2}{r^4}dy \wedge dx + \left(\frac{\partial}{\partial x}\frac{x}{r^2}dx + \frac{\partial}{\partial y} \frac{x}{r^2} dy\right) \wedge dy \\
			&= \frac{y^2 -x^2}{r^4}dy \wedge dx + \frac{\partial}{\partial x}\frac{x}{r^2}dx \wedge dy \\
			&= \frac{y^2 -x^2}{r^4}dy \wedge dx + \frac{y^2 - x^2}{r^4}dx \wedge dy = 0
	\end{aligned}	
	\end{equation}
	Substitution of $r\cos\theta$ for $x$ and similar for $y$ yields that the form is infact $d\theta$,  ($dx = \cos \theta dr - r\sin \theta d \theta).$ If the form were exact then its anti-exterior derivative should be $\theta,$ therefore its evaluation along a $1$ cell should be its net change along its end points.
	
	Consider the curve which takes the unit circle counter clockwise around the origin. 
	\begin{equation}
	 d\theta(c)= \int_0^\tau -\sin t dx + \cos t dy = \int_0^\tau dt = \tau.
	\end{equation}
	And $\tau = 2\pi \neq 0,$ but the net change in $\theta = 0$ So it could not be that this form is exact. 
\end{proof}
	The name $d\theta$ is totally misleading, it implies that $d\theta = d(\theta)$ which is false. 
	\setcounter{enumi}{67}
	\item Closedness of scalar multiplication.
	 \begin{theorem}
		If $\omega$ is closed then $f\omega$ is not necisarrily closed.	
	\end{theorem}
	\begin{proof}
		Take the exterior derivative of the expression and get
		\begin{equation}
		d(f\omega) = df \wedge \omega + f \wedge d\omega = df \wedge \omega.
		\end{equation}
		So if the differential of $f$ is $0$ then $f\omega$ is closed. Otherwise, no.
	\end{proof}
	\begin{theorem}
		
	\end{theorem}

	\setcounter{enumi}{70}
	\item Hairy ball theorem! I'll do part (a) seperately, but follow my logic for the remaining parts all in part(b).
	\begin{menumerate}
		\item 
		\begin{hairy}
 			If $X: S^2 \to \mathbb{R}^3$ is a continuous vector field so that every point $p\in S^2$, $X(p)$
 			is tangent to $S,$ then there exists a $q$ so that $X(q) = 0.$
 		\end{hairy}
 		\begin{hairy}
 			Other ball theorem.		
 		\end{hairy}
 		\begin{theorem}
 			Hairy ball Theorem 1 and Hairy Ball Theorem 2 are logically equivalent.
 		\end{theorem}
 		\begin{proof}
 			Lol.
 		\end{proof}
		\item We shall build a set of logical tools to prove the first hairy ball theorem.
		\begin{definition}
			If $X$ is a continuous vector field with no zero points, so that $X: U \subset \mathbb{R}^n \to \mathbb{R}^n,$
			we denote \textbf{the net winding of} $X$ along some curve $\phi \in C_1(\mathbb{R}^n),$
			$\omega_X \in \Omega^1(\mathbb{R}^n).$ We require that the for all  $p \in U$
			\begin{equation}
				\langle p , X(p) \rangle = 0,
			 \end{equation} or intuitiveley $X$ is tangential to points in $U,$ and $U$ \emph{simply connected.} 
		\end{definition}
		\begin{theorem}
			If $\omega_X$ is a continuously tangent to $S^2$ and has no zeroes, its net winding is $0.$
		\end{theorem}
		\begin{proof}
			First consider the point set of a $1$-form $\phi$ on the $2$-sphere. If $\phi$ is a a closed loop,
			then there is a point on the sphere, $p_0$ where $p_0$ is not in the point set. Consider the sterographic projection about 
			$p_0$ on the sphere to the plane. This mapping is a conformal diffeomorphism. Furthermore if we homotop $\phi$ homeomorphically
			(with respect to its pointset), we can likewise change $p_0$ smoothly so that any discussion we make about homotopy
			of $T(\phi)$ makes sense on $S^2.$

			It is therefore true that when $\phi$ is projected using $T$ we get a closed curve. Analagously take $X$ satisfying the winding
			definition above, and we build an analygous vector field in the image of $T$ as follows: 
			\begin{menumerate}
			 	\item For every $p \in S^2,$ take $X(p)$ and project it on $S^2$.
			 	\item The curved pointset produced is $\overline{p \mathrm{proj}_S^2(X(p))} = L_p.$
			 	\item Define $Y(T(p)) = T(L_p)$ in the sense that $Y(T(p))$ is the vector whose head is at the head
			 	of the projection of $L_p$ into the plane. 
			 \end{menumerate} 

			 The net winding of $X$ along some curve is the $1$-form, 
			 \begin{equation}
			 	\omega_Y = \frac{Y_1}{\sqrt{Y_1^2 + Y^2}}dy_1 + \frac{Y_2}{\sqrt{Y_1^2 + Y^2}} dy_2 := \left\langle Y, \begin{pmatrix}
			 		dx_1 \\
			 		dx_2
			 	\end{pmatrix} \right\rangle.
			 \end{equation}
			 This definion makes sense, since it accumulates the sum of the net change in $\cos\theta$ over the vectore field 
			 against the vector $(1,\dots,1)^T.$ So we are concerned with the field not the curve over which we accumulate the winding.

			 We show the net winding over $Y$ is $0$. First off, continuity of $Y$ implies that winding must be \emph{integer proportional} to $2\pi$.
			 If it were not then at a point infintesmally close to the initial parameter (start) of a loop $\gamma$ on $Y$, there would be 
			 largeley different (not infintesmally different) vector $Y$ than that of the initial parameter. If it is not at this inial point
			 then it must be somewhere else. However the mear existence of this 'jump' violates the continuity of $X.$ 

			 Now by the simply connected
			 nature of the projection, $T(S^2)$ imagine a sufficiently small homotopy of $\gamma$. It could not be the case that $\omega_Y$ changes
			 much, that is for this homotopy $H.$ Then $|\omega_Y(H(0,\phi)) - \omega_Y(H(1,\gamma))| < \epsilon,$ by continuity of $Y.$  However
			 this difference can only be \emph{integer proportional} to $2pi$, so it must be zero. Finally consider the homotopy of $\gamma$ to
			 a constant map (a point) $\mu \in T(S^2).$ The net winding of $Y$ in this map is nameley $0,$ and since $\omega_Y$ is invariant
			 to homotopy, we have that $\omega_Y(\gamma) = 0 $ for all $ \gamma$ homotopic to a point (closed curves). \emph{Note: the homotopy
			 of $\gamma$ to a point makes sense in terms of winding if and only if $T(S^2)$ is simply connected, which it is. Furthermore, 
			 we will be using this an homolygous homotopy on the sphere but recall the argument of the smooth map $T$ changing smoothly with homotopy.}

			 Now we wish to pull back the winding of curves in $Y$ to the tangential vector field on $S^2.$ In other terms, $T^*\omega_Y = \omega_X.$
			 Consider the followuing derivation,
			 \begin{equation}
			 	\begin{aligned}
			 		T^*\omega_Y &= T^*(\langle Y, (dy_1, dy_2)\rangle) \\
			 		&= T^*(\cos \theta_Y dy_\Theta),
			 	\end{aligned}
			 \end{equation}
			 where $dx_\Theta$ is the reparameterization of $(dy_1, dy_2)$ in polar coordinates. We use the conformality of
			 $T$ and get
			 \begin{equation}
			 	\omega_X = \cos \theta_X dx_\Theta \implies T^*(\omega_Y) = (T^* \cos \theta_Y) dT_\Theta = \omega_X. 
			 \end{equation}
			 Finally by the definition of pullback we get $T^*(\omega_Y): \phi \mapsto \omega(T\circ \phi) = 0 $ for all $ \phi$ 
			 that are simple closed $1$-cell loops. So $\omega_X(\phi) = 0$ and the proof is complete.
		\end{proof}
		\begin{definition}
			The winding of $\phi \in C_1(U)$ against a vector field 
			$X: U \to \mathbb{R}^n$ is the net angle between the tangent vectors of the curve and 
			the $X$ denoted $w(X) \in \omega^1(X(U)),$ so that
			\begin{equation}
				w(X) = \omega_X + d\theta,
			\end{equation}
			where $d \theta$ calculates the angle accumulated on $\phi.$
		\end{definition}
		\begin{corollary}
			If $\phi \in C_1(S^2)$ is a simple closed loop, then its net winding against a continuous vector field $X$
			on $S^2$ with no holes is $\pm 2\pi$ according to its orientation. 
		\end{corollary}
		\begin{proof}
			Using the previous theorem, by definition we have
			\begin{equation}
				w(X)(\phi) = \omega_X(\phi) + d\theta(\phi) = 0 + d\theta(\phi).
			\end{equation}
			Lastly, $d\theta(\phi) = \pm2\pi$ from class. This completes the proof.
		\end{proof}
		\begin{corollary}
			Homeomorphic homotopy of a simple closed curve, $\phi$ in $U$ does not change $\omega(\phi, X).$
 		\end{corollary}
 		\begin{proof}
 			Homeomorphic homotopy is meant in the sense that $\phi$ must have a loopy point set after deformation. 
 			By the same argument in theorem 1, continuity of $X$ implies that $w(X)(\phi)$ is a continuous functional
 			of small homotopic changes of $\phi$ whose value set has a discrete topology. The only continuous mappings
 			from some set $D \to C$ where $C$ has the discrete topology are constant maps. So the theorem holds.
 		\end{proof}

 		We now prove the hairy ball theorem!\\
 		\noindent
 		\emph{Proof of Hairy Ball Theorem 1.} Suppose that there does exist such a beautiful wonderful vector field $X$
 		with no zeroes!

 		Let $\phi$ be a nice artic circle closed loop $1$-form. We call the homotopy from Figure $135$
 		\begin{equation}
 			H_r: C_1(U) \times [0,1] \to C_1(U), \gamma, 0 \mapsto \phi \wedge \gamma, 1 \mapsto \phi_o. 
 		\end{equation}
 		Where $\phi_0$ is the same curve but with its orientation reversed. Let $w(X)(\phi) = a.$
 		Then clearly $w(X)(\phi_0) = -a.$ But by the previous corollary, the homeomorphic homotopy equivalence of
 		$\phi_0$ and $\phi$ implies that $a = -a = 0,$ which contradicts the corollary before that.
 		Therefore it could not possibly be that $X$ has no zeroes, and the proof is complete!
	\end{menumerate}
\end{menumerate}

\end{document}