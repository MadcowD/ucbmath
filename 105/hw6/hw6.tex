%%%%%%%%%%%%%%%%%%%%%%%%%%%%%%%%%%%%%%%%%%%%%%%%%%%%%%%%%%%%%%%%%%
%%%                      Homework _                            %%%
%%%%%%%%%%%%%%%%%%%%%%%%%%%%%%%%%%%%%%%%%%%%%%%%%%%%%%%%%%%%%%%%%%

\documentclass[letter]{article}

\usepackage{lipsum}
\usepackage[pdftex]{graphicx}
\usepackage[margin=1.5in]{geometry}
\usepackage[english]{babel}
\usepackage{listings}
\usepackage{amsthm}
\usepackage{amssymb}
\usepackage{framed} 
\usepackage{amsmath}
\usepackage{titling}
\usepackage{fancyhdr}

\pagestyle{fancy}


\newtheorem{theorem}{Theorem}
\newtheorem{hairy}{Hairy Ball Theorem}
\newtheorem{lemma}{Lemma}
\newtheorem{definition}{Definition}
\newtheorem{corollary}{Corollary}

\newenvironment{menumerate}{%
  \edef\backupindent{\the\parindent}%
  \enumerate%
  \setlength{\parindent}{\backupindent}%
}{\endenumerate}







%%%%%%%%%%%%%%%
%% DOC INFO %%%
%%%%%%%%%%%%%%%
\newcommand{\bHWN}{6}
\newcommand{\bCLASS}{MATH 105}

\title{\bCLASS: Homework \bHWN}
\author{William Guss\\26793499\\wguss@berkeley.edu}

\fancyhead[L]{\bCLASS}
\fancyhead[CO]{Homework \bHWN}
\fancyhead[CE]{GUSS}
\fancyhead[R]{\thepage}
\fancyfoot[LR]{}
\fancyfoot[C]{}
\usepackage{csquotes}

%%%%%%%%%%%%%%

\begin{document}
\maketitle
\thispagestyle{empty}


%%%%%%% Be sure to set the counter and use menumerate
\begin{menumerate}
\item Show some things.
	\begin{menumerate}
		\item \emph{Show that the definition of linear outer measure is unaeffected if we demand that the intervals $I_k$ in the coverings be closed instead of open.}
		\begin{definition}
			The linear outer measure of a set $A \subset \mathbb{R}$ is given by
			\begin{equation}
				m^*A = \inf \left\{\sum_k |I_k| : \{I_k\}\text{ is a covering of A by open intervals}\right\}.
			\end{equation}
		\end{definition}
		\begin{definition}
			The closed linear outer measure of a set $ \subset \mathbb{R}$ is given by
			\begin{equation}
				\bar m^*A = \inf \left\{\sum_k |\bar I_k| : \{\bar I_k\}\text{ is a covering of A by \textbf{closed} intervals}\right\}.
			\end{equation}
		\end{definition}
		\begin{theorem}
			Definition $1$ and definition $2$ give equivalent measures.
		\end{theorem}
		\begin{proof}
			Take some set $A$ and obtain its linear outer measure $m^*A$. By the definition of infimum,
			$m^* A$ is the limit of outer measures of finer and finer countable 
			coverings of $A.$ The same argument can be made for $\bar m^* A$, except for $\bar I_k$ closed.

			Let the two respective sequences of coverings be given by $\mathcal{C}_i$ and $\bar{\mathcal{C}_i}$. Clearly 
			\begin{equation}
				m^*A \leftarrow m_i^* A  = \sum_{C \in \mathcal{C}_i} |C| = \bar m_i^* A  =
				 \sum_{\bar C \in  \bar{\mathcal{C}_i}} |\bar C|\to \bar m^*A
			\end{equation}
			And so $m^*A = \bar m^*A.$ This follows subtly from $m(I) = m(\bar I) = b-a$. The proof is complete.
		\end{proof}
		\item The middle thirds cantor set has a covering by closed intervals 
		$C_i$ whose constituent area is $1/3^i$ and so the infimum has area $0.$
		\item How open should I really be? \begin{theorem}
			The outer measure of an interval can be taken without without conditions onc closedness/openess.
		\end{theorem}
		\begin{proof}
			Consider that any other covering of $A$ besides that depicted in definition 1 and definition 2, has area in between those two coverings by monotonicity of outer measure. Therefore $m^*A \leq \nu A \leq \bar m^* A \implies \nu A = m^* A.$
		\end{proof}

		\item The same thing holds for planar outer measure, since effectiveley 
		$S$ as a rectangle is the product of $n$ intervals. Furthermore, we can approximate any recatngle (open, closed, clopen, or neither) $\pm \epsilon$ by a bunch of squares. 
	\end{menumerate}
	\setcounter{enumi}{2}
	\item
	\begin{theorem}
		All lines are zero sets.
	\end{theorem}
	\begin{proof}
		Recall that (from the book) all rigid transformations $T: \mathbb{R}^n \to \mathbb{R}^n$ are meseometries. Take any rotation and translation $\phi.$ By the exercise $m(\mathbb{R} \times \{a\}) = 0$ implies that
		$m(\phi(\mathbb{R} \times  \{a\})) = m(\mathbb{R} \times \{a\}) = 0.$ 
	\end{proof}
	\begin{theorem}
		All $n-1$ hyperplanes are zero sets in $\mathbb{R}^n.$
	\end{theorem}
	\begin{proof}
		Recall proposition $2$ (from the book) then without loss of generality apply the meaeomorphism in the previous proof.
	\end{proof}
	\item Higher dimensional Lemmas!
	\begin{lemma}
		The boundary of an $n$-dimensional ball is an $n$-dimensional zero set.
	\end{lemma}
	\begin{proof}
		If $\Delta$ is the closed unit ball in $\mathbb{R}^n$, then $0 < m\Delta < \infty$ since $[-1/\sqrt{2},1/\sqrt{2}]^n \subset [-1,1]^0n.$ The
		unit sphere $S^{n-1}$ is the boundry of $\Delta.$ It is sandwhiched
		between balls $\Delta_-$ of radius $1 - \epsilon$ and $\Delta_+$ of radius $1+\epsilon.$ Corollary $8$ implies
		\begin{equation}
			m(\Delta_-) = (1- \epsilon)^n m\Delta < m \Delta < (1+ \epsilon)^n m \Delta = m(\Delta_+).	
		\end{equation}
		Measurability implies that $m(\Delta_+ \setminus \Delta_-) = m(\Delta_+) - m(\Delta_-) = ((1 + \epsilon)^n - (1 - \epsilon)^n)m \Delta.$
		This gives us 
		\begin{equation}
			m\left(S^{n-1}\right) \leq ((1 + \epsilon)^n - (1 - \epsilon)^n) m\Delta =2\left(\sum_{i=0}^n \begin{pmatrix}
				n \\
				i
			\end{pmatrix} \epsilon^{n-i} \right) m\Delta.
		\end{equation}
		Since $\epsilon > 0$ is arbitrary, we get $m(S^{n-1}) = 0.$
	\end{proof}
	\begin{lemma}
		Every open cube is a countable disjoint union of open balls plus a zero set.
	\end{lemma}
	\begin{proof}
		Let $S \subset \mathbb{R}^n$ be an open cube. It contains a compact ball $\Delta$ whose volume is greater than $1/2^n$ of the volume of the cube. This follows from 
		\begin{equation}
			\frac{m(\Delta)}{m(S)} = \frac{\pi^{n/2}}{\Gamma(\frac{n}{2} + 1)} > \frac{1}{2^{n}}. 	
		\end{equation}
		The difference $U_1 = S \setminus \Delta$ is an open subset of $S$ with $m(U_1) < m(S)((2^n - 1)/2^n).$ It is therefore the disjoint countable union of small open cubes $S_i$ plus a zero set. Each cube contains a ball is volume is greater than $1/2^n$ of the volume of each cube, and so the total volume of the small balls are more than $1/2^n$ the volume of the small cubes.  So we get that the difference is $U_2$ whose total volume is less than $m(U_1)(((2^n - 1)/2^n)) = ((2^n - 1)^2/2^{2n}).$

		Repeating this process we get \begin{equation*}
			m(U_k) = \frac{(2^n -1)^k}{2^{kn}} \implies \ln(m(U_k)) =
			\ln((2^n - 1)^k) - \ln(2^{kn}) = k(\ln(2^n - 1)) - n\ln(2)) \to 0  
		\end{equation*}
		since $\ln(2^n - 1) \to n \ln (2).$ In other words, repition gives
		smaller and smaller compact balls with total measure equal to $m(S)$.
		Lemma $10$ implies that the measure of a closed ball is the same as the measure of its interor, which completes the proof that $S$ consists of countably many disjoint open cubes plus a zero set.
	\end{proof}
	\begin{theorem}
		An affine motion $T: \mathbb{R}^n \to \mathbb{R}^n$ is a meseomorphism. It multiplies measure by $|\det T|.$
	\end{theorem}
	\begin{proof}
		Assume that $Tv = Mv$ where $M$ is an invertible matrix. 
		We first claim that if $Z$ is  a zero set then so is $TZ$. Given any $\epsilon > 0$ there is a countable covering of $Z$ by boxes  $R_k$ with total volume $< \epsilon.$ Each $R_k$ can be covered by cubes with total volume $m(R_k) + \epsilon/2^k.$ Hence $Z$ can be covered by countably many cubes $S_i$ with volume $2\epsilon.$ The $T$ image of each $S_i$ is contained in a cube with edge length $\|T\| diam S_i.$ This finally gives, $TZ$ contained by cubes whose total volume is
		\begin{equation}
			\sum(\|T\| diam S_i)^n = \sum n^{n/2} \|T\|^n |S_i| \leq 2n^{n/2} \|T\|^2\epsilon.
		\end{equation}
		Since $\epsilon > 0$ is as  small as we like, we have $m(TZ) = 0.$

		Next we claim that orthogonal transformations are meseometries. Let $O:\mathbb{R}^n \to \mathbb{R}^n$ be orthogonal. It carries the ball $B(r,p),$ to the ball $B(r, Op),$ which is a translate of $B(r,p).$ Let $S$ be a cube. The previous lemma implies that $S = \bigsqcup B_i \cup Z$ where $B_i$ are n-balls and $Z$ is a zero set. The O-image of $B_i$ is
		a ball of equal measure, and the $O$-image of $Z$ is a zeroset. Hence,
		$m(OS) = mS.$ Given $\epsilon > 0,$ there is a countable covering of $A$ by cubes $S_i$ with $\sum |S_i| < m^*A + \epsilon$. Thus $\{O(S_i)\}$ covers $OA$ and has total area $< m^*A + \epsilon.$ We therefore get
		\begin{equation}
		  	m^*(OA) \leq m^*A.
		  \end{equation}  

		Since $O^{-1}$ is also orthogonal, it too does not increase outer measure. Theorem $7$ implies that $O$ is a meseometry.

		Finally, we use Polar Form to write
		\begin{equation}
			M = O_1 D O_2
		\end{equation}
		where $O_1, O_2$ are orthogonal and $D$ is diagonal. Since $O_1$ and $O_2$ are meseometries and by Corrolary $8$ D is a meseomorphism which multiplies measire bt $|det D| =|det T|,$ the proof is complete. 
	\end{proof}
	\item Interesting general stuff for $\mathbb{R}!$
	\begin{theorem}
		Every closed set in $\mathbb{R}^n$ is a $G_\delta$ set, furthermore
		every open set is a $F_\sigma$ set.
	\end{theorem}
	\begin{proof}
		Take $S \subset N$ to be some closed set. Then for every $n\in \mathbb{N}$
		let 
		\begin{equation}
			O_n = \bigcup_{x \in S} B\left(x, \frac{1}{n}\right),	
		\end{equation}
		where $B(p,r),$ is the open ball of radius $r$ at $p.$
		Then clearly \begin{equation}
			\bigcap_{n=1}^\infty O_n = S,
		\end{equation}
		and $S$ is a $G_\delta$ set. Let $Y$ be some open set in $N.$ Then
		$Y^c$ is closed and therefore is an $G_\delta$ set. That is, there exist
		some open family $\{O_n\}$ so
		that 
		\begin{equation}
				Y^c = \bigcap_{n=1}^\infty O_n \implies {Y^c}^c = \bigcup_{n=1}^\infty O_n^c
		\end{equation}
		and $Y$ is an $F_\sigma$ set.
	\end{proof}
\end{menumerate} 
\end{document}