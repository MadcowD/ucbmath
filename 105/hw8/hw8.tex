%%%%%%%%%%%%%%%%%%%%%%%%%%%%%%%%%%%%%%%%%%%%%%%%%%%%%%%%%%%%%%%%%%
%%%                      Homework _                            %%%
%%%%%%%%%%%%%%%%%%%%%%%%%%%%%%%%%%%%%%%%%%%%%%%%%%%%%%%%%%%%%%%%%%

\documentclass[letter]{article}

\usepackage{lipsum}
\usepackage[pdftex]{graphicx}
\usepackage[margin=1.5in]{geometry}
\usepackage[english]{babel}
\usepackage{listings}
\usepackage{amsthm}
\usepackage{amssymb}
\usepackage{framed} 
\usepackage{amsmath}
\usepackage{titling}
\usepackage{fancyhdr}

\pagestyle{fancy}


\newtheorem{theorem}{Theorem}
\newtheorem{lemma}{Lemma}
\newtheorem{definition}{Definition}

\newenvironment{menumerate}{%
  \edef\backupindent{\the\parindent}%
  \enumerate%
  \setlength{\parindent}{\backupindent}%
}{\endenumerate}







%%%%%%%%%%%%%%%
%% DOC INFO %%%
%%%%%%%%%%%%%%%
\newcommand{\bHWN}{8}
\newcommand{\bCLASS}{MATH 105}

\title{\bCLASS: Homework \bHWN}
\author{William Guss\\26793499\\wguss@berkeley.edu}

\fancyhead[L]{\bCLASS}
\fancyhead[CO]{Homework \bHWN}
\fancyhead[CE]{GUSS}
\fancyhead[R]{\thepage}
\fancyfoot[LR]{}
\fancyfoot[C]{}
\usepackage{csquotes}

%%%%%%%%%%%%%%

\begin{document}
\maketitle
\thispagestyle{empty}

\begin{menumerate}
    \setcounter{enumi}{28}
    \item Upper semicontinuity.
    \begin{menumerate}
    \item A graph of an upper semicontinuous graph here:
    \\[3cm]
    \item Show the following.
    \begin{definition}
        We say that a function $f: M \to \mathbb{R}$ is
        $(\epsilon , \delta)$-upper semicontinuous if and only if
        for every $\epsilon > 0$ there is a $\delta > 0$ so that
        \begin{equation}
            0< d(x,y) < \delta \implies f(y) < f(x) + \epsilon
        \end{equation}
    \end{definition}
    \begin{lemma}
        Upper semicontinuity is equivalent to the $(\epsilon, \delta)$-upper semicontinuity.
    \end{lemma}
    \begin{proof}
        Observe the following fact about $\limsup.$ 
        \begin{equation}
            \limsup_{y\to x} g(y) = \alpha = \lim_{\epsilon \to 0} \sup \{g(y)\ :\ y \in M \cap M_{\epsilon}(x) \setminus \{x\}\}.
        \end{equation}
        Therefore $f$ is upper semicontinuous if and only if 
        \begin{equation}
            \limsup_{y\to x} f(y) \leq f(x) \iff \lim_{\epsilon \to 0}
            \sup\{f(y) : y \in M \cap M_{\epsilon}(x) \setminus \{x\} \} \leq f(x).       
        \end{equation}
        We then know for every $\epsilon >0$ there exists a $\delta $
        so that 
        \begin{equation}
            \sup \{f(y) \ :\ y \in M \cap M_\delta (x) \setminus \{x\} \} < f(x) + \epsilon.
        \end{equation}
        This is true if and only if 
        \begin{equation}
            d(y,x) < \delta \implies f(y) < f(x) + \epsilon.       
        \end{equation}
        Therefore $f$ is  $(\epsilon, \delta)$-upper semicontinuous.
    \end{proof}
    \begin{theorem}
        The function $f: M \to \mathbb{R}$ if upper semicontinuous 
        if and only if for every $a \in \mathbb{R}$, 
        \begin{equation}
            U_a = \{ x \ :\ f(x) < a \}       
        \end{equation}
        is an open subset of $M$.
    \end{theorem}
    \begin{proof}
        Take some $x \in U_a$. Then upper semicontinuity implies that
        for every $\epsilon> 0$ there is a $\delta$ so that
        \begin{equation}
            0 < d(x,y) < \delta \implies f(y) < f(x) + \epsilon.       
         \end{equation}
         We know that $f(x) < a$, so take $\epsilon = f(x) - a.$
         Then for every $y$ with $d(x,y) < \delta$, 
         \begin{equation}
         f(y) < f(x) + a - f(x) = a,
         \end{equation}
         and $y \in U_a$. Therefore for all $u \in U_a$ there exists a $\delta$ so that $d(u,v) < \delta \implies v \in U_a$, and
         $U_a$ is open.

         In the opposite direction suppose that $U_a$ is open.
         Then, for every $x \in U_a$ there exists a $\delta$ so that
         $d(y,x) < \delta \implies y \in U_a. $ Therefore $f(y) < a.$
         Since we can do this for any arbitrary $a,$ take any $\gamma \in M,$ then consider $U_{f(\gamma)}.$ It follows for every $\epsilon > 0$ there is a $\delta$ so that
         \begin{equation}
            0 < d(y, \gamma) < \delta, y \in U_{f(\gamma)} \implies f(y) < f(\gamma) + \epsilon
         \end{equation}
         What can be said about $y \notin U_{f(\gamma)}.$ Take the arg max of
         those $y$ subject to $f(y) \leq f(\gamma)  + \epsilon, y \neq \gamma$ (this is possible since $U_{f(\gamma)}^C$ is closed and there is an $a > \gamma$ so that every $x \in U_a \supset U_{f(\gamma)}$ is a point of upper semicontinuity) and we get $y'$ Then take a new 
         \begin{equation}
            \delta' = \min\{\delta, d(y',\gamma)\}
         \end{equation}
         and get $f$ upper semicontinuous.
    \end{proof}
    \item Negative semicontinuity.
    \begin{definition}
        We say that a function $f: M \to \mathbb{R}$ is negative semicontinuous if and only if $-f$ is upper semicontinuous.   
    \end{definition}
    \begin{theorem}
        A function is negative semicontinuous if and only if 
        \begin{equation}
            \lim_{y \to x} f(y) \geq f(x).    
        \end{equation}   
    \end{theorem}
    \begin{proof}
        Suppose that $-f$ is upper semicontinuous, then
        \begin{equation}
            \limsup_{y \to x} -f(y) \leq -f(x) \iff
            -\liminf_{y \to x} f(y) \leq -f(x),   
        \end{equation}   
        by the definition of $\liminf.$ Then we negate the inequality and get 
        \begin{equation}
            \liminf_{y\to x} f(y) \geq f(x).
        \end{equation}
        This completes the proof.
    \end{proof}
     \end{menumerate} 
     \item
     \item 

     \setcounter{enumi}{32}
     \item
     \item Prove the following

     \begin{theorem}
         Suppose that $f_n : \mathbb{R} \to [0, \infty)$ is a
         sequence of integrable functions, $f_n \downarrow f$ a.e. as $n \to \infty$ and $\int f_n \downarrow 0$, then $f = 0$ almost everywhere.
     \end{theorem}
     \begin{proof}
        Suppose that that $f \neq 0$ almost everywhere. Then the undergraph of $f$ would have nonzero measure. If this is the case 
        then it is not true that $\int f_n \downarrow 0$ since if it were the case then not $f_n \downarrow f$ since the undergraph of $f$ is not a zeroset. Therefore $f = 0.$ This completes the proof.
     \end{proof}
     \item 
\end{menumerate}
%%%%%%% Be sure to set the counter and use menumerate

\end{document}