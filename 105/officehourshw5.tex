%%%%%%%%%%%%%%%%%%%%%%%%%%%%%%%%%%%%%%%%%%%%%%%%%%%%%%%%%%%%%%%%%%
%%%                      Homework _                            %%%
%%%%%%%%%%%%%%%%%%%%%%%%%%%%%%%%%%%%%%%%%%%%%%%%%%%%%%%%%%%%%%%%%%

\documentclass[letter]{article}

\usepackage{lipsum}
\usepackage[pdftex]{graphicx}
\usepackage[margin=1.5in]{geometry}
\usepackage[english]{babel}
\usepackage{listings}
\usepackage{amsthm}
\usepackage{amssymb}
\usepackage{framed} 
\usepackage{amsmath}
\usepackage{titling}
\usepackage{fancyhdr}

\pagestyle{fancy}


\newtheorem{theorem}{Theorem}
\newtheorem{lemma}{Lemma}
\newtheorem{definition}{Definition}

\newenvironment{menumerate}{%
  \edef\backupindent{\the\parindent}%
  \enumerate%
  \setlength{\parindent}{\backupindent}%
}{\endenumerate}







%%%%%%%%%%%%%%%
%% DOC INFO %%%
%%%%%%%%%%%%%%%
\newcommand{\bHWN}{1}
\newcommand{\bCLASS}{MATH H105}

\title{\bCLASS: Homework \bHWN}
\author{William Guss\\26793499\\wguss@berkeley.edu}

\fancyhead[L]{\bCLASS}
\fancyhead[CO]{Homework \bHWN}
\fancyhead[CE]{GUSS}
\fancyhead[R]{\thepage}
\fancyfoot[LR]{}
\fancyfoot[C]{}
\usepackage{csquotes}

%%%%%%%%%%%%%%

\begin{document}
\maketitle
\thispagestyle{empty}


\begin{menumerate}
    \setcounter{enumi}{54}
    \item Take a map $f$ which is sufficiently close, that is 
    the max norm of the euclidean distance between.
    Take the line which takes $f(p)$ to infinity and take a tangent $Tp$ from the point $p$
    to a point on the line from $f(p).$ Consider this to be our tangent vector field. 
    If the hairy ball theorem is true then this field must be zero at some $p$. If this is the case
    then $f(p) = p$ since the line from the origin to $f(p)$ to infinity need be the line from 
    the origin to $p$ to infinity, otherwise we contradict that the length of this tangent vector is $0.$

    Therefore the map $f$ has a fixed point.

    Consider the converse. Suppose all maps which are sufficiently close to the identity with a fixed point. 
    Consider a vector field built in a similar fashion using the vectors $f(p) - p$ and homotopically moving them along the line
    from $f(p)$ to infinity untill you find a tangent vector to the sphere at $p$ clearly this exists, and since this construction is smooth
    and $p-f(p)$ is a smooth function, we get that this vector field is continuous and so there is a point where the vector
    field is $0$ since at one point the line from the origin to $f(p)$ is the same as the line from the origin
    to $p.$


    \item Part b use line integral with differential ewquaiton
    $w = F_1dx_1 + F_2dx_2 + F_3dx_3.$ Then we know that the form applied to a closed loop
    must be $0$ since it is the weighted change of $dx$ from start to finish. Observe that 
    the form evaluates the dot product of $F$ and the tangent of the $1-cell$ parameterization,
    so we get that the net change in $\cos\theta$ where $\theta$ is the angular difference between the curve
    and the vector field, is zero and so the total net change in angle is proportional to $2\pi.$
    The proportion is furthermore only dependent on orientation of the curve.






    \item 71**. Hairy Ball Theorem Proof
    \begin{theorem}
        Any continuous vector field $X: S \to \mathbb{R}^3$ that is every where tangent to $S$
        must be zero at some point.
    \end{theorem}
    Before we give the proof. We first show some lemmas.

    \begin{lemma}
        If $X$ is a continuous vector field on a closed simply connected
        domain so that $X \neq 0,$ then the winding number of a
        closed simple curve in the vector field must be zero.
    \end{lemma}
    \begin{proof}
        The winding number for a curve in $X$ referrs to the number 
        of times the vector field turns around itself on the curve
        as $t \to 1$ from $0.$ It is clear that the winding number of a curve must be an
        integer since it would violate the continuity of $X$ if 
        at a point infintesmally close the the start of the loop (near the end), the vector
        in $X$ at that point were a real number different than that of the start point.
        Therefore, at least the vector field must wind upon itself a whole number of times.

        Consider a sufficiently small homotopy of the smooth 1-cell curve we previously 
        referred to. It could not be that $H$ applied to the curve has 
        a winding number much larger than that of the cotarget. Since the winding 
        number itself is a continuous function of the vector field (which is continuous and whose
        vectors would not change angle much under small homotopy,) it follows that
        the winding number is invariant to homotopy.

        Finally, the winding number of a point is $0$ and
        so by the simple connectedness of the domain, recall that any
         simple closed curve is homotopic to the point. Therefore, the winding number of 
         any closed curve is $0.$
    \end{proof}

    \begin{lemma}
        The net winding of a closed curve in a simple closed domain over a continuous vector
        field $X$, which is nowhere $0,$ is invariant to the field. Furthermore
        if the closed curve is a loop, then it has winding number $\pm 1$ corresponding to
        its orientation in space. 
    \end{lemma}
    \begin{proof}
        The net winding of a closed simple $1$-cell, $C$ in the domain is exactly 
        the net angle between the tangent $C$ and the the vectors $X$ at every point along
        $C.$ So the net winding of $C$ is the sum of the winding number of
        $C$ in $X$ and the winding number of $C$ in general. So the net winding of $C$
        is $\pm 1$ contingent upon the orientation of $C.$
    \end{proof}

    It is important to be carefult when considering the winding number of a closed curve when it is homotoped to a
    point. At that instant it does not make sense to discuss the net winding, but only the winding number of $C$ in $X;$
    \emph{net winding is not winding in a field.}

    \begin{lemma}
        Homotopy does not change the net winding of a curve in a continuous vector field $X$ with no $0$ points, 
    \end{lemma}
    \begin{proof}
           
    \end{proof}



    \begin{lemma}
        If $X$ is a continuous vector field with no zeroes and $C$ is a simple closed curve. Then if 
        $\omega = X_1 dx_1 + X_2 dx_2,$ 
        \begin{equation}
            \oint_C \omega =0 \implies d\theta(C,X) = z2\pi, z \in \mathbb{Z}
        \end{equation}
        \begin{proof}
            By the definition of the form evaluation of a 1-cell,
            we have that $fdx_i$ is the $f$-weighted net change in $x$ 
            from the beginning of the parameterization of the 1-cell
            to its end. In paricular, $\omega = \langle X, dx_I\rangle$ for 
            $I = \{1,2\}.$ Geometrically we have the following interpretation
            for
             \begin{equation}
                \oint_C \omega = 0.
            \end{equation}
            Essentially, $\omega = \cos \theta |X||dx_I| (C).$ This gives the intuition
            that if the contour is closed, then the net angle between the tangents of $C$
            and the vector field $X$ is $0.$ 

            Suppose we break the $C$ 1-cell into the sum of two  (possibly infinite) 
            families of cells, $\{\gamma^+_t)_{t\in J^+}, \{\gamma^-_s\}_{t\in J^-}$ so that
            \begin{equation}
                \sum_{t\in J^+}  \int_{\gamma^+_t} \omega = -\sum_{s \in J^-} \int_{\gamma^-_s} \omega. 
            \end{equation}
            There exists some sequence $\Gamma = \{\gamma^*_t\}$ such that
            \begin{equation}
                \oint_C \omega = \sum_{t\in J^- \cup J^+} \int_{\gamma_t^*} \omega = 0.
            \end{equation}
            The continuity of $X$ implies that
        \end{proof}
    \end{lemma}
\end{menumerate}

%%%%%%% Be sure to set the counter and use menumerate

\end{document}