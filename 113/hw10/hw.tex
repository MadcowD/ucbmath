\documentclass[11pt]{amsart}

\usepackage{amsmath,amsthm}
\usepackage{amssymb}
%\usepackage{graphicx}
%\usepackage{enumerate}
\usepackage{fullpage}
% \usepackage{euscript}
% \makeatletter
% \nopagenumbers
\usepackage{verbatim}
\usepackage{color}
\usepackage{hyperref}
%\usepackage{times} %, mathtime}

\textheight=600pt %574pt
\textwidth=480pt %432pt
\oddsidemargin=15pt %18.88pt
\evensidemargin=18.88pt
\topmargin=10pt %14.21pt

\parskip=1pt %2pt

\def\reals{{\mathbb R}}
\def\torus{{\mathbb T}}
\def\integers{{\mathbb Z}}
\def\rationals{{\mathbb Q}}
\def\naturals{{\mathbb N}}
\def\complex{{\mathbb C}\/}
\def\distance{\operatorname{distance}\,}
\def\support{\operatorname{support}\,}
\def\dist{\operatorname{dist}\,}
\def\Span{\operatorname{span}\,}
\def\degree{\operatorname{degree}\,}
\def\kernel{\operatorname{kernel}\,}
\def\dim{\operatorname{dim}\,}
\def\codim{\operatorname{codim}}
\def\trace{\operatorname{trace\,}}
\def\dimension{\operatorname{dimension}\,}
\def\codimension{\operatorname{codimension}\,}
\def\kernel{\operatorname{Ker}}
\def\Re{\operatorname{Re\,} }
\def\Im{\operatorname{Im\,} }
\def\eps{\varepsilon}
\def\lt{L^2}
\def\bull{$\bullet$\ }
\def\det{\operatorname{det}}
\def\Det{\operatorname{Det}}
\def\diameter{\operatorname{diameter}}
\def\symdif{\,\Delta\,}
\newcommand{\norm}[1]{ \|  #1 \|}
\newcommand{\set}[1]{ \left\{ #1 \right\} }
\newcommand{\group}[2]{\left\langle #1, #2\right\rangle}
\def\one{{\mathbf 1}}
\def\cl{\text{cl}}

\def\newbull{\medskip\noindent $\bullet$\ }
\def\nobull{\noindent$\bullet$\ }



\def\scriptf{{\mathcal F}}
\def\scriptq{{\mathcal Q}}
\def\scriptg{{\mathcal G}}
\def\scriptm{{\mathcal M}}
\def\scriptb{{\mathcal B}}
\def\scriptc{{\mathcal C}}
\def\scriptt{{\mathcal T}}
\def\scripti{{\mathcal I}}
\def\scripte{{\mathcal E}}
\def\scriptv{{\mathcal V}}
\def\scriptw{{\mathcal W}}
\def\scriptu{{\mathcal U}}
\def\scriptS{{\mathcal S}}
\def\scripta{{\mathcal A}}
\def\scriptr{{\mathcal R}}
\def\scripto{{\mathcal O}}
\def\scripth{{\mathcal H}}
\def\scriptd{{\mathcal D}}
\def\scriptl{{\mathcal L}}
\def\scriptn{{\mathcal N}}
\def\scriptp{{\mathcal P}}
\def\scriptk{{\mathcal K}}
\def\scriptP{{\mathcal P}}
\def\scriptj{{\mathcal J}}
\def\scriptz{{\mathcal Z}}
\def\scripts{{\mathcal S}}
\def\scriptx{{\mathcal X}}
\def\scripty{{\mathcal Y}}
\def\frakv{{\mathfrak V}}
\def\frakG{{\mathfrak G}}
\def\frakB{{\mathfrak B}}
\def\frakC{{\mathfrak C}}

\newtheorem{corollary}{Corollary}
\newtheorem{theorem}{Theorem}
\newtheorem{lemma}{Lemma}
\newtheorem{definition}{Definition}

\def\soln{\noindent {\bf Solution.}\ }

\begin{document}

\begin{center}{\bf Math 113 --- Problem Set 10 --- William Guss} \end{center}


\bigskip


\medskip \noindent {\bf (P174. 6)}\
\medskip \noindent {\bf (P175. 12)}\

\medskip \noindent {\bf (P182. 2)}\ Solve the equation $3x =2$ in the field\\
(a) $\mathbb{Z}_7$.
\begin{proof}	
	We must find $x$ so that $x \mod 7 = 2$ and $3 | x$.
	First $3\times 1 = 3 \mod 7 = 3$, then $3 \times 3 = 9 \mod 7 = 2,$ thus $x = 2$ in $\mathbb{Z}_7.$
\end{proof}
(b) $\mathbb{Z}_23$
\begin{proof}
	We msut find $x$ so that $x \mod 23 = 2$ and $3 | x$.
	Take $x = 16$, then $3x = 48.$ Finally $23\times 2 = 46$ so $48 \mod 46 = 2$ and $3x = 2.$ We could have found this by showing that $3y = 1 $ if $y = 3^{-1}$ and thus $3\times 8 = 24 \mod 23 = 1$ so $y = 8$.
	Then $3x \equiv 2$ is solved by $x \equiv y3x  \equiv y \times 2 = 16.$ 
\end{proof}
\medskip \noindent {\bf (P182. 3)}\ Find all solutions of the equation $x^2 + 2x + 2= 0$ in $\mathbb{Z}_6.$
\begin{proof}
	First $\mathbb{Z}_6$ is not a field since $2 \times 3 = 6 \equiv 0$ so $\mathbb{Z}_6$ is not an integral domain. We factor the polynomial however and get
	$(x+1)(x+1) = 1 \in \mathbb{Z}_6.$ Thus we find all $y$ so that $y^2 = 1.$
\end{proof}
\medskip \noindent {\bf (P182. 14)}\
\end{document}\end
