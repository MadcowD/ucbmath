%%%%%%%%%%%%%%%%%%%%%%%%%%%%%%%%%%%%%%%%%%%%%%%%%%%%%%%%%%%%%%%%%%
%%%                      Homework _                            %%%
%%%%%%%%%%%%%%%%%%%%%%%%%%%%%%%%%%%%%%%%%%%%%%%%%%%%%%%%%%%%%%%%%%

\documentclass[letter]{article}

\usepackage{lipsum}
\usepackage[pdftex]{graphicx}
\usepackage[margin=1.5in]{geometry}
\usepackage[english]{babel}
\usepackage{listings}
\usepackage{amsthm}
\usepackage{amssymb}
\usepackage{framed} 
\usepackage{amsmath}
\usepackage{titling}
\usepackage{fancyhdr}

\pagestyle{fancy}


\newtheorem{theorem}{Theorem}
\newtheorem{lemma}{Lemma}
\newtheorem{definition}{Definition}

\newenvironment{menumerate}{%
  \edef\backupindent{\the\parindent}%
  \enumerate%
  \setlength{\parindent}{\backupindent}%
}{\endenumerate}







%%%%%%%%%%%%%%%
%% DOC INFO %%%
%%%%%%%%%%%%%%%
\newcommand{\bHWN}{2}
\newcommand{\bCLASS}{MATH 113}

\title{\bCLASS: Homework \bHWN}
\author{William Guss\\26793499\\wguss@berkeley.edu}

\fancyhead[L]{\bCLASS}
\fancyhead[CO]{Homework \bHWN}
\fancyhead[CE]{GUSS}
\fancyhead[R]{\thepage}
\fancyfoot[LR]{}
\fancyfoot[C]{}
\usepackage{csquotes}

%%%%%%%%%%%%%%

\begin{document}
\maketitle
\thispagestyle{empty}

\begin{menumerate}
\setcounter{enumi}{31}
	\item Let $\mathcal{R}$ be the following relation. We say that $x \mathcal{R} y \in \mathbb{R}$ if and only if $|x -y| \leq 3$. We claim that the relation is not an equivalence relation.

	\begin{proof}
		Let $x = 0$, $y = 3$, $z =6.$ It is obvious that $x\mathcal{R}y$ and $y\mathcal{R}z$, but since $|z-x| = 6 \not\leq 3$ so it is not the case that $x \mathcal{R} z$. Therefore the relation is not transitive by counter example, and thereby is not an equivalence relation.
	\end{proof}

	\setcounter{enumi}{3}
	\item We compute the result of $(-i)^{35}.$ 
	\begin{equation*}
		(-i)^{35} = (-1)^{35}i^{35} = -i^3 = -i^2i = i.
	\end{equation*}
	\setcounter{enumi}{7}
	\item We compute the result of $(i+1)^3$ by first establishing the coefficients of pascals triangle as follows.
	\begin{equation*}
		\begin{array}{ccccccc}
			{(a+b)}^0&	&&1 \\
			{(a+b)}^1&      & 1 & &1 &   &  \\
			{(a+b)}^2&  & 1 & 2 & 1 &    \\
			{(a+b)}^3& 1 & 3 & & 3 & 1 & \\
			{(a+b)}^4&     1 & 4 & 6 & 4 & 1
		\end{array}
	\end{equation*}
	Therefore we apply the rule to our equation and yield
	\begin{equation*}
		(i+1)^3 = 1 +3i + 3i^2 + i^3 = -2 + 3i - i = -2 +2i.
	\end{equation*}

	\setcounter{enumi}{18}
	\item We find all solutions to $z^3 = -27i.$ First let $z = re^{i\theta}.$ Then $r^3e^{i3\theta} = 27e^{-i\pi/2}$. Therefore $r = 3$ and $\3\theta = -\pi/2$ gives the principle solution $\theta^* = -\pi/6.$ It also follows however that $\theta = -\pi/6 +2k\pi/3$ are all valid solutions since in the cube 
	\begin{equation*}
		{e^{i\theta}}^3 = {e^{-i\pi/6 +i2k\pi/3}}^3 = e^{-i\pi/2 + i2k\pi} = e^{-i\pi/2}.
	\end{equation*}

	Hence the following set satisfies $z^3 = -27i$
	\begin{equation*}
		S = \left\{3\exp\left({i\pi\left(\frac{2k}{3} - \frac{1}{6}\right)}\right)\ \Big|\ k \in \mathbb{Z}.\right\}.
	\end{equation*}

	\setcounter{enumi}{20}
	\item We find all solutions to $z^6 = -64$ using the same logic as before. Observe that $2^6 = 64,$ and $e^{i\pi} +1 = 0$ (\emph{magic!!!}). Then
	\begin{equation}
		S = \left\{2\exp\left({i\pi\frac{1 + 2k}{6}}\right)\ \Big|\ k \in \mathbb{Z}.\right\}
	\end{equation}
	satisfies $z \in S \implies z^6 = -64.$


\end{menumerate}

%%%%%%% Be sure to set the counter and use menumerate

\end{document}	