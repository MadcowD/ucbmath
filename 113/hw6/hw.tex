\documentclass[11pt]{amsart}

\usepackage{amsmath,amsthm}
\usepackage{amssymb}
%\usepackage{graphicx}
%\usepackage{enumerate}
\usepackage{fullpage}
% \usepackage{euscript}
% \makeatletter
% \nopagenumbers
\usepackage{verbatim}
\usepackage{color}
\usepackage{hyperref}
%\usepackage{times} %, mathtime}

\textheight=600pt %574pt
\textwidth=480pt %432pt
\oddsidemargin=15pt %18.88pt
\evensidemargin=18.88pt
\topmargin=10pt %14.21pt

\parskip=1pt %2pt

\def\reals{{\mathbb R}}
\def\torus{{\mathbb T}}
\def\integers{{\mathbb Z}}
\def\rationals{{\mathbb Q}}
\def\naturals{{\mathbb N}}
\def\complex{{\mathbb C}\/}
\def\distance{\operatorname{distance}\,}
\def\support{\operatorname{support}\,}
\def\dist{\operatorname{dist}\,}
\def\Span{\operatorname{span}\,}
\def\degree{\operatorname{degree}\,}
\def\kernel{\operatorname{kernel}\,}
\def\dim{\operatorname{dim}\,}
\def\codim{\operatorname{codim}}
\def\trace{\operatorname{trace\,}}
\def\dimension{\operatorname{dimension}\,}
\def\codimension{\operatorname{codimension}\,}
\def\kernel{\operatorname{Ker}}
\def\Re{\operatorname{Re\,} }
\def\Im{\operatorname{Im\,} }
\def\eps{\varepsilon}
\def\lt{L^2}
\def\bull{$\bullet$\ }
\def\det{\operatorname{det}}
\def\Det{\operatorname{Det}}
\def\diameter{\operatorname{diameter}}
\def\symdif{\,\Delta\,}
\newcommand{\norm}[1]{ \|  #1 \|}
\newcommand{\set}[1]{ \left\{ #1 \right\} }
\newcommand{\group}[2]{\left\langle #1, #2\right\rangle}
\def\one{{\mathbf 1}}
\def\cl{\text{cl}}

\def\newbull{\medskip\noindent $\bullet$\ }
\def\nobull{\noindent$\bullet$\ }



\def\scriptf{{\mathcal F}}
\def\scriptq{{\mathcal Q}}
\def\scriptg{{\mathcal G}}
\def\scriptm{{\mathcal M}}
\def\scriptb{{\mathcal B}}
\def\scriptc{{\mathcal C}}
\def\scriptt{{\mathcal T}}
\def\scripti{{\mathcal I}}
\def\scripte{{\mathcal E}}
\def\scriptv{{\mathcal V}}
\def\scriptw{{\mathcal W}}
\def\scriptu{{\mathcal U}}
\def\scriptS{{\mathcal S}}
\def\scripta{{\mathcal A}}
\def\scriptr{{\mathcal R}}
\def\scripto{{\mathcal O}}
\def\scripth{{\mathcal H}}
\def\scriptd{{\mathcal D}}
\def\scriptl{{\mathcal L}}
\def\scriptn{{\mathcal N}}
\def\scriptp{{\mathcal P}}
\def\scriptk{{\mathcal K}}
\def\scriptP{{\mathcal P}}
\def\scriptj{{\mathcal J}}
\def\scriptz{{\mathcal Z}}
\def\scripts{{\mathcal S}}
\def\scriptx{{\mathcal X}}
\def\scripty{{\mathcal Y}}
\def\frakv{{\mathfrak V}}
\def\frakG{{\mathfrak G}}
\def\frakB{{\mathfrak B}}
\def\frakC{{\mathfrak C}}

\newtheorem{corollary}{Corollary}
\newtheorem{theorem}{Theorem}
\newtheorem{lemma}{Lemma}
\newtheorem{definition}{Definition}

\def\soln{\noindent {\bf Solution.}\ }

\begin{document}

\begin{center}{\bf Math 113 --- Problem Set 6 --- William Guss} \end{center}


\bigskip


\medskip \noindent {\bf (P67. 41)}\ Find the primitive roots of $U_{12}.$
\begin{proof}
	We claim that $\{exp(i1\times 2\pi/12), exp(i5\times 2\pi/12),exp(i7\times 2\pi/12, exp(i11\times 2\pi/12)\}$ are the primitive roots of $U_{12}.$ First observe that $\{1,2,3,4,6,12\}$ all divide $12$ and there is an isomorphism between $\mathbb{Z}_{12}$ and $U_{12}$ so that the strict subgroups generated by each of those integers do not isomorphically describe the full group $U_{12}$, again using the same isomorphism argument, any multiple of those numbers (by themselves) are just a part of the same subgroup and themselves generate potentially a smaller subgroup then those subgroups. Therefore we look for elements in $\mathbb{Z}_{12}$ which are not multiples of the divisors of $12$ but are less than 12 (except for $1$). We get $1,5,7,11$. Therefore using the cannonical isomorphism we get the claimed set of primitive roots. 
\end{proof}
\medskip \noindent {\bf (83. 2)}\ We compute the permutation. 
\begin{proof}
	(I need to use a \emph{Solution.} environment.)  First 
	\begin{equation*}
		\tau \sigma = \begin{pmatrix}
			1 & 2 & 3 & 4 & 5 & 6 \\
			3 & 1 & 4 & 5 & 6 & 2 \\
			1 & 2 & 3 & 6 & 5 & 4 \\
		\end{pmatrix}
	\end{equation*}
	so we get therefore
	\begin{equation*}
		\tau^2 \sigma = 
		\begin{pmatrix}
			1 & 2 & 3 & 4 & 5 & 6 \\
			3 & 1 & 4 & 5 & 6 & 2 \\
			1 & 2 & 3 & 6 & 5 & 4 \\
			2 & 4 & 1 & 5 & 6 & 3
		\end{pmatrix} =
		\begin{pmatrix}
			1 & 2 & 3 & 4 & 5 & 6 \\
			2 & 4 & 1 & 5 & 6 & 3
		\end{pmatrix}
	\end{equation*}

\end{proof}
\medskip \noindent {\bf (84. 16)}\ Find the number of elements in the set $\{\sigma \in S_4 \ |\  \sigma(3)=3\}$.
\begin{proof}
	Any permutation is permitted as long as $\sigma(3) = 3.$ Therefore we accept 3 choices and restrict the first. Therefore there must be $3 \times 2 \times 1 = 6$ permutations.
\end{proof}
\medskip \noindent {\bf (84. 30)}\ Show that $f_1 : \mathbb{R} \to \mathbb{R}$ such that $f_1(x)  = x + 1$ is a bijection (and therefore a permutation.)
\begin{proof}
	First for every $x \in \mathbb{R}$ there exists a unique sucessor, nameley $x+ 1$, by the archimedian property. Therefore we have an injection. Furthermore $x + 1 =y$ if and only if $y -1 = x$ since $\mathbb{R}$ is an Abelian group on addition. Thus $f_1: x \mapsto x+1$ is a bijection.
\end{proof}
\medskip \noindent {\bf (84. 31)}\ Show that $f_2 : \mathbb{R} \to \mathbb{R}$ such that $f_2(x)  = x^2$ is not bijection (and therefore not permutation.)
\begin{proof}
	First observe that $x^2 = 1$ is $x = -1$ OR if $x=1$. Therefore the mapping is not injective. Therefore the map is not a bijection.
\end{proof}
\medskip \noindent {\bf (84. 32)}\  Show that $f_3 : \mathbb{R} \to \mathbb{R}$ such that $f_3(x)  = -x^3$ is a bijection (and therefore a permutation.)
\begin{proof}	
	Clearly $-(\cdot)$ is a permutation and the composition of permutations is a permutation by the definition. We now show that $x^3$ is a permutation. 
	First if $x \neq y$ then without loss of generality $x > y$. Furthermore by the montonicty of $x \mapsto x^3$ (basic calculus) we have that $x^3 > y^3 \iff x^3 \neq y^3$ by the well ordering of $\mathbb{R}.$ Finally take any $z \in \mathbb{R}$. We claim that there is an $x$ say $x = \sqrt[3]{z}$ so that $x^3 = z.$ First by the completeness of $\mathbb{R}$ and the cauchy approximation sequence of $\sqrt[3]{z}$ in $\mathbb{Q}$ we have the existence of $x.$ Finally $x^3 = z^{1/3\times 3} = z.$ Therefore the map is a surjection. This completes the proof.
\end{proof}
\end{document}\end
