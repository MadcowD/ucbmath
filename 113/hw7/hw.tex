\documentclass[11pt]{amsart}

\usepackage{amsmath,amsthm}
\usepackage{amssymb}
%\usepackage{graphicx}
%\usepackage{enumerate}
\usepackage{fullpage}
% \usepackage{euscript}
% \makeatletter
% \nopagenumbers
\usepackage{verbatim}
\usepackage{color}
\usepackage{hyperref}
%\usepackage{times} %, mathtime}

\textheight=600pt %574pt
\textwidth=480pt %432pt
\oddsidemargin=15pt %18.88pt
\evensidemargin=18.88pt
\topmargin=10pt %14.21pt

\parskip=1pt %2pt

\def\reals{{\mathbb R}}
\def\torus{{\mathbb T}}
\def\integers{{\mathbb Z}}
\def\rationals{{\mathbb Q}}
\def\naturals{{\mathbb N}}
\def\complex{{\mathbb C}\/}
\def\distance{\operatorname{distance}\,}
\def\support{\operatorname{support}\,}
\def\dist{\operatorname{dist}\,}
\def\Span{\operatorname{span}\,}
\def\degree{\operatorname{degree}\,}
\def\kernel{\operatorname{kernel}\,}
\def\dim{\operatorname{dim}\,}
\def\codim{\operatorname{codim}}
\def\trace{\operatorname{trace\,}}
\def\dimension{\operatorname{dimension}\,}
\def\codimension{\operatorname{codimension}\,}
\def\kernel{\operatorname{Ker}}
\def\Re{\operatorname{Re\,} }
\def\Im{\operatorname{Im\,} }
\def\eps{\varepsilon}
\def\lt{L^2}
\def\bull{$\bullet$\ }
\def\det{\operatorname{det}}
\def\Det{\operatorname{Det}}
\def\diameter{\operatorname{diameter}}
\def\symdif{\,\Delta\,}
\newcommand{\norm}[1]{ \|  #1 \|}
\newcommand{\set}[1]{ \left\{ #1 \right\} }
\newcommand{\group}[2]{\left\langle #1, #2\right\rangle}
\def\one{{\mathbf 1}}
\def\cl{\text{cl}}

\def\newbull{\medskip\noindent $\bullet$\ }
\def\nobull{\noindent$\bullet$\ }



\def\scriptf{{\mathcal F}}
\def\scriptq{{\mathcal Q}}
\def\scriptg{{\mathcal G}}
\def\scriptm{{\mathcal M}}
\def\scriptb{{\mathcal B}}
\def\scriptc{{\mathcal C}}
\def\scriptt{{\mathcal T}}
\def\scripti{{\mathcal I}}
\def\scripte{{\mathcal E}}
\def\scriptv{{\mathcal V}}
\def\scriptw{{\mathcal W}}
\def\scriptu{{\mathcal U}}
\def\scriptS{{\mathcal S}}
\def\scripta{{\mathcal A}}
\def\scriptr{{\mathcal R}}
\def\scripto{{\mathcal O}}
\def\scripth{{\mathcal H}}
\def\scriptd{{\mathcal D}}
\def\scriptl{{\mathcal L}}
\def\scriptn{{\mathcal N}}
\def\scriptp{{\mathcal P}}
\def\scriptk{{\mathcal K}}
\def\scriptP{{\mathcal P}}
\def\scriptj{{\mathcal J}}
\def\scriptz{{\mathcal Z}}
\def\scripts{{\mathcal S}}
\def\scriptx{{\mathcal X}}
\def\scripty{{\mathcal Y}}
\def\frakv{{\mathfrak V}}
\def\frakG{{\mathfrak G}}
\def\frakB{{\mathfrak B}}
\def\frakC{{\mathfrak C}}

\newtheorem{corollary}{Corollary}
\newtheorem{theorem}{Theorem}
\newtheorem{lemma}{Lemma}
\newtheorem{definition}{Definition}

\def\soln{\noindent {\bf Solution.}\ }

\begin{document}

\begin{center}{\bf Math 113 --- Problem Set 7 --- William Guss} \end{center}


\bigskip


\medskip \noindent {\bf (P94. 2)}\ Find all orbits of the permutation 
\begin{equation*}
	\sigma = \begin{pmatrix}
		1 & 2 & 3 & 4 & 5 & 6 & 7 & 8 \\
		5 & 6 & 2 & 4 & 8 & 3 & 1 & 7
	\end{pmatrix}.
\end{equation*}
\begin{proof}
(Solution \emph{really}). Consider the following orbit of $1$,
\begin{equation*}
	1 \to 5 \to 8 \to7 \to 1
\end{equation*}
which gives $O^1 = \{1, 5, 7, 8\}$. We now consider the orbit of $2$,
\begin{equation*}
	2 \to 6 \to 3 \to 2
\end{equation*}
which gives $O^2 = \{2,3,6\}.$ We now consider the orbit of $4$,
\begin{equation*}
	4 \to 4
\end{equation*}
which gives $O^4 = \{4\}$ as a fixed point. We conclude that $\bigsqcup O^n = G$
gives a partition so all the orbits are found.
\end{proof}

\medskip \noindent {\bf (P94. 5)}\ Find all the orbits of $\sigma: \mathbb{Z} \to \mathbb{Z}$ where $\sigma(n) = n +2.$
\begin{proof}
(Solution \emph{really}). First consider that $\sigma^{-1}(n) = n -2.$ Now the first orbit $O^{0} = \{2n\mathrel{}|\mathrel{}n\in\mathbb{Z}\}$ is the set of all even integers. Then $O^{1} =  \{1+2n\mathrel{}|\mathrel{}n\in\mathbb{Z}\}$ is exactly the set of odd integers. Clearly $O^1 \sqcup O^0 = \mathbb{Z}$ and so the orbits form a partition of the space therefore all orbits are found.
\end{proof}

\medskip \noindent {\bf (P94. 7)}\ Compute the product of cycles $(1, 4, 5)(7,8)(2,5,7)$ on $\{1, \dots, 8\}.$
\begin{proof}
(Solution \emph{really}). We iterate on each cycle through composition. First
\begin{equation*}
	\sigma_1 = \begin{pmatrix}
		1 & 2 & 3 & 4 & 5 & 6 & 7 & 8 \\
		1 & 5 & 3 & 4 & 7 & 6 & 2 & 8
	\end{pmatrix}
\end{equation*}
which is composed into
\begin{equation*}
	\sigma_2\sigma_1 = \begin{pmatrix}
		1 & 2 & 3 & 4 & 5 & 6 & 7 & 8 \\
		1 & 5 & 3 & 4 & 8 & 6 & 2 & 7
	\end{pmatrix}.
\end{equation*}
Finally we take the last composition, thus giving
\begin{equation*}
	\sigma_3\sigma_2\sigma_1 = \begin{pmatrix}
		1 & 2 & 3 & 4 & 5 & 6 & 7 & 8 \\
		4 & 1 & 3 & 5 & 8 & 6 & 2 & 7
	\end{pmatrix}.
\end{equation*}
\end{proof}

\medskip \noindent {\bf (P94. 10)}\ Write the following permutation as a product of disjoint cycles and then
transpositions when
\begin{equation*}
	\sigma = \begin{pmatrix}
		1 & 2 & 3 & 4 & 5 & 6 & 7 & 8 \\
		8 & 2 & 6 & 3 & 7 & 4 & 5 & 1
	\end{pmatrix}.
\end{equation*}
\begin{proof}
(Solution \emph{really}). First we gather the constituent cycles. This can be done by obtaining each orbit and composing the cyclic product as follows. Take $O^1$ and yield
\begin{equation*}
	1 \to 8 \to 1.
\end{equation*}
Take $O^2$ and yield \begin{equation*}
	2 \to 2
\end{equation*}
Take $O^3$ and yield
\begin{equation*}
	3 \to 6 \to 4 \to 3.
\end{equation*}
Take $O^5$ and yield
\begin{equation*}
	5 \to 7 \to 5.
\end{equation*}
Now we get the following orbits $O^1 = \{1,8\}$, $O^2 = \{2\}$, $O^3 = \{3,6,4\}$, $O^5 = \{5,7\}$ such that $\bigsqcup O^n = G.$ Ignoring fixed point orbits given is thus a disjoint product of cycles $\sigma = (1,8)(3,6,4)(5,7).$

Next the decomposition of $\sigma$ into transpositions is achieved as $(3,6,4) = (3,4)(3,6)$ giving  
\begin{equation*}
	 (3 \mapsto 4 \mapsto 3)\circ(3 \mapsto 6 \mapsto 3) =\begin{pmatrix}
	 	3 & 4 & 6\\
	 	4 & 3 & 6
	 \end{pmatrix} \begin{pmatrix}
	 	3 & 4 & 6\\
	 	6 & 4 & 3
	 \end{pmatrix} = \begin{pmatrix}
	 	3 & 4 & 6 \\
	 	6 & 3 & 4
	 \end{pmatrix}.
\end{equation*}	
Therefore $\sigma = (1,8)(3,4)(3,6).$
\end{proof}
\end{document}\end
