%%%%%%%%%%%%%%%%%%%%%%%%%%%%%%%%%%%%%%%%%%%%%%%%%%%%%%%%%%%%%%%%%%
%%%                      Homework _                            %%%
%%%%%%%%%%%%%%%%%%%%%%%%%%%%%%%%%%%%%%%%%%%%%%%%%%%%%%%%%%%%%%%%%%

\documentclass[letter]{article}

\usepackage{lipsum}
\usepackage[pdftex]{graphicx}
\usepackage[margin=1.5in]{geometry}
\usepackage[english]{babel}
\usepackage{listings}
\usepackage{amsthm}
\usepackage{amssymb}
\usepackage{framed} 
\usepackage{amsmath}
\usepackage{titling}
\usepackage{fancyhdr}

\pagestyle{fancy}


\newtheorem{theorem}{Theorem}
\newtheorem{fact}{Fact}
\newtheorem{definition}{Definition}
\newtheorem{proposition}{Proposition}

\newenvironment{menumerate}{%
  \edef\backupindent{\the\parindent}%
  \enumerate%
  \setlength{\parindent}{\backupindent}%
}{\endenumerate}







%%%%%%%%%%%%%%%
%% DOC INFO %%%
%%%%%%%%%%%%%%%
\newcommand{\bHWN}{ }
\newcommand{\bCLASS}{MATH 113}

\title{\bCLASS: Notes }
\author{Scribe: William Guss}

\fancyhead[L]{\bCLASS}
\fancyhead[CO]{Notes \today}
\fancyhead[CE]{GUSS}
\fancyhead[R]{\thepage}
\fancyfoot[LR]{}
\fancyfoot[C]{}
\usepackage{csquotes}

%%%%%%%%%%%%%%

\begin{document}
\maketitle
\thispagestyle{empty}

Recall multiplying a complex number in trigonometric form
	\begin{equation*}
		\begin{aligned}
		z_k &= r(\cos \theta_k + i \sin \theta_k),\;\;\;k=1,2 \\ 
		z_1z_2 &= r_1r_2(\cos(\theta_1 + \theta_2) + i\sin(\theta_1 + \theta_2))\\
		z_k^n &= r_k^n(\cos(n\theta_k) + i\sin(n\theta_k))
		\end{aligned}	
	\end{equation*}
	The advanced student might observe that the trigonometric parameterizaiton of $z$ is \emph{homomorphic} under complex number multiplication.
	Furthermore $z_1 = z_2$ if and only if $|z_1| = |z_2|$, $\theta_1 \equiv \theta_2 mod 2\pi$.
	Using the complex conjugate we also get $z \overline{z} = |z|^2$. 
	Notationally we denote the real part and the imaginary part of a complex number $z$ by $Re(z), Im(z)$ respectively.


	Complex numbers also have the property that for any $z$, there exist $r, \theta$ such that
	\begin{equation*}
		z = re^{i\theta} = r(\cos \theta + i\sin \theta), \;\;\;\; r = |z|, \theta = Arg(z).
	\end{equation*}
	We can derive the rleation as follows. Consider the taylor series of $exp$.
	\begin{equation*}
		e^{z}= \sum_{n=1}^\infty \frac{z^n}{n!} := \sum_{n=1}^\infty \frac{Re(z^n)}{n!} + i\sum_{n=1}^\infty \frac{Im(z^n)}{n!}.
	\end{equation*}
	\begin{definition}
		A complex series $\sum_k z_k$ is absolutely convergent iff
		\begin{equation*}
			\sum_{k=1}^\infty |Re(z_k)| < \infty, \sum_{k=1}^\infty |Im(z_k)| < \infty
		\end{equation*}
	\end{definition}
	\begin{fact}
		A complex series converges if it absolutely converges.
	\end{fact}
	\begin{proposition}
		For any $z \in \mathbb{C}$ the series $e^z$ converges absolutely.
	\end{proposition}
	\begin{proof}
		Recall that $|a| \leq |a + bi|$ and $|b| \leq |a + bi$. Now consider that
		\begin{equation*}
			\begin{aligned}
				\left|\frac{Re(z^n)}{n!}\right| \leq \left|\frac{z^n}{n!}\right| \\
				\left|\frac{Im(z^n)}{n!}\right| \leq \left|\frac{z^n}{n!}\right|
			\end{aligned},
		\end{equation*}
		Therefore we need show that the series $\sum_n |z|^n/n!$ is convergent which implies that $e^z$ is absolutely convergent.
		Recall that $\sum_n |z|^n/n!$ is just $e^|z|$ which converges since $|z| \in \mathbb{R}.$ Therefore
		$e^z$ converges to a compelx number.
	\end{proof}
	\begin{fact}
		If $z_1, z_2 \in \mathbb{C}$ then $e^{z_1}e^{z_2}=e^{z_1 + z_2}$; that is, $exp$ is a homomorphism.
	\end{fact}
	\begin{proposition}
		If $a, b \in \mathbb{R}$, then $e^{a + bi} = e^a(\cos(b) + i\sin(b)).$
	\end{proposition}
	\begin{proof}
		By fact 1, we have that $e^{a + bi} = e^ae^{bi}.$ We claim that $e^{ib} = \cos b + i\sin b$. Recall the series definition of
		$e^z$,
		\begin{equation*}
			e^{ib} = \sum_{n=0}^\infty \frac{(ib)^n}{n!} = \sum_{n=0}^\infty \frac{i^n b^n}{n!}
		\end{equation*}
		Using that $i^2 = -1$, we have
		\begin{equation*}
			\begin{aligned}
			e^{ib} &= 1 + \frac{ib}{1} + \frac{-b^2}{2!} + \frac{-ib^3}{3!} + \frac{b^4}{4!} + \frac{ib^5}{5!} + \cdots \\
				   &= \sum_{k=0}^\infty \frac{(-1)^kb^{2k}}{(2k)!} + i \sum_{k=0}^\infty \frac{(-1)^k b^{2k+1}}{(2k+1)!} \\
				   &= \cos(b) + i\sin(b).
			\end{aligned}
		\end{equation*}
		This completes the proof.
	\end{proof}

	\emph{What are the complex numbers of $|\cdot| = 1$?} They must be $z = \cos \theta + i\sin \theta = e^{i\theta}$, $\theta \in \mathbb{R}$.
	We can use such an intuition to compute roots of complex numbers. 
	\begin{proposition}
		For a complex number $z$, let $R_z = \{w :\ ^n = z\ \in \mathbb{C}\}$ be th set of $n^{th}$ roots of $z$. We claim that
		\begin{equation*}
			R_z = \left\{|z|^{1/n}\exp\left(\frac{i(Arg(z) + 2\pi k)}{n}\right)\ \Big|\ k \in \mathbb{Z}\right\} 
		\end{equation*}
		and $R_z \cong \mathbb{Z}/n$ if $|z| = 1$.
	\end{proposition}
	\begin{proof}
		Take any $w \in R_z$, then there is a $k \in \mathbb{Z}$ such that
		\begin{equation*}
			w^n = \left(|z|^{1/n}\exp\left(\frac{i(Arg(z) + 2\pi k)}{n}\right)\right)^n = |z|\exp\left(i(Arg(z) + 2k\pi\right) = z.
		\end{equation*}
		Furthermore define an index set for $R_z$, such that $w_j = |z|^{1/n}\exp\left(\frac{i(Arg(z) + 2\pi j)}{n}\right)$. Then
		define the mapping $\phi: \mathbb{Z}_n \to R_z$ such that $j \mapsto w_j$. Clearly such a map is injective since there are $n$
		elements of both $\mathbb{Z}_n$ and $R_z$, and each $j$, $w_j$ in those sets respectively are unique up to $n$. Therefore $\phi$ is bijective.

		Now we claim that if $|z| = 1$ $\phi^{-1} = \gamma$ is a \emph{homomorphism}, that is $\gamma(w_kw_j) \equiv \gamma(w_k) + \gamma(w_j) \mod n$. We do simple algebra
		\begin{equation*}
			w_kw_j = |z|^{1/n}|z|^{1/n}e^{\frac{i(Arg(z) + 2\pi k)}{n}}e^{\frac{i(Arg(z) + 2\pi j)}{n}} = 1e^{\frac{i(Arg(z) + 2\pi (k+ j)}{n}} =e^{\frac{i(Arg(z) + 2\pi (k+ j))}{n} \mod 2\pi} 
		\end{equation*}
		so it follows that
		\begin{equation*}
			\gamma(w_kw_j) \equiv (2\pi(k + j) \mod 2\pi) \mod n \equiv (k + j) \mod n.
		\end{equation*}

	\end{proof}



%%%%%%% Be sure to set the counter and use menumerate

\end{document}	