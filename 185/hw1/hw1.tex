%%%%%%%%%%%%%%%%%%%%%%%%%%%%%%%%%%%%%%%%%%%%%%%%%%%%%%%%%%%%%%%%%%
%%%                      Homework 1                            %%%
%%%%%%%%%%%%%%%%%%%%%%%%%%%%%%%%%%%%%%%%%%%%%%%%%%%%%%%%%%%%%%%%%%

\documentclass[letter]{article}

\usepackage{lipsum}
\usepackage[pdftex]{graphicx}
\usepackage[margin=1.5in]{geometry}
\usepackage[english]{babel}
\usepackage{listings}
\usepackage{amsthm}
\usepackage{amssymb}
\usepackage{framed} 
\usepackage{amsmath}
\usepackage{titling}
\usepackage{fancyhdr}

\pagestyle{fancy}


\newtheorem{theorem}{Theorem}
\newtheorem{definition}{Definition}

\newenvironment{menumerate}{%
  \edef\backupindent{\the\parindent}%
  \enumerate%
  \setlength{\parindent}{\backupindent}%
}{\endenumerate}







%%%%%%%%%%%%%%%
%% DOC INFO %%%
%%%%%%%%%%%%%%%
\newcommand{\bHWN}{1}
\newcommand{\bCLASS}{MATH 185}

\title{\bCLASS: Homework \bHWN}
\author{William Guss\\26793499\\wguss@berkeley.edu}

\fancyhead[L]{\bCLASS}
\fancyhead[CO]{Homework \bHWN}
\fancyhead[CE]{GUSS}
\fancyhead[R]{\thepage}
\fancyfoot[LR]{}
\fancyfoot[C]{}
\usepackage{csquotes}

%%%%%%%%%%%%%%

\begin{document}
\maketitle
\thispagestyle{empty}


%%%%%%% Be sure to set the counter and use menumerate
\begin{menumerate}
    \item \emph{Show that multiplication of complex numbers satisfies the associative,
commutative, and distributive laws.}
    \begin{theorem}
        Given that $\mathbb{C}$ is Abelian under addition, $\mathbb{C}$ is a field.
    \end{theorem}
    \begin{proof}
        Let $a,b,c \in \mathbb{C}.$ Then recall that for any $z \in \mathbb{C},$  
        $z = |z|e^{i\theta_z},$ where $\theta_z = Arg z$. We show that $\mathbb{C}$
        satisfies associative, commutative, and distributive laws.

        Using that $\mathbb{R}$ is a field, it follows that 
        \begin{equation}
            \begin{aligned}
                (ab)c &= (|a|e^{i\theta_a}|b|e^{i\theta_b})|c|e^{i\theta_c} \\
                    &= |a||b|e^{i(\theta_a+\theta_b)}|c|e^{i\theta_c} \\ 
                    &= |a||b||c|e^{i(\theta_a+\theta_b +\theta_c)} \\
                    &= |a|e^{i\theta_a}|b||c|e^{i(\theta_b + \theta_c)} \\
                    &= a(bc).
            \end{aligned}
        \end{equation}
        Without the assumption of eulers identity  , we have that
        \begin{equation}
            \begin{aligned}
                (ab)c &= ((a_1+ia_2)(b_1 +ib_2))(c_1+ic_2) \\
                      &= ((a_1b_1 - a_2b_2) + (a_1b_2 + a_2b_1)i)(c_1+ic_2) \\
                      &= ((a_1b_1 - a_2b_2)c_1 - (a_1b_2 + a_2b_1)c_2) \\
                      &\;\;\;\;\;\;+ ((a_1b_1 - a_2b_2)c_2 + (a_1b_2 + a_2b_1)c_1)i \\
                      &= a_1b_1c_1 - a_2b_2c_1 - a_1b_2c_2 + a_2b_1c_2 \\
                      &\;\;\;\;\;\;+ (a_1b_1c_2 - a_2b_2c_2 + a_1b_2c_1 + a_2b_1c_1)i \\
                      &= a_1(b_1c_1 - b_2c_2) - a_2(b_2c_1  + b_1c_2) \\
                      &\;\;\;\;\;\;+ (a_1(b_1c_2 + b_2c_1) - a_2(b_2c_2 + b_1c_1))i \\
                      &=(a_1+a_2i)((b_1c_1 - b_2c_2) + (b_1c_2 + b_2c_1)i) \\
                      &= a(bc).
            \end{aligned}
        \end{equation}

        In a similar fashion, consider the following rearrangement which follows by the field 
        properties of $\mathbb{R}$:
        \begin{equation}
            \begin{aligned}
                ab &= (a_1b_1 - a_2b_2) + (a_1b_2 +a_2b_1)i \\
                &= (b_1a_1 - b_2a_2) + (b_2a_1 + b_1a_2)i \\
                &= ba.          
            \end{aligned}
        \end{equation}
        Lastly we show the distributive property: 
        \begin{equation}
            \begin{aligned}
                a(b+c) &= a(b_1+b_2i + c_1 + c_2i) \\
                &= a((b_1+c_1) + (b_2+c_2)i) \\
                &= (a_1(b_1+c_1) - a_2(b_2+c_2)) + (a_1(b_2+c_2) + a_2(b_1+c_1))i \\
                &= (a_1b_1 - a_2b_2)  + (a_1c_1 - a_2c_2) + (a_1b_2 + a_2b_1 )i + (a_1c_2 + a_2c_1)i   \\         
                &= ab + ac
            \end{aligned}
        \end{equation}
        Therefore $\mathbb{C}$ is a ring.
    \end{proof}


    \item Gamelin Exercise I.1.7 (Chapter I, Section 1, Exercise 7)  
    \begin{theorem}
        Let $\rho >1, \rho \neq 1$ and fix $z_0, z_1 \in \mathbb{C}.$ Then 
        $$S = \{|z-z_0| = \rho |z-z_1| : z \in \mathbb{C}\}$$ 
        is isometric to some $S^1_r \subset \mathbb{R}^2$ for some $r.$
    \end{theorem}
    \begin{proof}
        Since all $s \in S$ satisfy the above equation, we have that 
        \begin{equation}
            \sqrt{(s_1 - z_{01})^2 + (s_2 - z_{02})^2} = 
            \rho\sqrt{((s_1 - z_{11})^2 + (s_2 - z_{12})^2}.   
        \end{equation}
        The form of $(5)$ is identical to a distance meterization in $\mathbb{R}^2;$
        that is, take the isometry $\phi : \mathbb{C} \to \mathbb{R}^2, ((x + iy) \mapsto (x,y)$ and
        \begin{equation}
            \begin{aligned}
                d(\phi(s),\phi(z_0)) &= \rho d(\phi(s), \phi(z_1))
                    \frac{d(S,Z_0)}{d(S,Z_1)} = \rho,
            \end{aligned}
        \end{equation}
        which from high school geometry one might recognize as the equation of the circle
        of Appolonius. 
    \end{proof}

        The geometric proof of a equivalency between Appolonius' circle and the
        Euclidean circle is omitted.

        However, if we take the euclidean distance on $\mathbb{R}^2,$ we have the following theorem.
        \begin{theorem}
            Suppose that $P,Q \in \mathbb{R}^2$ and $S$ such that 
            $$\frac{\overline{PS}}{\overline{QS}} = k \in (0,1)[WLOG],$$
            then $S$ is a point on a circle.
        \end{theorem}
        \begin{proof}
            Observe the following algebraic derivation using the parallelagram law inspired by J Wilson at the University of Georgia:
            \begin{equation}
                \begin{aligned}
                    \frac{|P-S|^2}{|Q-S|^2} &= k^2 \\
                     |P|^2 + |S|^2 -2\langle P, S \rangle &= k^2(|Q|^2 + |S|^2 - 2\langle Q,S\rangle ) \\
                     0&= |P|^2 + |S|^2 -2\langle P, S \rangle - k^2(|Q|^2 + |S|^2 - 2\langle Q,S\rangle ) \\
                     &= (1-k^2)|S|^2 + |P|^2 -k^2|Q|^2 -2\langle P - Q, k^2S \rangle
                     &= |S|^2 + \frac{|P|^2}{1-k^2} - \frac{}k^2|Q|
                \end{aligned}
            \end{equation}
        \end{proof}
        TODOTODOTODOTODOTODOTODOTODOTODOTODOTODOTODOTODOTODOTODOTODOTODOTODOTODOTODOTODOTODOTODOTODOTODOTODOTODOTODOTODOTODOTODOTODOTODOTODOTODOTODOTODOTODOTODOTODOTODOTODOTODOTODOTODOTODOTODOTODOTODOTODOTODOTODO

        \item Gamelin Exercise I.2.5

        \begin{theorem}
            For $n\geq 1$ and $z \in \mathbb{C}$ such that $z \neq 1,$ we have that
            \begin{equation}
                1 +z + z^2 + \dots + z^n = (1-z^{n+1})/(1-z).
            \end{equation}
        \end{theorem}
        \begin{proof}
            Observe that for $z \in \mathbb{C}$ we have that, $z = e^{i\theta}.$ Therefore,
            \begin{equation}
                \begin{aligned}
                    e^{i0} + e^{i\theta} + e^{i2\theta} + \dots + e^{in\theta} &= 1 +z + z^2 + \dots + z^n
                \end{aligned}
            \end{equation}
            Multiplication by $(1 -z)$ gives, 
            \begin{equation}
                \begin{aligned}
                    (1-e^{i\theta})e^{i0} + e^{i\theta} + e^{i2\theta} + \dots + e^{in\theta} &= e^{i0} + e^{i\theta} + e^{i2\theta} + \dots + e^{in\theta}\\
                    &\;\;\; - e^{i(0 + \theta)} + e^{i(\theta + \theta)} + e^{i(2\theta+\theta)} + \dots + e^{i(n\theta + \theta)} \\
                    &= e^{i0} - e^{i(n\theta + \theta)} \\
                    &= 1 - z^{n+1}.
                \end{aligned}
            \end{equation}

            Reducing using eulers identity it follows that,
            \begin{equation}
                \begin{aligned}
                    (1-z)(1 +z + z^2 + \dots + z^n) &= (1-z^{n+1})\\
                    1 +z + z^2 + \dots + z^n &= (1-z^{n+1})/(1-z),
                \end{aligned}
            \end{equation}
            when $z \neq 1.$ This completes the proof.
        \end{proof}

        \begin{theorem}
            For $n\geq 1$ and $z \in \mathbb{C}$ such that $z \neq 1,$ we have that
            \begin{equation}
                1 +\cos \theta + \cos 2 \theta + \dots + \cos n \theta = \frac12 + \frac{\sin(n + \frac12)}{2 \sin \theta/2}
            \end{equation}
        \end{theorem}
        \begin{proof}
            Recall  that $z = r cis\theta.$ Take in particular all such $z$ whose absolute magnitude is unity. 
            Then Theorem $4$ implies that 
            \begin{equation}
                1 + cis \theta + cis 2 \theta + \dots + cis n \theta = (1-z^{n+1})/(1-z).
            \end{equation}
            A little algebra gives us 
            \begin{equation}
                \begin{aligned}
                    \frac{Re(1 - cis (n+1)\theta}{Re(1 - cis \theta)} &= \frac{Re(1 - e^{(n+1)\theta})Re(1-e^{-i\theta})}{Re(1 - e^{i\theta})Re(1-e^{-\theta i})} \\
                    &= \frac{Re(1 - e^{i\theta} - e^{i(n+1)\theta} + e^{in\theta})}{Re(2 - 2cos\theta)} \\
                    &= \frac{1 - \cos\theta - \cos (n+1)\theta + \cos n \theta}{4 \sin^2(\theta/2)} \\
                    &= \frac{2\sin^2{\theta/2}- \sin(n  + 1/2)\sin(\theta/2)}{4 \sin^2(\theta/2)} \\
                    &= \frac12 -\frac{ \sin(n  + 1/2)}{2\sin(\theta/2)} 
                \end{aligned}
            \end{equation}
            Since the above was the real part of $1 +z + z^2 + \dots + z^n,$ the theorem holds.
        \end{proof}
\end{menumerate}



\end{document}