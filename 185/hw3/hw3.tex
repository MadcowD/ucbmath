%%%%%%%%%%%%%%%%%%%%%%%%%%%%%%%%%%%%%%%%%%%%%%%%%%%%%%%%%%%%%%%%%%
%%%                      Homework _                            %%%
%%%%%%%%%%%%%%%%%%%%%%%%%%%%%%%%%%%%%%%%%%%%%%%%%%%%%%%%%%%%%%%%%%

\documentclass[letter]{article}

\usepackage{lipsum}
\usepackage[pdftex]{graphicx}
\usepackage[margin=1.5in]{geometry}
\usepackage[english]{babel}
\usepackage{listings}
\usepackage{amsthm}
\usepackage{amssymb}
\usepackage{framed} 
\usepackage{amsmath}
\usepackage{titling}
\usepackage{fancyhdr}

\pagestyle{fancy}


\newtheorem{theorem}{Theorem}
\newtheorem{definition}{Definition}

\newenvironment{menumerate}{%
  \edef\backupindent{\the\parindent}%
  \enumerate%
  \setlength{\parindent}{\backupindent}%
}{\endenumerate}







%%%%%%%%%%%%%%%
%% DOC INFO %%%
%%%%%%%%%%%%%%%
\newcommand{\bHWN}{3}
\newcommand{\bCLASS}{MATH: 185}

\title{\bCLASS: Homework \bHWN}
\author{William Guss\\26793499\\wguss@berkeley.edu}

\fancyhead[L]{\bCLASS}
\fancyhead[CO]{Homework \bHWN}
\fancyhead[CE]{GUSS}
\fancyhead[R]{\thepage}
\fancyfoot[LR]{}
\fancyfoot[C]{}
\usepackage{csquotes}

%%%%%%%%%%%%%%

\begin{document}
\maketitle
\thispagestyle{empty}

\begin{menumerate}

%%%%%%% Be sure to set the counter and use menumerate
\item II.4.5
\begin{theorem}
    The different branches of $cos^{-1}(z)$ have the same derivative.
\end{theorem}
\begin{proof}
    Let $f = cos^{-1}(z).$ Then $f'$ is determined by the following derivation;
    \begin{equation*}
        \begin{aligned}
            f'(z) = \frac{d}{dz} -i\log[z \pm \sqrt{z^2 - 1}] &= -\frac{i}{z \pm \sqrt{z^2 -1}}\frac{d}{dz}(z \pm \sqrt{z^2 -1}) \\
            &= -\frac{i}{z \pm \sqrt{z^2 -1}}(1 \pm \frac{d}{dz}\sqrt{z^2 -1})) \\
            &= -\frac{i}{z \pm \sqrt{z^2 -1}}(1 \pm \frac{1}{2\sqrt{z^2 -1}}\frac{d}{dz}(z^2 -1))) \\
            &= -\frac{i}{z \pm \sqrt{z^2 -1}}(1 \pm \frac{2z}{2\sqrt{z^2 -1}})) \\
            &= -\frac{-z\sqrt{z^2-1}}{z \sqrt{1-z^2}\sqrt{z^2-1}} \\
            &= \frac{1}{\sqrt{1-z^2}}
            &= \frac{\sqrt{1-z^2}}{1-z^2}.
        \end{aligned}
    \end{equation*}

        And so, the derivative has branches corresponding to that of $\sqrt{\gamma(z)}.$ Since this function's riemann surface
        is not regular in the sence that $\log'(z)$ is. So we have that the derivative of $\cos$ is different on different branches.

\end{proof}
\item II.4.7
\begin{theorem}
 Let $f(z)$ be a bounded analytic injective function. Then let $D \subset \mathbb(C)$ be its domain. 
 It follows that 
 \begin{equation}
    Area(f(D)) = \iint_D |f'(z)|^2 dxdy.
 \end{equation}
\end{theorem}
\begin{proof}
    The area of a region $A$ is given by the riemann integral over that region,
    $Area(A) = \int_A du.$ for $u \in \mathbb{R}^2.$ If $\phi$ is a $2$-cell, that is 
    $\phi: I^2 \to A$ is a diffeomorphism where $I^2$ is the unit square. We have that the $dx\wedge dy$
    $2$-form area is given by 
    \begin{equation}
        Area(A) = \int_\phi dx \wedge dy = \int_{I^2} \frac{\partial(\phi)}{\partial(u)} du.    
    \end{equation}    

    With thjat in mind, we can assume that $f$ is lopcally diffeomorphic by its injectivity and the inverse function theorem.
    So we assert that if $D$ is the image of a smooth $2$-cell, $\gamma$, then $f(D) = d(\gamma(i^2)).$
    Therefore, we get

    \begin{equation}
        \begin{aligned}
            Area(f(D) = \int_{f\circ \gamma} dx\wedge dy &= \int_{I^2} \frac{\partial(f \circ \gamma)}{\partial(u)} du \\
            &= \int_{I^2} \frac{\partial(f)}{\partial(v)}_{v=\gamma(u)}\frac{\partial(\gamma)}{\partial(u)}  du \\
            &= \int_{D} \frac{\partial(f)}{\partial(v)} dv\;(\mathrm{c.o.v}) \\
            &= \int_{D} |f'(v)|^2 dv\;(\mathrm{C.R.}) \\
            &= \iint_{D} |f'(v)|^2 dxdy\;(\mathrm{notation}) \\
        \end{aligned}
    \end{equation}
    And this completes the proof.
\end{proof}
\item II.5.1
I(*b) The second derivative of $xy + 3x^2y - y^3$ with $x$ is $6y$, and with $y$ is $-6y$ so their sum is
$0$ and the harmonic equation is satisfied. For the harmonic conjugate we use $u_x = v_y.$
So $u_x = y + 6xy = v_y$ so $v = \frac 12 y^2 + 3xy^2 + h(x).$ Then $u_y = -v_x.$ So 
$v_x = 3y^2 + h'(x) = -x - 3x^2 + 3y^2$ which implioes $h'(x) = -x -3x^2$ so $h(x) = -\frac12 x^2 -x^3$,
giving $v =  \frac12 y^2 + 3xy^2 - \frac12 x^2 - x^3 + C.$ 

(c) The second derivative with $x$ is $\sin hx \sin y$ and with y is $-\sin h x \sin x$ so the sum is zero and the
equation is harmonic  In this case we have that $u = \sin hx \sin y$ so it follows that $u_x = cosh(x)\sin y = v_y$ 
so $v = -\cos h(x) cos(y) + h(x).$ Then $v_x = -\sin h(x) \cos(y) + h'(x) = -\sin hx \cos y $ so $h'$ = 0 and $h = C$
Therefore the harmonic conjugate is $v = -\cos h \cos(y) + C.$
\item The proof is roughly as follows. Take a region on which the set of discontinuities of $f$ is a zero set,
In particualr for the punctured plane we could take the unit rectangle aropund the puncture. Then integration of $v$ as
determined by the harmonic conjuage method is valid in the real direction since the discontinuity set on every line is 
at most a zero set (a single point) and Fubilinilinili's theorem says intergration in this fashion is valid.
However, integration of such an $h(x)$ function fails to give a satisfactory harmonic conjugate (for the line along $Im(z) = 0)$
. In other words the equation for $v(x,y)$ does not satisfy the Laplace equations.

However such a line in that region could be integrated (for lack of existing) in the slit plane. Since the line is a zero set 
in $\mathbb{C}$ removing it from the integrand does not affect th[e result of Fublbinilili's theorem and so in this case there are
no jumps and this would suggest that the harmonic conjugate naturally satisfies the laplace equations. 

\item
\item Take the map $-z^2$ from the first quadrant complex planme and observe that its range is the lower half plane. $z^2$ is
a conformal map so its submapping on a suibmetric space is also conformal. Therefore this mapping is a conformal map.
\item Suppose that some order-derivative of $f$, say $g$, vanished. Then, we have the following argument. 
Since $f'(\gamma)$ is the tangent vector to $\gamma$ at some point, say the intersection of $\gamma,$ with $\phi$
then $f'(\gamma)$ should be orthpogonal to $g'(\gamma).$ This holds for all orders of dderivatives. Since $g$ is
$0$ at this point, we have that the curves $g'(\gamma), g'(\phi)$ are not orthogonal which is a contradiction
to the angle preserving property of $f.$ So $f'$ must not vanish.
\end{menumerate}    
\end{document}