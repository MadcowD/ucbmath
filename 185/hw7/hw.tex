\documentclass[11pt]{amsart}

\usepackage{amsmath,amsthm}
\usepackage{amssymb}
\usepackage{graphicx}
\usepackage{enumerate}
\usepackage{fullpage}
% \usepackage{euscript}
% \makeatletter
% \nopagenumbers
\usepackage{verbatim}
\usepackage{color}
\usepackage{hyperref}
%\usepackage{times} %, mathtime}

\textheight=600pt %574pt
\textwidth=480pt %432pt
\oddsidemargin=15pt %18.88pt
\evensidemargin=18.88pt
\topmargin=10pt %14.21pt

\parskip=1pt %2pt

% define theorem environments
\newtheorem{theorem}{Theorem}    %[section]
%\def\thetheorem{\unskip}
\newtheorem{proposition}[theorem]{Proposition}
%\def\theproposition{\unskip}
\newtheorem{conjecture}[theorem]{Conjecture}
\def\theconjecture{\unskip}
\newtheorem{corollary}[theorem]{Corollary}
\newtheorem{lemma}[theorem]{Lemma}
\newtheorem{sublemma}[theorem]{Sublemma}
\newtheorem{fact}[theorem]{Fact}
\newtheorem{observation}[theorem]{Observation}
%\def\thelemma{\unskip}
\theoremstyle{definition}
\newtheorem{definition}{Definition}
%\def\thedefinition{\unskip}
\newtheorem{notation}[definition]{Notation}
\newtheorem{remark}[definition]{Remark}
% \def\theremark{\unskip}
\newtheorem{question}[definition]{Question}
\newtheorem{questions}[definition]{Questions}
%\def\thequestion{\unskip}
\newtheorem{example}[definition]{Example}
%\def\theexample{\unskip}
\newtheorem{problem}[definition]{Problem}
\newtheorem{exercise}[definition]{Exercise}

\numberwithin{theorem}{section}
\numberwithin{definition}{section}
\numberwithin{equation}{section}

\def\reals{{\mathbb R}}
\def\torus{{\mathbb T}}
\def\integers{{\mathbb Z}}
\def\rationals{{\mathbb Q}}
\def\naturals{{\mathbb N}}
\def\complex{{\mathbb C}\/}
\def\distance{\operatorname{distance}\,}
\def\support{\operatorname{support}\,}
\def\dist{\operatorname{dist}\,}
\def\Span{\operatorname{span}\,}
\def\degree{\operatorname{degree}\,}
\def\kernel{\operatorname{kernel}\,}
\def\dim{\operatorname{dim}\,}
\def\codim{\operatorname{codim}}
\def\trace{\operatorname{trace\,}}
\def\dimension{\operatorname{dimension}\,}
\def\codimension{\operatorname{codimension}\,}
\def\nullspace{\scriptk}
\def\kernel{\operatorname{Ker}}
\def\p{\partial}
\def\Re{\operatorname{Re\,} }
\def\Im{\operatorname{Im\,} }
\def\ov{\overline}
\def\eps{\varepsilon}
\def\lt{L^2}
\def\curl{\operatorname{curl}}
\def\divergence{\operatorname{div}}
\newcommand{\norm}[1]{ \|  #1 \|}
\def\expect{\mathbb E}
\def\bull{$\bullet$\ }
\def\det{\operatorname{det}}
\def\Det{\operatorname{Det}}
\def\rank{\mathbf r}
\def\diameter{\operatorname{diameter}}

\def\t2{\tfrac12}

\newcommand{\abr}[1]{ \langle  #1 \rangle}

\def\newbull{\medskip\noindent $\bullet$\ }
\def\field{{\mathbb F}}
\def\cc{C_c}

\newenvironment{solution}
  {\begin{proof}[Solution]}
  {\end{proof}}



% \renewcommand\forall{\ \forall\,}

% \newcommand{\Norm}[1]{ \left\|  #1 \right\| }
\newcommand{\Norm}[1]{ \Big\|  #1 \Big\| }
\newcommand{\set}[1]{ \left\{ #1 \right\} }
%\newcommand{\ifof}{\Leftrightarrow}
\def\one{{\mathbf 1}}
\newcommand{\modulo}[2]{[#1]_{#2}}

\def\bd{\operatorname{bd}\,}
\def\cl{\text{cl}}
\def\nobull{\noindent$\bullet$\ }

\def\scriptf{{\mathcal F}}
\def\scriptq{{\mathcal Q}}
\def\scriptg{{\mathcal G}}
\def\scriptm{{\mathcal M}}
\def\scriptb{{\mathcal B}}
\def\scriptc{{\mathcal C}}
\def\scriptt{{\mathcal T}}
\def\scripti{{\mathcal I}}
\def\scripte{{\mathcal E}}
\def\scriptv{{\mathcal V}}
\def\scriptw{{\mathcal W}}
\def\scriptu{{\mathcal U}}
\def\scriptS{{\mathcal S}}
\def\scripta{{\mathcal A}}
\def\scriptr{{\mathcal R}}
\def\scripto{{\mathcal O}}
\def\scripth{{\mathcal H}}
\def\scriptd{{\mathcal D}}
\def\scriptl{{\mathcal L}}
\def\scriptn{{\mathcal N}}
\def\scriptp{{\mathcal P}}
\def\scriptk{{\mathcal K}}
\def\scriptP{{\mathcal P}}
\def\scriptj{{\mathcal J}}
\def\scriptz{{\mathcal Z}}
\def\scripts{{\mathcal S}}
\def\scriptx{{\mathcal X}}
\def\scripty{{\mathcal Y}}
\def\frakv{{\mathfrak V}}
\def\frakG{{\mathfrak G}}
\def\aff{\operatorname{Aff}}
\def\frakB{{\mathfrak B}}
\def\frakC{{\mathfrak C}}

\def\suchthat{\mathrel{}:\mathrel{}}
\def\symdif{\,\Delta\,}
\def\mustar{\mu^*}
\def\muplus{\mu^+}

\def\soln{\noindent {\bf Solution.}\ }


%\pagestyle{empty}
%\setlength{\parindent}{0pt}

\begin{document}

\begin{center}{\bf Math 185 --- UCB, Fall 2016 --- William Guss}
\\
{\bf Problem Set 7, due November 4th}
\end{center}
\medskip \noindent {\bf (68.1)}\ Find the Laurent series that represents the function
\begin{equation*}
	f(z) = z^2 \sin\left(\frac{1}{z^2}\right)
\end{equation*}
\begin{solution}
	We first recall the Mclauren series for $\sin$ is given by
	\begin{equation*}
		\sin z = \sum_{n=0}^\infty (-1)^n \frac{z^{2n+1}}{(2n+1)!}	
	\end{equation*}
	Then composing $\sin$ with $1/z^2$ we get
	\begin{equation*}
		\sin 1/z^2 = \sum_{n=0}^\infty (-1)^n \frac{{\left(z^{-2}\right)}^{2n+1}}{(2n+1)!}	=  \sum_{n=0}^\infty (-1)^n \frac{{z}^{-4n-2}}{(2n+1)!}	
	\end{equation*}
	TBy multiplying by the $z^2$ (already in Lauren form) we get
	\begin{equation*}
	z^2 \sin 1/z^2 = z^2 \sum_{n=0}^\infty (-1)^n \frac{{z}^{-4n-2}}{(2n+1)!}	 = 1 + \sum_{n=1}^\infty (-1)^n \frac{1}{(2n+1)!}\cdot {(z-0)}^{-4n}.	
	\end{equation*}
\end{solution}

\medskip \noindent {\bf (68.2)}\ Find a representation of the function
\begin{equation*}
	f(z) = \frac{1}{z+1} = \frac{1}{z}\cdot \frac{1}{1 + (1/z)}
\end{equation*}
in negative powers of $z$ that is valid when $1 < |z| < \infty$.
\begin{solution}
	First we consider the function $\frac{1}{1 + (1/z)}$ when $1 < |z|.$ In this case $|1/z| < 1$
	so \begin{equation*}
		\frac{1}{1 + (1/z)} = \frac{1}{1 - (-1/z)} = \sum_{n=0}^\infty (-1)^n \left(\frac{1}{z}\right)^n.
	\end{equation*}
	Then multiplying by $1/z$ we get that
	\begin{equation*}
		f(z) = \frac{1}{z}\sum_{n=0}^\infty (-1)^n \left(\frac{1}{z}\right)^n = \sum_{n=0}^\infty (-1)^n \frac{1}{z^{n+1}} = \sum_{n=1}^\infty \frac{ (-1)^{n+1}}{z^{n}}.
	\end{equation*}
\end{solution}

\medskip \noindent {\bf (68.3)}\ Find the Laurent series that represents the function
\begin{equation*}
	f(z) = \frac{1}{z(1+z^2)} = \frac{1}{z} \cdot \frac{1}{1 + z^2}
\end{equation*}
when $1 < |z| < \infty.$
\begin{solution}
	First observe that 
	\begin{equation*}
		\frac{1}{z^2}\frac{1}{1 - (-\frac{1}{z^2})} = \frac{1}{z^2 - z^2(-z^2)} = \frac{1}{z^2+ 1}
	\end{equation*}
	Then 
	\begin{equation*}
	\begin{aligned}
		f(z) &= \frac{1}{z} \left(\frac{1}{1 + z}\circ z^2\right) 
		= \frac{1}{z} \frac{1}{z^2}\left(\frac{1}{1 -\frac{1}{z}} \circ -z^2\right) \\
		&= \frac{1}{z^3} \left[\sum_{n=0}^\infty\ \frac{1}{z^n} \circ -z^2 \right] = \frac{1}{z^3} \sum_{n=0}^\infty \frac{(-1)^n}{z^{2n}} \\
		&= \sum_{n=0}^\infty \frac{(-1)^n}{z^{2n + 3}} = \sum_{n=1}^\infty \frac{(-1)^{n+1}}{z^{2n+1}}.
	\end{aligned}
	\end{equation*}
	This completes the exercise.
\end{solution}


\medskip \noindent {\bf (68.4)}\  Give two Laurent series expansions in pwoers of $z$ for the function
\begin{equation*}
	f(z)  = \frac{1}{z^2(1-z)}
\end{equation*} and specify the regions in which the expansions are valid.
\begin{solution}
	First we consider the region $0 < |z| < 1$. Then
	\begin{equation*}
		f(z) = \frac{1}{z^2} \sum_{n=0}^\infty z^n =  \sum_{n=1}^\infty z^{n-2} + \frac{1}{z} + \frac{1}{z^2}.
	\end{equation*}
	Substituting the indices we get
	\begin{equation*}
		f(z) = \sum_{n=0}^\infty z^{n} + \frac{1}{z} + \frac{1}{z^2}.
	\end{equation*}
	Outside of the unit disk,
	\begin{equation*}
		f(z) = \frac{1}{z^2} \frac{-1}{z}\frac{1}{1 - \frac{1}{z}} = -\frac{1}{z^3} \sum_{n=0}^\infty \frac{1}{z^n} = -\sum_{n=3}^\infty \frac{1}{z^n}.
	\end{equation*}
	This completes the proof.
\end{solution}

\medskip \noindent {\bf (68.6)}\ Show that when $0 < |z-1| < 2$,
\begin{equation*}
 	\frac{z}{(z-1)(z-3)} = -3 \sum_{n=0}^\infty \frac{(z-1)^n}{2^{n+2}} - \frac{1}{2(z-1)}.
 \end{equation*} 
 \begin{proof}
 	Starting from the right hand side we yield \begin{equation*}
 		\begin{aligned}
 			-3 \sum_{n=0}^\infty \frac{(z-1)^n}{2^{n+2}} - \frac{1}{2(z-1)} &= \frac{-3}{4} \sum_{n=0}^\infty \frac{(z-1)^n}{2^{n}} - \frac{1}{2(z-1)} \\
 			&=  \frac{-3}{4} \frac{1}{1 - \frac{z-1}{2}}- \frac{1}{2(z-1)} \\
 			&=  \frac{-3}{2} \frac{1}{2 - (z-1)}- \frac{1}{2(z-1)} \\
 			&=  \frac{-3}{2} \frac{1}{3 - z}- \frac{1}{2(z-1)} \\
 			&=  \frac{1}{2} \left(\frac{-3}{3 - z}- \frac{1}{z-1}\right) \\
 			&=  \frac{1}{2} \left(\frac{3}{z-3}- \frac{1}{z-1}\right) \\
 			&=  \frac{1}{2} \left(\frac{3z-2 -z + 3}{(z-3)(z-1)}\right)\\
 			&= \frac{z}{(z-3)(z-1)}.\\
 		\end{aligned}
 	\end{equation*}
 	The replacement of the series above is given by $0 < |z-1| < 2.$
 \end{proof}

 \medskip \noindent {\bf (72.1)}\ By differentiating the Maclaurin series representation
 \begin{equation*}
 \frac{1}{1 - z} = \sum_{n=0}^\infty z^n \;\;\;\;\;\;(|z| < 1)
 \end{equation*}
 obtain the expansions 
 \begin{equation*}
 	\frac{1}{(1-z)^2} = \sum_{n=0}^\infty (n+1)z^n \;\;\;\;\;\;(|z| < 1)
 \end{equation*} and 
 \begin{equation*}
 	\frac{2}{(1-z)^3} = \sum_{n=0}^\infty (n+1)(n+2)z^n \;\;\;\;\;\;(|z| < 1)
 \end{equation*}
 \begin{proof}
 	First recall that differentiation of a uniformly convergent sequence of functions is the limit of the derivatives of those functions. By the book Laurent series converge uniformly in their radius of convergence, thus
 	\begin{equation*}
 	\frac{(-1)^2}{(1-z)^2}= \frac{d}{dz}	\frac{1}{1-z} =\frac{d}{dz} \sum_{n=0}^\infty z^n = \sum_{n=1}^\infty n z^{n-1} = \sum_{n=0}^\infty (n+1) z^n
 	\end{equation*}
 	when $|z| < 1$. Differentiation again (valid by the analyicity of $1/1-z$ on $\mathbb{C} \setminus \{1\}$) gives 
 	\begin{equation*}
 		\frac{(-1)(-2)}{(1-z)^3} \frac{d}{dz} \frac{1}{(1-z)^2} = \frac{d}{dz} \sum_{n=0}^\infty (n+1) z^n = \sum_{n=1}^\infty n(n+1) z^{n-1} = \sum_{n=0}^\infty (n+1)(n+2) z^{n}
 	\end{equation*}
 	when $|z| < 1$. This completes the proof
 \end{proof}


 \medskip \noindent {\bf (72.3)}\ Find the taylor series expansion for the function 
 \begin{equation*}
  	\frac{1}{z} = \frac{1}{2 + (z-2)} = \frac{1}{2} \cdot \frac{1}{1 + (z-2)/2}
\end{equation*} 
about the point $z_0 = 2.$ Then by differentiating the series term by term, show that
\begin{equation*}
	\frac{1}{z^2} = \frac{1}{4} \sum_{n=0}^\infty (-1)^n(n+1) \left(\frac{z-2}{2}\right)^n.
\end{equation*}
\begin{proof}
	First we observe that when $z_0 = 2$, $|z-2|/2 < 1$ when $|z-z_0| < 2$ so we consider that region. Recalling the standard geometric series we get
	\begin{equation*}
		\frac{1}{2}\frac{1}{1 + (z-2/2)} = \frac{1}{2} \frac{1}{1 - (-\frac{z-2}{2})} = \sum_{n=0}^\infty (-1)^n\frac{(z-2)^n}{2^{n+1}}.
	\end{equation*}
	Then differentation of the series gives
	\begin{equation*}
		\frac{1}{z^2}= -\frac{d}{dz}\  \frac{1}{z} = -\frac{d}{dz}  \sum_{n=0}^\infty (-1)^n\frac{(z-2)^n}{2^{n+1}} =  \sum_{n=1}^\infty (-1)^{n+1}n\frac{(z-2)^{n-1}}{2^{n+1}}. 
	\end{equation*}
	Replacing $n$ with $n+1$ we get
	\begin{equation*}
	\frac{1}{z^2} =  \sum_{n=0}^\infty (-1)^n(n+1)\frac{(z-2)^{n}}{2^{n+2}} = \frac{1}{4} \sum_{n=0}^\infty (-1)^n(n+1)\frac{(z-2)^{n}}{2^{n}}.
	\end{equation*}
	This completes the proof.
\end{proof}
\medskip \noindent {\bf (72.4)}\ Show that the function defined by means of the equations
\begin{equation*}	
	f(z) = \begin{cases}
		\frac{1-\cos z}{z^2} & \text{if }\ z \neq 0 \\
		1/2 &\text{otherwise}
	\end{cases} 
\end{equation*}
is entire.
\begin{proof}
	First we will show that when $z \neq 0$, $z$ is analytic, and that $f(0) = 1/2$ is 
	the natural analytic extension in the sense that the series for $f(z)$ when $z \neq 0$
	converges at $z = 0.$

	Observe that when $z \neq 0$, $f$ is the composition of elementary functions. Next
	the series for $\cos z$ is given by
	\begin{equation*}
		\cos z = \sum_{n=0}^\infty (-1)^n \frac{z^{2n}}{(2n)!}
	\end{equation*}
	and thus we apply the definition of $f$ in terms of $\cos$
	\begin{equation*}
	\begin{aligned}
		\frac{1- \cos z}{z^2} &= \frac{1}{z^2}\left(1 - \sum_{n=0}^\infty (-1)^n \frac{z^{2n}}{(2n)!}\right) \\
		&= \frac{1}{z^2} + \sum_{n=0}^\infty (-1)^{n+1} \frac{z^{2n-2}}{(2n)!} \\
		&= \frac{1}{z^2} - \frac{1}{z^2} + \sum_{n=1}^\infty (-1)^{n+1} \frac{z^{2n-2}}{(2n)!} \\
		&= \sum_{n=0}^\infty (-1)^{n} \frac{z^{2n}}{(2n+2)!}.
	\end{aligned}
	\end{equation*}
	Now when $z = 0$, this series converges to $1/2$ since every term but the first is $0$, and thus $f$ is analyitc at $z = 0.$
	This completes the proof.
\end{proof}	

\medskip \noindent {\bf (72.5)}\ Show that the function defined by means of the equations
\begin{equation*}	
	f(z) = \begin{cases}
		-\frac{1}{\pi} & \text{if }\ z = \pm \pi/2, \\
		\frac{\cos z}{z^2 - (\pi/2)^2} &\text{otherwise}
	\end{cases} 
\end{equation*}
is entire.
\begin{proof}
	We write $f$ as its Laurent series for its otherwise condition. Clearly when $z \neq \pm \pi/2$ $f$ is the composition of analytic functions and is therefore analytic; thus it is uniqueley determined by its Laurent series.
	First recall that
	\begin{equation*}
		\cos z = \sum_{n=0}^\infty (-1)^n \frac{z^{2n}}{(2n)!}.
	\end{equation*}
	Then we use this composition to describe $f$ as follows
	\begin{equation*}
		\frac{\cos z}{(z- \pi/2)(z+ \pi/2)} = \cos z \left(\frac{A}{z - \pi/2} +  \frac{B}{z+ \pi/2}\right)
	\end{equation*}
	where 
	\begin{equation*}
		1 = A(z + \pi/2) + B(z - \pi/2),\ \implies (z = -\pi/2); B = -2/\pi \wedge (z = \pi/2); A = 2/\pi.
	\end{equation*}
	Appling the derivation for partial fractions we have
	\begin{equation*}
		f(z) = \cos z \left(\frac{2/\pi}{z - \pi/2} - \frac{2/\pi}{z + \pi/2}\right).
	\end{equation*}
	Now  we have that under substitution on $z$
	\begin{equation*}
		\begin{aligned}
		f(z) &= \sum_{n=0}^\infty (-1)^n \frac{z^{2n}}{(2n)!} \left(\frac{1}{z}\frac{2/\pi}{1 - \frac{\pi}{2z}} - \frac{1}{z}\frac{2/\pi}{1 + \frac{\pi}{2z}}\right) \\
		&= \frac{2}{\pi}\sum_{n=0}^\infty (-1)^n \frac{z^{2n-1}}{(2n)!} \left(\sum_{m=0}^\infty \frac{\pi^m}{(2z)^m} - \left[\sum_{m=0}^\infty \frac{\pi^m}{(2z)^m} \circ -z \right]\right) \\
		&=\frac{2}{\pi}\sum_{n=0}^\infty (-1)^n \frac{z^{2n-1}}{(2n)!} \left(\sum_{m=0}^\infty \frac{\pi^m}{(2z)^m} + \sum_{m=0}^\infty \frac{(-1)^{m+1}\pi^m}{(2z)^m}\right) \\
		&=\frac{2}{\pi} \sum_{m=0}^\infty  \sum_{n=0}^\infty  (-1)^n \frac{z^{2n-1}}{(2n)!} \frac{\pi^{2m-1}}{(2z)^{2m-1}} \\
		&=\frac{2}{\pi} \sum_{m=0}^\infty  \left(\sum_{n=0}^m  (-1)^n \frac{\pi^{2m-1}}{(2n)!z^{2(m- n)} 2^{2(m-n)}} + \sum_{n=m+1}^\infty  (-1)^n \frac{\pi^{2m-1}z^{2(n-m)} }{(2n)!2^{2(n-m)}} \right) \\
		\end{aligned}
	\end{equation*}
	Then we evaluate series when $z = \pm\pi/2.$ In the positive case
	\begin{equation*}
		\begin{aligned}
			f(\pi/2) &= \frac{2}{\pi} \sum_{m=0}^\infty  \left(\sum_{n=0}^m  (-1)^n \frac{\pi^{2m-1}}{(2n)!\pi^{2(m- n)}} + \sum_{n=m+1}^\infty  (-1)^n \frac{\pi^{2m-1}\pi^{2(n-m)} }{(2n)!2^{4(n-m)}} \right) \\
			 &= \frac{2}{\pi} \sum_{m=0}^\infty  \left(\sum_{n=0}^m  (-1)^n \frac{\pi^{2n-1}}{(2n)!} + \sum_{n=m+1}^\infty  (-1)^n \frac{\pi^{2n -1} }{(2n)!2^{4(n-m)}} \right) \\
			 &= \frac{2}{\pi} \sum_{m=0}^\infty  \left(\sum_{n=0}^\infty  (-1)^n \frac{\pi^{2n-1}}{(2n)!2^{\chi_{n>m}4(n-m)}}  \right) \\
			 &= \frac{2}{\pi} \sum_{m=0}^\infty \sum_{n=0}^\infty 2^{\chi_{n>m}4(m-n)}  (-1)^n \frac{\pi^{2n-1}}{(2n)!}  \\
			 &= \frac{2}{\pi} \sum_{n=0}^\infty  (-1)^n \frac{\pi^{2n-1}}{(2n)!}
			 \sum_{m=0}^\infty 2^{\chi_{n>m}4(m-n) } \\
			 &= \frac{2}{\pi} \sum_{n=0}^\infty  (-1)^n \frac{\pi^{2n-1}}{(2n)!} \left(n + \sum_{k=1}^\infty \frac{1}{2^{4k}}\right)  \\
		\end{aligned}
	\end{equation*}
	thus convergence is given as desired. The same verification holds for the negative case.
	
\end{proof}

\medskip \noindent {\bf (72.11)}\ Show that the function
\begin{equation*}
	f_2(z) = \frac{1}{z^2 + 1}
\end{equation*}
is the analytic continuation of the function
\begin{equation*}
	f_1(z) = \sum_{n=0}^\infty (-1)^n z^{2n}.
\end{equation*}
\begin{proof}
	To show that $f_2$ is the analytic continuation of the function we need only show that the functions agree (in their unique Laurent seires representation)
 	on the intersection of their analytic domains. As stated in the book, that intesection is the unit disk $|z| < 1$.

 	Observe that
 	\begin{equation*}
 		\frac{1}{1 + z^2} = \frac{1}{1 - z} \circ -z^2	
 	\end{equation*}
 	and in particular if $|z| < 1, |z|^2 < 1$, and thus
 	\begin{equation*}
 		\frac{1}{1 + z^2} = \sum_{n=0}^\infty z^n \circ -z^2 = \sum_{n=0}^\infty z^{2n}(-1)^n = f_1(z)
 	\end{equation*}
 	as desired. Therefore $f_2$ is the analytical continuation of $f_1.$
\end{proof}
\medskip \noindent {\bf (73.1)}\ Use the multiplication of series to show that
\begin{equation*}
	\frac{e^z}{z(z^2+1)} = \frac{1}{z} + 1 - \frac{1}{2}z - \frac{5}{6}z^2.
\end{equation*}
\begin{proof} 
First the series for $e^z = \sum_{n=0}^\infty z^n/n!$, then the series expansion for $e^z/z$ is namely
\begin{equation*}
	\frac{e^z}{z} = \sum_{n=0}^\infty \frac{z^{n-1}}{n!}.
\end{equation*}
Next we cite the above exercise and manually compute 
the multiplication. First
\begin{equation*}
	\begin{aligned}
		1/(1+z^2) = \sum_{n=0}^\infty z^{2n}(-1)^n \\
		e^z = \sum_{n=0}^\infty z^n/n! \\
	\end{aligned}
\end{equation*}
Then we multiply
\begin{equation*}
	\begin{aligned}
		&1 + z + \frac{1}{2}z^2 + \frac{1}{6}z^3 + \cdots\\
		\times\;\;\; &1 -z^2 + z^4 - z^6 + z^8 -\cdots \\
		=\;\;\;& 1 + z + \frac{1}{2}z^2 + \frac{1}{6}z^3 \\
		+\;\;\;& -z^2 + -z^3 - \frac{1}{2}z^4 + -\frac{1}{6}z^5 \\
		+\;\;\;&z^4 + z^5 + \frac{1}{2}z^6 + \frac{1}{6}z^7 \\
		\vdots\;\;\;\; \\
		=\;\;\;&1 + z - \frac{1}{2}z^2 - \frac{5}{6}z^3 + \cdots
	\end{aligned}
\end{equation*}
Then division by $z$ gives
\begin{equation*}
	\frac{e^z}{z}\cdot  \frac{1}{1+z^2} = \frac{1}{z} + 1 - \frac{1}{2}z - \frac{5}{6}z^2 + \cdots
\end{equation*}

\end{proof}

\medskip \noindent {\bf (73.2)}\ By multiplying the two Maclauren series term byy term show that:\\
(a) $e^z \sin z = z + z^2 + 1/3 z^3 + \cdots$
\begin{solution}
	First recall 
	\begin{equation*}
		e^z = \sum_{n=0}^\infty \frac{z^n}{n!}\;\;\wedge\;\;\sin z = \sum_{n=0}^\infty (-1)^n \frac{z^{2n+1}}{(2n+1)!}
	\end{equation*}
	Then
	\begin{equation*}
	\begin{aligned}
		&1 + z + \frac{1}{2}z^2 + \frac{1}{6}z^3 + \cdots\\
		\times\;\;\; &z - \frac{1}{6}z^3 + \frac{1}{120}z^5\cdots \\
		=\;\;\;& z + z^2 + \frac{1}{2}z^3 + \frac{1}{6}z^4 + \cdots\\
		+\;\;\;& - \frac{1}{6}z^3  -\frac{1}{6}z^4 + -  \frac{1}{12}z^5 -  \frac{1}{36}z^6 - \cdots\\
		+\;\;\;& - \frac{1}{6}z^3  -\frac{1}{6}z^4  -  \frac{1}{12}z^5 -  \frac{1}{36}z^6 - \cdots\\
	\end{aligned}
	\end{equation*}
	Thus
	\begin{equation*}
		e^z \sin z = z +z^2 + \frac{1}{3}z^3 + \cdots 
	\end{equation*}
\end{solution}
(b) $e^z/(1+z) = 1 + 1/2z^2 - 1/3z^3 + \cdots$
\begin{solution}
First recall 
	\begin{equation*}
		e^z = \sum_{n=0}^\infty \frac{z^n}{n!}\;\;\wedge\;\;1/(1+z) = \sum_{n=0}^\infty z^{n}(-1)^n 
	\end{equation*}
	Then
	\begin{equation*}
	\begin{aligned}
		&1 + z + \frac{1}{2}z^2 + \frac{1}{6}z^3 + \cdots\\
		\times\;\;\; &1 -z + z^2 + z^3 - \cdots \\
		=\;\;\;& 1 + z + \frac{1}{2}z^2 + \frac{1}{6}z^3 + \cdots\\
		+\;\;\;& -z - z^2 - \frac{1}{2}z^3 - \frac{1}{6}z^4 + \cdots\\
		+\;\;\;& z^2 + z^3 + \frac{1}{2}z^4 + \frac{1}{6}z^5 + \cdots\\
		+\;\;\;& -z^3 - z^4 - \frac{1}{2}z^5 - \frac{1}{6}z^6 + \cdots\\
	\end{aligned}
	\end{equation*}
	Thus
	\begin{equation*}
		e^z/(1+z) = 1 + \frac{1}{2}z^2 - \frac{1}{3}z^3 + \cdots
	\end{equation*}
\end{solution}

\medskip \noindent {\bf (73.4)}\ Use division to obtain the laurent series representation
\begin{equation*}
 	\frac{1}{e^z - 1} = \frac{1}{z} - \frac{1}{2 } + \frac{1}{12} - \frac{1}{720}z^3 + \cdots.
 \end{equation*} 
 \begin{solution}
 	First $e^z -1 = \sum_{n=1}^\infty z^n/n!$. Next 
 	we apply the division algoirithm as depicted in the last example and when $0 < |z| < 2\pi$ get
 	\begin{equation*}
 	\begin{aligned}
 		1/(e^z - 1) &= \frac{1}{x + \frac{x^2}{2} + \frac{x^3}{6} \cdots } \;\;\;\;\;\;\;\;\;\;\;\;\;\;\;\;\;\;\;\;\;\;\;\;\;\;\;\;\;\;\;\;\;\;\;\;\;\;\;\;\;\;\;\;\;\;\;\;\;\;\;\;\;\;\;\;\;\;\;\;\;\;\;\;\;\;\;\;\;\;\;\;\;\;\;\;\;\;\;\;\;\;\;\;\;\;\;\;\;\;\;\;\;\;\;\;\;\;\;\;\hfill\\
 		&=
 	\end{aligned}
 	\end{equation*} \\[4in]
 \end{solution}

\medskip \noindent {\bf (73.5)}\ Show that \begin{equation*}
	\int_C \frac{dz}{z^2 \sinh z} = -\frac{\pi i}{3}
\end{equation*} 
\begin{proof}
	First observe that as an immediate consequence of the Laurent series (8) in Sec 73 \begin{equation*}
		\frac{1}{z^2 \sinh} = \frac{1}{z^3 } - \frac{1}{6}\cdot \frac{1}{z} + \frac{7}{360}\cdot z + \cdots
	\end{equation*}
	when $0 < |z| < 2\pi.$ Then obsercing that first  coefficient of the inverse polynomial part of the laurent series uniqueley determines
	\begin{equation*}
		b_1 = \frac{1}{2\pi i} \int_C \frac{dz}{z^2 \sinh z} = -\frac{1}{6} \implies \int_C \frac{dz}{z^2 \sinh z} = -\frac{\pi i}{3}.
	\end{equation*}
\end{proof}

\medskip \noindent {\bf (73.9)}\ The \textbf{Euler numbers} are the numbers $E_n (n = 0, 1, 2, \dots)$ in the Maclaurin series representation
\begin{equation*}
 	\frac{1}{\cosh z} = \sum_{n=0}^\infty \frac{E_n}{n!} z^n
 \end{equation*} 
 Point out why this representation is valid in the indicated disk ($|z| < \pi/2$) and why $E_{2n+1} = 0$ for all $n \geq 0.$
 Then show that 
 \begin{equation*}
 	E_0 = 1, E_2 = -1, E_4 = 5, E_6=  -61.
 \end{equation*}
 \begin{solution}
 	First observe that $\frac{1}{cosh} = \frac{2}{e^x + e^{-x}}.$ Then in the disk of radius $\pi/2$ (such that $e^{-z} \neq -e^{z}$) we have that the inversion of the series 
 	\begin{equation*}
 		\frac{e^z + e^{-z}}{2} = \frac{\sum_{n=0}^\infty \frac{z^n}{n!} + \sum_{n=0}^\infty (-1)^n \frac{z^n}{n!}}{2} = \sum_{n=0}^\infty \frac{z^{2n}}{(2n)!}
 	\end{equation*}
 	is valid. It is immediate that there are no terms of odd power in the derivative; that is $E_{2n+1} = 0.$ Next it follows that by long division:
 	\\[5in]
 \end{solution}
\end{document}\end
