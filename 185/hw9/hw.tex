 \documentclass[11pt]{amsart}

\usepackage{amsmath,amsthm}
\usepackage{amssymb}
\usepackage{graphicx}
\usepackage{enumerate}
\usepackage{fullpage}
\usepackage{tikz}
\usetikzlibrary{calc,decorations.markings}
% \usepackage{euscript}
% \makeatletter
% \nopagenumbers
\usepackage{verbatim}
\usepackage{color}
\usepackage{hyperref}
%\usepackage{times} %, mathtime}

\textheight=600pt %574pt
\textwidth=480pt %432pt
\oddsidemargin=15pt %18.88pt
\evensidemargin=18.88pt
\topmargin=10pt %14.21pt

\parskip=1pt %2pt

% define theorem environments
\newtheorem{theorem}{Theorem}    %[section]
%\def\thetheorem{\unskip}
\newtheorem{proposition}[theorem]{Proposition}
%\def\theproposition{\unskip}
\newtheorem{conjecture}[theorem]{Conjecture}
\def\theconjecture{\unskip}
\newtheorem{corollary}[theorem]{Corollary}
\newtheorem{lemma}[theorem]{Lemma}
\newtheorem{sublemma}[theorem]{Sublemma}
\newtheorem{fact}[theorem]{Fact}
\newtheorem{observation}[theorem]{Observation}
%\def\thelemma{\unskip}
\theoremstyle{definition}
\newtheorem{definition}{Definition}
%\def\thedefinition{\unskip}
\newtheorem{notation}[definition]{Notation}
\newtheorem{remark}[definition]{Remark}
% \def\theremark{\unskip}
\newtheorem{question}[definition]{Question}
\newtheorem{questions}[definition]{Questions}
%\def\thequestion{\unskip}
\newtheorem{example}[definition]{Example}
%\def\theexample{\unskip}
\newtheorem{problem}[definition]{Problem}
\newtheorem{exercise}[definition]{Exercise}

\numberwithin{theorem}{section}
\numberwithin{definition}{section}
\numberwithin{equation}{section}

\def\reals{{\mathbb R}}
\def\torus{{\mathbb T}}
\def\integers{{\mathbb Z}}
\def\rationals{{\mathbb Q}}
\def\naturals{{\mathbb N}}
\def\complex{{\mathbb C}\/}
\def\distance{\operatorname{distance}\,}
\def\support{\operatorname{support}\,}
\def\dist{\operatorname{dist}\,}
\def\Span{\operatorname{span}\,}
\def\degree{\operatorname{degree}\,}
\def\kernel{\operatorname{kernel}\,}
\def\dim{\operatorname{dim}\,}
\def\codim{\operatorname{codim}}
\def\trace{\operatorname{trace\,}}
\def\dimension{\operatorname{dimension}\,}
\def\codimension{\operatorname{codimension}\,}
\def\nullspace{\scriptk}
\def\kernel{\operatorname{Ker}}
\def\p{\partial}
\def\Re{\operatorname{Re\,} }
\def\Im{\operatorname{Im\,} }
\def\ov{\overline}
\def\eps{\varepsilon}
\def\lt{L^2}
\def\curl{\operatorname{curl}}
\def\divergence{\operatorname{div}}
\newcommand{\norm}[1]{ \|  #1 \|}
\def\expect{\mathbb E}
\def\bull{$\bullet$\ }
\def\det{\operatorname{det}}
\def\Det{\operatorname{Det}}
\def\rank{\mathbf r}
\def\diameter{\operatorname{diameter}}

\def\t2{\tfrac12}

\newcommand{\abr}[1]{ \langle  #1 \rangle}

\def\newbull{\medskip\noindent $\bullet$\ }
\def\field{{\mathbb F}}
\def\cc{C_c}

\newenvironment{solution}
  {\begin{proof}[Solution]}
  {\end{proof}}



% \renewcommand\forall{\ \forall\,}

% \newcommand{\Norm}[1]{ \left\|  #1 \right\| }
\newcommand{\Norm}[1]{ \Big\|  #1 \Big\| }
\newcommand{\set}[1]{ \left\{ #1 \right\} }
\newcommand{\parens}[1]{ \left( #1 \right) }
%\newcommand{\ifof}{\Leftrightarrow}
\def\one{{\mathbf 1}}
\newcommand{\modulo}[2]{[#1]_{#2}}

\def\bd{\operatorname{bd}\,}
\def\cl{\text{cl}}
\def\nobull{\noindent$\bullet$\ }

\def\scriptf{{\mathcal F}}
\def\scriptq{{\mathcal Q}}
\def\scriptg{{\mathcal G}}
\def\scriptm{{\mathcal M}}
\def\scriptb{{\mathcal B}}
\def\scriptc{{\mathcal C}}
\def\scriptt{{\mathcal T}}
\def\scripti{{\mathcal I}}
\def\scripte{{\mathcal E}}
\def\scriptv{{\mathcal V}}
\def\scriptw{{\mathcal W}}
\def\scriptu{{\mathcal U}}
\def\scriptS{{\mathcal S}}
\def\scripta{{\mathcal A}}
\def\scriptr{{\mathcal R}}
\def\scripto{{\mathcal O}}
\def\scripth{{\mathcal H}}
\def\scriptd{{\mathcal D}}
\def\scriptl{{\mathcal L}}
\def\scriptn{{\mathcal N}}
\def\scriptp{{\mathcal P}}
\def\scriptk{{\mathcal K}}
\def\scriptP{{\mathcal P}}
\def\scriptj{{\mathcal J}}
\def\scriptz{{\mathcal Z}}
\def\scripts{{\mathcal S}}
\def\scriptx{{\mathcal X}}
\def\scripty{{\mathcal Y}}
\def\frakv{{\mathfrak V}}
\def\frakG{{\mathfrak G}}
\def\aff{\operatorname{Aff}}
\def\frakB{{\mathfrak B}}
\def\frakC{{\mathfrak C}}

\def\suchthat{\mathrel{}:\mathrel{}}
\def\symdif{\,\Delta\,}
\def\mustar{\mu^*}
\def\muplus{\mu^+}

\def\soln{\noindent {\bf Solution.}\ }


%\pagestyle{empty}
%\setlength{\parindent}{0pt}

\begin{document}

\begin{center}{\bf Math 185 --- UCB, Fall 2016 --- William Guss}
\\
{\bf Problem Set 9, due November 29th}
\end{center}
\medskip \noindent {\bf (86.10)}\ Let $m$ and $n$ be integers, where $0 \leq m \leq n$. Show that
\begin{equation*}
	\int_0^\infty \frac{z^{2m}}{z^{2n}+1}\ dz= \frac{\pi}{2n} \csc \parens{\frac{2m + 1}{2n} \pi}.
\end{equation*}
\begin{proof}
	We will use Residue theory to evaluate the integral. Therefore we first show that the zeroes of the divisor (and therefore singularities of the integrand), $z^{2n}+1$ lying above the real axis are
	\begin{equation*}
		c_k = exp\parens{i\frac{(2k+1)\pi}{2n}}\;\;\;(k = 0, 1, 2, \dots, n-1).
	\end{equation*}
	To see this first observe that $c_k$ are solutions to $z^{2n} = -1 = e^{i\pi}.$ Then writing
	$z^{2n}$ in euler form we get $re^{2ni\theta} = e^{i\pi}$ and therefore $r = 1$ and 
	$2n i \theta = i \pi \mod 2\pi i$. Thus $\theta = \pi/(2n) + 2k\pi/(2n) = (2k +1)\pi/(2n).$
	Finally $c_k = exp\parens{i \frac{(2k+1)\pi}{2n}}$ for all $k$ up to $n-1$, for at $k = n$ we 
	yield $\theta = (2n +1)\pi/(2n) = 1/(2n) \pi + \pi$ which is lies below the real axis.
	Then at $k = 2n,$ we yield the solution for $k =0$, $\theta = (4n+1)\pi/(2n) = 2\pi + \pi/2n$.
	Therefore our characterization of the roots hold.


	Next observe that $(z^{2n} +1)' = 2n(z^{2n -1}) \neq 0$ when $z = c_k$ and therefore the divisor of the integrand
	is has simple zeros at all $c_k$; ie. the integrand has simple poles at all $x = c_k.$

	Now we use the method of residue integration along the half semi-circle to finally evaluate the integral. Observe
	that by the evenness of the integrand
	\begin{equation*}
		\int_0^\infty \frac{x^{2m}}{x^{2n}+1}\ dx = \lim_{R\to\infty }1/2\int_{-R}^R \frac{z^{2m}}{z^{2n}+1}\ dz
	\end{equation*}
	Additionally, that for a upper semicircle of radius $R$, 
	\begin{equation*}
		 \int_0^\infty \frac{x^{2m}}{x^{2n}+1}\ dx = \pi i \sum_{k=0}^{n-1}\ Res_{z= c_k}\ \parens{\frac{z^{2m}}{z^{2n}+1}} - \lim_{R\to\infty } \frac{1}{2}\int_{C_R} \frac{z^{2m}}{z^{2n}+1}\ dz.
	\end{equation*}
	Therefore we use our conclusion that the integrand has simple poles at all $c_k$ to just evaluate the residue as 
	\begin{equation*}
	\begin{aligned}
	\pi i \sum_{k=0}^{n-1}\ Res_{z= c_k}\ \parens{\frac{z^{2m}}{z^{2n}+1}} &= \pi i \sum_{k=0}^{n-1}\frac{c_k^{2m}}{2n c_k^{2n-1}}  \\
	&=  \frac{\pi i}{2n} \sum_{k=0}^{n-1} \frac{exp\parens{i\frac{(2k+1)2m\pi}{2n}}}{exp\parens{i\frac{(2k+1)(2n-1)\pi}{2n}}}\\
	&=  \frac{\pi i}{2n} \sum_{k=0}^{n-1} exp\parens{i\pi \frac{2k+1}{2n}(2m  - (2n - 1))}\\
	&=  \frac{\pi i}{2n} \sum_{k=0}^{n-1} exp\parens{i\pi \frac{2k+1}{2n}(2m + 1) - i\pi \frac{2n(2k+1)}{2n}}\\
	&=  \frac{\pi i}{2n} \sum_{k=0}^{n-1} exp\parens{i\pi \parens{\frac{2k+1}{2n}(2m + 1) - 2k+1}}\\
	&=  \frac{\pi i}{2n} \sum_{k=0}^{n-1} \frac{exp\parens{i\pi \parens{\frac{2k+1}{2n}(2m + 1)}} }{exp(i\pi)} \\
	&=  -\frac{\pi i}{2n} \sum_{k=0}^{n-1} exp\parens{i\pi \parens{\frac{2k+1}{2n}(2m + 1)}}.
	\end{aligned}
	\end{equation*}
	To simplify we use the summation formula $\sum_0^{n-1} w^{2k+1} = w\sum_{0}^{n-1} {(w^2)^k} = w(1-(w^2)^n)/(1-(w^2))$ where $z \neq 1$ and apply it to our expression where we let $w = exp(i\pi(2m+1)/(2n))$ and then
	\begin{equation*}
	\begin{aligned}
		-\frac{\pi i}{2n} \sum_{k=0}^{n-1} exp\parens{i\pi \parens{\frac{2k+1}{2n}(2m + 1)}} 
			&= - \frac{w \pi i}{2n} \sum_{k=0}^{n-1} w^{2k}. \\
			&= - \frac{\pi i}{2n} \frac{w(1-w^{2n})}{1- w^2} \\
			&= - \frac{\pi i}{2n} \frac{w(1-exp(i\pi (2m + 1)))}{1- w^2} = - \frac{\pi i}{n} \frac{w}{1- w^2}.
	\end{aligned}
	\end{equation*}
	Warranting that $w^{2n} = 1$, $|w| = 1$ algebra yields
	\begin{equation*}
	\begin{aligned}
		- \frac{\pi i}{n} \frac{w}{1- w^2} &= - \frac{\pi i}{n} \frac{1}{w(w^{2n-2}- 1)} \\
		&= - \frac{\pi i}{n} \frac{1}{w(w^{2n-2}- w^{2n})} 
		= - \frac{\pi i}{n} \frac{1}{w(w^{-2}- {1})} \\
		&= - \frac{\pi i}{n} \frac{1}{w^{-1}- w}
		= \frac{\pi i}{n} \frac{1}{w - \overline{w}}
	\end{aligned}
	\end{equation*}
	Now $w - \overline{w} = 2i Im(w) = 2i \sin\left((2m+1)\pi/2n\right).$ Therefore
	\begin{equation*}
			\pi i \sum_{k=0}^{n-1}\ Res_{z= c_k}\ \parens{\frac{z^{2m}}{z^{2n}+1}}  = \frac{\pi}{ 2n \sin\left(\frac{(2m+1)\pi}{2n}\right)}.
	\end{equation*}

	Lastly to show the statement we must finally show that $\int_{C_R} f(z)\ dz \to 0$ as $R \to \infty.$
	When $m < n$, it is clear that $\frac{|z^{2m}|}{|z^{2n}+1|} \leq  \frac{R^{2m}}{R^{2n}-1}$ as $|z^{2n} + 1| \geq ||z|^{2n} - 1| = R^{2n} -1.$ Since $R^{2n}$ subsumes $R^{2m}$ in the limit ($m < n$), $\frac{R^{2m}}{R^{2n}-1} \to 0$
	and by the Maximul Modulus principle $|\int_{C_R} \frac{z^{2m}}{z^{2n}+1}\ dz| \leq 2\pi R \frac{R^{2m}}{R^{2n}-1} $. Then since $2m +1 < 2n,$ there is yet subsumption in the limit and $ 2\pi R \frac{R^{2m}}{R^{2n}-1} \to 0$ implies that from the original equation
	\begin{equation*}
		 \int_0^\infty \frac{x^{2m}}{x^{2n}+1}\ dx = \pi i \sum_{k=0}^{n-1}\ Res_{z= c_k}\ \parens{\frac{z^{2m}}{z^{2n}+1}} - \lim_{R\to\infty } \frac{1}{2}\int_{C_R} \frac{z^{2m}}{z^{2n}+1}\ dz = \frac{\pi}{ 2n} \csc\left(\frac{2m+1}{2n}\pi\right).
	\end{equation*}
\end{proof}
\medskip \noindent {\bf (86.1)}\ Use residues to derive the integration formula
\begin{equation*}
	\int_0^\infty \frac{dx}{x^2 + 1} = \frac{\pi}{2}
\end{equation*}
\begin{proof}
	Applying the result in \textbf{(86.10)} above we get that
	\begin{equation*}
	\int_0^\infty \frac{dx}{x^2 + 1} = \int_0^\infty \frac{x^{2\cdot 0}dx}{x^{2\cdot 1} + 1} = \frac{\pi}{2} \csc(\pi/2) = \frac{\pi}{2}.
	\end{equation*}
\end{proof}
\medskip \noindent {\bf (86.4)}\ Use residues to derive the integration formula
\begin{equation*}
	\int_0^\infty \frac{x^2}{x^6  + 1}\ dx = \frac{\pi}{6}
\end{equation*}
\begin{proof}
	Applying the result in \textbf{(86.10)} above, we get that $m = 1, n =3$ and so
	\begin{equation*}
	\int_0^\infty \frac{x^2}{x^6  + 1}\ dx = \int_0^\infty \frac{x^{2m}dx}{x^{2n} + 1} = \frac{\pi}{6} \csc({3}\pi/{6}) = \frac{\pi}{6}.
	\end{equation*}
\end{proof}
\medskip \noindent {\bf (86.6)}\ Use residues to derive the integration formula
\begin{equation*}
	\int_0^\infty \frac{x^2}{(x^2 + 9)(x^2 + 4)^2}\ dx = \frac{\pi}{200}
\end{equation*}
\begin{proof}
We will use Residue theory to evaluate the integral. Therefore we first show that the zeroes of the divisor (and therefore singularities of the integrand), $(z^2 + 9)(z^2 + 4)^2$ lying above the real axis are
	\begin{equation*}
		c_{0} =  3i, c_{1} =  2i, c_{2} = 2i
	\end{equation*}
	To see this first observe that
	\begin{equation*}
	\frac{z^2}{(z^2 + 9)(z^2 + 4)^2} = \frac{z^2}{(z^2-(3i)^2)(z^2 - (2i)^2)^2} = \frac{z^2}{(z-3i)(z+3i)(z-2i)(z+2i)(z-2i)(z+2i)}
	\end{equation*}
	Then the zeroes of the divisor are clearly $\pm 3i$ and $\pm 2i$ (multiplicity $2$), but those quantities only are above the real line
	in their positive imaginary form, so our characterization of the zeroes of the divisor hold. We want to
	characterize the residue, so we will therefore characterize the zeroes of the divsor. First let $u = (z^2 + 9)$,
	$v = (z^2 + 4).$ Then we know that $u' = 2z$, $u'' = 2$ and clearly $u', u'' \neq 0$, at $c_k$. On the other hand $v' = 2z$, $v'' = 2$ and $v' , v'' \neq 0$ at $z = c_k$. So we must differentiate the divsior untill it does not assume a zero value at $c_k,$ recalling that if all of the terms contain $u$ or $v$, the divisors $n$-derivative is still zero at $c_0$ or $c_{1,2}$ respectively..
	Next, observe that
	\begin{equation*}
	\begin{aligned}
		((z^2 + 9)(z^2 + 4)^2)'' = (uv^2)''&= (u'v^2 + 2uvv')' = u''v^2 + 2u'vv' + 2u'vv' + 2u(v'v' + vv'') \\
	\end{aligned}
	\end{equation*}
	and so the second derivative always has non-zero terms at each $c_k$. Thus the integrand has a pole of order $1$ at $z= c_0$ and a pole of order two at $z = c_{1,2}$.

	Now we use the method of residue integration along the half semi-circle to finally evaluate the integral. Observe
	that by the evenness of the integrand
	\begin{equation*}
		\int_0^\infty \frac{z^2}{(z^2 + 9)(z^2 + 4)^2}\ dz = \lim_{R\to\infty }1/2\int_{-R}^R \frac{z^2}{(z^2 + 9)(z^2 + 4)^2}\ dz
	\end{equation*}
	Additionally, for a upper semicircle of radius $R$, 
	\begin{equation*}
		 \int_0^\infty \frac{z^2}{(z^2 + 9)(z^2 + 4)^2}\ dz = \pi i \sum_{k=0}^{1}\ Res_{z= c_k}\ \parens{\frac{z^2}{(z^2 + 9)(z^2 + 4)^2}} - \lim_{R\to\infty } \frac{1}{2}\int_{C_R} \frac{z^2}{(z^2 + 9)(z^2 + 4)^2}\ dz.
	\end{equation*}
	Letting $p = z^2, q= (z^2 + 9)(z^2 + 4)^2$, then
	to calculate the residue at $z = c_0$, we know that $q(z)$ has a pole of order $1$ at $z = z_0$, thus
	the residue is equal to $p(c_0)/q'(c_0)$; that is
	\begin{equation*}
		Res_{z = c_0} p(z)/q(z) = \frac{(3i)^2}{[u'v^2 + 2uvv']_{z = c_0}} = -\frac{9}{2(3i)((3i)^2 + 4)^2 + 2\cdot 0} = -\frac{4}{2i(-9 + 4)^2} = \frac{3i}{50}.
	\end{equation*}
	For $c_{1,2}$, let $\phi(z) = \frac{p(z)}{((z^2 + 9)(z+2i)^2)},$ then $\phi(z)$ is analytic at $z = c_{1,2}$ and
	$p(z)/q(z)$ has a pole of order two at $z = 2i$ thus letting $g = (z^2 + 9)(z+2i)^2$, we can calculate $g' = 
	2z(z+2i)^2 + 2(z+2i)(z^2+9)$ and evaluate at $z = 2i$ yielding $g'(c_{1,2}) = 4i(4i)^2 + 2(4i)((2i)^2 + 9) = 4i(-16 + 2(-4 + 9)) = -24i.$ Additionally $g(c_{1,2}) = 5\cdot(4i)^2 = -80.$ 
	\begin{equation*}
	\begin{aligned}
		Res_{z = c_{1,2}} \frac{p(z) }{q(z)}= \phi'(c_{1,2}) &= \frac{p'(c_{1,2})g(c_{1,2}) - p(c_{1,2})g'(c_{1,2}) }{g(c_{1,2})^2}  = \frac{4i(-80) - (-4)(-24i)}{(-80)^2} = \frac{-416i}{6400} = \frac{-13i}{200}.
	\end{aligned}
	\end{equation*}
	Finally 
	\begin{equation*}
		\pi i \sum_{k=0}^1 Res_{z = c_k}\ \frac{p(z)}{q(z)} = \pi i \frac{(12i - 13i)}{200} = \frac{\pi}{200}.
	\end{equation*}

	Lastly to show the statement we must finally show that $\int_{C_R} f(z)\ dz \to 0$ as $R \to \infty.$
	When $m < n$, it is clear that $p(z)/q(z) \leq  \frac{R^{2}}{(R^{2}-9)(R^2 -4)^2}$ as $|(z^2 + 9)(z^2 + 4)^2| = |z^2 + 9||z^2 + 4|^2 \geq ||z|^{2} - 9|||z^2| - 4|^2 = (R^{2}-9)(R^2 -4)^2.$ Since $R^{3}$ is order three it is  subsumed by $(R^{2}-9)(R^2 -4)^2$ in the limit , $ \frac{R^{2}}{(R^{2}-9)(R^2 -4)^2} \to 0$
	and by the Maximul Modulus principle $$\left|\int_{C_R} p(z)/q(z)\ dz\right| \leq 2\pi  \frac{R^{2}}{(R^{2}-9)(R^2 -4)^2} \to 0.$$ Therefore we conclude
\begin{equation*}
	\int_0^\infty \frac{x^2}{(x^2 + 9)(x^2 + 4)^2}\ dx = \frac{\pi}{200}.
\end{equation*}
\end{proof}








\medskip \noindent {\bf (91.1)}\  Use the function $f(z) = (e^{iaz} - e^{ibz})/z^2$ and the indented contour in Fig 109 (Sec 89) to derive the integration formula
\begin{equation*}
	\int_0^\infty \frac{cos(ax) - cos(bx)}{x^2}\ dx = \frac{\pi}{2}(b-a)\;\;\;(a \geq 0, b\geq 0).
\end{equation*} 
Then show that 
\begin{equation*}
	\int_0^\infty \frac{\sin^2(x)}{x^2}\ dx = \frac{\pi}{2}.
\end{equation*}
\begin{proof}
	Assuming that $a \geq 0$ and $b \geq 0$ we will use the contours provided below:
\begin{center}
\begin{tikzpicture}
% Configurable parameters
\def\gap{0.2}
\def\bigradius{3}
\def\littleradius{0.5}

% Axes
\draw [help lines,->] (-1.25*\bigradius, 0) -- (1.25*\bigradius,0);
\draw [help lines,->] (0, -0.1*\bigradius) -- (0, 1.25*\bigradius);
% Red path
\draw[line width=1pt,   decoration={ markings,
  mark=at position 0.2455 with {\arrow[line width=1.2pt]{>}},
  mark=at position 0.765 with {\arrow[line width=1.2pt]{>}},
  mark=at position 0.87 with {\arrow[line width=1.2pt]{>}},
  mark=at position 0.97 with {\arrow[line width=1.2pt]{>}}},
  postaction={decorate}]
  let
     \n1 = {asin(0/2/\bigradius)},
     \n2 = {asin(0/2/\littleradius)}
  in (0:\bigradius) arc (0:180-\n1:\bigradius)
  -- (-180:\littleradius) arc (180:0:\littleradius)
  -- cycle;

% The labels
\node at (3.6,-0.2){$\mathbb{R}$};
\node at (-0.24,3.53) {$i\mathbb{R}$};
\node at (-0.6,0.63) {$C_{\rho}$};
\node at (+1.9,2.8) {$C_R$};
\node at (1.9,0.29) {$L_1$};
\node at (-1.9,0.29) {$L_2$};

\end{tikzpicture}
\end{center}
	and let $f(z) = (e^{iaz} - e^{ibz})/z^2$. Then the function has a pole of order two at $z = 0$ and is analytic elsewhere. Thus
	\begin{equation*}
	\begin{aligned}
		\int_{C_R} f(z)\ dz + \int_{C_\rho} f(z)\ dz + \int_{L_1} f(z)\ dz  + \int_{L_2} f(z)\ dz = 0 \\
		\int_{L_1} f(z)\ dz  + \int_{L_2} f(z)\ dz  = -\int_{C_R} f(z)\ dz  -\int_{C_\rho} f(z)\ dz 
	\end{aligned}
	\end{equation*}
	In particular we manipulate the left hand side by changing the orientation of $L_2$ so that
	\begin{equation*}
	\begin{aligned}
		\int_{L_1} f(z)\ dz  + \int_{L_2} f(z)\ dz &= \int_{L_1} f(z)\ dz  - \int_{-L_2} f(z)\ dz\\
		&= \int_\rho^R f(re^{i0})(re^{i0})'\ dr - \int_\rho^R f(re^{i\pi})(re^{i\pi})'\ dr \\
		&= \int_\rho^R f(r)\ dr + \int_\rho^R f(-r)\ dr \\
		&= \int_\rho^R (e^{iar} - e^{ibr})/r^2 + (e^{-iar} - e^{-ibr})/(-r)^2\ dr \\
		&= \int_\rho^R ((e^{iar}  + e^{-iar}) - (e^{-ibr} +e^{ibr}))/r^2 \ dr \\
		&= 2\int_\rho^R \frac{\cos(ar) - \cos(br)}{r^2} \ dr.
	\end{aligned}
	\end{equation*}
	Consequently we need only evaluate the semicircular contour integrals; that is,
	\begin{equation*}
		2\int_\rho^R \frac{\cos(ar) - \cos(br)}{r^2} \ dr =  -\int_{C_R} f(z)\ dz  -\int_{C_\rho} f(z)\ dz.
	\end{equation*}
	Now observe that $f(z)$ only has a pole of order two at $z = 0.$
	Therefore \begin{equation*}
		f(z) = \frac{\phi(z)}{z^2}; \phi(z) \text{ analytic} \implies f(z) = \frac{c_{-2}}{z^2} + \frac{c_{-1}}{z} + \sum_{n=0}^\infty c_{n}z^n.
	\end{equation*}
	Now observe that since $\sum_{n=0}^\infty c_{n}z^n$ is analytic and $z = 0$ is an isolated singular point of $f$, there is an $\epsilon$-ball small enough so that $\sum_{n=0}^\infty c_{n}z^n$ is analytic on a closed subdomain $D$ with no singularities. Thus it achieves a maximum $M$, so that $$\left|\int \sum_{n=0}^\infty c_{n}z^n\ dz\right| \leq  ML = M\pi \rho \to 0.$$
	Therefore
	\begin{equation*}
		\lim_{\rho \to 0} \int_{C_\rho} f(z)\ dz = \lim_{\rho \to 0} \int_{C_\rho} \frac{c_{-2}}{z^2} + \frac{c_{-1}}{z}\ dz = \lim_{\rho \to 0} \int_{C_\rho} \frac{c_{-2}}{z^2} \ dz + \pi i Res_{z=0}f(z).
	\end{equation*}
	Finally we evaluate the remaining integral and yield that
	\begin{equation*}
	\lim_{\rho \to 0} \int_{C_\rho} \frac{c_{-2}}{z^2} \ dz = \lim_{\rho \to 0} \int_0^\pi \frac{c_{-2}}{\rho^2 e^{i2\theta}} \rho i e^{i\theta} \ d\theta = \lim_{\rho \to 0} \frac{ic_{-2}}{\rho} \int_0^\pi e^{-i\theta} \ d\theta
	\end{equation*}
	But then because $|\int_0^\pi e^{-i\theta} \ d\theta| \leq \pi$ we have that the integral tends to 0, and so
	\begin{equation*}
		\int_{C_\rho} f(z)\ dz \to \pi i Res_{z=0}\ f(z).
	\end{equation*}

	Now we expand the Larent series representation of $f(z)$ and get
	\begin{equation*}
	\begin{aligned}
		f(z) = \frac{1}{z^2}(e^{iaz} - e^{ibz}) &= \frac{1}{z^2}\parens{1 + \frac{(iaz)}{1!} + \frac{(aiz)^2}{2!} + \cdots} - \frac{1}{z^2}\parens{1 + \frac{(ibz)}{1!} + \frac{(ibz)^2}{2!} + \cdots}\\
		&= \parens{\frac{1}{z^2} + \frac{(ia)}{z1!} + \frac{(ai)^2}{2!} + \cdots} - \parens{\frac{1}{z^2} + \frac{(ib)}{z1!} + \frac{(ib)^2}{2!} + \cdots}\\
		&= \frac{ia-ib}{z} +  \frac{b-a}{2!} + \cdots \\
		\implies& \pi i Res_{z=0}\ f(z) =  \pi(b-a). 
	\end{aligned}
	\end{equation*}

	Lastly we will show that $\int_{C_R}\ f(z)\ dz \to 0.$ Consider that $e^{iaz} - e^{ibz}$. Additonally $|z| = R^2$ and so $|1/z^2| \leq \frac{1}{R^2}$. Therefore by Jordans lemma, $a,b \geq 0$, and the triange inequality
	\begin{equation*}
		\left|\int_{C_R} f(z)\ dz \right|	\leq \left|\int_{C_R} \frac{ e^{az}}{z^2}\ dz \right| + \left|\int_{C_R} \frac{e^{bz}}{z^2}\ dz \right| \to 0.
	\end{equation*}
	Therefore $\int_{C_R} f(z)\ dz \to 0$, and we have the following statement.
	\begin{equation*}
		\int_0^\infty \frac{\cos(ax) - \cos(bx)}{x^2} \ dx =  \lim_{\rho \to 0, R \to \infty} \int_\rho^R \frac{\cos(ar) - \cos(br)}{r^2} \ dr = \frac{ \pi(b-a)}{2} + 0.
	\end{equation*}

	It follows immediately that $\int_0^\infty \frac{\sin^2(x)}{x^2}\ dx = \pi/2$ because $\sin^2 (x) = \frac{\cos(0x) - \cos(2x)}{2}$; in other words
	\begin{equation*}
	\int_0^\infty \frac{\sin^2(x)}{x^2}\ dx = \frac{1}{2}\int_0^\infty \frac{\cos(0) - \cos(2x)}{x^2} \ dx = \frac{1}{2}\frac{\pi(2- 0)}{2}.
	\end{equation*}
	This completes the proof.
\end{proof}














\medskip \noindent {\bf (91.2)}\ Derive the integral formula
\begin{equation*}
	\int_0^\infty \frac{dx}{\sqrt{x} (x^2 + 1)} = \frac{\pi}{\sqrt{2}}.
\end{equation*}
\begin{proof}
	Assuming that $\rho < 1$ and $R > 1$ we will use the contours provided below:
\begin{center}
\begin{tikzpicture}
% Configurable parameters
\def\gap{0.2}
\def\bigradius{3}
\def\littleradius{0.5}

% Axes
\draw [help lines,->] (-1.25*\bigradius, 0) -- (1.25*\bigradius,0);
\draw [help lines,->] (0, -0.1*\bigradius) -- (0, 1.25*\bigradius);
% Red path
\draw[line width=1pt,   decoration={ markings,
  mark=at position 0.2455 with {\arrow[line width=1.2pt]{>}},
  mark=at position 0.765 with {\arrow[line width=1.2pt]{>}},
  mark=at position 0.87 with {\arrow[line width=1.2pt]{>}},
  mark=at position 0.97 with {\arrow[line width=1.2pt]{>}}},
  postaction={decorate}]
  let
     \n1 = {asin(0/2/\bigradius)},
     \n2 = {asin(0/2/\littleradius)}
  in (0:\bigradius) arc (0:180-\n1:\bigradius)
  -- (-180:\littleradius) arc (180:0:\littleradius)
  -- cycle;

% The labels
\node at (3.6,-0.2){$\mathbb{R}$};
\node at (-0.24,3.53) {$i\mathbb{R}$};
\node at (-0.6,0.63) {$C_{\rho}$};
\node at (+1.9,2.8) {$C_R$};
\node at (1.9,0.29) {$L_1$};
\node at (-1.9,0.29) {$L_2$};

\end{tikzpicture}
\end{center}
	and let $f(z) = \frac{z^{-1/2}}{z^2 + 1}$. Where the particular branch in contention is $|z| > 0$, and $- \pi/2 < \arg z < 3\pi /2$.

	Then the function has a singularity inside the domain of the contour (at $i$) since $i^2 + 1 = 0$. Thus
	Cauchy's residue theorem gives us
	\begin{equation*}
	\begin{aligned}
		\int_{C_R} f(z)\ dz + \int_{C_\rho} f(z)\ dz + \int_{L_1} f(z)\ dz  + \int_{L_2} f(z)\ dz = 2\pi i Res_{z = i}\ f(z) \\
		\int_{L_1} f(z)\ dz  + \int_{L_2} f(z)\ dz  = 2\pi i Res_{z = i}\ f(z)   -\int_{C_R} f(z)\ dz  -\int_{C_\rho} f(z)\ dz 
	\end{aligned}
	\end{equation*}

	In particular we manipulate the left hand side by chaning the orientation of $L_2$ so that
	\begin{equation*}
	\begin{aligned}
		\int_{L_1} f(z)\ dz  + \int_{L_2} f(z)\ dz &= \int_{L_1} f(z)\ dz  - \int_{-L_2} f(z)\ dz\\
		&= \int_\rho^R f(re^{i0})(re^{i0})'\ dr - \int_\rho^R f(re^{i\pi})(re^{i\pi})'\ dr \\
		&= \int_\rho^R f(r)\ dr + \int_\rho^R f(-r)\ dr \\
		&= \int_\rho^R \frac{e^{-i0/2}+ e^{-i\pi/2}}{\sqrt{r}(r^2 + 1)}\ dr \\
		&= (1 -i)\int_\rho^R \frac{1}{\sqrt{r}(r^2 + 1)}\ dr \\
	\end{aligned}
	\end{equation*}
	Consequently we need only evaluate the semicircular contour integrals and the residue; that is,
	\begin{equation*}
		(1 -i)\int_\rho^R \frac{1}{\sqrt{r}(r^2 + 1)}\ dr  =  2\pi i Res_{z = i}\ f(z) -\int_{C_R} f(z)\ dz  -\int_{C_\rho} f(z)\ dz.
	\end{equation*}


	Next observe that $|z^2| = r^2$ when $z = re^{i\theta}.$ Thus $|z^2 + 1| \geq ||z|^2 -1| = 1- \rho^2$. When $\rho < 1.$ Now the contour integral is bounded using the maximum modulus principle and we get that
	\begin{equation*}
		\left|\int_{C_\rho} f(z)\ dz\right| \leq \frac{\rho^{-1/2}}{1 - \rho^2} \pi \rho \leq \frac{\pi \rho^{1/2}}{1 - \rho^2} \to 0
	\end{equation*}
	since $1- \rho^2 \to 1$ and $\sqrt{\rho} \to 0$ as $\rho \to 0.$ Therefore  $\int_{C_\rho} f(z)\ dz \to 0.$
	
	In the case that $z$ is on $C_R$ then $|z^2 + 1| \geq ||z|^2 - 1| = |R^2 - 1| = R^2 -1$ when $R > 1.$ Now we can bound the contour integral using the maximum modulo principle and we get
	\begin{equation*}
		\left|\int_{C_R} f(z)\ dz\right| \leq \frac{R^{-1/2}}{R ^2-1} \pi R \leq \frac{\pi R^{1/2}}{R^2 - 1}.
	\end{equation*}
	Since $R^{1/2}$ is subsumed by $R^2$ as $R \to \infty$ we have that the ML bound tends to $0$. In fact
	$\int_{C_R} f(z)\ dz \to 0$ as $R \to \infty. $


	So it remains to evaluate the residue of $f$ at $z = i$. Since $f(z) = z^{-1/2}/(z+i)(z-i),$ $f$ has a
	pole of order $1$ at $z = i$ and so we may evaluate the residue as $\phi(i)$ where $\phi(i)$ removes the singularity. Therefore
	$$
	 Res_{z =i} f(z) = \frac{i^{-1/2}}{[(z+i)]_{z = i}} = \frac{i^{-1/2}}{[(z+i)]_{z = i}} = -\frac{e^{-1/2 (Log|1| + iArg(i))}}{2i}= \frac{1-i}{2i\sqrt{2}}.$$
	 Then $2 \pi i Res_{z= i}\ f(z) = (1-i)\pi/\sqrt{2}$. Putting everything together we yield that in the limit
	 \begin{equation*}
	\int_0^\infty \frac{dx}{\sqrt{x} (x^2 + 1)} = \frac{\pi}{\sqrt{2}}. 
	 \end{equation*}
\end{proof}



















\medskip \noindent {\bf (91.4)}\ Derive the integration formula
\begin{equation*}
	\int_0^\infty \frac{\sqrt[3]{x}}{(x+a)(x+b)}\ dx = \frac{2\pi}{\sqrt{3}} \cdot \frac{\sqrt[3]{a} - \sqrt[3]{b}}{a -b} \;\;\;(a > b > 0).
\end{equation*}
\begin{proof}

	Assuming that $\rho < b$ and $R > a$ we will use the contours provided below:
\begin{center}
\begin{tikzpicture}
% Configurable parameters
\def\gap{0.2}
\def\bigradius{3}
\def\littleradius{0.5}

% Axes
\draw [help lines,->] (-1.25*\bigradius, 0) -- (1.25*\bigradius,0);
\draw [help lines,->] (0, -0.1*\bigradius) -- (0, 1.25*\bigradius);
% Red path
\draw[line width=1pt,   decoration={ markings,
  mark=at position 0.2455 with {\arrow[line width=1.2pt]{>}},
  mark=at position 0.765 with {\arrow[line width=1.2pt]{>}},
  mark=at position 0.87 with {\arrow[line width=1.2pt]{>}},
  mark=at position 0.97 with {\arrow[line width=1.2pt]{>}}},
  postaction={decorate}]
  let
     \n1 = {asin(0/2/\bigradius)},
     \n2 = {asin(0/2/\littleradius)}
  in (0:\bigradius) arc (0:360-\n1:\bigradius)
  -- (0:\littleradius) arc (0:360:\littleradius)
  -- cycle;

% The labels
\node at (3.6,-0.2){$\mathbb{R}$};
\node at (-0.24,3.53) {$i\mathbb{R}$};
\node at (-0.6,0.63) {$C_{\rho}$};
\node at (+1.9,2.8) {$C_R$};
\node at (1.9,0.29) {$L_1$};
\node at (1.9,-0.29) {$L_2$};

\end{tikzpicture}
\end{center}
	and let $f(z) = \frac{z^{1/3}}{(z+b)(z+a)}$. Where the particular branch in contention is $|z| > 0$, and $0 < \arg z < 2\pi$. (We referr to Problem 6 for a rigorous derivation of integration along $L_1$.)

	Then the function has a singularity inside the domain of the contour (at $i$) since $i^2 + 1 = 0$. Thus
	Cauchy's residue theorem gives us
	\begin{equation*}
	\begin{aligned}
		\int_{C_R} f(z)\ dz + \int_{C_\rho} f(z)\ dz + \int_{L_1} f(z)\ dz  + \int_{L_2} f(z)\ dz = 2\pi i \parens{Res_{z = -a}\ f(z) +   Res_{z = -b}\ f(z)}  \\
		\int_{L_1} f(z)\ dz  + \int_{L_2} f(z)\ dz  = 2\pi i \parens{Res_{z = -a}\ f(z) +   Res_{z = -b}\ f(z)}   -\int_{C_R} f(z)\ dz  -\int_{C_\rho} f(z)\ dz 
	\end{aligned}
	\end{equation*}

	Recall that  in the principle branch $f(z) = \frac{exp[1/3(\ln r + i0)]}{(r+a)(r+b)}$ when $z = re^{i0}.$ 
	Then when $z = re^{i2\pi}$, $f(z) = \frac{exp[1/3\ln r + i 2\pi]}{(r+a)(r+b)}$, since $L_2$ is the reverse parameterization then we get
	\begin{equation*}
	\begin{aligned}
		\int_{L_1} f(z)\ dz  - \int_{-L_2} f(z)\ dz 
		&= \int_{\rho}^R \frac{\sqrt[3]{r}}{(r+a)(r+b)}\ dr - \int_{\rho}^R \frac{\sqrt[3]{r}e^{i2\pi/3}}{(re^{i2\pi}+a)(re^{i2\pi}+b)}\ dr\\
		&=(1 - e^{i2\pi/3})\int_{\rho}^R \frac{\sqrt[3]{r}}{(r+a)(r+b)}\ dr.
	\end{aligned}
	\end{equation*}
	Consequently we need only evaluate the semicircular contour integrals and the residue; that is,
	\begin{equation*}
		(1 - e^{i2\pi/3})\int_{\rho}^R \frac{\sqrt[3]{r}}{(r+a)(r+b)}\ dr  =  2\pi i \parens{Res_{z = a}\ f(z) +   Res_{z = b}\ f(z)}   -\int_{C_R} f(z)\ dz  -\int_{C_\rho} f(z)\ dz.
	\end{equation*}


	Next observe that $|z^{1/3}| = R^{1/3}$ when $z = re^{i\theta}$ is on $C_R$. Thus when $b < R$, $|(z+a)(z+b)| \geq |z+a||z+b| \geq ||z| - a|||z| - b| = (R-a)(R-b).$ Now we can bound the contour integral using the maximum modulo principle and we get
	\begin{equation*}
		\left|\int_{C_R} f(z)\ dz\right| \leq \frac{R^{1/3}}{(R-a)(R-b)} \pi R \leq \frac{R^{4/3}\pi}{(R-a)(R-b)} .
	\end{equation*}
	Since $R^{4/3}$ is subsumed by $R^2$ as $R \to \infty$ we have that the ML bound tends to $0$. In fact
	$\int_{C_R} f(z)\ dz \to 0$ as $R \to \infty. $

	Next observe that when $z = re^{i\theta}$ is on $C_\rho$, $r = \rho$ so $|z^{1/3}| \leq \rho^{1/3}$ and
	when $\rho < a$, $|(z+a)(z+b)| \geq |z+a||z+b| \geq ||z| - a|||z| - b| = |a - |z|||b - |z|| = (a - \rho)(b-\rho).$ Now we can bound the contour integral using the maximum modulo principle and we get
	\begin{equation*}
		\left|\int_{C_\rho} f(z)\ dz\right| \leq \frac{\rho^{1/3}}{(a- \rho)(b - \rho)} \pi \rho \leq \frac{\rho^{4/3}\pi}{(a - \rho)(b-\rho)} .
	\end{equation*}
	Since $\rho^{4/3} \to 0 $  and the divisor tends to $ab$ as $\rho \to 0$ we have that the ML bound tends to $0$. In fact
	$\int_{C_\rho} f(z)\ dz \to 0$ as $\rho \to 0.$



	So it remains to evaluate the residue of $f$ at $z = -a$ and $z = - b$. Since $f$ has a
	pole of order $1$ at both $z= -a,-b$ we may evaluate the residue as
	\begin{equation*}
		Res_{z = -a} f(z) = \phi_1(-a);\;\; \phi_1(z) = \frac{z^{1/3}}{(z+b)}. \;\;\;Res_{z = -b} f(z) = \phi_2(-b);\;\; \phi_2(z) = \frac{z^{1/3}}{(z+a)}
	\end{equation*}
	Since $a, b$ are positive real numbers we get that in the principle branch.
	\begin{equation*}
	\phi_1(-a) = \frac{e^{i\pi/3}\sqrt[3]{a}}{(b-a)};\;\;\;\phi_1(-b) = \frac{e^{i\pi/3}\sqrt[3]{b}}{(a-b)};
	\end{equation*}

	Returning to the derivation, we have then that
	\begin{equation*}
	\begin{aligned}
		\lim_{\rho \to 0, R \to \infty}(1 - e^{i2\pi/3})\int_{\rho}^R \frac{\sqrt[3]{r}}{(r+a)(r+b)}\ dr  &=  2\pi i \parens{Res_{z = -a}\ f(z) +   Res_{z = -b}\ f(z)}  \\ 
		&=2\pi i e^{i\pi/3}\parens{\frac{\sqrt[3]{a}}{(b-a)} + \frac{\sqrt[3]{b}}{(a-b)} }  \\
		\implies	\int_{0}^\infty \frac{\sqrt[3]{x}}{(x+a)(x+b)}\ dx &= \frac{2\pi i e^{i\pi/3}}{1 - e^{i\pi2/3}} \parens{\frac{\sqrt[3]{a}}{(b-a)} + \frac{\sqrt[3]{b}}{(a-b)} }  \\
		&=\frac{2\pi i^2}{\sqrt[2]{3}} \parens{\frac{\sqrt[3]{a}}{(b-a)} + \frac{\sqrt[3]{b}}{(a-b)} }   \\
		&=-\frac{2\pi }{\sqrt[2]{3}} \parens{\frac{-\sqrt[3]{a}}{(a-b)} + \frac{\sqrt[3]{b}}{(a-b)} }  \\
		&=\frac{2\pi}{\sqrt[2]{3}} \parens{\frac{\sqrt[3]{a} - \sqrt[3]{b}}{(a-b)}  }  \\
	\end{aligned}
	\end{equation*} 
	This completes te derivation.
\end{proof}
\medskip \noindent {\bf (91.5)}\  The beta function is this function of two real variables:
\begin{equation*}
	B(p, q) = \int_0^1 t^{p-1}(1-t)^{q-1}\ dt\;\;\;\;(p > 0, q > 0).
\end{equation*}
Make the subsitution $t =1/(x+1)$ and use the result obtained in the example in Sec. 91
to show that
\begin{equation*}
	B(p, 1-p) = \frac{\pi}{\sin(p\pi)}\;\;\;(0 < p < 1)
\end{equation*}
\begin{proof}
	As stated above, let $t = 1/(x+1)$, then $\int \phi(t)\ dt = \int \phi(t(x))\ t'(x) \ dx$, via the change
	of variable formula. Tne $t'(x) = -1/(x+1)^2$ and thuis
	\begin{equation*}
	\begin{aligned}
		B(p, 1- p) = \int_{t^{-1}(0)}^{t^{-1}(1)} -\frac{1}{(x+1)^{p-1}}\parens{1 - \frac{1}{x+1}}^{1-p -1} \frac{1}{(x+1)^2}\ dx &= \int_{t^{-1}(1)}^{t^{-1}(0)} \frac{1}{(x+1)^{p+1}} \parens{\frac{x}{x+1}}^{-p}\ dx \\
		&= \int_0^\infty\frac{x^{-p}}{x+1}\ dx.
	\end{aligned}
	\end{equation*}
	Then since $0 < p < 1$ then by the exaple of the previous seciton we yield
	\begin{equation*}
	 	B(p, 1-p) = \int_0^\infty\frac{x^{-p}}{x+1}\ dx = \frac{\pi}{\sin(p \pi)}.
	 \end{equation*} 
\end{proof}






%%%%%%%%%%%%%%%%%%%%%%%%%%%%%%%%%%%%%%%%%%%%%
%%%%%%%%%%%%%%%%%%%%%%%%%%%%%%%%%%%%%%%%%%%%%%
%%%%%%%%%%%%%%%%%%%%%%%%%%%%%%%%%%%%%%%%%%%%%%%



\medskip \noindent {\bf (88.1)}\ Use residues to derive the integration formula
\begin{equation*}
	\int_{-\infty}^\infty \frac{\cos x\ dx}{(x^2 + a^2)(x^2 + b^2)} = \frac{\pi}{a^2 - b^2} \left(\frac{e^{-b}}{b} - \frac{e^{-a}}{a}\right) \;\;\; (a > b > 0).
\end{equation*}
\begin{proof}
	Let $f: \mathbb{C} \to \mathbb{C}$ so that 
	\begin{equation*}
		f: z \mapsto \frac{1}{(z^2 + a^2)(z^2 + b^2)}
	\end{equation*}
	Then on the real axis, the real part of the  function $fe^{iz}$ assumes values $Re(e^{iz}f(x + 0i)) = Re(e^{ix}/((x^2 + a^2)(x^2 + b^2))) = (\cos x)/(x^2 + x^2)(z^2 + b^2)$. Therefore we adopt the methodology of the example in seciton 88 and thus calculate the principle value integral (and by eveness, this is equivalent to the sum of both indefnite integrals.)

	By the Cauchy Residue theorem under the standard semi-circle contour of radius $R$ we have that
	\begin{equation*}
		\int_{-R}^R f(z)e^{iz}\ dz + \int_{C_R} f(z) e^{iz}\ dz = 2\pi i \sum_{c_k \in S(f) \cap C_R^o} Res_{z=c_k}\ e^{iz} f(z)
	\end{equation*}
	where $S(f)$ is the set of singularities of $f$ in $\mathbb{C}.$

	First to calculate the residues at $S \cap \mathbb{C}^H = \{ai, bi\}$, we observe that $f$ has poles of order 1 at both singulariites in $S \cap \mathbb{C}^H$. Therefore we need only evaliuate $\phi_1(z) = e^{iz}/((x^2 + b)(x+ ia)), \phi_2(z) = e^{iz}/((x^2 + a)(x + ib))$ at the respective $c_k$. First we yield $\phi_1(ia) = e^{-a}/((b^2 -a^2)(2ia)),$ then we yield $\phi_2(ib) = e^{-b}/((a^2 - b^2)(2ib))$. Returning to the derivation
	\begin{equation*}
		\int_{-R}^R f(z)e^{iz}\ dz + \int_{C_R} f(z) e^{iz}\ dz = 2\pi i \parens{\frac{e^{-a}}{(2ia)(b^2 - a^2)} - \frac{e^{-b}}{(2ib)(a^2 - b^2)}} = -\frac{-2 \pi i}{a^2 - b^2} \parens{\frac{e^{-b}}{2ib} - \frac{e^{-a}}{2ia}} 
	\end{equation*}
	Lastly we need show that the $C_R$ contour integral tens to $0$.  In particular we will  bound $f(z)$ on $C_R$
	and then apply Jordan's lemma.  We know that $|(z^2 + a^2)(z^2 + b^2)| = |z^2 + a^2||z^2 + b^2| \geq ||z^2| - a^2|||z^2| - b^2|$ Then for $R$ large enough $||z^2| - a^2|||z^2| - b^2| = (R^2 -a)(R^2 - b).$
	Therefore $|f(z)| \leq 1/((R^2 -a)(R^2 - b))$ and $1/((R^2 -a)(R^2 - b)) \to 0$ as $R \to \infty.$
	Therefore under Jordan's lemma ($a ,b > 0$) the integral is bounded by
	\begin{equation*}
		\lim_{R \to \infty} \int_{C_R} f(z) e^{iz}\ dz = 0.
	\end{equation*}
	Finally we yield that
	\begin{equation*}
		\int_{-\infty}^\infty \frac{\cos x\ dx}{(x^2 + a^2)(x^2 + b^2)}  = (P.V)\int_{-\infty}^\infty \frac{\cos x\ dx}{(x^2 + a^2)(x^2 + b^2)}  =  \frac{\pi }{a^2 - b^2} \parens{\frac{e^{-b}}{b} - \frac{e^{-a}}{a}} 
	\end{equation*}
	and this completes the derivation.
\end{proof}

\medskip \noindent {\bf (88.2)}\ Use residues to derive the integration formula
\begin{equation*}
	\int_0^\infty \frac{\cos ax}{x^2 + 1}\ dx = \frac{\pi}{2}e^{-a}. \;\;\;(a > 0)
\end{equation*}
\begin{proof}
		Let $f: \mathbb{C} \to \mathbb{C}$ so that 
	\begin{equation*}
		f: z \mapsto \frac{1}{z^2 + 1} = \frac{1}{(z+i)(z-i)}
	\end{equation*}
	Then on the real axis, the real part of the  function $fe^{iaz}$ assumes values $Re(e^{iz}f(x + 0i)) = Re(e^{ix}/(x^2 + 1) = \cos(ax)/(x^2 + 1)$. Therefore we adopt the methodology of the example in seciton 88 and thus calculate the principle value integral (and by eveness, this is equivalent to the sum of both indefnite integrals.) We do not presume the existence of the integral untill we have shown the formula is valid.

	By the Cauchy Residue theorem under the standard semi-circle contour of radius $R$ we have that
	\begin{equation*}
		\int_{-R}^R f(z)e^{iaz}\ dz + \int_{C_R} f(z) e^{iaz}\ dz = 2\pi i \sum_{c_k \in S(f) \cap C_R^o} Res_{z=c_k}\ e^{iaz} f(z)
	\end{equation*}
	where $S(f)$ is the set of singularities of $f$ in $\mathbb{C}.$

	First to calculate the residues at $S \cap \mathbb{C}^H = \{i\}$, we observe that $f$ has a pole of order 1 at the element in $S \cap \mathbb{C}^H$. Therefore we need only evaliuate $\phi_1(z) = e^{iaz}/(z + i)$ at $z = i$. We yield $\phi_1(i) = e^{-a}/2i.$ Returning to the derivation
	\begin{equation*}
		\int_{-R}^R f(z)e^{iz}\ dz + \int_{C_R} f(z) e^{iz}\ dz = 2\pi i \frac{e^{-a}}{2i} = \pi e^{-a}
	\end{equation*}
	Lastly we need show that the $C_R$ contour integral tends to $0$.  In particular we will  bound $f(z)$ on $C_R$
	and then apply Jordan's lemma as $a > 0$.  We know that $|z^2 + 1| \geq ||z|^2 - 1|.$ Then for $R$ large enough $||z|^2 - 1| = R^2 - 1.$
	Therefore $|f(z)| \leq 1/(R^2 - 1)$ and $1/(R^2 - 1) \to 0$ as $R \to \infty.$
	Therefore under Jordan's lemma the integral is bounded by
	\begin{equation*}
		\lim_{R \to \infty} \int_{C_R} f(z) e^{iz}\ dz = 0.
	\end{equation*}
	Finally we yield that through the evenness of the integrand in the original statement,
	\begin{equation*}
		\int_0^\infty \frac{\cos ax}{x^2 + 1}\ dx = \frac{1}{2}  \int_{-R}^R f(z)e^{iz}\ dz = \frac{\pi}{2} e^{-a}
	\end{equation*}
	and this completes the derivation.
\end{proof}

\medskip \noindent {\bf (88.3)}\ Use residues to derive the integration formula
\begin{equation*}

\end{equation*}

\medskip \noindent {\bf (88.12)}\ 
\medskip \noindent {\bf (92.1)}\ 
\medskip \noindent {\bf (92.6)}\ 
\medskip \noindent {\bf (95.1)}\ 
\medskip \noindent {\bf (95.2)}\ 

\end{document}\end