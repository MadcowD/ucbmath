%%%%%%%%%%%%%%%%%%%%%%%%%%%%%%%%%%%%%%%%%%%%%%%%%%%%%%%%%%%%%%%%%%
%%%                      Homework _                            %%%
%%%%%%%%%%%%%%%%%%%%%%%%%%%%%%%%%%%%%%%%%%%%%%%%%%%%%%%%%%%%%%%%%%

\documentclass[letter]{article}

\usepackage{lipsum}
\usepackage[pdftex]{graphicx}
\usepackage[margin=1.5in]{geometry}
\usepackage[english]{babel}
\usepackage{listings}
\usepackage{amsthm}
\usepackage{amssymb}
\usepackage{framed} 
\usepackage{amsmath}
\usepackage{titling}
\usepackage{fancyhdr}

\pagestyle{fancy}


\newtheorem{theorem}{Theorem}
\newtheorem{definition}{Definition}

\newenvironment{menumerate}{%
  \edef\backupindent{\the\parindent}%
  \enumerate%
  \setlength{\parindent}{\backupindent}%
}{\endenumerate}







%%%%%%%%%%%%%%%
%% DOC INFO %%%
%%%%%%%%%%%%%%%
\newcommand{\bHWN}{2}
\newcommand{\bCLASS}{MATH 185}

\title{\bCLASS: Homework \bHWN}
\author{William Guss\\26793499\\wguss@berkeley.edu}

\fancyhead[L]{\bCLASS}
\fancyhead[CO]{Homework \bHWN}
\fancyhead[CE]{GUSS}
\fancyhead[R]{\thepage}
\fancyfoot[LR]{}
\fancyfoot[C]{}
\usepackage{csquotes}

%%%%%%%%%%%%%%

\begin{document}
\maketitle
\thispagestyle{empty}


%%%%%%% Be sure to set the counter and use menumerate
\begin{menumerate}
	\item
	\begin{definition}
		A set $S \subset \mathbb{C}$ is bounded if and only if there exists $ z \in \mathbb{C}$ such that for every $s \in S,$ $|s| \leq |z|$
	\end{definition}
	\begin{definition}
		Alternativeley, a set $S \subset \mathbb{C}$ is bounded if and only if there is an $r$  such that $S \subset B_r(0),$ where $B_s(z)$ is the ball of radius $s$ with center $z.$
	\end{definition}
	\begin{theorem}
		If $(z_n)^\infty_{n=1}$ is a convergent sequence of compex numbers,
		then the sequence is bounded.
	\end{theorem}
	\begin{proof}
		Take the value set $S = \{z_n\}.$ Then suppose there were no $r$ such that $S \subset B_r(0)$. If this is the case, the countability of $S$ implies that for every $n$, $S \cap B_n(0)$ is finite.
		Since $z_n \to z,$ take $N \in \mathbb{N}$ such that $N > |z|.$ Such an $n$ exists by the archimedian principle of $\mathbb{R}.$ Then $S \cap B_N(0)$ must be finite. 

		Take $\epsilon = N - |z|,$ then there is an $M$ such that for all $m > M,$ $d(z_n,z) < \epsilon.$ That is there are infinite elements within $\epsilon$ of $z,$ and thereby there are infinite elements in $S \cap B_N(0).$ This is a contradiction to its finiteness.

		Therefore it must be that the value set is contained within the $N$ ball, and therefore,  $(z_n)$ is bounded. 
	\end{proof} 

	\item Exercise II.1.11
	\begin{theorem}
	The function $Arg : \mathbb{C} \to \mathbb{R}$ is continuous except for along the line $L = \{z : Im(z) = 0 \wedge Re(z) < 0\}.$
	\end{theorem}
	\begin{proof}
		A function is continuous if and only if it preserves limits. Specifically, if $\lim_{h \to x} f(h) = f(x) $ implies that $f$ is continuous at $h$. Consider the restricted $Arg$ function, say $A:\mathbb{C} \setminus L \to \mathbb{R}.$ Then it is clear that $\lim_{\mathbb{C}\setminus L} A(h) = (-\pi,\pi),$ since if a point is
		within an $\epsilon$ neighborhood of another point, its gradial distance is proportionate to $\sin^{-1}$ of its $\epsilon$ distance, (a continuous function).

		However consider any $z \in L$ Such that $h \to z$ approaches from the upper half plane and $g \to z$ from the lower. Clearly $Arg(h) \to \pi$ and $Arg(g) \to -\pi,$ so no limit exists and the function is not continuous at $z.$ This completes the proof.
	\end{proof}

	\item Exercise II.1.16
	\begin{theorem}
		The punctured plane $\mathbb{C} \setminus L = \mathbb{C}_P$ is star shaped but not convex.
	\end{theorem}
	\begin{proof}
		Take any $z \in \mathbb{C}_P.$ Then for any $r \geq 1,$ $z/r$ is
		clearly in $\mathbb{C}_P$ since $r$ is always positive and the imaginary part of $z$ is always non-zero or its real 
		part is non-negative. In the first case $z/r$ is never in $L$ for all finite $r,$ and when $r \to \infty,$ then $r = 0 \in \mathbb{C}_P.$  In the second case, its real part is always positive or $0$ until it reaches $0$ by the same logic. In the case that both are true, we consider again the same logic. If $z = 0$, we are done.

		Clearly, $\mathbb{C}_P$ is not convex when considering the line, $B = \{x + iy : x = -1\}$ which contains $-1 \in L.$
 	\end{proof}
 	\begin{definition}
 		A space $X$ is contractible if the identity map is homotopic to some constant map.
 	\end{definition}
 	\begin{definition}
 		A homotopy between two continuous functions $f,g$ from a topological space $X$ to a topological space $Y$ is a continuous function $H: X \times [0,1] \to Y$, such that if $x \in X$ then,
 		$H(x,0) = f(x), H(x,1) = g(x).$ [Wikipedia]
 	\end{definition}
 	\begin{theorem}
 		Every homeomorphism is a homotopy equivalence.
 	\end{theorem}
 	\begin{theorem}
 		A star-shaped space $X$ is homotopic to a point.
 	\end{theorem}
 	\begin{proof}
 		Let $H(x,t) = x(1-t)+z_0t$, then $H(x,0) = id_X,$ and $H(x,1)$ is the constant identity. $H$ is continuous by the definition of $H$ as a star shaped space. Therefore, the star-shaped space is homotopic to a point.
 	\end{proof}
 	\begin{theorem}
 		The space $\gamma = \mathbb{C}\setminus[-1,1]$ is not star shaped.
 	\end{theorem}
 	\begin{proof}
 		The set $\gamma$ is not homeomorphic to the open unit ball $B^2$ since it is homeomorphic to the open annalus. Therefore, $\gamma$ is not homotopic to
 		$B^2$ which is homotopic to a point since $B^2$ is star shaped.
 		The space $\gamma$ could not be star shaped since if it were it would be homotopic to a point which it is not. Therefore, $\gamma$
 		is not star shaped. 
 	\end{proof}
 	\begin{theorem}
 		The punctured disk is not star shaped.
 	\end{theorem}
 	\begin{proof}
 		The punctured disk is not homeomorphic to $B^2$ for the same reason as the previous proof. Therefore it is not homeotopic, and by the logic of
 		the above proof, it is not homeotopic to a point, and so it could not possibly be star shaped as that would lead to a contradiction. This completes the proof.
 	\end{proof}
 	\item 
 	\begin{theorem}
 		The functions $x= Re z$ and $y = Im z$ are not complex differentiable
 		at any point.
 	\end{theorem}
 	\begin{proof}
 		Suppose those functions were differentiable. Then it follows that
 		there partials as functions of $\mathbb{R}^2$ should be 
 		\begin{equation}
 			Dx(p) = \begin{bmatrix}
 				1 & 0 \\
 				0 & 0
 			\end{bmatrix},
 			Dy(p) = \begin{bmatrix}
 				0 & 1 \\
 				0 & 0
 			\end{bmatrix}.
 		\end{equation}
 		This is a contradiction to the Cauchy-Rieman equations.
 	\end{proof}
 	\item We take the derivative as follows
 	\begin{equation}
 		\begin{aligned}
 			f' &= a2z + b\bar z + bz(\bar z)' + 2c\bar z (\bar z)'.
 		\end{aligned}
 	\end{equation}
 	This only makes sense where the terms containing $\bar z ' $ are not a part of
 	the equation, since the complex conjugate is not complex differentiable.
 	This occurs when $\bar z'(bz + 2c\bar z) = 0.$ So it must be that
 	$bx +2cx = 0, by - 2cy = 0$ which implies that the differentiability of
 	$f$ does not depend on $z$ but on $b,c.$

 	Setting up a linear system, we get 
 	\begin{equation}
 	Aa = 0 = 
 		\begin{bmatrix}
 			1 & 2 \\
 			1 & -1
 		\end{bmatrix}
 		\begin{bmatrix}
 			a \\
 			b
 		\end{bmatrix} =
 		\begin{bmatrix}
 			0 \\ 0
 		\end{bmatrix} \implies
 		\begin{bmatrix}
 			a \\ b
 		\end{bmatrix} \in Nul(A).
 	\end{equation}
 	And $f$ is analytic every where in this case!
 	\item Exercise II.2.5
 	\begin{theorem}
 		Suppose that $f:\mathbb{C} \to \mathbb{C}$ is analytic on $D.$ Then
 		let $g = \overline{f}$ is analytic on $D^* = \{\bar z : z \in \mathbb{D}\}.$ 
 		It follows that, $g'(w) = \overline{f(\bar w)}.$
 	\end{theorem}
 	\begin{proof}
 		Take $w \in D^*,$ so that there exists a $z \in D$ so that $\bar z = w, \bar w = z.$
 		 Consider the standard limit definition of the derivative.
 		\begin{equation}
 			\begin{aligned}
 					\lim_{\Delta w \to w} \frac{g(w + \Delta w) - g(w)}{\Delta w} &= 
					\lim_{\Delta w \to w} \frac{\overline{f(\overline{w + \Delta w})} - \overline{f(\bar w)}}{\Delta \bar w} \\
					&= \lim_{\Delta z \to z} \frac{\overline{f({z + \Delta z})} - \overline{f(z)}}{\Delta z} \\
					&= \lim_{\Delta z \to z} \frac{\overline{f({z + \Delta z}) - f(z)}}{\Delta z} \\
					&= \overline{\lim_{\Delta z \to z} \frac{{f({z + \Delta z}) - f(z)}}{\Delta z}} \\
					&= \overline{f'(z)}
					&= \overline{f'(\bar w)},
 			\end{aligned}
 		\end{equation}
 		which is the statement of the theorem.
 	\end{proof}
 	\item Validation:
 	\begin{equation}
 		Df(p) = \begin{bmatrix}
 			\cos x \sinh y & \sin x \cosh y \\
 			-\sin x \cosh y & \cos x \sinh y
 		\end{bmatrix}
 	\end{equation}
 	The complex function is $f = \ = sin x \sinh y, v = \cos x \cosh y$. It follows that, $f = ie^z +e^{-z};$ So I estimate this fucntion is a rotated cosine.
 	\item Exercise II.3.3
 	\begin{theorem}
 		If $f$ and $\bar f$ are analytic on $D,$ then $f$ is constant.
 	\end{theorem}
 	\begin{proof}
 		If $f$ and $\bar f$ are analytic for all $z \in D \subset \mathbb{C},$
 		\begin{equation}
 			Df(z) = \begin{bmatrix}
 				u_x(z) & u_y(z) \\
 				-u_y(z) & u_x(z)
 			\end{bmatrix},
 			 D\bar f(z)= \begin{bmatrix}
 			 	u_x(z) & -v_y(z) \\
 			 	u_y(z) &  -v_y(z)
 			 \end{bmatrix}
 		\end{equation}
 		So by the cauchy riemann equations we have that,
 		$\bar f$ and $f$ analytic implies 
 		\begin{align}
 			-v_y = u_y, -v_y = u_x \\
 			v_x = u_y, v_y = u_x
 		\end{align}
 		So, $-v_y = v_y$ and $-v_x = v_x$ gives
 		\begin{equation}
 			Df(z) = \begin{bmatrix}
 				0 & 0\\
 				0 & 0.
 			\end{bmatrix} 
 		\end{equation}
 		Which from multivariable calculus, we know is true if and only if
 		$f$ is a constant map!
 	\end{proof}
 	\item Exercise II.3.8
 	\begin{theorem}
 		Let $f: \mathbb{C} \to \mathbb{C}$ be an analytic function. Then if
 		$z = re^{i\theta}$ $f$ satisfies
 		\begin{equation}
 		u_r = \frac{1}{r} v_\vartheta, u_\vartheta = -rv_r.
 		\end{equation}
 	\end{theorem}
 	\begin{proof}
 		By the chain rule of partial derivatives,
 		\begin{equation}
 			\begin{aligned}
 				u_r =u_xx_r + u_yy_r = u_x\cos \vartheta + u_y \sin \vartheta	
 			\end{aligned}	
 		\end{equation}	
 		and identically by the Cauchy-Riemann  equations,
 				\begin{equation}
 			\begin{aligned}
 				v_\vartheta = v_xx_\vartheta + v_yy_\vartheta &= -v_xr\sin\vartheta + v_yr\cos\vartheta \\
 				&= r(-v_x\sin \vartheta + v_y \cos \vartheta) \\
 				&= r(u_x\cos \vartheta + u_y \sin \vartheta	) \\
 				&= r u_r.
 			\end{aligned}	
 		\end{equation}	
 		So it follows that $u_r = \frac{1}{r} v_\vartheta.$
 		Likewise, 
 		\begin{equation}
 			\begin{aligned}
 				u_\vartheta =r(u_xx_\vartheta + u_yy_\vartheta) = r(-u_x\sin\vartheta + u_y\cos \vartheta))
 			\end{aligned}	
 		\end{equation}	
		and identically by the Cauchy-Riemann equations,
 		\begin{equation}
 			\begin{aligned}
 				v_r = v_xx_r +v_yy_r = v_x\cos\vartheta + v_y\sin\vartheta &= -u_y\cos\vartheta + u_x \sin\vartheta 			\end{aligned}	
 		\end{equation}	
 		So it follows that $-r v_r = u_\vartheta.$ This completes the proof.
 	\end{proof}

\end{menumerate}

\end{document}