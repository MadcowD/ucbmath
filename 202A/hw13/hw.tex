\documentclass[11pt]{amsart}

\usepackage{amsmath,amsthm}
\usepackage{amssymb}
\usepackage{graphicx}
\usepackage{enumerate}
\usepackage{fullpage}
 \usepackage{euscript}
% \makeatletter
% \nopagenumbers
\usepackage{verbatim}
\usepackage{color}
\usepackage{hyperref}

\usepackage{fullpage,tikz,float}
\usepackage{tikz-cd}
%\usepackage{times} %, mathtime}

\textheight=600pt %574pt
\textwidth=480pt %432pt
\oddsidemargin=15pt %18.88pt
\evensidemargin=18.88pt
\topmargin=10pt %14.21pt

\parskip=1pt %2pt

% define theorem environments
\newtheorem{theorem}{Theorem}    %[section]
%\def\thetheorem{\unskip}
\newtheorem{proposition}[theorem]{Proposition}
%\def\theproposition{\unskip}
\newtheorem{conjecture}[theorem]{Conjecture}
\def\theconjecture{\unskip}
\newtheorem{corollary}[theorem]{Corollary}
\newtheorem{lemma}[theorem]{Lemma}
\newtheorem{sublemma}[theorem]{Sublemma}
\newtheorem{fact}[theorem]{Fact}
\newtheorem{observation}[theorem]{Observation}
%\def\thelemma{\unskip}
\theoremstyle{definition}
\newtheorem{definition}{Definition}
%\def\thedefinition{\unskip}
\newtheorem{notation}[definition]{Notation}
\newtheorem{remark}[definition]{Remark}
% \def\theremark{\unskip}
\newtheorem{question}[definition]{Question}
\newtheorem{questions}[definition]{Questions}
%\def\thequestion{\unskip}
\newtheorem{example}[definition]{Example}
%\def\theexample{\unskip}
\newtheorem{problem}[definition]{Problem}
\newtheorem{exercise}[definition]{Exercise}

\numberwithin{theorem}{section}
\numberwithin{definition}{section}
\numberwithin{equation}{section}

\def\reals{{\mathbb R}}
\def\torus{{\mathbb T}}
\def\integers{{\mathbb Z}}
\def\rationals{{\mathbb Q}}
\def\naturals{{\mathbb N}}
\def\complex{{\mathbb C}\/}
\def\distance{\operatorname{distance}\,}
\def\support{\operatorname{support}\,}
\def\dist{\operatorname{dist}\,}
\def\Span{\operatorname{span}\,}
\def\degree{\operatorname{degree}\,}
\def\kernel{\operatorname{kernel}\,}
\def\dim{\operatorname{dim}\,}
\def\codim{\operatorname{codim}}
\def\trace{\operatorname{trace\,}}
\def\dimension{\operatorname{dimension}\,}
\def\codimension{\operatorname{codimension}\,}
\def\nullspace{\scriptk}
\def\kernel{\operatorname{Ker}}
\def\p{\partial}
\def\Re{\operatorname{Re\,} }
\def\Im{\operatorname{Im\,} }
\def\ov{\overline}
\def\eps{\varepsilon}
\def\lt{L^2}
\def\curl{\operatorname{curl}}
\def\divergence{\operatorname{div}}
\newcommand{\norm}[1]{ \|  #1 \|}
\def\expect{\mathbb E}
\def\bull{$\bullet$\ }
\def\det{\operatorname{det}}
\def\Det{\operatorname{Det}}
\def\rank{\mathbf r}
\def\diameter{\operatorname{diameter}}

\def\t2{\tfrac12}

\newcommand{\abr}[1]{ \langle  #1 \rangle}

\def\newbull{\medskip\noindent $\bullet$\ }
\def\field{{\mathbb F}}
\def\cc{C_c}



% \renewcommand\forall{\ \forall\,}

% \newcommand{\Norm}[1]{ \left\|  #1 \right\| }
\newcommand{\Norm}[1]{ \Big\|  #1 \Big\| }
\newcommand{\set}[1]{ \left\{ #1 \right\} }
%\newcommand{\ifof}{\Leftrightarrow}
\def\one{{\mathbf 1}}
\newcommand{\modulo}[2]{[#1]_{#2}}

\def\bd{\operatorname{bd}\,}
\def\cl{\text{cl}}
\def\nobull{\noindent$\bullet$\ }

\def\scriptf{{\mathcal F}}
\def\scriptq{{\mathcal Q}}
\def\scriptg{{\mathcal G}}
\def\scriptm{{\mathcal M}}
\def\scriptb{{\mathcal B}}
\def\scriptc{{\mathcal C}}
\def\scriptt{{\mathcal T}}
\def\scripti{{\mathcal I}}
\def\scripte{{\mathcal E}}
\def\scriptv{{\mathcal V}}
\def\scriptw{{\mathcal W}}
\def\scriptu{{\mathcal U}}
\def\scriptS{{\mathcal S}}
\def\scripta{{\mathcal A}}
\def\scriptr{{\mathcal R}}
\def\scripto{{\mathcal O}}
\def\scripth{{\mathcal H}}
\def\scriptd{{\mathcal D}}
\def\scriptl{{\mathcal L}}
\def\scriptn{{\mathcal N}}
\def\scriptp{{\mathcal P}}
\def\scriptk{{\mathcal K}}
\def\scriptP{{\mathcal P}}
\def\scriptj{{\mathcal J}}
\def\scriptz{{\mathcal Z}}
\def\scripts{{\mathcal S}}
\def\scriptx{{\mathcal X}}
\def\scripty{{\mathcal Y}}
\def\frakv{{\mathfrak V}}
\def\frakG{{\mathfrak G}}
\def\aff{\operatorname{Aff}}
\def\frakB{{\mathfrak B}}
\def\frakC{{\mathfrak C}}

\def\symdif{\,\Delta\,}
\def\mustar{\mu^*}
\def\muplus{\mu^+}

\def\soln{\noindent {\bf Solution.}\ }


%\pagestyle{empty}
%\setlength{\parindent}{0pt}

\begin{document}

\begin{center}{\bf Math 202A --- UCB, Fall 2016 --- M.~Christ}
\\
{\bf Problem Set 13, due Wednesday November 30}
\end{center}

\medskip \noindent {\bf (13.1)}\ (Folland problem 4.22)\ 
Let $X$ be a topological space, an d let $(Y, d)$ be a complete metric space.
Let $(f_n)$ be a seuqnece of functions in $Y^X$ that satisfies
$\sup_{x\in X} \lim_{m,n\to\infty} d(f_m(x), f_n(x)) = 0$. \\
\noindent \textbf{(a)} Show that there exists an $f \in Y^X$ so that
$\sup_{x\in X} d(f_n(x), f(x)) \to 0.$
\begin{proof}
	We will first construct $f$ and show that the the uniform norm distance tends to $0$. 
	As $Y$ is a complete metric space under $d$ and for every $x$, $\lim_{m,n\to\infty} d(f_m(x), f_n(x))$
	is bounded by $\sup_{x\in X} \lim_{m,n\to\infty} d(f_m(x), f_n(x)) = 0$, the sequence $(f_m(x))_m$ is cauchy
	and thus has a limit, say $f(x).$ Now define $f: X \to Y$ so that $x \mapsto f(x)$ as defined previously for every
	$x$. We have shown that this function is well defined.

	We now claim that $\sup_{x\in X} d(f_n(x), f(x)) \to 0.$ First\footnote{We will abuse notation and 
	write $\lim_n \sup_{x \in X} d(f_n(x), f(x)) = \lim_n\sup_{j\geq n} \sup_{x \in X} d(f_j(x), f(x))$, but we 
	do not assume the limit exists. In fact we will show that the the $\lim \sup_n$ is bounded above by $0$ and the sequence is bounded below by $0$ so the limit is $0.$ }
 	\begin{equation*}
		\begin{aligned}
		\lim_{n\to \infty }\sup_{x \in X} d(f_n(x), f(x)) &=  \lim_{n\to \infty }\sup_{x \in X} \lim_{m\to \infty}d(f_n(x), f_m(x)) \\
		&= \lim_{n\to \infty }\sup_{x \in X} \lim_{m\to \infty}  \sup_{k \geq m} d(f_n(x), f_k(x)) \\
		&\leq \lim_{n\to \infty }\lim_{m\to \infty}  \sup_{x \in X}  \sup_{k \geq m} d(f_n(x), f_k(x)) \;\;\;\text{(sup of dec. seq at x)} \\
		&\leq \lim_{n\to \infty }\lim_{m\to \infty}  \sup_{k \geq m} \sup_{x \in X}   d(f_n(x), f_k(x)) \;\;\;\text{(sup of maximal distances over x.)} \\
		& = \lim_{m,n\to \infty }  \sup_{x \in X}   d(f_n(x), f_k(x)) = 0 \;\;\;\text{(existence by hypothesis)}
		\end{aligned}
	\end{equation*}
	This completes the proof. 
\end{proof}
\noindent \textbf{(b)} Show that $f$ is unique. 
\begin{proof}
	Suppose that $f$ is not unique, that is there is a $g \neq f$ so that $\sup_{x\in X} d(f_n(x), g(x)) \to 0.$ Then
	for every $n$
	\begin{equation*}
		\sup_x d(f(x), g(x)) \leq  \sup_x \left(d(f(x), f_n(x)) +  d(f_n(x), g(x))\right) \leq  \sup d(f(x), f_n(x)) + \sup d(g(x), f_n(x)).
	\end{equation*}
	But then $\sup d(g(x), f_n(x)) \to 0$ and $\sup d(f(x), f_n(x)) \to 0$ implies that $\sup_x d(f(x), g(x)) = 0$;
	thus for every $x, $ $f(x) = g(x)$, and $f = g$; a contradiction to the existence of $g$. Therefore $f$ is unique.
\end{proof}
\noindent \textbf{(c)} Show that if every function $f_n$ is continuous then so is $f$.
\begin{proof}
	Let $f_n$ and $f$ be given as above. Then for every $\epsilon > 0$, there is an $N$ so that for all $n \geq N$
	$\sup_{x\in X} d(f_n(x), f(x)) < \epsilon/3.$ Then by continuity of $f_n$ there is an open neighborhood $U_n \subset X$ of $x$ so that for all $y \in U_n$ $d(f_n(x), f_n(y)) < \epsilon/3.$ Finally
	\begin{equation*}
	\begin{aligned}
		d(f(x), f(y)) &\leq d(f(x), f_n(x)) + d(f_n(x), f_n(y)) + d(f_n(y), f(y)) \\
		&\leq d(f_n(x), f_n(y)) + 2\sup_{x\in X} d(f_n(x), f(x)) < 2/3\epsilon + \epsilon/3 = \epsilon.
	\end{aligned}
	\end{equation*}
	Thus for every $\epsilon$, there is an open neighborhood of $x$ so that for all $y$ in that neighborhood, $d(f(x), f(y)) < \epsilon$.

\end{proof}
\medskip \noindent {\bf (13.2)}\ Give an elementary proof of the Tietze Extension Theorem for the special case in which
$X = \mathbb{R}.$
\begin{proof}
	If $A$ is a closed subset of $\mathbb{R}$ and $f: A \to [a,b]$ is continuous, we wish to show the existence of a continuous function $F: \mathbb{R} \to [a,b]$ so that $F|_A = f.$ The main idea of our proof is that since $f$ is
	defined on a closed subset of $\mathbb{R}$ we need only work to define $F$ on the open compliment of $A$ which
	from elementary real analysis is just the countable union of open intervals.

	In the trivial case that $A = \mathbb{R}$ or $A = \emptyset$, let $F = f$ or $F = c$ respectively, the first is obviously a continuous extension and the second is a trivial extension since $f$ is undefined for every $x \in \mathbb{R}$. 

	Otherwise recall from real analysis that $\mathbb{R} \setminus A = B$ is an open subset of $\mathbb{R}$ and so is
	the countable union of disjoint open intervals $O_n$. We will essentially define $F$ as the affine interpolation of $f$ on these open intervals. First assume that $O_n = (a_n, b_n)$ where $a_n, b_n$ are finite; we will address the nonfinite cases later. Define $$F|_{[a_n,b_n]}: x \mapsto \frac{f(b_n) - f(a_n)}{b_n - a_n}(x - a_n) + f(a_n).$$
	$F|_{[a_n,b_n]}(a_n) = f(a_n)$ and $F|_{[a_n,b_n]}(b_n) = f(b_n)$. Furthermore from $104$, $F|_{[a_n,b_n]}$ is continuous for all $x \in {[a_n,b_n]}$ as $F|_{[a_n,b_n]}$ is just the affine linear interpolation of $f(a_n)$ and $f(b_n)$. If there are $O_n$ so that $a_n = -\infty$ or $b_n = \infty$ then  let $F|_{(a_n,b_n]}: x \mapsto f(b_n)$
	or $F|_{[a_n,b_n)}: x \mapsto f(a_n)$ respectively. In this case these functions are just constant and so $F|_{O_n}$ is also continuous. Lastly let $F|_A = f,$ and then $F: \mathbb{R} \to [a,b]$ so that $F$ satisfies all of the previous restrictions\footnote{This is defined since we described the restriction of $F$ on all $x$ in $\mathbb{R} \setminus A \cup A = \mathbb{R}.$}

	By the continuity of $f$ on $A$ and the previous arguments $F$ is continuous with respect to the whole topology on $\mathbb{R}$ at least on $A^o$ and $B^o = B.$ 
	As for continuity at $A \setminus A^o = C$, we observe that $cl(A) = A$ and so $C = \partial A$. From real analysis the boundary of a closed set cannot contain any open intervals as $A^o$ is the countable union of open intervals, and if $\partial A$ contained an open interval, then $A^o$ would not be the largest open set containing $A$, therefore $\partial A = \{y \in A\}$ so that if $v,w \in \partial A$ $v < w$ or $w < v$. Now as $\delta \to 0,  \sup f((v - \delta, v+ \delta) \cap A) - \inf f((v - \delta, v+ \delta) \cap A) \to 0$. Observe that now $\sup F((v - \delta, v+ \delta)) = f((v - \delta, v+ \delta) \cap A)$ since on $(v - \delta, v+ \delta) \setminus A$
	$F$ is defined on some collection of $O_n$ to never achieve values larger than $f$ on the end points on which it must be defined. Likewise for the infimum $\inf F((v - \delta, v+ \delta)) = \inf f((v - \delta, v+ \delta) \cap A)$. Therefore $\sup F((v - \delta, v+ \delta)) - \inf F((v - \delta, v+ \delta)) \to 0$ and $F$ is continuous at $v \in \partial A$.

	 Therefore $F$ is continuous on $A^o$ and $\partial A$ and $A^c$ with respect to the global topology so it is continuous. This completes the proof.
	\end{proof}
\medskip \noindent {\bf (13.3)}\ Show that any product of connected topological spaces is connected.
\begin{proof}
	Let $X = \prod_{\alpha} X_\alpha.$ Then we wish to show that if for every $\alpha$ $X_\alpha$ connected implies that $X$ connected. In the contrapositive if there is an $X_\alpha$ disconnected then we must show that the product is disconnected.

	If $X_\alpha$ disconnected then there is a disjoint non-trivial open partion $A \cup B = X_\alpha$. We claim that sets $\pi^{-1}_\alpha(A), \pi^{-1}_\alpha(B)$ form an open disjoint non-trivial partition of $X$. First by definition
	of the product topology, $\pi^{-1}_\alpha(A)$ and $\pi^{-1}_\alpha(B)$ are both open. Now suppose that $\pi^{-1}_\alpha(A) \cap \pi^{-1}_\alpha(B) \neq \emptyset.$ Then take $x$ in the intersection. It follows that $\pi_\alpha(x) \in A$ and $\pi_\alpha(x) \in B$, but this would contradict $A \cap B = \emptyset.$ Therefore $\pi^{-1}_\alpha(A) \cap \pi^{-1}_\alpha(B) = \emptyset.$ Now observe that $X = \pi^{-1}_\alpha(X_\alpha)$ by definition and so $$X = \pi^{-1}_\alpha(A \cup B) = \pi^{-1}_\alpha(A) \cup \pi^{-1}_\alpha(B) $$. It remains to show that the both sets are non-empty. Since $\pi_\alpha$ is surjective as the cannonical projection mapping from $X$ to $X_\alpha$ the preimage of any non-empty set cannot be empty, therefore the partition is not trivial. Thus $X$ is disconnected.

	By contraposition any product of connected spaces is connected under the product topology.
\end{proof}

\medskip \noindent {\bf (13.4)}\ Let $X$ be a topological space equipped with with an equivalence realtion, $\tilde X$ the set of equivalence classes, $\pi: X \to \tilde X$ the map taking each $x \in X$ to its equivalence class, and $\scriptt = \{U \subset \tilde X\ :\ \pi^{-1}(U) \in \scriptt_X\}$. \\
\noindent \textbf{(a)} Show that $\scriptt$ is a topology on $X$.
\begin{proof}
	First we show that $\tilde X \in \scriptt$. Since $\pi$ is a surjection onto $\tilde X$ it follows that $\pi^{-1}(\tilde X) = X \in \scriptt_X$ and so $\tilde X \in \scriptt$. Additionally $\pi^{-1}(\emptyset) = \emptyset \in \scriptt_X$ so $\emptyset \in \scriptt$. Now take any $U_\alpha \in \scriptt$ with an arbitrary index set $\alpha \in A$. We claim that $\bigcup_{\alpha} U_\alpha \in \scriptt$. First by the definition of preimage 
	\begin{equation*}
	\pi^{-1}\left(\bigcup_{\alpha \in A} U_\alpha\right) = \bigcup_{\alpha \in A} \pi^{-1}\left( U_\alpha\right) \in \scriptt_X
	\end{equation*}
	since $\pi^{-1}\left( U_\alpha\right) \in \scriptt_X$ are open in $X$ and the topology $\scriptt_X$ is closed under arbitrary union. Lastly consider the finite intersection of $U_m \in \scriptt$; that is, 
	We claim that $\bigcap_{1}^n U_m \in \scriptt$. First by the definition of preimage 
	\begin{equation*}
	\pi^{-1}\left(\bigcap_{m=1}^n U_m\right) = \bigcap_{m=1}^n  \pi^{-1}\left( U_m\right) \in \scriptt_X
	\end{equation*}
	since $\pi^{-1}\left( U_\alpha\right) \in \scriptt_X$ are open in $X$ and the topology $\scriptt_X$ is closed under finite intersection.

	Thus $\scriptt$ satisfies the definition of a topology.
\end{proof}

\noindent \textbf{(b)} Show that if $Y$ is a topological space, $f: \tilde X \to Y$ is continuous if and only if 
$f \circ \pi$ is continuous.
\begin{proof}
Suppose that $f: \tilde X \to Y$ is continuous and $\scriptt_Y$ denotes the topology of $Y$. Then for any open $V \in \scriptt_Y$, $f^{-1}(V) \in \scriptt,$ but then $\scriptt_X \ni \pi^{-1}( f^{-1}(V)) = (f\circ \pi)^{-1}(V).$ Thus the preimage mapping $(f\circ \pi)^{-1}$ preserves the topology $\scriptt_Y$ and so $f \circ \pi: X \to Y$ is continuous. 

In the other direction if $f \circ \pi: X \to Y $ is continuous then for every open neighborhood of $(f \circ \pi)(x),$ say $V$ there is an open neighborhood of $x$, say $U$, so that $(f \circ \pi)(U) \subset V.$ Then $\pi(U) = W \ni \pi(x)$ is a neighboorhood of $\pi(x)$ so that $f(W) \subset V$. Thus take $W^o$ and then $f(W^o) \subset V$. Now since $\pi$ is surjective for every $\tilde x \in \tilde X$ pick any $x \in \pi^{-1}(x)$. Then for any $V$ as above there is an open neighboorhood $W^o$  of $\tilde x$ so that $f(W^o) \subset C$. Therefore $f: \tilde X \to Y$ is continuous.
\end{proof}

\noindent \textbf{(c)} Show that $\tilde X$ is $T_1$ iff every equivalence class is closed.
\begin{proof}
	Suppose that $(\tilde X, \scriptt)$ is $T_1$. Then $X \setminus \{\tilde x\} \in \scriptt$ for every $\tilde x \in \tilde X$ and so
	$\pi^{-1}(\tilde X \setminus \{\tilde x\}) \in \scriptt_X.$ Now since $\pi$ maps every $x$ into its exact equivalence class and the fundamental theorem of equivalence relations states that the equivalence classes of $x$ form a partiiton of the space $\pi^{-1}(\tilde X \setminus \{\tilde x\}) \cap \pi^{-1}(\{\tilde x\}) = \emptyset$
	and $\pi^{-1}(\tilde X \setminus \{\tilde x\}) \cup \pi^{-1}(\{\tilde x\}) = X$. Therefore $X \setminus  \pi^{-1}(\tilde X \setminus \{\tilde x\}) =  \pi^{-1}(\{\tilde x\}) = [x]_\sim$ for some $x;$ that is $[x]_\sim$ is the compliment of an open set and so $[x]_\sim$ is closed. Since we argued this for every $\tilde x$ and $\pi$ is surjective, we get that for every $x \in X$, $[x]_\sim$ is closed.

	Suppose that every equivalence class is closed. Then for every $x \in X$, $X \setminus [x]_\sim \in \scriptt_X.$  
	Now take any $\{\tilde x\} \subset \tilde X.$ It follows that $\pi^{-1}(X \setminus \{\tilde x\}) = \pi^{-1}(\tilde X) \setminus \pi^{-1}(\tilde x) = X \setminus [x]_\sim \in \scriptt_X$ and so $\tilde X \setminus \{\tilde x\}\in \scriptt$ for every $\tilde x$ and so singletons are closed in $\tilde X$ and thus $(\tilde X, \scriptt)$ is $T_1$.
\end{proof}

\medskip \noindent {\bf (13.5)}\ Let $(X, \scriptt)$ be compact and Hausdorff. \\ 
\noindent \textbf{(a)} Let $\scriptt'$ be strictly stronger than $\scriptt$ on $X$. Show that $(X, \scriptt')$ is not compact.
\begin{proof}
	Suppose that $\scriptt' \supset \scriptt$ and $\scriptt' \neq \scriptt$ and $\scriptt'$ compact. Let $
	id: (X, \scriptt') \to (X, \scriptt)$ so that $id(x) = x$. Then if $V \in \scriptt$ $id^{-1}(V) = V \in \scriptt'$
	by our assumption that $\scriptt'$ is a stronger topology. Therefore $id$ is a continuous bijection. Then by Proposition
	4.28, $(X, \scriptt')$ compact and $(X, \scriptt)$ Hausdorff implies that $id$ is a homeomorphism. Therefore $\scriptt' = \scriptt.$ This is a contradiction ahd so $\scriptt$ cannot be compact.
\end{proof}
\noindent \textbf{(b)} Let $\scriptt'$ be strictly weaker than $\scriptt$ on $X$. Show that $(X, \scriptt')$ is not Hausdorff.
\begin{proof}
	Suppose that $\scriptt' \subset \scriptt$ and $\scriptt' \neq \scriptt$ and $\scriptt'$ Hausdorff.  Let $
	id: (X, \scriptt) \to (X, \scriptt')$ so that $id(x) = x$. Then if $V \in \scriptt'$ $id^{-1}(V) = V \in \scriptt$
	by our assumption that $\scriptt'$ is a weaker topology. Therefore $id$ is a continuous bijection. Then by Proposition
	4.28, $(X, \scriptt)$ compact and $(X, \scriptt')$ Hausdorff implies that $id$ is a homeomorphism. Therefore $\scriptt' = \scriptt.$ This is a contradiction ahd so $\scriptt$ cannot be Hausdorff.
\end{proof}
\medskip \noindent {\bf (13.6)}\ Let $X$ be a topological space. Show that following are equivalent: (i) $X$ is normal; (ii) $X$ satisfies the conclusion of Urysohn's Lemma; (iii) $X$ satisfies the conclusion of the Tietze Extension Theorem.
\begin{proof}
	Trivially\footnote{$X$ normal is a hypothesis of both propositions.} (i) implies (ii) and (iii). We will now show that (ii) implies (i) and (iii) implies (ii).

	First suppose that $X$ satisfies the conclusions of Urysohn's Lemma, that is for $A, B$ closed disjoint in $X$ then
	there is a continuous function $f: X \to [0,1]$ so that $f(A)= \{1\}$ and $f(B) = \{0\}.$ Now when $0< \epsilon < 1$ consider the open set $V_\epsilon = (\epsilon, 1]$ then $A \subset f^{-1}(V_\epsilon)$.  Additionaly consider the open set $U_\epsilon = (\epsilon, 1]$ then $B \subset f^{-1}(U_\epsilon)$. We claim that $f^{-1}(U_\epsilon) \cap f^{-1}(V_\epsilon) = \emptyset.$ By the set operation homomorphism property of the preimage, we have that $$ \emptyset = f^{-1}(\emptyset)  =f^{-1}(U_\epsilon \cap V_\epsilon) = f^{-1}(U_\epsilon) \cap f^{-1}(V_\epsilon).$$
	Therefore $A, B$ are contained in open sets which are disjoint. Since this holds for every $A, B$, the space $X$ is normal.

	Now suppose that $X$ satisfies the conclusion of the Tietze Extension Theorem. Then take any two closed disjoint sets $A, B$ in $X$. Let $f: A \cup B \to [0,1]$ so that $f(A) = \{0\}$ and $f(B) = \{1\}.$ Since $A, B$, $A \cup B$ is closed. Now for every $x \in A \cup B$, $x \in A$ or $x \in B$ but not both. Assume that $x \in A$. Then
	$f(x) = 0$ and for every open neighborhood $V$ of $f(x)$ in $[0,1]$ we claim that $f^{-1}(V)$ is open. since $f(A \cup B) = \{0, 1\}$ then $V$ must contain $0$ and need not contain $1$. In the case that $0, 1 \in V$ we have that
	$f^{-1}(V) = A \cup B$ which is the whole space $A \cup B$ and therefore is open in the relative subspace topology. Otherwise $f^{-1}(V) = A$ which is closed in the relative subspace topology since $A \cap A \cup B = A$ and $A$ is closed globally. Since $B$ is closed in the global topology and $B \cap A \cup B = B$, then $B$ is closed in the subspace topology. By the disjointness of $A$ and $B$, $A \cup B \setminus B = A$ and so $A$ is open in the subspace topology. Therefore $f^{-1}(V) = A$ which is an open neighborhood of $x \in A$. The same argument can be made by replacing symbols to show that $f$ is continuous for $x \in B$.

	Then the Tietze extension theorem gives a continuous extension $F: X \to [0,1]$ so that $F|_{A \cup B} = f;$ that is $f|_A = 0$ and $f|_B = 1$, therefore the conclusion of Urysohn's Lemma is satisfied.

	We have thus shown that $(ii) \implies (i) \implies (iii) \implies (i)$ and thus all of the conditions are equivalent.
\end{proof}


\medskip \noindent {\bf (13.7)}\ Show that every sequentially compact topological space is countably compact.
\begin{proof}
	Suppose that $X$ is sequentially compact. Then given any sequence $(x_n)_n$ then there is a subsequence which converges to some $x \in X;$ namely $x_{n_k} \to x.$ If $\scriptv$ is a countable open cover of $X$, then we would like to show that there is a finite subcover. Proposition 4.21 implies that $X$ is countable compact if and only if for every countable family $\{F_n\}$ of
	closed sets with the finite intersection property ($\bigcap_{n\in B\subset\mathbb{N}} F_n \neq \emptyset$ ), it follows that
$\bigcap_{n=1}^\infty F_n \neq \emptyset.$

	To show compactness we will show the dual property with closed sets. Let $\{F_n\}$ be given with the above property and then let choose $x_n$ so that $x_n \in \bigcap_{m=1}^n F_m.$ Such an $x_n$ exists for every $n$ by the finite intersection property, and there is a subsequence $x_{n_k}$ which converges to $x$ in $X$. Suppose $x \not\in \bigcap_{n=1}^\infty F_n$, then there is an $m$ so that $x \not\in F_m$. Therefore by $F_m$ closed $x \not\in acc(F_m).$ By definition for all $j >m$, $x_{n_j} \in \bigcap_{p=1}^{x_{n_j}} F_p \subset F_m$. Therefore the sequence $(x_{n_k})_{k=m}^\infty$ is entirely contained in $F_m$.
	Now since $x \notin acc(F_m)$, it follows that $x \notin acc\left(\{x_{n_k}\}_{k=m}^\infty\right)$ and so
	there is a neighborhood $V$ of $x$ so that $\{x_{n_k}\}_{k=m}^\infty \cap (V \setminus \{x\}) = \emptyset$, 
	and therefore exists a neighborhood so that for all $k \geq m$, and for all $j > k$ $x_{n_j} \notin V$ and so $x_{n_k} \not\to x,$ (unless of course $x_{n_k} = x$ becomes constant at somepoint in the sequence, in which case $x \in F_m$, but this is not possible.) This is a contradiction, and so $x \in \bigcap_{m=1}^\infty F_m$ and thus $X$ is countably compact.
\end{proof}

\medskip \noindent {\bf (13.8)}\ Show that if $X$ is countably compact, then every sequence in $X$ has a cluster point and additionally if $X$ is also first countable then $X$ is sequentially compact.
\begin{proof}
	Let $(x_n)$ be some sequence in $X$. Let $S_m = \{x_n\}_{n=m}^\infty$ be the value set of the sequence starting at $m$. Finally let $G = \bigcap_{n=1}^\infty acc(S_n).$  When $n \leq m$, $S_m \subset S_n$ and so $acc(S_m) \subset acc(S_n)$. Therefore take any finite $B \subset\mathbb{N}.$ There is a maximal element say $k = \max B$, and thus $acc(S_j) \supset acc(S_k)$ so the intersection of accumulation points of the partial sequence value sets is non-empty; that is the family $(acc(S_n))_{n\in \mathbb{N}}$ has the finite intersection property and every member is closed in $X$. Hence by the countable compactness of $X$, $G$ is non-empty. We claim that if $x \in G$ then $x$ is a cluster point of the sequence.

	Taking $x \in G$ then forall $m$ and for all neighborhoods $V$ of $x$ then $V \cap S_m$ is non-tivial. Hence forall neighborhoods and forall $m$, there exists an $n > m$ so that $x_n \in S_m$ and $x_n \in V$. Therefore $(x_n)$ is in $V$ infinitely often and so $x$ is a cluster point of the sequence.

	Supposing additionally that $X$ is a first countable space, then in Homework 11, exercize $11.4$ (or equivalently Folland 4.7) we showed that if $x$ is a cluster point of a sequence in a first countable sequence, then there is a subsequence which converges to $x$. Adopting the results of this exercise to the previous proof, $x \in G$ is a cluster point of $(x_n)$ and so there is a subsequence which converges to it. Therefore for any sequence in $X$ there is a convergent subsequence (the existence of a cluster point guarenteed), and so $X$ is sequentially compact.

	This completes the proof.
\end{proof}

\medskip \noindent {\bf (13.9)}\ If $X$ is normal, then $X$ is countably compact iff $C(X) = BC(X).$
\begin{proof}
	Suppose that $X$ is countably compact. Now if $f \in C(X)$, then $f: X \to \mathbb{C}$ has the property that $f(X)$
	is countably compact. Since $\mathbb{C}$ is second countable then $f(X)$ is sequentially compact. Additionally since $\mathbb{C}$ is a metric space, from the theory of metric spaces we know that sequentiall compactness implies compactness which implies total boundedness and closedness (Heine Borel in $\mathbb{R}^2 \simeq \mathbb{C})$\footnote{Closedness comes from $\mathbb{C}$ metric spaces being Hausfordd.} Therefore $f(X)$ is bounded and so $f$ is bounded; that is $C(X) \subset B(X).$ Then $BC(X) = B(X) \cap C(X) = C(X).$

	Now suppose that $X$ is not countably countably compact, then by 13.7, $X$ is not sequentially compact, and 
	so there is a sequence with no convergent subsequence. Therefore there is a sequence without a cluster point.
	Take such a sequence, say $(x_n).$ Then we claim that the value set $S = \{x_n\}_{n=1}^\infty$ is closed. If $x \in acc(S)$ we wish to show that $x \in S$. 

	Suppose for the sake of contradiciton that $x \notin S$, then because the sequence has no cluster points, there is a neighborhood of $x$ say $W$ so that there is an $N$ so that $x_n \notin W$ when $n > N.$ Then if $S_n = S \setminus \{x_{n}\}_{n=1}^m$ it follows that $S_n \cap W = \emptyset$ for all $n > N.$ Since $x$ is an accumulation point it must then be that $S \cap W \ni x_k \neq x$ when $k \leq N.$ Now consider any neighborhood $V \subset W$ of $x$, then $\emptyset \neq V \cap S = V \cap S \setminus S_N$ and so $x \in acc(S \setminus S_N).$ However, $x$ 
	is $T_4$, and therefore $S \setminus S_N$ is  closed because it is a finite set of points. Therefore $acc(S \setminus S_N) \subset S \setminus S_N \subset S$, which contradicts $x \notin S$. Therefore $acc(S) \subset S$ and so $S$ is closed.

	Now let $f : S \to \mathbb{C}$ so that $f(x) = \max \{n \in \mathbb{N}\ :\ x_n = x \} + 0i$. Such a maximum is defined because $x_n$ cannot cluster at any $x \in S$; that is, $\{n \in \mathbb{N}\ :\ x_n = x\}$ must be finite since $(x_n)$ has no cluster points and cannot visit $x$ infiniteley many times. We claim the function is not bounded. Suppose that there were an $N$ so that 
	for all $x \in S$ $|\max \{n \in \mathbb{N}\ :\ x_n = x\}| \leq N$. Then if $m > N$, $x_m \in S$ since $N$ is finite 
	and additionally $N \geq |f(x_m)| = \max \{n \in \mathbb{N}\ :\ x_n = x\} = m > N$ which is a contradiction so, $f$
	cannot be bounded. 

	To show $f$ is continuous on $S$ with respect to its subspace topology take any closed subset $K$ of $\mathbb{C}$, then it suffices to show that $f^{-1}(K)$ is a closed subset of $X$. Since $S \sim \mathbb{N}$ there are two cases: If $f^{-1}(K)$ is finite then it is closed by $X$ a $T_4$ topological space. If it is not then it is countable. For each $y \in f^{-1}(K)$ clearly the integer $f(y) = \max \{ n : x_n = y\}$ is unique as the natural numbers have a linear order and if $x_f(y) = y$ and $y \neq z$ then $x_f(y) \neq z$. If $y_k$ is an enumeration of $f^{-1}(K)$ then let $x_{f(y_k)}$ be a subsequence of $x_n$. As $f(y_k)$ does not repeat integers by the aforementioned uniqeuness, if $k \neq j$ then $y_k \neq y_j$ and so $f(y_k) \neq f(y_j)$ and so $x_{f(y_k)} \neq x_{f(y_j)}$ and so finally $(x_{f(y_k)})_{k=1}^\infty$ does not repeat elements infinitely many times. Because $(x_n)$ does not have any cluster points the non-repeatability of $x_{f(y_k)}$ implies that $(x_{f(y_k)})$ has no cluster points. Observing that the value set $S(K) = \{x_{f(y_k)}\}_{k=1}^\infty$ is exactly $f^{-1}(K)$, we use the logic showing that $(x_n)$ was closed to conclude that $S(K)$ is closed in $X;$ therefore $S(K)$ is closed in $S$. Hence $f$ is continuous.

	Now by Corrolary $4.17$, it follows that $f$ can be extended to some $F \in C(X)$ so that $F|_S = f.$ Therefore $F$
	is an unbounded continuous function on $X$ and so $BC(X) \neq C(X).$
\end{proof}
\end{document}\end
	