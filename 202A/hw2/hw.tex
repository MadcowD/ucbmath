\documentclass[11pt]{amsart}

\usepackage{amsmath,amsthm}
\usepackage{amssymb}
%\usepackage{graphicx}
%\usepackage{enumerate}
\usepackage{fullpage}
% \usepackage{euscript}
% \makeatletter
% \nopagenumbers
\usepackage{verbatim}
\usepackage{color}
\usepackage{hyperref}
%\usepackage{times} %, mathtime}

\textheight=600pt %574pt
\textwidth=480pt %432pt
\oddsidemargin=15pt %18.88pt
\evensidemargin=18.88pt
\topmargin=10pt %14.21pt

\parskip=1pt %2pt

\def\reals{{\mathbb R}}
\def\torus{{\mathbb T}}
\def\integers{{\mathbb Z}}
\def\rationals{{\mathbb Q}}
\def\naturals{{\mathbb N}}
\def\complex{{\mathbb C}\/}
\def\distance{\operatorname{distance}\,}
\def\support{\operatorname{support}\,}
\def\dist{\operatorname{dist}\,}
\def\Span{\operatorname{span}\,}
\def\degree{\operatorname{degree}\,}
\def\kernel{\operatorname{kernel}\,}
\def\dim{\operatorname{dim}\,}
\def\codim{\operatorname{codim}}
\def\trace{\operatorname{trace\,}}
\def\dimension{\operatorname{dimension}\,}
\def\codimension{\operatorname{codimension}\,}
\def\kernel{\operatorname{Ker}}
\def\Re{\operatorname{Re\,} }
\def\Im{\operatorname{Im\,} }
\def\eps{\varepsilon}
\def\lt{L^2}
\def\bull{$\bullet$\ }
\def\det{\operatorname{det}}
\def\Det{\operatorname{Det}}
\def\diameter{\operatorname{diameter}}
\def\symdif{\,\Delta\,}
\newcommand{\norm}[1]{ \|  #1 \|}
\newcommand{\set}[1]{ \left\{ #1 \right\} }
\def\one{{\mathbf 1}}
\def\cl{\text{cl}}

\def\newbull{\medskip\noindent $\bullet$\ }
\def\nobull{\noindent$\bullet$\ }



\def\scriptf{{\mathcal F}}
\def\scriptq{{\mathcal Q}}
\def\scriptg{{\mathcal G}}
\def\scriptm{{\mathcal M}}
\def\scriptb{{\mathcal B}}
\def\scriptc{{\mathcal C}}
\def\scriptt{{\mathcal T}}
\def\scripti{{\mathcal I}}
\def\scripte{{\mathcal E}}
\def\scriptv{{\mathcal V}}
\def\scriptw{{\mathcal W}}
\def\scriptu{{\mathcal U}}
\def\scriptS{{\mathcal S}}
\def\scripta{{\mathcal A}}
\def\scriptr{{\mathcal R}}
\def\scripto{{\mathcal O}}
\def\scripth{{\mathcal H}}
\def\scriptd{{\mathcal D}}
\def\scriptl{{\mathcal L}}
\def\scriptn{{\mathcal N}}
\def\scriptp{{\mathcal P}}
\def\scriptk{{\mathcal K}}
\def\scriptP{{\mathcal P}}
\def\scriptj{{\mathcal J}}
\def\scriptz{{\mathcal Z}}
\def\scripts{{\mathcal S}}
\def\scriptx{{\mathcal X}}
\def\scripty{{\mathcal Y}}
\def\frakv{{\mathfrak V}}
\def\frakG{{\mathfrak G}}
\def\frakB{{\mathfrak B}}
\def\frakC{{\mathfrak C}}


\def\soln{\noindent {\bf Solution.}\ }

\begin{document}

\begin{center}{\bf Math 202A --- Problem Set 2 --- William Guss} \end{center}


\bigskip


\medskip \noindent {\bf (2.1)}\ 
Show that
if $(X^*_j,\rho^*_j,\phi_j)$ are completions of $(X,\rho)$ for $j=1,2$
then there exists a bijection $\psi:X_1^*\to X_2^*$
such that $\psi\circ\phi_1=\phi_2$.

\begin{proof}
	If $X_1, X_2$ are completions, then for every cauchy sequence in $X$, say $(x_n)$, there is a \emph{unique limit} say $a_1$  ($a_2$ resp.) in the completion $X_1$ and in the completion $X_2$. Define $\psi:X_1 \to X_2$ to be the identity map when $\psi$ restricted to $X.$ Otherwise for every cauchy sequence $(x_n)$, define $\psi$ on the limit in the completion so that $a_1 \mapsto a_2.$ If $(x_n)$ is cauchy then it must have a limit in both $X_1$ and $X_2$, therefore $a_1$ exists if and only if $a_2$ exists. Furthermore since $x_n \to a_1 \in X_1$, $a_1$ is unique in $X_1$. Without loss of generality, the same argument holds for $a_2 \in X_2$. Therefore the mapping $\psi$ is a bijection. It is immediate that $\psi \circ \phi_1 = \phi_2$ since what was meant by $x_n \to a_1, a_2$ in the previous arguments was that $\phi_1(x_n) \to a_1$ and visa versa.
\end{proof}

\medskip \noindent {\bf (2.2)}\ 
(a) Show that the function $\rho^*$ defined in the proof of Theorem~1.17 of the
course lecture notes is a metric on $X^*$. 
\begin{proof}
	Define $\rho^*$ as follows,
	\begin{equation*}
		\rho^*([x], [y]) = \lim \rho(x_n, y_n).
	\end{equation*}
	We claim that $\rho^*$ is a metric. Clearly $\rho^*: X \to [0, \infty]$ since for any two $x_n, y_n$, $\rho(x_n,y_n) \geq 0$ for all $n$. Now we show that $[x] = [y]$ if and only if $\rho^*([x], [y]) = 0.$ By definition $[x] = [y]$ if and only if for all $(x_n), (y_n) \in [x], [y]$ 
	\begin{equation*}
		0 = \lim \rho(x_n, y_n)  = \rho^*([x],[y]).
	\end{equation*}
	So $\rho^*$ is positive definite. Next, symmetry follows directly from symmetry of $\rho;$ that is, $\rho(x_n, y_n) = \rho(y_n, x_n).$  Finally we need to show the triangle equality.

	We claim that $\rho^*([x],[z]) \leq \rho^*([x], [y]) \leq \rho^*([y], [z]).$ Taking the definition, we observe that for every $n$
	\begin{equation*}
		\rho(x_n, z_n) \leq \rho(x_n, y_n) + \rho(y_n, z_n).
	\end{equation*}
	Since  non-strict inequalities hold in the limit we get
	\begin{equation*}
		\begin{aligned}
			\lim_{n\to\infty}\rho(x_n, z_n) &\leq \lim_{n\to\infty}\rho(x_n, y_n) + \lim_{n\to\infty}\rho(y_n, z_n).\\
			\rho^*(x_n, z_n) &\leq \rho^*(x_n, y_n) + \rho^*(y_n, z_n). 
		\end{aligned}
	\end{equation*}
	This completes the proof, $\rho^*$ is a metric.
\end{proof}

 (b) Show that $\phi$ is distance-preserving,
that is, $\rho^*(\phi(x),\phi(y))=\rho(x,y)$ for all $x,y\in X$.
\begin{proof}
	Recall that $\phi: X \to X^*$ such that $x \mapsto [(x,x,x,\dots)]$. Take any $x, y \in X$. Then the constant sequence $(x_n) =(x, \dots)$ and $ (y_n) = (y, \dots)$ have limits, $x, y$ respectively. Observe that $\rho(x_n, y_n) = \rho(x,y)$ for every $n$. Therefore $\lim \rho(x_n, y_n) = \rho(x,y)$ and so $\rho^{*}(\phi(x), \phi(y)) = \rho(x,y);$ that is, $\phi$ is an isometry.
\end{proof}

\medskip \noindent {\bf (2.3)}\ 
(Folland problem 1.4)\
An algebra $\scripta$ is a $\sigma$-algebra if and only if $\scripta$
is closed under countable {\em ascending} unions.
\begin{proof}
	If $\scripta$ is a $\sigma$-algebra, then it is closed under countable unions. A union $\bigcup F_n$ is ascending if for $n > m$ $F_n \supset F_m$ with $F_j \in \scripta$ for all $j.$ 
	Clearly $\bigcup F_n \in \scripta$ since it is a $\sigma$-algebra.

	If $\scripta$ is an algebra closed under countable ascending unions, we claim it is a $\sigma$-algebra. Take any countable subfamily of $\scripta$, say $(F_n)$. Then define $E_1 \subset E_2 \subset \cdots$ such that $E_1 = F_1, E_2 = E_1 \cup F_2, \dots, E_n = E_{n-1} \cup F_n$. For every $n$ $E_n \in \scripta$ and $\bigcup E_n$ is an ascending countable union containing every element of $F_n$ so it is in $\scripta$ by the hypothesis.  Furthermore $\bigcup E_n = \bigcup F_n.$ Therefore for every countable family $(F_n) = f:\mathbb{N} \to \scripta$ the countable union $\bigcup F_n \in \scripta$ and $\scripta$ is thereby a $\sigma$-algebra.
\end{proof}

\medskip \noindent {\bf (2.4)}\ 
(Folland problem 1.5)\
Let $X$ be a set and let $\scripte\subset\scriptp(X)$.
Show that $\scriptm(\scripte) = \bigcup_\scriptf \scriptm(\scriptf)$,
where the union is taken over all countable subsets $\scriptf$ of $\scripte$.
(Recall that $\scriptm(\scripte)$ denotes the smallest $\sigma$--algebra
of subsets of $X$ that contains $\scripte$.)
\qed

(A hint is given in our text. This problem is interesting because it illustrates
a general method for establishing properties of the $\sigma$--algebra generated
by a collection of sets: Prove instead
that the collection of sets possessing some desired property is a $\sigma$--algebra.)
\begin{proof}
	To prove the fact, we need only show that $\bigcup_\scriptf \scriptm(\scriptf) \supset \scriptm(\scripte)$ since by Lemma 1.1 $\scriptm(\scriptf) \subset \scriptm(\scripte).$ We need show that $\bigcup_\scriptf \scriptm(\scriptf)$ contains $\scripte$ and is a $\sigma$-algebra, and it will follow immediately that $\bigcup_\scriptf \scriptm(\scriptf) \supset \scriptm(\scripte),$ by definition of $\scriptm(\scripte).$

	First define $F_X = \{X\}$ for all $X \in \scripte.$ Clearly $F_X$ are countable subfamilies of $\scripte$ sinc they contain only one element, then $\bigcup_\scriptf \scriptm(\scriptf) \supset \bigcup_{X \in \scripte} F_X \supset \scripte$. 

	Now take $E_1, E_2 \dots  \in \bigcup_\scriptf \scriptm(\scriptf).$ We claim that $\bigcup_n E_n \in \bigcup_\scriptf \scriptm(\scriptf).$ Since $E_j$ from some $\scriptm(\scriptf_j)$. Take $I_j$ to be the minimal set of $j$ such that there are $E_k$ from $  \scriptm(\scriptf_j).$ Also, let $J$ be an index set so that 
	\begin{equation*}
		\bigcup_{j \in J} \bigcup_{i \in I_j} E_j = \bigcup_{n=1}^\infty E_n.
	\end{equation*}
	Then for every $j \in J,$  $\bigcup_{i \in I_j}E_i := \mathfrak{E}_j \in \scriptm(\scriptf_j)$. In the worst case $J$ is countable, and $\scriptf_J = \{\mathfrak{E}_j\ :\ j \in J\} \subset \scripte $ since each $\mathfrak{E} \in \scriptf_J$ is generated by a countable union of elements of $\scriptm(\scriptf_j) \subset \scriptm(\scripte)$ for all $j \in J$. Therefore $\scriptf_J \subset \scripte$ and so $\scriptm(\scriptf) \subset \bigcup_\scriptf \scriptm(\scriptf).$ 

	 If $E \in \bigcup_\scriptf \scriptm(\scriptf)$ then $E$ must be contained in some $\scriptm(\scriptf)$ so its compliment is also therein contained and $\scriptm(\scriptf)$ so its compliment is in the union. 

	 Therefore $\bigcup_\scriptf \scriptm(\scriptf)$ is a $\sigma$-algebra contained in $\scriptm(\scripte)$, but since $\scriptm(\scripte)$ is the smallest such $\sigma$-algebra which contains $\scripte$ then $\scriptm(\scripte) \supset \bigcup_\scriptf \scriptm(\scriptf) \supset \scriptm(\scripte)$ implies $\scriptm(\scripte) = \bigcup_\scriptf \scriptm(\scriptf).$ This completes the proof.
 \end{proof}

\medskip \noindent {\bf (2.5)}\ 
(Folland problem 1.6)\
Complete the proof of Theorem~1.9 of our text.
\begin{proof}
	We will show that $\overline{\mu}$ is a complete measure and uniquely extends $\mu.$ First observe that for any $E \cup F \in \overline{\scriptm}$ that $\overline{\mu}(E)\leq \overline{\mu}(E\cup F) \leq \overline{\mu}(E) + \overline{\mu}(F) = \overline{\mu}(E)$. So $\overline{\mu}(E \cup F) = \overline{\mu}(E)$ where $E$ is any set in $\scriptm$ and $F$ is a subset of a $\mu$-null set. So take $E$ to be $\emptyset$ and then for every $F \subset N$ for all $N$ $\overline{\mu}(F) = 0.$ Therefore $\overline{\mu}$ is complete.

	Now suppose that $\overline{\mu}$ is not unique. Call two extensions $\overline{\mu}_1$ and $\overline{\mu}_2$. Then $\overline{\mu}_1$ and $\overline{\mu}_2$ can only differ on $F \subset N$ where $\mu$ undefined. However these measures must be complete so for any $F \subset N$, $\overline{\mu}_1(F) = 0 = \overline{\mu}_2(F)$ which is a contradiction. Therefore the extension is unique.
\end{proof}

\medskip \noindent {\bf (2.6)}\ 
(Folland problem 1.10)\
Let $(X,\scriptm,\mu)$ be a measure space. Let $E\in\scriptm$.
Define $\nu(A)=\mu(A\cap E)$ for all $A\in\scriptm$.
Prove that $\nu$ is a measure on $\scriptm$.
\begin{proof}
	We show countable additivity and that the emptyset is a $\nu$-nullset. First $\nu(\emptyset) = \mu(\emptyset \cap E) = \mu(\emptyset) = 0$ by $\mu$ a measure. Now consider a set of disjoint sets in $\scriptm$, say $Q_1, Q_2, \dots$. Then
	\begin{equation*}
	\begin{aligned}
		\nu\left(\bigcup_{n=1}^\infty Q_n\right) =\mu\left(\bigcup_{n=1}^\infty Q_n \cap E\right) = \sum_{n=1}^\infty \mu\left(Q_n \cap E\right)  = \sum_{n=1}^\infty \nu(Q_n).
	\end{aligned}
	\end{equation*}
	So $\nu$ is countably additive and $\nu$ is a measure.
\end{proof}

\medskip \noindent {\bf (2.7)}\ 
(Folland problem 1.11)\
See text for the problem statement.
(This concerns the connection between finitely additive measures, and genuine measures.
It helps to emphasize that countable additivity is connected with limiting operations.)
\begin{proof}
	If $\mu$ is a finite addative measure and is continuous from below, then take any family of disjoint sets $(E_j)$ in $\scriptm$. Then define $Q_1 = E_1,$ $Q_2 = Q_1 \cup E_2$, $\dots$ to be another subfamily of $\scriptm$. It follows that 
	\begin{equation*}
		\mu\left(\bigcup_{n=1}^\infty Q_j\right) = \lim_{n\to\infty} \sum_{j=1}^n \mu(E_j) = \sum_{h=1}^\infty \mu(E_j)
	\end{equation*}
	Therefore $\mu$ is a countably addative measure on $\scriptm$.

	If $ \mu(X) < \infty$ and $\mu$ is downward continuous. Then $\mu(\emptyset) = 0$ since one could easily define a monotone nonincreasing family of sets which terminates finiteley at towards $\emptyset$ for which the measure of each set in the family is less than half of the previous set. Furthermore if we assume that $\mu$ is a finite addative measure then take a sequence of disjoint $E_1, \dots E_n, \dots $ and define $Q_n = X \setminus \bigcup_{j=1}^n E_n$, then 
	
	\begin{equation*}
			\mu\left(X \setminus \bigcup_{n=1}^\infty Q_j\right) = \mu\left(\bigcap_{n=1}^\infty X \setminus Q_j\right) = \lim_{n\to\infty}  \mu\left(X \setminus \left(X \setminus \bigcup_{j=1}^nE_j\right)\right) = \sum_{j=1}^\infty \mu(E_j).
	\end{equation*}
	So $\mu$ is a measure. This completes the proof.
\end{proof}

\medskip \noindent {\bf (2.8)}\ 
(Folland problem 1.12)\
Let $(X,\scriptm,\mu)$ be a measure space. Recall that the symmetric difference of two sets
is $E\symdif F = (E\cup F)\setminus (E\cap F)$.

(a) Show that if $E,F\in\scriptm$ satisfy $\mu(E\symdif F)=0$
then $\mu(E)=\mu(F)$.
\begin{proof}
	If $\mu(E \symdif F) = 0$ then the measure of the disjoint union $C_E \sqcup C_F$ is $0$ where $C_E$ and $C_F$ are the contribution of $E$ and $F$ to $E \symdif F$ respectively. Since the union is disjoint it follows that $\mu(C_E \sqcup C_F) = \mu(C_E) + \mu(C_F) = 0$ implies that both sets are $\mu$-null sets.
\end{proof}

(b) Define $E\sim F$ if $\mu(E\symdif F)=0$.
Show that $\sim$ is an equivalence relation on $\scriptm$.
\begin{proof}
	First, $E \symdif F = F \symdif E$ implies that if $\mu(E \symdif F) = 0$ then $\mu(F \symdif E) = 0$ and $\sim$ is symmetric. Furthermore $E \symdif E = \emptyset$ and $\mu(E \symdif E) = 0 $ implies $ E \sim E$, so $\sim$ is reflexive. Lastly suppose $E \sim F$ and $F \sim G$  and not $E \sim G$. Then there is some set, say $J$, in $E \symdif G$ with positive measure. The set $J$ cannot be contained in $E \cap F \cap G$ so it must be partially in $E \symdif F$ or $F \symdif G$. Either $\mu(J \cap (E \symdif F)) > 0$ or $\mu(J \cap (F \symdif G)) > 0$. In either case we contradict $E \sim F$ or $F\sim G$, therefore it must be that $J$ does not exist and $E \sim G$.
\end{proof}

(c) Suppose that $\mu(X)<\infty$. Define $\rho(E,F) = \mu(E\symdif F)$ for $E,F\in\scriptm$.
Show that $\rho$ defines a metric on the set $\scriptm/\sim$ of equivalence
classes of elements of $\scriptm$.
\begin{proof}
	Clearly $E \sim F$ if and only if $\mu(E \symdif F) = 0$. and $\mu: \scriptm \to [0, \infty]$ so $\rho(E,F) = \mu(E \symdif F)$ is  positive definite. Next $\mu(E \symdif F) = \mu(F \symdif E)$ so $\rho$ is symmetric. Lastly, 
	\begin{equation*}
		\rho(E,F) = \mu(E \symdif F) \leq \mu(E \symdif G) + \mu(G\symdif F) = \rho(E,G) + \rho(G,F).
	\end{equation*}
	This is easy to see since $G = (G \setminus (F \cap E)) \cup F \cap E \cap G.$ so if $F \cap G$ or $G \cap E$ has positive measure then $\mu(E \symdif G) + \mu(G\symdif F)  > \mu(E \cup F \setminus(E \cap F)).$ Therefore $\mu$ is a measure.
\end{proof}

\medskip \noindent {\bf (2.9)}\ 
(Folland problem 1.14)\
Let $\mu$ be a semifinite measure on $\scriptm$.
Show that if $E\in\scriptm$ and $\mu(E)=\infty$
then $\sup\{\mu(F): F\subset E \text{ and } \mu(F)<\infty \}=\infty$.
(The supremum is taken over subsets of $E$ that belong to $\scriptm$.)
\begin{proof}
	If $\mu$ is semifinite then there is a set $F_1 \subset E$ so that $\mu(F_1)$ is finite. Then consider the set $E \setminus F_1 = Q_1$ This set has $\mu(Q_1) = \infty$, so there is a set $F_2 \subset Q_1$ with $\mu(F_2) < \infty$.  Let $Q_2 = Q_1 \setminus F_2$. Repeat this process and yield the families $(Q_n), (F_n)$. We claim that for any $C > 0$ there is an $N$ so that $\mu\left(\bigcup_n^N F_n\right) > C. $  

	Since $\bigcup_j^m F_j \subset \bigcup_j^n F_j$ if $n > m$ strictly, then measure continuity implies that $\bigcup_n^\infty \bigcup_j^n F_j = E$ has measure equal to the limit of the unions; that is, $\mu\left(\bigcup_j^NF_j\right) \to \infty$ So there is a collection of sets of finite measure whose measure tends towards infinity.
\end{proof}

\medskip \noindent {\bf (2.10)}\ 
Denote by $\ell(I)$ the length of an interval $I\subset\reals$.
Let $a,b\in\reals$ satisfy $a\le b$.
Prove
that if $I_j$ are open intervals, and if $[a,b]\subset\bigcup_{j=1}^N I_j$,
then $b-a\le \sum_{j=1}^N \ell(I_j)$.
Do not invoke any results from Chapter~1.
\begin{proof}
	For each $I_j$ denote $r_j, l_j$ to be the left and right ends of the interval. Then define $x_1 = l_1$ and $x_N = l_N$. Then define $x_j = r_j$. It follows that the family of intervals $([x_j, x_{j+1}])$ have sum length less than $(I_j)$ since if there are any $j$ so that $l_j < r_{j-1}$ then $\ell(I_{j}) + \ell(I_{j-1}) = r_j - l_j + r_{j-1} -  l_{j-1} \geq \ell([x_{j-2}, x_j-1]) + \ell([x_{j-1}, x_j]).$ Finally without loss of generality order $I_j$ so that $l_1 \leq a, l_N \geq b$. This gives $\bigcup [x_{j-1}, x_{j}] = [x_1, x_N] \supset I$ and $x_N - x_1 \geq b-a$. Therefore $b-a \leq \sum_{j}^N \ell(I_j)$.
\end{proof}

\end{document}\end
