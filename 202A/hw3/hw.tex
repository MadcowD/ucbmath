\documentclass[11pt]{amsart}

\usepackage{amsmath,amsthm}
\usepackage{amssymb}
\usepackage{graphicx}
\usepackage{enumerate}
\usepackage{fullpage}
% \usepackage{euscript}
% \makeatletter
% \nopagenumbers
\usepackage{verbatim}
\usepackage{color}
\usepackage{hyperref}
%\usepackage{times} %, mathtime}

\textheight=600pt %574pt
\textwidth=480pt %432pt
\oddsidemargin=15pt %18.88pt
\evensidemargin=18.88pt
\topmargin=10pt %14.21pt

\parskip=1pt %2pt

% define theorem environments
\newtheorem{theorem}{Theorem}    %[section]
%\def\thetheorem{\unskip}
\newtheorem{proposition}[theorem]{Proposition}
%\def\theproposition{\unskip}
\newtheorem{conjecture}[theorem]{Conjecture}
\def\theconjecture{\unskip}
\newtheorem{corollary}[theorem]{Corollary}
\newtheorem{lemma}[theorem]{Lemma}
\newtheorem{sublemma}[theorem]{Sublemma}
\newtheorem{fact}[theorem]{Fact}
\newtheorem{observation}[theorem]{Observation}
%\def\thelemma{\unskip}
\theoremstyle{definition}
\newtheorem{definition}{Definition}
%\def\thedefinition{\unskip}
\newtheorem{notation}[definition]{Notation}
\newtheorem{remark}[definition]{Remark}
% \def\theremark{\unskip}
\newtheorem{question}[definition]{Question}
\newtheorem{questions}[definition]{Questions}
%\def\thequestion{\unskip}
\newtheorem{example}[definition]{Example}
%\def\theexample{\unskip}
\newtheorem{problem}[definition]{Problem}
\newtheorem{exercise}[definition]{Exercise}

\numberwithin{theorem}{section}
\numberwithin{definition}{section}
\numberwithin{equation}{section}

\def\reals{{\mathbb R}}
\def\torus{{\mathbb T}}
\def\integers{{\mathbb Z}}
\def\rationals{{\mathbb Q}}
\def\naturals{{\mathbb N}}
\def\complex{{\mathbb C}\/}
\def\distance{\operatorname{distance}\,}
\def\support{\operatorname{support}\,}
\def\dist{\operatorname{dist}\,}
\def\Span{\operatorname{span}\,}
\def\degree{\operatorname{degree}\,}
\def\kernel{\operatorname{kernel}\,}
\def\dim{\operatorname{dim}\,}
\def\codim{\operatorname{codim}}
\def\trace{\operatorname{trace\,}}
\def\dimension{\operatorname{dimension}\,}
\def\codimension{\operatorname{codimension}\,}
\def\nullspace{\scriptk}
\def\kernel{\operatorname{Ker}}
\def\p{\partial}
\def\Re{\operatorname{Re\,} }
\def\Im{\operatorname{Im\,} }
\def\ov{\overline}
\def\eps{\varepsilon}
\def\lt{L^2}
\def\curl{\operatorname{curl}}
\def\divergence{\operatorname{div}}
\newcommand{\norm}[1]{ \|  #1 \|}
\def\expect{\mathbb E}
\def\bull{$\bullet$\ }
\def\det{\operatorname{det}}
\def\Det{\operatorname{Det}}
\def\rank{\mathbf r}
\def\diameter{\operatorname{diameter}}

\def\t2{\tfrac12}

\newcommand{\abr}[1]{ \langle  #1 \rangle}

\def\newbull{\medskip\noindent $\bullet$\ }
\def\field{{\mathbb F}}
\def\cc{C_c}



% \renewcommand\forall{\ \forall\,}

% \newcommand{\Norm}[1]{ \left\|  #1 \right\| }
\newcommand{\Norm}[1]{ \Big\|  #1 \Big\| }
\newcommand{\set}[1]{ \left\{ #1 \right\} }
%\newcommand{\ifof}{\Leftrightarrow}
\def\one{{\mathbf 1}}
\newcommand{\modulo}[2]{[#1]_{#2}}

\def\bd{\operatorname{bd}\,}
\def\cl{\text{cl}}
\def\nobull{\noindent$\bullet$\ }

\def\scriptf{{\mathcal F}}
\def\scriptq{{\mathcal Q}}
\def\scriptg{{\mathcal G}}
\def\scriptm{{\mathcal M}}
\def\scriptb{{\mathcal B}}
\def\scriptc{{\mathcal C}}
\def\scriptt{{\mathcal T}}
\def\scripti{{\mathcal I}}
\def\scripte{{\mathcal E}}
\def\scriptv{{\mathcal V}}
\def\scriptw{{\mathcal W}}
\def\scriptu{{\mathcal U}}
\def\scriptS{{\mathcal S}}
\def\scripta{{\mathcal A}}
\def\scriptr{{\mathcal R}}
\def\scripto{{\mathcal O}}
\def\scripth{{\mathcal H}}
\def\scriptd{{\mathcal D}}
\def\scriptl{{\mathcal L}}
\def\scriptn{{\mathcal N}}
\def\scriptp{{\mathcal P}}
\def\scriptk{{\mathcal K}}
\def\scriptP{{\mathcal P}}
\def\scriptj{{\mathcal J}}
\def\scriptz{{\mathcal Z}}
\def\scripts{{\mathcal S}}
\def\scriptx{{\mathcal X}}
\def\scripty{{\mathcal Y}}
\def\frakv{{\mathfrak V}}
\def\frakG{{\mathfrak G}}
\def\aff{\operatorname{Aff}}
\def\frakB{{\mathfrak B}}
\def\frakC{{\mathfrak C}}

\def\symdif{\,\Delta\,}
\def\mustar{\mu^*}
\def\muplus{\mu^+}

\def\soln{\noindent {\bf Solution.}\ }


%\pagestyle{empty}
%\setlength{\parindent}{0pt}

\begin{document}

\begin{center}{\bf Math 202A--- UCB, Fall 2016 --- M.~Christ}
\\
{\bf Problem Set 3, due Wednesday September 14\footnote{\copyright{Michael Christ, August 2016}}}
\end{center}

Please read section 1.5 of our text. 

\medskip \noindent {\bf (3.1)}\ 
(Folland problem 1.17)\ 
Let $\mustar$ be an outer measure on $X$ and let $A_j$ be pairwise disjoint
$\mustar$--measurable sets. Show that for any $E\subset X$,
$\mustar(E\cap(\cup_{j=1}^\infty A_j)) = \sum_{j=1}^\infty \mustar(E\cap A_j)$.
\begin{proof}
	First observe that $E \cap \bigcup A_j = \bigcup (A_j \cap E)$, and the sets $A_j \cap E$ are clearly a pairwise disjoint cut of $E$. Let $B_j = A_j \cap E$. Then now consider the finite union of $B_j$. We first show finite addativity by induction. Clearly $\mu^*(B_j) = \mu^*(B_j)$ so the base case holdes. Now suppose that $\mu^*(\bigcup_{j=1}^{n-1}B_j) = \sum^{n-1}_{j=1} \mu^*(B_j)$. Then it follows that
	$$ \mu^*\left(\bigcup_{j=1}^n B_j\right) =\mu^*\left(\bigcup_{j=1}^n B_j\cap A_j\right) + \mu^*\left(\bigcup_{j=1}^n B_j \cap X \setminus A_j\right)$$
	and by the disjointness of $A_j$ we have \begin{equation*}
		\mu^*\left(\bigcup_{j=1}^n B_j\right) = \mu^*(B_n) + \mu^*\left(\bigcup_{j=1}^n B_j\right) = \sum^{n}_{j=1} \mu^*(B_j)
	\end{equation*}
	for every n. This holds in the limit and $\mustar(E\cap(\cup_{j=1}^\infty A_j)) = \sum_{j=1}^\infty \mustar(E\cap A_j)$.
\end{proof}

\medskip \noindent {\bf (3.2)}\ 
(Folland problem 1.18)\ 
Let $\mu_0$ be a premeasure on an algebra $\scripta\subset\scriptp(X)$.
Let $\mustar$ be the outer measure induced\footnote{This means that
$\mustar$ is defined in terms of $\mu_0$ by (1.12) of our text.} by $\mu_0$.
Let $\scripta_\sigma$ be the collection of all countable unions of elements
of $\scripta$, and let $\scripta_{\sigma\delta}$ be the collection of
all countable intersections of elements of $\scripta_\sigma$.
Prove the following:
\newline {\bf (a)}\
For any $E\subset X$ and $\eps>0$ there exists $A\in\scripta_\sigma$
such that $E\subset A$ and $\mustar(A)\le\mustar(E)+\eps$.
\begin{proof}
	If $\scripta$ is an algebra 
\end{proof}
{\bf (b)}\
If $\mustar(E)<\infty$ then $E$ is $\mustar$--measurable if and only if
there exists $B\in\scripta_{\sigma\delta}$ such that $\mustar(B\setminus E)=0$.
\newline {\bf (c)}\
If $\mu_0$ is $\sigma$--finite then the hypothesis $\mustar(E)<\infty$
in part (b) is superfluous.
\qed

\medskip \noindent {\bf (3.3)}\ 
(Folland problem 1.22(a))\ 
Let $(X,\scriptm,\mu)$ be a measure space, let $\mustar$ be the outer measure induced
by $\mu$, let $\scriptm^*$ be the $\sigma$--algebra of measurable sets, 
and let $\bar\mu=\mustar|_{\scriptm^*}$ be the associated measure.
Assume that $\mu$ is $\sigma$--finite. Show that $\bar\mu$ is the completion\footnote{This means
that $\bar\mu$ coincides with the measure constructed from $\mu$ in Theorem~1.9 of our text.} of $\mu$. 
\qed

\medskip \noindent {\bf (3.4)}\ 
(Folland problem 1.23)\ 
Let $\scripta$ be the collection of all finite unions of sets $(a,b]\cap\rationals$
with $-\infty\le a<b\le\infty$. Prove the following:
\newline {\bf (a)}\
$\scripta$ is an algebra of subsets of $\rationals$.
\newline {\bf (b)}\
The $\sigma$--algebra generated by $\scripta$ is equal to $\scriptp(\rationals)$.
\newline {\bf (c)}\
The set function defined on $\scripta$ by $\mu_0(\emptyset)=0$
and $\mu_0(A)=\infty$ for all nonempty $A\in\scripta$ is a premeasure on $\scripta$.
There exists more than one measure on $\scriptp(\rationals)$ whose restriction to
$\scripta$ is equal to $\mu_0$.
\qed

\medskip \noindent {\bf (3.5)}\ 
Complete the proof of Theorem~1.19 of our text. With the notation of that theorem, show that:
(a) $E\in\scriptm_\mu$. (b) $E = A\setminus N_1$ where $A$ is a $G_\delta$
set and $N_1$ is a $\mu$--null set.  (c) $E = B\cup N_2$ where $B$ is an $F_\sigma$
set and $N_2$ is a $\mu$--null set.
\qed

\medskip \noindent {\bf (3.6)}\ 
Prove Proposition~1.20 of our text:
If $E\in\scriptm_\mu$ and $\mu(E)<\infty$ then for every $\eps>0$
there exist finitely many disjoint open intervals $I_j$ such that
$\mu(E\symdif \cup_j I_j)<\eps$.
\qed

\medskip \noindent {\bf (3.7)}\ 
(Folland problem 1.29)\ 
Let $E\subset\reals$ be a Lebesgue measurable set. 
Let $N$ be the nonmeasurable set constructed in \S1.1 of our text and in
lecture (where it was denoted by $\scripte$). 
(a) Show that $m(E)=0$ if $E\subset N$.
(b) Show that if $E\in\scriptl$ and $\mu(E)>0$ then $E$ contains a nonmeasurable set.
(In part (b) it is not assumed that $E\subset N$.)
\begin{proof}
If $E \subset N$ is a measurable set then $\mu^*(E \cap N) + \mu^*(E^c \cap N) = \mu^*(N) = \mu(E) + \mu^*(N \setminus E).$ If $\mu^*(N) = 0$ the proof is complete, but this cannot be the case since $\mu^*(N)$ gives $N$ measurabe; that is for any set $\mu^*(X \cap N) + \mu^*(X^c \cap N) \leq 2\mu^*(N) \leq 0$, therefore $\mu^*(N) > 0$. If $\mu(E) > 0$, then $\mu^*(N) - \mu^*(N \setminus E) = \mu^*(E)$ which gives $N$ measurable. Contradiction. 
\end{proof}

\medskip \noindent {\bf (3.8)}\ 
(Folland problem 1.30)\ 
Let $E\in\scriptl$ and $0<\alpha<1$. Show that if $m(E)>0$
then there exists an open interval $I$ satisfying $m(E\cap I) > \alpha m(I)$.
Since $E$ is Labesgue measurable, there is an openset such that $E$ is strictly contained in that openset say $U$. Furthermore, since $m(\mathbb{R}) = \infty$ then by a previous exercise/theorem we can find opensets which are larger than $U$ (contain it) and have larger measure than $U$. Now take any $\alpha \in (0,1)$ since $m(E)$ is the infimum of the measures of open $U$ containing it, we can take $U$ small enough that $m(E \cap U) = m(E)$ and $m(U) - m(E) < \epsilon.$ In particular  there are $U$ such that $m(E)/m(U) = 1- \epsilon$, so take $\epsilon \in (0, 1)$ and we have $\alpha$ so that $m(E)\alpha = m(E \cap U)\alpha = m(U)$. 

\medskip \noindent {\bf (3.9)}\ 
(Folland problem 1.33)\ 
Show that there exists a Borel set $A\subset[0,1]$
such that for every subinterval $I\subset[0,1]$
of positive length, $0<m(A\cap I)<m(I)$.
\begin{proof}
	Consider the following construction, using knowledge from Pugh. Denote $E_1 = [0,1]$. Then let $E_2 = E_1 \setminus (1/3, 2/3)$. Then remove the middle open set of length $1/9$ from each connected segment of $E_2$. Then remove the middle open set of length $1/3^n$ from each segment of $E_{n-1}$. The limiting process is a countable intersection process so thhe set $E^{n\to\infty} = F_C$ is Borel measurable. Now take any interval $I \subset [0,1]$ and intersect $F_C$. We claim that there are no intervals in $F_C$ and so $F_C$ mudst be strictly smaller than $I$. Suppose that there were an $I$ in $F_C$. This means that for every $n$, $I$ must lie in some segment of $E_n$. If it did not for every $n$ then the symmetric difference of $E_n$ and $I$ would be an interval of finite length. Now observe that as $n\to \infty$ the  difference endpoints of segments in $E_n \to 0$, so $I$ must be a point, a contradiction to $I$ an interval (in this problem we are sure that intervals are not points). Therefore $m(F_C \cap I) \leq m(I)$.

	Now we must modify our construction so that $m(F_C \cap I) > 0$ for every $I$. For every $n$, construct an $F_C^n$ such that every gap $I \setminus E_n$ which has not already been filled in a previous iteration is filled by a set $F_c$ by the same construction as before except where the initial set $E_1^n = [c,g]$ where $c,g$ are the end points of a gap in this construction. This looks like a fractal. Repeat this process forever, and since it is countable we still have a Boral set, $A$.

	Again take any interval in $[a,b]$ we claim that the intersection has area $p(b-a)$. First recall that the $F_C$, fat cantor set has area $m([0,1]) - 1/3 - 2/27 - 4/27*27 -\cdots -2^n/3^{2n+2}$ which gives the set a postive proportion of $[0,1]$ say $p$. Then for any $F_C^n$ we can scale the original set to fit in that interval giving us (by the same argument) a set of area $p(c-g)$ Then any interval $[a,b] \subset [0,1]$ intersects a number of these $F_C^n$ whose area could never be $[a,b]$ since $0 < p < 1$.

	This completes the proof
\end{proof}


\end{document}\end

