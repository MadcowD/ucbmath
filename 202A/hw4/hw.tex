\documentclass[11pt]{amsart}

\usepackage{amsmath,amsthm}
\usepackage{amssymb}
\usepackage{graphicx}
\usepackage{enumerate}
\usepackage{fullpage}
% \usepackage{euscript}
% \makeatletter
% \nopagenumbers
\usepackage{verbatim}
\usepackage{color}
\usepackage{hyperref}
\usepackage{tikz-cd}
%\usepackage{times} %, mathtime}

\textheight=600pt %574pt
\textwidth=480pt %432pt
\oddsidemargin=15pt %18.88pt
\evensidemargin=18.88pt
\topmargin=10pt %14.21pt

\parskip=1pt %2pt

% define theorem environments
\newtheorem{theorem}{Theorem}    %[section]
%\def\thetheorem{\unskip}
\newtheorem{proposition}[theorem]{Proposition}
%\def\theproposition{\unskip}
\newtheorem{conjecture}[theorem]{Conjecture}
\def\theconjecture{\unskip}
\newtheorem{corollary}[theorem]{Corollary}
\newtheorem{lemma}[theorem]{Lemma}
\newtheorem{sublemma}[theorem]{Sublemma}
\newtheorem{fact}[theorem]{Fact}
\newtheorem{observation}[theorem]{Observation}
%\def\thelemma{\unskip}
\theoremstyle{definition}
\newtheorem{definition}{Definition}
%\def\thedefinition{\unskip}
\newtheorem{notation}[definition]{Notation}
\newtheorem{remark}[definition]{Remark}
% \def\theremark{\unskip}
\newtheorem{question}[definition]{Question}
\newtheorem{questions}[definition]{Questions}
%\def\thequestion{\unskip}
\newtheorem{example}[definition]{Example}
%\def\theexample{\unskip}
\newtheorem{problem}[definition]{Problem}
\newtheorem{exercise}[definition]{Exercise}

\numberwithin{theorem}{section}
\numberwithin{definition}{section}
\numberwithin{equation}{section}

\def\reals{{\mathbb R}}
\def\torus{{\mathbb T}}
\def\integers{{\mathbb Z}}
\def\rationals{{\mathbb Q}}
\def\naturals{{\mathbb N}}
\def\complex{{\mathbb C}\/}
\def\distance{\operatorname{distance}\,}
\def\support{\operatorname{support}\,}
\def\dist{\operatorname{dist}\,}
\def\Span{\operatorname{span}\,}
\def\degree{\operatorname{degree}\,}
\def\kernel{\operatorname{kernel}\,}
\def\dim{\operatorname{dim}\,}
\def\codim{\operatorname{codim}}
\def\trace{\operatorname{trace\,}}
\def\dimension{\operatorname{dimension}\,}
\def\codimension{\operatorname{codimension}\,}
\def\nullspace{\scriptk}
\def\kernel{\operatorname{Ker}}
\def\p{\partial}
\def\Re{\operatorname{Re\,} }
\def\Im{\operatorname{Im\,} }
\def\ov{\overline}
\def\eps{\varepsilon}
\def\lt{L^2}
\def\curl{\operatorname{curl}}
\def\divergence{\operatorname{div}}
\newcommand{\norm}[1]{ \|  #1 \|}
\def\expect{\mathbb E}
\def\bull{$\bullet$\ }
\def\det{\operatorname{det}}
\def\Det{\operatorname{Det}}
\def\rank{\mathbf r}
\def\diameter{\operatorname{diameter}}

\def\t2{\tfrac12}

\newcommand{\abr}[1]{ \langle  #1 \rangle}

\def\newbull{\medskip\noindent $\bullet$\ }
\def\field{{\mathbb F}}
\def\cc{C_c}



% \renewcommand\forall{\ \forall\,}

% \newcommand{\Norm}[1]{ \left\|  #1 \right\| }
\newcommand{\Norm}[1]{ \Big\|  #1 \Big\| }
\newcommand{\set}[1]{ \left\{ #1 \right\} }
%\newcommand{\ifof}{\Leftrightarrow}
\def\one{{\mathbf 1}}
\newcommand{\modulo}[2]{[#1]_{#2}}

\def\bd{\operatorname{bd}\,}
\def\cl{\text{cl}}
\def\nobull{\noindent$\bullet$\ }

\def\scriptf{{\mathcal F}}
\def\scriptq{{\mathcal Q}}
\def\scriptg{{\mathcal G}}
\def\scriptm{{\mathcal M}}
\def\scriptb{{\mathcal B}}
\def\scriptc{{\mathcal C}}
\def\scriptt{{\mathcal T}}
\def\scripti{{\mathcal I}}
\def\scripte{{\mathcal E}}
\def\scriptv{{\mathcal V}}
\def\scriptw{{\mathcal W}}
\def\scriptu{{\mathcal U}}
\def\scriptS{{\mathcal S}}
\def\scripta{{\mathcal A}}
\def\scriptr{{\mathcal R}}
\def\scripto{{\mathcal O}}
\def\scripth{{\mathcal H}}
\def\scriptd{{\mathcal D}}
\def\scriptl{{\mathcal L}}
\def\scriptn{{\mathcal N}}
\def\scriptp{{\mathcal P}}
\def\scriptk{{\mathcal K}}
\def\scriptP{{\mathcal P}}
\def\scriptj{{\mathcal J}}
\def\scriptz{{\mathcal Z}}
\def\scripts{{\mathcal S}}
\def\scriptx{{\mathcal X}}
\def\scripty{{\mathcal Y}}
\def\frakv{{\mathfrak V}}
\def\frakG{{\mathfrak G}}
\def\aff{\operatorname{Aff}}
\def\frakB{{\mathfrak B}}
\def\frakC{{\mathfrak C}}

\def\symdif{\,\Delta\,}
\def\mustar{\mu^*}
\def\muplus{\mu^+}

\def\soln{\noindent {\bf Solution.}\ }


%\pagestyle{empty}
%\setlength{\parindent}{0pt}

\begin{document}

\begin{center}{\bf Math 202A--- UCB, Fall 2016 --- M.~Christ}
\\
{\bf Problem Set 4, due Wednesday September 14 - William Guss}
\end{center}
\medskip \noindent {\bf (4.1)}\ Let $f_n: X \to \mathbb{R}^*$ be measurable. \\
(a) Show that $\set{x \in X\mathrel{}\middle|\mathrel{} \lim_{n\to\infty}f_n(x)\;\mathrm{exists}}$ is a measurable set.
\begin{proof}
	First we consider $\set{x \in X\mathrel{}\middle|\mathrel{} |\lim_{n\to\infty}f_n(x)| < \infty} = J$, if $x \in J$ then $g(x) - h(x) = z(x) = 0$ iff $h = \lim \inf f_n, g = \lim \sup f_n$, which are measurable so $z^{-1}(0)$ is also measurable. Now for infinite values, $h = g = \infty$ and again by m easurability $h^{-1}(\infty) \cap g^{-1}(\infty)$ measurable, the same aregument ghoes for $-\infty.$
\end{proof}
\medskip \noindent {\bf (4.2)}\ Show that if $f: X \to \mathbb{R}^* $ and if $f^{-1}((q, \infty]) \in \scriptm$ for every $q\in\mathbb{Q}$ then $f$ is measurable.
\begin{proof}
 	We need to show that algebra of half half-infinite rays $\scripte$ can be generated by $\scriptq = \{(q, \infty]\}$. Take any $x \in \reals^*$, there is a sequence of ascending $q \in \mathbb{Q}$, say $q_n \to x$. T hen consider their manifestation in $\scriptq,$ say $Q_1 = (q_1, \infty], Q_n =(q_n, \infty], \dots$. Then
 	\begin{equation*}
 		\bigcap_{n=1}^\infty Q_n = \set{y \in \mathbb{R} \mathrel{}\middle|\mathrel{} y > q_n, n \in \mathbb{N}} = [x, \infty].
 	 \end{equation*}
 	 Then since $f^{-1}(q, \infty]) \in \scriptm,$ it's clear that $f^{-1}(\bigcap Q_n) \in \scriptm$, and so by Proposition 2.3, $f$ is measurable.
 \end{proof}  

 \medskip \noindent {\bf (4.3)}\ Let $(X, \scriptm, \mu)$ be a \emph{complete} measure space, then
 (a) If $f$ is measurable and $f=g$ $\mu$-a.e. then $g$ is measurable.
 \begin{proof}
 	Take some set in $\scriptn$,  say $E$, then $f^{-1}(E) \cap f^{-1}(E) \in \scriptm$ and $g^{-1}(E) = f^{-1}(E) \setminus J \cup F$ where
 	$J = \{x \in f^{-1}(E)\mathrel{}|\mathrel{}f(x) \neq f(g)\}$ and $F = \{x \in X\ |\ g(x) \in E, g(x) \neq f(x)\},$ both of which are subsets of $D = \set{x \in X\mathrel{}\middle|\mathrel{}f(x) \neq g(x)}.$ Because $\mu(D) = 0$ and $\mu$ complete, then $J, F$ are measurable. Then by measurability of $f$, $g^{-1}(E) \in \scriptm$. Therefore $g$ measurable and this completes the proof.
  \end{proof}
  (b) If $f_n \to f$ almost every where and $f_n$ measurable then $f$ measurable.
  \begin{proof}
  	Consider $h_n = \sup_{k\geq n} f_n$ and $g_n = \inf_{k\geq n} f_n$. From a proposition of the text $h_n$ and $g_n$ measurable for all $n$ and $\lim h_n = h$ and $\lim g_n = g$ measurable. Furthermore, $h(x)-g(x) = 0 \iff f_n(x) \to f(x)$ for those $x$. Therefore $D = \{|h(x)-g(x)| > 0\}$ is a zeroset. 
  	Furthermore $g(x) = f(x) =h(x)$ on $D$ and by the previous proposition $f = g$ a.e is measurable
  	\end{proof}
 \medskip \noindent {\bf (4.4)} If $f \in L^+$ and $\int f < \infty$ then $\{x:f(x) = \infty\}$ is a nullset and $\{x : f(x) > 0\}$ is $\sigma$-finite.
  \begin{proof}
    We prove the contrapositive. Suppose that $G = \{x:f(x) = \infty\}$ has measure $m > 0$. Then we have by measure outward continuity (and the results of the section that) $\int_X f > \int_G f$ since $X \supset G$. However, $f|_G$ can be described by a simple function in standard from $f|_G \geq c\chi_G$ where $c = \infty$. Therefore $\int_G f \geq c\mu(\chi) = \infty\times m, $ where $m > 0$. Therefore $\int f \geq \infty.$ Now suppose that
     $F = \set{x\mathrel{}\middle|\mathrel{}f(x) > 0}$ is not $\sigma$-finite. In such a case, in any countable union forming $F$ has a member set with measure $\infty$. For example take, $F_n = \set{x\mathrel{}\middle|\mathrel{}f(x) > 1/n}$. Clearly $\bigcup_{n=1}^\infty F_n = F$ and there exists an $N$ so that $\mu(F_N) = \infty$. Again $f|_{F_N} > 1/N$ so we know that $\int_{F_N} f \geq \int_{F_N} 1/N\chi_{F_N} = 1/N\mu(F_N) = \infty/N = \infty$. Therefore $\int_X f > \int_{F_N} f = \infty.$ So the contrapositive holds.

     This completes the proof.
  \end{proof} 

  \medskip \noindent {\bf (4.5)} Suppose that $f_n \in L^+$ and $f_n$ decreases pointwise to $f$ and $\int f_1 < \infty$. Show that $\lim{n \to \infty} \int f_n = \int f$.
  \begin{proof} 
  Define a sequence of simple functions $\mathfrak{F}_n$ so that  for every $n$,  $\mathfrak{F}_n \leq f_n$ and $|\int \mathfrak{F}_n - \int f_n| <2^{-n}$.  Such a sequence exist since $\int f_n < \infty$ so the subtraction does not violate any rules of the extended reals. Furthermore there are such simple functions since $\int f_n < \infty$ gives $\set{x: f(x) > 0}$ $\sigma$-finite, so partitioning the domain into constants and indicator functions gives a finite integral for such a simple function $\leq f_n$. First we know that $\int \mathfrak{F}_n \geq \int f$ for every $n$ since $\mathfrak{F}_n \geq f$. Furthermore $a_n = \int \mathfrak{F}_n$ is Cauchy since $|\int \mathfrak{F}_n - \int \mathfrak{F}_m| \leq |\int \mathfrak{F}_n - \int f_n| + |\int \mathfrak{F}_m - \int f_m| + - |\int f_m - \int f_n| < \epsilon  + 2^{-n} +2^{-m} \to 0$ as $m,n \to \infty$.

    Next we use the sandwhich theorem and
    \begin{center}
        \begin{tikzcd}
          f\arrow{d}{n}\arrow[dash]{r}{\leq} & \mathfrak{F}_n \arrow[dashed]{d}{n}\arrow[dash]{r}{\leq} & f_n \arrow{d}{n} \\
          f \arrow[equal]{r}& f\arrow[equal]{r} & f
        \end{tikzcd}
     \end{center}   
    Let $\epsilon > 0$. Consider the sequence $d_n = 2\sup_x \mathfrak{F}_n - f$. By the above limit diagram, $d_n \to 0$ and $\mathfrak{F}_n - d_n = \mathfrak{G}_n(x) = \sum_{y_n \in range(\mathfrak{F})} (y_n - d_n)\chi_{y_n}(x)$ is an element of a family of siple functions so that $\mathfrak{G}_n \uparrow f$ pointwise. By the \emph{upward} monotone convergence theorem of measure theory $\int \mathfrak{G}_n \uparrow \int f$. Now observe that
    \begin{equation*}
      \left|\int \mathfrak{G}_n - \int \mathfrak{F}_n\right| = \sum\left(y_n - (y_n - d_n)\right)\mu(\set{x:\mathfrak{F}(x) = y_n}) = \sum\left(d_n\right)\mu(\set{x:\mathfrak{F}(x) = y_n}) \to 0
    \end{equation*}
    So $\int \mathfrak{G}_n \to \int f$ implies that $\int \mathfrak{F}_n \to \int f$. Finally we can now take $n$ large enough that $2^{-n} < \epsilon/2$ and $\left|\int \mathfrak{F}_n - \int f\right| < \epsilon/2$.
    \begin{equation*}
      \left|\int f_n - \int f\right| \leq \left|\int f_n - \int \mathfrak{F}_n\right| + \left|\int \mathfrak{F}_n - \int f\right| < \epsilon/2 + \epsilon/2  < \epsilon.
    \end{equation*}
  \end{proof}
  This completes the proof.

  \medskip \noindent {\bf (4.6)} Let $C$ be the Cantor ternary set, and let $f: [0,1] \to [0,1]$ be the Devil's staircase, as defined in $1.5$ of our text. Define $g(x) = f(x) + x$. Prove the following:
  (a) $g$ is homeomorphism from $[0,1] \to [0,2]$.
  \begin{lemma}
  	If $f: E \subset\mathbb{R} \to F \subset \mathbb{R}$ non decreasing and $E$, compact then $f +id$ is bijective on its range.
  \end{lemma}
  \begin{proof}
  	Take any $x \neq y$, then without loss of generality $x >y$, so $f(x) \geq f(y)$ and $id(x) > id(y)$ so $f(x) + id(x) > id(y) + f(y)$ and $(f +id)(x) \neq (f+id)(y)$ and $f + id$ is injective, and so $f + id$ bijective on its range from $E$.
  \end{proof}
  \begin{proof} (Of 4.6.a)
  The function is bijective by the previous lemma. The sum of to continuous functions is continuous. Every closed subset of $[0,1]$ is compact. Since the function is continuous the image of a compact set is a compact and closed subset of $[0,2]$ so the map is closed so the map is open and bijective continuous and so $g$ is a homeo.
  \end{proof}

  (b) $m(g(C)) = 1$ even though $C$ is a null set.
  \begin{proof}
  		We know that $[0,1] \setminus C = U =\bigsqcup_j B_j$ is the union of open intervals.  Then take $$g(U) = \bigsqcup_{n=1}^j c_j + B_j$$ and $m(B_j) = m(B_j + c_j)$ because $B_j \in \scriptb_\mathbb{R}$ open and $B_j + c_j \in \scriptb_\mathbb{R}$ so $$m(U) = \sum_{j=1}^\infty m(B_j) = \sum_{j=1}^\infty m(B_j + c_j) = m(g(U)).$$

  		Therefore $m([0,2]) = m(g(U)) + m(g(C)) \implies m(g(C)) = 1$ and $m(C) = 0.$
  \end{proof}
  (c) Let $A$ be any subset of $g(C)$ that is not Lebesgue measurable. Show that $B = g^{-1}(A)$ is Lebesgue measurable, but is not Borel measurable.
  \begin{proof}
  	Let $A \subset g(C)$ non measurable. $B = g^{-1}(A)$ is a subset of $C$ which is a Lebesgue null-set. Since labesgue $\mu$ is complete then any subset of $C$ is measurable with measure $0$. Since $g^{-1}$ is a homeomorphism it preserves the Borel $\sigma$-algebra, and so since $A$ is not Lebesgue measurable it is not Borel measurable and so if $B$ were Borel measurable it would conteradict the topological invariance of $g.$
  \end{proof}
  (d) There exist a Lebesgue measurable function $F: \mathbb{R} \to \mathbb{R}$ and a continuous function $G: \mathbb{R} \to \reals$ such that $F \circ G$ is not Lebesgue measurable. 
  \begin{proof}	
 		Let $F = \chi_B$ then $G = g^{-1}$ when $x \in [0,2]$ other wise if $x <0$, $G(x) = 0$, or if $x>2$, $G(x)=1$. Then $(F \circ G)^{-1}  = g \circ F^{-1}(1) = g(B) =A$.
  \end{proof}

\end{document}\end

