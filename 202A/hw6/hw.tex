\documentclass[11pt]{amsart}

\usepackage{amsmath,amsthm}
\usepackage{amssymb}
\usepackage{graphicx}
\usepackage{enumerate}
\usepackage{fullpage}
% \usepackage{euscript}
% \makeatletter
% \nopagenumbers
\usepackage{verbatim}
\usepackage{color}
\usepackage{hyperref}
%\usepackage{times} %, mathtime}

\textheight=600pt %574pt
\textwidth=480pt %432pt
\oddsidemargin=15pt %18.88pt
\evensidemargin=18.88pt
\topmargin=10pt %14.21pt

\parskip=1pt %2pt

% define theorem environments
\newtheorem{theorem}{Theorem}    %[section]
%\def\thetheorem{\unskip}
\newtheorem{proposition}[theorem]{Proposition}
%\def\theproposition{\unskip}
\newtheorem{conjecture}[theorem]{Conjecture}
\def\theconjecture{\unskip}
\newtheorem{corollary}[theorem]{Corollary}
\newtheorem{lemma}[theorem]{Lemma}
\newtheorem{sublemma}[theorem]{Sublemma}
\newtheorem{fact}[theorem]{Fact}
\newtheorem{observation}[theorem]{Observation}
%\def\thelemma{\unskip}
\theoremstyle{definition}
\newtheorem{definition}{Definition}
%\def\thedefinition{\unskip}
\newtheorem{notation}[definition]{Notation}
\newtheorem{remark}[definition]{Remark}
% \def\theremark{\unskip}
\newtheorem{question}[definition]{Question}
\newtheorem{questions}[definition]{Questions}
%\def\thequestion{\unskip}
\newtheorem{example}[definition]{Example}
%\def\theexample{\unskip}
\newtheorem{problem}[definition]{Problem}
\newtheorem{exercise}[definition]{Exercise}

\numberwithin{theorem}{section}
\numberwithin{definition}{section}
\numberwithin{equation}{section}

\def\reals{{\mathbb R}}
\def\torus{{\mathbb T}}
\def\integers{{\mathbb Z}}
\def\rationals{{\mathbb Q}}
\def\naturals{{\mathbb N}}
\def\complex{{\mathbb C}\/}
\def\distance{\operatorname{distance}\,}
\def\support{\operatorname{support}\,}
\def\dist{\operatorname{dist}\,}
\def\Span{\operatorname{span}\,}
\def\degree{\operatorname{degree}\,}
\def\kernel{\operatorname{kernel}\,}
\def\dim{\operatorname{dim}\,}
\def\codim{\operatorname{codim}}
\def\trace{\operatorname{trace\,}}
\def\dimension{\operatorname{dimension}\,}
\def\codimension{\operatorname{codimension}\,}
\def\nullspace{\scriptk}
\def\kernel{\operatorname{Ker}}
\def\p{\partial}
\def\Re{\operatorname{Re\,} }
\def\Im{\operatorname{Im\,} }
\def\ov{\overline}
\def\eps{\varepsilon}
\def\lt{L^2}
\def\curl{\operatorname{curl}}
\def\divergence{\operatorname{div}}
\newcommand{\norm}[1]{ \|  #1 \|}
\def\expect{\mathbb E}
\def\bull{$\bullet$\ }
\def\det{\operatorname{det}}
\def\Det{\operatorname{Det}}
\def\rank{\mathbf r}
\def\diameter{\operatorname{diameter}}

\def\t2{\tfrac12}

\newcommand{\abr}[1]{ \langle  #1 \rangle}

\def\newbull{\medskip\noindent $\bullet$\ }
\def\field{{\mathbb F}}
\def\cc{C_c}



% \renewcommand\forall{\ \forall\,}

% \newcommand{\Norm}[1]{ \left\|  #1 \right\| }
\newcommand{\Norm}[1]{ \Big\|  #1 \Big\| }
\newcommand{\set}[1]{ \left\{ #1 \right\} }
%\newcommand{\ifof}{\Leftrightarrow}
\def\one{{\mathbf 1}}
\newcommand{\modulo}[2]{[#1]_{#2}}

\def\bd{\operatorname{bd}\,}
\def\cl{\text{cl}}
\def\nobull{\noindent$\bullet$\ }

\def\scriptf{{\mathcal F}}
\def\scriptq{{\mathcal Q}}
\def\scriptg{{\mathcal G}}
\def\scriptm{{\mathcal M}}
\def\scriptb{{\mathcal B}}
\def\scriptc{{\mathcal C}}
\def\scriptt{{\mathcal T}}
\def\scripti{{\mathcal I}}
\def\scripte{{\mathcal E}}
\def\scriptv{{\mathcal V}}
\def\scriptw{{\mathcal W}}
\def\scriptu{{\mathcal U}}
\def\scriptS{{\mathcal S}}
\def\scripta{{\mathcal A}}
\def\scriptr{{\mathcal R}}
\def\scripto{{\mathcal O}}
\def\scripth{{\mathcal H}}
\def\scriptd{{\mathcal D}}
\def\scriptl{{\mathcal L}}
\def\scriptn{{\mathcal N}}
\def\scriptp{{\mathcal P}}
\def\scriptk{{\mathcal K}}
\def\scriptP{{\mathcal P}}
\def\scriptj{{\mathcal J}}
\def\scriptz{{\mathcal Z}}
\def\scripts{{\mathcal S}}
\def\scriptx{{\mathcal X}}
\def\scripty{{\mathcal Y}}
\def\frakv{{\mathfrak V}}
\def\frakG{{\mathfrak G}}
\def\aff{\operatorname{Aff}}
\def\frakB{{\mathfrak B}}
\def\frakC{{\mathfrak C}}

\def\suchthat{\mathrel{}:\mathrel{}}
\def\symdif{\,\Delta\,}
\def\mustar{\mu^*}
\def\muplus{\mu^+}

\def\soln{\noindent {\bf Solution.}\ }


%\pagestyle{empty}
%\setlength{\parindent}{0pt}

\begin{document}

\begin{center}{\bf Math 202A --- UCB, Fall 2016 --- William Guss}
\\
{\bf Problem Set 6, due Wednesday October 5}
\end{center}

\medskip
Throughout, $(X,\scriptm,\mu)$ denotes a measure space. 
$\int f$ is shorthand for $\int_X f\,d\mu$, where $\mu$ is a measure which
may not be explicitly specified. $m$ denotes Lebesgue measure on either $\scriptb_\reals$
or $\scriptl_\reals$. Unless otherwise indicated, ``$f$ is measurable'' means
that $f:X\to\complex$ and $f$ is measurable with respect to $\scriptm$.
$L^1$ refers to functions, rather than to equivalence classes of functions, unless
otherwise indicated.
proof

\medskip \noindent {\bf (6.1)}\ (Folland problem 2.33)\ 
Let $f_n,f$ be measurable.
Suppose that $f_n\ge 0$ and $f_n\to f$ in measure. 
Show that $\int f\le \liminf \int f_n$.

\begin{proof}
	First we can always construct a sequence $f_{n_k}$ so that $\int f_{n_k} \to \lim\inf \int f_{n_k}.$ Now $f_{n_k} \to f$ in measure by basic analysis ($(\mu(|f_{n_k} - f| \geq \epsilon))_k$ is a subsequence of $(\mu(|f_{n} - f| \geq \epsilon))_n$ ). 
	There is a sub subsequence $f_{n_{k_j}}$ which converges to $f$ almost everywhere by Proposition $2.30$.  
	So $\int f = \int \lim f_{n_{k_j}} = \int \lim \inf f_{n_{k_j}}  \leq \lim \inf \int {f_{n_{k_{j}}}}$ by Fatou's lemma.
	  Now $\lim \inf \int f_{n_{k_j}} \leq \lim \int f_{n_{k_j}}$. Now $\int f_{n_{k_j}}$ is a subsequence of the convergent sequence $\int f_{n_k}$, so it converges to that same limit. Therefore $\lim \inf \int {f_{n_{k_{j}}}} \leq \lim \inf \int f_n.$ Therefore $\int f \leq \lim \inf f_n$.
\end{proof}


\medskip \noindent {\bf (6.2)}\ (Folland problem 2.36)\ 
Suppose that $E_n\in\scriptm$ and $\mu(E_n)<\infty$ for each $n\in\naturals$.
Suppose that $\one_{E_n}\to f$ in $L^1$ (that is, $f$ is measurable and $\complex$--valued, 
and $\int |\one_{E_n}-f|\to 0$).
Show that there exists a measurable set $E$ such that $f=\one_E$ almost everywhere.
\begin{proof}
If $\chi_{E_n} \to f$ in the $L^1$ sense then $\chi_{E_n} \to f$ in measure since $\chi_{E_n} \in L^1$ ($\mu(E_n) < \epsilon$).  Let $$\Delta_{n, \epsilon} = \set{x\mathrel{}:\mathrel{}\left|\chi_{E_n}(x) -f \right|\geq \epsilon}.$$
Now for all $\epsilon$, $\mu(\Delta_{n,\epsilon}) \to 0$ as $n \to \infty$ implies that 
\begin{equation*}
	\mu(\set{x\mathrel{}:\mathrel{}x \in E_n, \left|1 -f \right|\geq \epsilon} \cup \set{x\mathrel{}:\mathrel{}x \notin E_n\left|0-f \right|\geq \epsilon}) \to 0
\end{equation*}
for all $\epsilon.$ So $\mu(\set{x\mathrel{}:\mathrel{}f(x) \notin \{0,1\}} = 0$, and call that set $B_f$. Partition $X$ so that $X = B_f \cup E \cup (X \setminus (E \cup B_f))$, where $E = \set{x\mathrel{}:\mathrel{}f(x) = 1}$ then $f(x) = \chi_{B_f}g(x) + \chi_E\cdot1 + \chi_{E^c \setminus B_f} \cdot 0$, where $g(x)$ is some crazy cooky measurable function. Now
\begin{equation*}
	\int |f - \chi_E| = \int |\chi_{B_f}g(x)| = 0
\end{equation*}
so $f = \chi_E$ up to a null set and $\chi_E$ is measurable so $\chi_E^{-1}(\{1\}) = E$ is measurable by measurability of the a.e. limit of measurable functions.
\end{proof}

\medskip \noindent {\bf (6.3)}\ (Folland problem 2.38)\ 
Assume that $f_n,g_n,f,g$ are measurable $\complex$--valued functions.
Suppose that $f_n\to f$ in measure, and $g_n\to g$ in measure.
\newline
{\bf (a)}\  Show that $f_n+g_n\to f+g$ in measure.
\begin{proof}
	If $f_n \to f$ and $g_n \to g$ in measure let
	$$\Delta_{\epsilon/2, n}(f) =  \set{x\mathrel{}:\mathrel{}\left|f_n -f \right|\geq \epsilon/2}$$
	$$\Delta_{\epsilon/2, n}(g) =  \set{x\mathrel{}:\mathrel{}\left|g_n -g \right|\geq \epsilon/2}$$
	Now observe that $|f_n +g_n - f - g| \leq |f_n -f| + |g_n - g| $ 
	and so in the worst case $\Delta(f)$ and $\Delta(g)$ are large enough in measure that (say $\delta/2 + \delta/2$, $\delta \to 0$) that the $x$ satisifying  $|f_n +g_n - f -g|  > \epsilon $ are either $x \in\Delta_{\epsilon, n}(g) $ or $x \in \Delta_{\epsilon, n}(f) $ or both then
	$$x \in \Delta_{\epsilon, n}(f,g) =  \set{x\mathrel{}:\mathrel{}\left|g_n + f_n -g -f \right|\geq \epsilon} \subset  \Delta_{\epsilon, n}(g) \cup \Delta_{\epsilon, n}(f)$$
	Then since these difference sets go to sets with null measure the difference sets for $f_n +g_n$ go to a null set in the limit.

\end{proof}
\noindent {\bf (b)}\  Show that if $\mu(X)<\infty$ then $f_ng_n\to fg$ in measure.
\begin{proof}
%Using exercise $32$. If $f_n \to f$ in measure and $g_n \to g$ in measure then
%$\rho(f_n, f) \to 0$ and $\rho(g_n, g) \to 0$. Using triangle inequality we get \begin{equation*}
%	\begin{aligned}
%		\rho(f_ng_n, fg) &\leq \rho(f_ng_n, f_ng) + \rho(f_ng, fg)  \\
%		&= \int \frac{|f_ng_n - f_ng|}{1 + |f_ng_n - f_ng|}\  d\mu + \int \frac{|f_ng - fg|}{1 + |f_ng - fg|}\  d\mu \\
%		&= \int \frac{|f_ng_n - f_ng|}{1 + |f_ng_n - f_ng|}+ \frac{|f_ng - fg|}{1 + |f_ng - fg|}\  d\mu \\
%		&= \int \frac{|f_n||g_n - g|}{1 + |f_n||g_n - g|}+ \frac{|g||f_n - f|}{1 + |g||f_n - f|}\  d\mu \\
%	\end{aligned}
%\end{equation*}
%Now $\int \frac{|g||f_n - f|}{1 + |g||f_n - f|}\  d\mu \to 0$ since if $\frac{|f_n - f|}{1 + |f_n - f|}\  d\mu \to 0$ and the denominator always positive


	If $f_n \to f$ and $g_n \to g$ in measure then using the previous notation, recall $\mu(\Delta_{\epsilon, n}(f))\to 0$, $\mu(\Delta_{\epsilon, n})(g) \to 0$. Now $|f_ng_n -fg| \geq \epsilon$ can be ammenable using the triangle inequality using the trick from homework one: $|f_ng_n -f_ng + f_ng -fg| = |f_ng_n - fg| \geq \epsilon$ has triangle inequality expansion $|f_ng_n -f_ng + f_ng -fg| \leq |g||f_n -f| + |f_n||g- g_n|$. So we need to show that $|g-g_n|$ is eventually small enough so that the set $|f_n||g-g_n| > \epsilon$ is small. First we know that $|f_n| > r$ on some set say $D_r$. Then $|f_n||g-g_n| > \epsilon $ on $D_r \cup \{x : |g-g_n| > \epsilon/r\} := D_r \cup G_{r,\epsilon}.$ If $p > q$ then $D_p \subset D_q$. and since $m(X) < \infty$ we have that $\bigcap D_r = \emptyset$ implies that $\mu(D_r) \to 0$ for large enough $r$ and furthermore that $D_r$ has finite measure. Additionally we have that $\mu(G_{r, \epsilon}) \to 0 $ as $n \to \infty \to 0$. So the set for which $|f_n||g-g_n| > \epsilon$ with fixed $r$ for all $r$ has measure $0$ as $n \to \infty.$ A similar argument applies to $|g||f_n -f| > \epsilon$. Therefore $f_ng_n \to fg$ in measure.
\end{proof}



{\bf (c)}\  Show by example that the conclusion of (b) need not hold
if $\mu(X)$ is not finite.
\begin{proof}
	Take any $g_n \to g$ on $\mathbb{R}$ in measure Take any increasing $f$ measurable such that $f>x$. Let $f_n = = f + 1/n = g_n$. Both functions tend to
	$f$ in measure since $|f_n -f| = 1/n$ and the set of $x$ so that $|f_m(x) - f(x)| \geq 1/n$ is infinite until $m > n$ at which case $|f_m -f| < 1/n$
	for all $x$ so the difference set for $\epsilon = 1/n$ is a zeroset. Then for any $\epsilon$ find the closest $1/n$ and wait for $m >n$ and the same argument gives that for every $\epsilon$ the difference set by $\epsilon$ becomes a zeroset.

	Now consider $f_ng_n$. Claim that $f_ng_n \not\to f^2$ in measure. The set of $x$ so that $|f_n(x)f_n(x) - f^2| > 1/m$ is the set 
	$|f(x)^2 + 2f(x)/n + 1/n^2  - f^2| > 1/m$. The LHS becomes $|2f(x)/n + 1/n^2| > 1/m.$ By the monotonicity of $f$, $2f(x)/n > 1/n^2$ eventually and
	then for all $x$ more than that eventuallity and increasing by more than $1/m$ since $f > x$, so that the set of $x$ for which $x > 1/m$ always has infinite measure. Therefore $f_n^2 \not\to f^2$ in measure.

\end{proof}

\medskip \noindent {\bf (6.4)}\ (Folland problem 2.40)\ 
Show that in Egoroff's theorem, the hypothesis that $\mu(X)$ is finite
can be replaced by the existence of $g\in L^1$ such that
$|f_n|\le g$ for all $n$.
\begin{proof}
	We adapt Egoroff's proof. Without loss of generality assume that $f_n 
	\to f$ everywhere on $X$. Then since $f_n$ dominated we get that $\int f_n \to \int f$ and 
	therefore $f_n \to f$ in the $L^1$ sense. Now consider the sets
	\begin{equation*}
		E_{n}(k) = \bigcup_{m=n}^\infty \set{x\suchthat |f_m(x) - f(x)| \geq \frac{1}{k}}.
	\end{equation*}
	When fixing $k$ these sets decrease as $n \to \infty$ since there are less elements in the union. Additionally 
	by the convergence of $f_n$ to $f$ pointwise we have $\bigcap_{n=1}^\infty E_n(k) = \emptyset.$  Let $\epsilon > 0$ be given. Since $f_n$ 
	converges in $L^1$ it therefore converges in measure and so $\mu(E_n(k)) \to 0$ for every $k$ as $n \to \infty$ and so for each $k$ take $n_k$ 
	large enough that $\mu(E_{n_k}(k)) < 2^{-k}\epsilon.$\footnote{To see that these measures are really decreasing, convergence in measure gives $E_{n}(k) \cap E_{n+1}(k)$ is eventually as small as we like. And so we sum infinitely many elements as small as we like (and possibly smaller as the sum continues), so eventually
	we sum infintieley many zeroes which must be zero $(\infty\cdot 0 = 0)$. This is a high level explaination of the phenomena occuring, for a detailed and rigourous treatment of decreasing infinite series of convergent sequences, see Charles Pugh's undergraduate text, \emph{Real Mathematical Analysis}.}
	Now $$\mu\left(\bigcup_{k=1}^\infty E_{n_k}(k)\right) < \sum_{k=1}^\infty 2^{-k}\epsilon < \epsilon.$$
	So the set $X \setminus E$, the compliment of the union above, has the property that for every $k > 0$ there is an $N$ so that $\|f_m -f\| < k^{-1}$ in
	the $\sup$, uniform norm. Additionaly $\mu(E) < \epsilon$ so $f$ is uniformly continuous (as $\epsilon \to 0$) except for a zero set.
\end{proof}

\medskip \noindent {\bf (6.5)}\ (Folland problem 2.44)\ 
Let $(X,\scriptm,\mu) = (\reals,\scriptl_\reals,m)$.
Let $[a,b]\subset\reals$ be a bounded closed interval. 
Let $f:[a,b]\to\complex$ be Lebesgue measurable. Show that for any $\eps>0$
there exist a compact set $K\subset[a,b]$ 
such that $m([a,b]\setminus K)<\eps$ and the restriction of $f$ to $K$
is a continuous function. 
\begin{proof}
	Let $B_r \subset \mathbb{C}$ be a ball of radius $r$ centered at the origin in $\mathbb{C}$. Now consider the \emph{finite restriction} of $f$, say $f_r$, which is clipped by the ball in its imaginary and compex parts. Then the preimage of this ball is a measurable set, which is contained by a compact say $K_r$ so that
	$\mu(K_r \cap f_r^{pre}(B_r)) = 0,$ by the measurability of $f_r^{pre}(B_r).$ On this compact set $f_r$ is measurable and $L^1$. Therefore there is a continuous $g_r$ so that $\int |g_r -f_r| = 0$ and $g_r$ vanishes outside of $[a,b]$ by a proposition of our book.

	Now take a sub family of those such continuous functions say $(g_n)_{n\in \mathbb{N}}.$ Every continuous function is itself $L^1$. Now we have a sequence of $g_n$ continuous $L^1$ functions converging to $f$ almost everywhere by transitivity of the approximation on $f_n \to f$ and $g_n \to f_n$. By Egoroff's Theorem for every $\epsilon$ there is a set $E \subset \mathbb{[a,b]}$ measurable such that $g_n$ converges to $f$ uniformly on $E$ and $\mu(E) = \mu([a,b]) - \epsilon.$ By uniform convergence and continuity of $g_n$ it must be that $f$ is continuous on the set $E$. Then by measurability of $E$ there is a set $K \supset E$ $\mu(K \cap E) = 0$. Lastly if $f$ continuous on $E$ then there is a unique continuous extension of $f$ to $K$ which is also continuous (from basic undergraduate real analysis, Royden Prop 35 Section 7.9). Therefore $f|K$ is continuous and $\mu([a,b] \setminus K) < \epsilon.$ 
\end{proof}

\end{document}\end
