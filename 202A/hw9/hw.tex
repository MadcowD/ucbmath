\documentclass[11pt]{amsart}

\usepackage{amsmath,amsthm}
\usepackage{amssymb}
\usepackage{graphicx}
\usepackage{enumerate}
\usepackage{fullpage}
% \usepackage{euscript}
% \makeatletter
% \nopagenumbers
\usepackage{verbatim}
\usepackage{color}
\usepackage{hyperref}
%\usepackage{times} %, mathtime}

\textheight=600pt %574pt
\textwidth=480pt %432pt
\oddsidemargin=15pt %18.88pt
\evensidemargin=18.88pt
\topmargin=10pt %14.21pt

\parskip=1pt %2pt

% define theorem environments
\newtheorem{theorem}{Theorem}    %[section]
%\def\thetheorem{\unskip}
\newtheorem{proposition}[theorem]{Proposition}
%\def\theproposition{\unskip}
\newtheorem{conjecture}[theorem]{Conjecture}
\def\theconjecture{\unskip}
\newtheorem{corollary}[theorem]{Corollary}
\newtheorem{lemma}[theorem]{Lemma}
\newtheorem{sublemma}[theorem]{Sublemma}
\newtheorem{fact}[theorem]{Fact}
\newtheorem{observation}[theorem]{Observation}
%\def\thelemma{\unskip}
\theoremstyle{definition}
\newtheorem{definition}{Definition}
%\def\thedefinition{\unskip}
\newtheorem{notation}[definition]{Notation}
\newtheorem{remark}[definition]{Remark}
% \def\theremark{\unskip}
\newtheorem{question}[definition]{Question}
\newtheorem{questions}[definition]{Questions}
%\def\thequestion{\unskip}
\newtheorem{example}[definition]{Example}
%\def\theexample{\unskip}
\newtheorem{problem}[definition]{Problem}
\newtheorem{exercise}[definition]{Exercise}

\numberwithin{theorem}{section}
\numberwithin{definition}{section}
\numberwithin{equation}{section}

\def\reals{{\mathbb R}}
\def\torus{{\mathbb T}}
\def\integers{{\mathbb Z}}
\def\rationals{{\mathbb Q}}
\def\naturals{{\mathbb N}}
\def\complex{{\mathbb C}\/}
\def\distance{\operatorname{distance}\,}
\def\support{\operatorname{support}\,}
\def\dist{\operatorname{dist}\,}
\def\Span{\operatorname{span}\,}
\def\degree{\operatorname{degree}\,}
\def\kernel{\operatorname{kernel}\,}
\def\dim{\operatorname{dim}\,}
\def\codim{\operatorname{codim}}
\def\trace{\operatorname{trace\,}}
\def\dimension{\operatorname{dimension}\,}
\def\codimension{\operatorname{codimension}\,}
\def\nullspace{\scriptk}
\def\kernel{\operatorname{Ker}}
\def\p{\partial}
\def\Re{\operatorname{Re\,} }
\def\Im{\operatorname{Im\,} }
\def\ov{\overline}
\def\eps{\varepsilon}
\def\lt{L^2}
\def\curl{\operatorname{curl}}
\def\divergence{\operatorname{div}}
\newcommand{\norm}[1]{ \|  #1 \|}
\def\expect{\mathbb E}
\def\bull{$\bullet$\ }
\def\det{\operatorname{det}}
\def\Det{\operatorname{Det}}
\def\rank{\mathbf r}
\def\diameter{\operatorname{diameter}}

\def\t2{\tfrac12}

\newcommand{\abr}[1]{ \langle  #1 \rangle}

\def\newbull{\medskip\noindent $\bullet$\ }
\def\field{{\mathbb F}}
\def\cc{C_c}



% \renewcommand\forall{\ \forall\,}

% \newcommand{\Norm}[1]{ \left\|  #1 \right\| }
\newcommand{\Norm}[1]{ \Big\|  #1 \Big\| }
\newcommand{\set}[1]{ \left\{ #1 \right\} }
%\newcommand{\ifof}{\Leftrightarrow}
\def\one{{\mathbf 1}}
\newcommand{\modulo}[2]{[#1]_{#2}}

\def\bd{\operatorname{bd}\,}
\def\cl{\text{cl}}
\def\nobull{\noindent$\bullet$\ }

\def\scriptf{{\mathcal F}}
\def\scriptq{{\mathcal Q}}
\def\scriptg{{\mathcal G}}
\def\scriptm{{\mathcal M}}
\def\scriptb{{\mathcal B}}
\def\scriptc{{\mathcal C}}
\def\scriptt{{\mathcal T}}
\def\scripti{{\mathcal I}}
\def\scripte{{\mathcal E}}
\def\scriptv{{\mathcal V}}
\def\scriptw{{\mathcal W}}
\def\scriptu{{\mathcal U}}
\def\scriptS{{\mathcal S}}
\def\scripta{{\mathcal A}}
\def\scriptr{{\mathcal R}}
\def\scripto{{\mathcal O}}
\def\scripth{{\mathcal H}}
\def\scriptd{{\mathcal D}}
\def\scriptl{{\mathcal L}}
\def\scriptn{{\mathcal N}}
\def\scriptp{{\mathcal P}}
\def\scriptk{{\mathcal K}}
\def\scriptP{{\mathcal P}}
\def\scriptj{{\mathcal J}}
\def\scriptz{{\mathcal Z}}
\def\scripts{{\mathcal S}}
\def\scriptx{{\mathcal X}}
\def\scripty{{\mathcal Y}}
\def\frakv{{\mathfrak V}}
\def\frakG{{\mathfrak G}}
\def\aff{\operatorname{Aff}}
\def\frakB{{\mathfrak B}}
\def\frakC{{\mathfrak C}}

\def\symdif{\,\Delta\,}
\def\mustar{\mu^*}
\def\muplus{\mu^+}

\def\soln{\noindent {\bf Solution.}\ }


%\pagestyle{empty}
%\setlength{\parindent}{0pt}

\begin{document}

\begin{center}{\bf Math 202A --- UCB, Fall 2016 --- William Guss}
\\
{\bf Problem Set 9, due Wednesday October 26}
\end{center}


\medskip \noindent {\bf (9.1)}\ (Folland problem 3.22)\ 
Let $f\in L^1=L^1(\reals,\scriptl,m)$. Assume that $f\ne 0$
in the sense that $\{x: f(x)\ne 0\}$ has strictly positive Lebesgue measure.
Show that there exist $c,R>0$ such that
$Hf(x)\ge c|x|^{-1}$ for all $x$ satisfying $|x|\ge R$.
\begin{proof}
	First observe that the proposition concerns the ball of radius $R$ around the origin.
	If $x$ is outside of that ball, ($|x| > R$) we hope to determine the value of the maximal function $Hf(x).$

	Now since $f$ is nonzero in measure there is an $R > 1$ so that $A_{R}|f|(0) m(B(R,0)) \geq \gamma > 0.$ Considering $x$ outside of the ball, we have that
	\begin{equation*}
		\int_{B(2|x|, x)} |f| dm \geq \int_{B(r, 0)} |f| dm \geq \gamma > 0
	\end{equation*}
	since $B(2|x|, x) \supset B(r, 0).$
	Then we know that $Hf(x) \geq A_{2|x|}|f|(x) = m(B(2|x|,x)) \int_{B(2|x|,x)} |f|\ dm$ so 
	\begin{equation*}
		Hf(x) \geq A_{2|x|}|f|(x) \geq \frac{m(B(R,0))}{m(B(2|x|, x))} A_R |f|(0) \geq \frac{\gamma}{m(B(2|x|, x))} > 0
	\end{equation*}
	Therefore $Hf(x) \geq 2\gamma|x|^{-1}$ when $|x| > R$
\end{proof}

\medskip \noindent {\bf (9.2)}\ (Folland problem 3.23)\ 
A variant $H^*$ of $H$ is defined by $H^*f(x) = \sup_I m(I)^{-1}
\int_I |f|$, where the supremum is taken over all bounded intervalsd
of positive lengths that contain the point $x$.
Show that for any $f\in L^1_{\rm loc}$ and any $x\in\reals$,
$Hf(x)\le H^*f(x)\le 2Hf(x)$.
\begin{proof} If $$Hf(x) = \sup_{r > 0} m(B_r(x))^{-1} \int_{B_r(x)} |f|,$$
then let $H_k$ be defined for $r_k > 0$ increasing to infinity so that $$H_kf(x) =  \sup_{r_k>r>0} m(B_r(x))^{-1} \int_{B_r(x)} |f|.$$ As $k \to \infty$ it follows that $H_kf(x) \to Hf(x).$ Next let $H^*_kf(x)$ be defined as 
$$\sup_{r_k > r > 0} \;\;\;\sup_{I \subset B_{r}(x),\; x\in I \text{ an interval}} m(I)^{-1} \int_{I} |f|$$
In the limit $H^*_k f(x) = H^* f(x).$ Additionally, since the set of balls overwhich which $H_kf(x)$ is the supremum is a subset of the set of intervals over which $H^*_kf(x)$ is the supremum it follows that for every $k$, 
$H^*_kf(x) \geq H_kf(x)$ and thus in the limit $H^*f(x) \geq Hf(x).$ 

Now consider take any $I$ with $m(I) = r.$
\begin{equation*}
	\frac{1}{m(I)} \int_I |f| = \frac{1}{r} \int_I |f| \leq \frac{1}{r}\int_{B_{r}(x)} |f| = \frac{2}{2r}\int_{B_{r}(x)} |f| = \frac{2}{m(B(r,x))} \int_{B(r,x)} |f|
\end{equation*}
Therefore for every $I$ we have that $\frac{1}{m(I)} \int_I |f| \leq 2 \frac{1}{B(m(I), x)} \int_{B(m(I), x)} |f| \leq 2 Hf(x)$. It then follows that the supremum must have the same property that $H^*f(x) \leq 2 Hf(x).$

Thus for any $f\in L^1_{\rm loc}$ and any $x\in\reals$,
$Hf(x)\le H^*f(x)\le 2Hf(x)$.
\end{proof}

\medskip \noindent {\bf (9.3)}\ (Folland problem 3.24)\ 
Show that if $f\in L^1_{\rm loc}$, and if $f$ is continuous at $x\in\reals$,
then $x$ belongs to the Lebesgue set of $f$.
That is, $\lim_{r\to 0^+} (2r)^{-1} \int_{[x-r,x+r]}|f(y)-f(x)|\,dm(y)=0$.
\begin{proof}
	We use a proof method similar to the book. If $f$ is continuous at $x$ then for every $\epsilon > 0$
	there is an $r >0$ so that for all $y \in B(r,x)$, $|f(y) - f(x)| < \epsilon.$ 
	Then 
	\begin{equation*}
	\begin{aligned}
		\frac{1}{B(r,x))}\int_{B(r,x))} |f(y) - f(x)|\ dm(y) &< \frac{1}{B(r,x))} \int_{B(r,x))} \epsilon dm(y) \\
		&\leq 
		\frac{\epsilon}{B(r,x))} \int_{B(r,x))} dm(y) = \epsilon
	\end{aligned}
	\end{equation*}
	Therefore 
	\begin{equation*}
		\lim_{r \to 0^+} \frac{1}{m(B(r,x))} \int_{B(r,x))} |f(y) - f(x)|\ dm(y) = 0
	\end{equation*}
	if $f$ is continuous at $f$ and so $x \in L_f$.
\end{proof}

\medskip \noindent {\bf (9.4)}\ (Folland problem 3.25)\ 
For any Lebesgue measurable set $E\subset\reals$
and any $x\in\reals$, the density $D_E(x)$
is defined to be the limit as $r\to 0^+$
of $\frac{m(E\cap [x-r,x+r])}{2r}$, provided that this limit exists.
\newline
(a) Prove that $D_E(x)=1$ for $m$--almost every $x\in E$, 
and $D_E(x)=0$ for $m$--almost every $x\in \reals\setminus E$.
\begin{proof}
	Let $f = \chi_E$ be a measurable $L^1_{loc}$ function. The function $f$ is well defined since $E$ is measurable and
	$\chi_E$ does not explode on finite sets. Then we have that for almost every $x \in E$,
	\begin{equation*}
		\chi_E(x) = \lim_{r\to 0}\frac{1}{m(B(r,x))} \int_{B(r,x)} \chi_E(y) dm(y) = \lim_{r\to 0}\frac{1}{m(B(r,x))} \int \chi_E(y) \chi_{B(r,x)} dm(y)
	\end{equation*}
	but then we have that
	\begin{equation*}
		\chi_E(x) = \lim_{r \to 0}\frac{m(E \cap B(r,x))}{m(B(r,x))} = D_E(x)
	\end{equation*}
	by definition of $\int \chi_F\ dm$ where $F$ measurable.
\end{proof}
(b) Let $\alpha\in(0,1)$.
Construct a set $E$ satisfying $D_E(0)=\alpha$.
\begin{proof}
(Worked with Lucas on this). Take the interval $[0,1]$ divide it into pieces with end point
 $[1/(n+1), 1/n]$. Then we will take the $\alpha$ proportion these pieces. Let $E_n = [1/(n+1), (1/n - 1/(n+1))\alpha +1/(n+1)].$
 Then $\mu(E_n) = (1/n - 1/(n+1))\alpha = \alpha/(n(n+1)).$ Luckily $\alpha \sum_1^m 1/(n(n+1)) = \alpha (1 - 1/(m+1))$ from math $1B$.
 Now take the union of these intervals and get the set $E$ so that $\mu(E) = \sum \mu(E_n) \to \alpha(1 - 0) = \alpha.$ As $r \to 0$
 we can pick an $ n > 0$ so that $1/(n-1) > r > 1/n$. Then $\mu(E \setminus \bigcup_{j=1}^nE_j) \leq m(E \cap B_r(0)) \leq \mu(E \setminus \bigcup_{j=1}^{n-1}E_j).$ But then $\mu(E \setminus \bigcup_{j=1}^nE_j) = \alpha - \alpha(1 - 1/(n+1))$ and $\mu(E \setminus \bigcup_{j=1}^{n-1}E_j) = \alpha - \alpha(1- 1/n)$ and so $\alpha - \alpha + \alpha(1 - 1/(n+1) - 1 + 1/n) = \alpha (1/n - 1/(n+1))$ is the size of the difference between the lower and upper bound on the measures  which clearly must tend to $0.$ $D_E(0)$ exists and is equal to one of its bounds
 so it must be equal to $\mu(E \setminus \bigcup_{j=1}^{n-1}E_j)/m(B_r(0)) =  (\alpha - \alpha(1 - 1/(n))n = n\alpha - n\alpha + \alpha = \alpha$. Thus $E$ satisfies the result.

\end{proof}

(c) Construct a set $E$ for which the above limit fails to exist at $x=0$.
\begin{proof}
We construct the fat cantorset. 
Start with $E_1 = [0,1]$. Then define the following recursive process. Construct $E_n+1$ by removing sub intervals of width $2^{2{n+1}}$
from the middle of each the $2^n$ intervals contained in $E_n$. Repeat this process infiniteley many times and call the final intersection $\scriptf_\mathfrak{C}^\alpha.$ This set contains the end points of every interval at each $E_n$ and is very similar to the cantor set except that applying downward measure continuity  (and then upward measure continuity to the compliment) to $E_1 \subset E_2 \subset \dots \subset E_n \subset \dots$, we get $ \mu(\cap E_n) = \lim \mu(E_n)$
and so $$\mu(\scriptf_\mathfrak{C}^\alpha) = \lim_{n\to\infty} \mu\left(E_n\right) = 1 - \mu(\bigcup_{n=1}^\infty [0,1] \setminus E_n) = 1 - \lim_{n \to \infty} \mu([0,1] \setminus E_n)$$
This right limit is the limit of the measure of the intervals removed from $E_n$ up and prior to step $n$ of the process as $n \to \infty.$ After $n$ steps we remove $2^{n-1}$ intervals of length $2^{2n+ 2}$ in addition to those removed at prior steps. Thus
\begin{equation*}
	 1 - \mu(\mathcal{F}_\mathfrak{C}^\alpha) = 1 - \sum_{n=1}^\infty \frac{2^n}{2^{2n+2}} = \sum_{n=1}^\infty \frac{1}{2^{-n}\cdot 2^{2n+2}} = 1 - \frac{1/2}{1- 1/2} = 1/2
\end{equation*}
Now we consider the limit of the density at $0$. Let $d = \lim\inf m(E_r \cap \scriptf_\mathfrak{C}^\alpha.)/m(E_r)$ and $D = \lim\sup  m(E_r \cap \scriptf_\mathfrak{C}^\alpha.)/m(E_r).$
We need only show that two different sequence of radii converge to a different limit. First take the sequence of dyadic partitions, $a_k = 1/2^k.$ Then $m(B(a_k,0) = 2a_k$. Let $k = 2f + \chi_{\mathbb{Z} \setminus 2\mathbb{Z}}(k)$ for some $n$.  Suppose that $\chi_{\mathbb{Z} \setminus 2\mathbb{Z}}(k) = 0$ Then $$m(B(a_k, 0) \cap \mathcal{F}_\mathfrak{C}^\alpha) = \sum_{2n \geq k } m[2^{-(2n + 1)}, 2^{-(2n)}] = \sum_{n \geq f + \chi_{\mathbb{Z} \setminus 2\mathbb{Z}}(k) = f} 2^{(2n + 1)} = 2^{-1}4^{-f}\cdot 4/3 = 3^{-1}\cdot 2^{-(k-1)}$$ and then the ratio is $1/3$. In the case that $\chi_{\mathbb{Z} \setminus 2\mathbb{Z}}(k) > 0$ we have that
$$m(B(a_k, 0) \cap \mathcal{F}_\mathfrak{C}^\alpha) = \sum_{2n \geq k } m[2^{-(2n + 1)}, 2^{-(2n)}] = \sum_{n \geq f+ \chi_{\mathbb{Z} \setminus 2\mathbb{Z}}(k) = f + 1} 2^{(2n + 1)} = 2^{-1}4^{-(f+1)}\cdot 4/3 = 3^{-1}\cdot 2^{-(k)}$$ 
and so the density ratio is $1/6$ in this case. Therefore the limit oscilates between $1/6$ and $1/3$ and so the sequence of average densities could never converge. 
\end{proof}
\medskip \noindent {\bf (9.5)}\ (Folland problem 3.26)\ 
Let $\lambda,\mu$ be mutually singular positive Borel measures on $\reals^n$. 
Assume that $\lambda(K)<\infty$ and $\mu(K)<\infty$
for every compact set $K\subset\reals^n$.
Prove that if $\lambda+\mu$ is outer regular, then so are $\lambda$ and $\mu$.
(A positive Borel measure $\nu$ that is finite on all compact sets is said to be outer regular 
if $\nu(E) = \inf_{E\subset\scripto}\lambda(\scripto)$ for every $E\in\scriptb_{\reals^n}$,
where the infimum is taken over all open sets $\scripto$ that contain $E$.)
\begin{proof}
	First $\lambda \perp \mu$ implies that there is a disjoint partition of the space where 
	$X = A \sqcup B$ and $B$ is $\lambda$ null and $A$ is $\mu$ null. Take $E \in \scriptm$ If $\lambda + \mu$ are outer regular
	then $(\lambda + \mu)(E) = \inf_{E\subset\scripto}(\mu(\scripto) + \lambda(\scripto))$. Now consider a sequence of openballls
	'decreasing' towards $E \cap A$, say $O^A_n$, and likewise a sequence 'decreasing' towards $E \cap B$, say $O^B_n$. We furthermore have
	a sequence of openballs $O_n = O_n^A \cup O_n^B \supset E$. We know that $(\mu+\lambda)(E) = \lim (\mu + \lambda)(O_n)$ by outer regularity.

	Now consider $\lambda(O^A_n \cup O^B_n) = \lambda((O_n^A \cup O_n^B) \cap A).$ Then this quanity is such that
	$\lambda((O_n^A \cup O_n^B) \cap A) \leq (\lambda+\mu)((O_n^A \cup O_n^B) \cap A) \leq (\lambda+\mu)(O_n^A \cap A) + (\lambda+\mu)(O_n^A \cap B).$ But the right hand side tends towards $(\lambda + \mu)(E \cap A) + (\lambda + \mu)(E \cap A \cap B) = (\lambda + \mu)(E \cap A).$
	But then by $\mu \perp \lambda,$ $(\lambda + \mu)(E \cap A) = \lambda(E \cap A) + \mu(E \cap A) = \lambda(E \cap A) = \lambda(E).$ Thus there is a sequence of opensets $O_n$ tending down to $E$ such that $\lambda(O_n) \to \lambda(E).$ This argument can be symmetricaly applied to $\mu$.


	If $\lambda + \mu$ are outer regular, then they are finite on every compact set $K$. If $K$ is compact
	then
	\begin{equation*}
		\lambda(K) \leq  (\lambda + \mu)(K) < \infty \\
		\mu(K) \leq  (\lambda + \mu)(K) < \infty \\
	\end{equation*}
	by the positivity of $\mu$ and $\lambda$.  Then $\lambda$ and $\mu$ are both respectiveley finite on all compact sets. 

	Therefore $\lambda, \mu$ are outer regular.
\end{proof}




\medskip \noindent {\bf (9.6)}\ (Folland problem 3.27)\ 
In our text, Example(s) 3.25 presents several functions, and makes statements about whether
each is of bounded variation. For each of these examples, prove that the statement is
correct. \\
(a) If $F : \mathbb{R} \to \mathbb{R}$ is bounded and increasing then $F \in BV$.
\begin{proof}
	If $F$ is bounded then there is an $M$ so that $\sup_x {|F|(x)} \leq M$. By monotonicity of $F$ for any $x_j < x_k$ we know
	that $|F(x_k) - F(x_j)| = F(x_k) - F(x_j)$. Therefore for any partition $-\infty < x_0 < x_1 < \cdots < x_n =x$,
	\begin{equation*}
		\sum_{1}^n |F(x_j) - F(x_{j-1})| = \sum_{1}^n F(x_j) - F(x_{j-1}) = F(x_n) - F(x_0) =  | F(x_n) - F(x_0)|
	\end{equation*}
	by the telescoping trick of finite summation. Then $T_F(x) = \sup{F(b)- F(a): b,a \in \mathbb{R}, x > b>a > -\infty}$
	and by monotonictiy $T_F(x) = F(x) - \lim_{x \to -\infty} \geq F(x) - M$ by the boundedness of $F$. In fact the limit converges by the monotone convergence theorem, so call the limit $F(-\infty)$ Then $\lim_{x\to \infty}$
	$T_F(x) = \lim_{x\to \infty}F(x) - F(-\infty) \leq M - F(-\infty)$ and the monotone convergence theoren gives convergence of $F(x)$ to
	say $F(\infty)$ as $x \to \infty.$ Therefore $F$ is of bounded variation on $\mathbb{R}$ and $T_F(x) = F(x) - F(-\infty).$
\end{proof}

(b) If $F, G \in BV$ and $a,b \in \mathbb{C}$ then $aF + bG \in BV.$
\begin{proof}
	First we show that scalar multiplication still closes $BV$. Take with out loss of generality $aF.$ Then 
	for any partition $-\infty < x_0 < x_1 < \cdots < x_n = x$
	\begin{equation*}
		\sum_{1}^n |aF(x_j) - aF(x_{j-1})| = \sum_{1}^n |a||F(x_j) - F(x_{j-1})| = |a|\sum_{1}^n |F(x_j) - F(x_{j-1})| \leq |a|T_F(x)
	\end{equation*}
	Thus since every sum in $T_{aF}(c)$ supremum is bounded by $|a|T_F(x)$ we have that $T_{aF}(x) \leq |a|T_F(x)$. Then since $ |a|T_F(\infty) < \infty$ by $F \in BV$ and $T_F$ increasing we have 
	\begin{equation*}
		T_{aF}(x) \leq   |a|T_F(x) \leq  |a|T_F(\infty) \implies \lim T_{aF}(x) < \infty
	\end{equation*}
	where the limit exists and is finite by the montone convergence theorem.

	Now consider the sum of two function $F +G$ where $F,G \in BV$
	Then 
	for any partition $-\infty < x_0 < x_1 < \cdots < x_n = x$
	\begin{equation*}
	\begin{aligned}
		\sum_{1}^n |F(x_j) + G(x_j)  - F(x_{j-1}) - G(x_{j-1})| &\leq  \sum_{1}^n |F(x_j) - F(x_{j-1})| + |G(x_j)   - G(x_{j-1})| \\
		&\leq \sum_{1}^n |F(x_j) - F(x_{j-1})| + \sum_{1}^n  |G(x_j)   - G(x_{j-1})| \\
		&\leq T_F(x) + T_G(x)
	\end{aligned}
	\end{equation*}
	by the triangle inequality and the positivity of both finite sums and their summands. As $T_F(x) + T_G(x)$ is an upperbound
	for every such sum over which the supremum is $T_{F+G}(x)$, we have that $T_{F+G}(x) \leq T_F(x) + T_G(x)$ for all $x$ and since 
	$\lim_{x \to \infty} T_F(x) + T_G(x) < \infty$ and $T_{F+G}(x)$ is monotone increasing $\lim T_{F+G}(x)$ exists and is finite so
	$F, G \in BV$.
\end{proof}

(c) If $F$ differentiable on $\mathbb{R}$ and $F'$ is bounded then $F \in BV([a,b])$ for $-\infty < a < b < \infty$. 
\begin{proof}
	Fix $[a,b] \subset \mathbb{R}$. Then if $F'$ is bounded then take $|F'| \leq M$ to be that bound. It follows that by the
	fundamental theorem of calculus \begin{equation*}
		|F(y) - F(x)| = \left|\int_{x}^y F'(t)\ dt\right| \leq ||y-x|2M| = 2|y-x|M.
	\end{equation*}
	and so $F$ is $2M$-Lipschitz. Then 
	for any partition $a\leq x_0 < x_1 < \cdots < x_n = x \leq b$
	\begin{equation*}
		\sum_{1}^n |F(x_j) - F(x_{j-1})| \leq \sum_{1}^n 2M|x_j - x_{j-1}| = 2M \sum_{1}^n x_j - x_{j-1} = 2M(x_n - x_0)
	\end{equation*}
	by the telescoping tricjk of summation and $x_{j-1} < x_j$. Then 
	$$T_F(b) - T_F(a) = \sup\{\ 2M(x_n - x_0) : n\in\mathbb{N}\  a\leq x_0 < x_1 < \cdots < x_n = x = b\}$$
	and by the inner regularity of the Lebesgue measure, $T_F(b) - T_F(a) = 2M m(a,b) < \infty$ and thus $F \in BV([a,b]).$
\end{proof}
(d) If $F(x) = \sin x$ then $F \in BV([a,b])$ but $F \not\in BV$. 
\begin{proof}
	Undergraduate analysis gives that $F$ is differentiable on $\mathbb{R}$. Additionally $F' = \cos x $ and so for all $x,$ $|F'(x)| \leq 1$
	and thus $F$ is $2$-Lipschitz.  Thus by the previous subexercise $F \in BV([a,b])$ for any  $-\infty < a < b < \infty$. 

	Now consider the following sum
	\begin{equation*}
		T_F(n\pi) \geq \sum_{k=1}^n |\sin(k\pi/2) - \sin((k-1)\pi/2)| = \sum_{k=1}^n 1 = n.
	\end{equation*}
	Then as $n \to \infty$ $T_F(n \pi) > n \to \infty$ and thus $T_F(x) \to \infty$ and $F \notin BV.$
\end{proof}

(e) If $F(x) = x\sin(x^{-1})$ when $x \neq 0$ and $F(x) = 0$ when $x = 0$ then show that $F \notin BV([a,b])$ for $a \leq 0 < b$ or $a < 0 \leq b$.
\begin{proof}
	We consider the first case where $a \leq 0 < b$. Without loss of generality let $b = 2$, the algebra gets messy otherwise. 
	One could reparameterized the following construction by shifting a sequence $x_j$ we will construct to do more preciseley that which we would like to be accomplished. Then 
	\begin{equation*}
	 	T_F(b) - T_F(a) = \sup\left\{ \sum_{j=1}^n |F(x_j) - F(x_{j-1}) |: n \in\mathbb{N},  a=  x_0 < x_1 < \cdots < x_n = b \right\}.
	 \end{equation*} 
	 Let $x_{j} = 2/{(n-j)\pi} \leq 2 = b$ Then 
	 \begin{equation*}
	 \begin{aligned}
	 	T_F(b) - T_F(a) &\geq \sum_{j=1}^n |F(x_j) - F(x_{j-1})|\\
	 	&= \sum_{j=1}^n \left|\frac{1}{{(n-j)\pi}} \sin((n-j)\pi/2)  - \frac{1}{{(n-j-1)\pi}} \sin((n-j-1)\pi/2))\right| \\
	 	&\geq \sum_{j=1}^n \left|\frac{1}{{(n-j-1)\pi}} \sin((n-j)\pi/2)  - \frac{1}{{(n-j-1)\pi}} \sin((n-j-1)\pi/2))\right| \\
	 	&\geq  \sum_{j=1}^n  \left|\frac{1}{{(n-j-1)\pi}}\right| = \frac{1}{\pi}\sum_{k=0}^{n} \frac{1}{n} + 1
	 \end{aligned}
	 \end{equation*}
	 Since the above inequality holds for all $n$, $T_F(b) - T_F(a) \geq \sum_{i=1}^\infty (1/n\pi)  = \infty$ and so $F$ is not in $B([a,b]).$

 	We consider the second case where $a < 0 \leq b$. Without loss of generality let $a = -2$, the algebra gets messy otherwise. 
	One could reparameterized the following construction by shifting a sequence $x_j$ we will construct to do more preciseley that which we would like to be accomplished. Then 
	\begin{equation*}
	 	T_F(b) - T_F(a) = \sup\left\{ \sum_{j=1}^n |F(x_j) - F(x_{j-1}) |: n \in\mathbb{N},  a = x_0 < x_1 < \cdots < x_n = b \right\}.
	 \end{equation*} 
	 Let $x_{j} = -2/{(n-j)\pi} \geq -2 = a$ Then 
	 \begin{equation*}
	 \begin{aligned}
	 	T_F(b) - T_F(a) &\geq \sum_{j=1}^n |F(x_j) - F(x_{j-1})|\\
	 	&= \sum_{j=1}^n \left| \frac{1}{{(n-j-1)\pi}} \sin(-(n-j-1)\pi/2)) - \frac{1}{{(n-j)\pi}} \sin(-(n-j)\pi/2) \right| \\
	 	&= \sum_{j=1}^n \left|  \frac{1}{{(n-j)\pi}} \sin(-(n-j)\pi/2) - \frac{1}{{(n-j-1)\pi}} \sin(-(n-j-1)\pi/2))\right| \\
	 	&\geq \sum_{j=1}^n \left|\frac{1}{{(n-j-1)\pi}} \sin(-(n-j)\pi/2)  - \frac{1}{{(n-j-1)\pi}} \sin(-(n-j-1)\pi/2))\right| \\
	 	&\geq  \sum_{j=1}^n  \left|\frac{1}{{(n-j-1)\pi}}\right| = \frac{1}{\pi}\sum_{k=0}^{n} \frac{1}{n} + 1
	 \end{aligned}
	 \end{equation*}
	 Since the above inequality holds for all $n$, $T_F(b) - T_F(a) \geq \sum_{i=1}^\infty (1/n\pi)  = \infty$ and so $F$ is not in $B([a,b]).$
\end{proof}


\medskip \noindent {\bf (9.7)}\ (Folland problem 3.30)\ 
Construct a nondecreasing function $f:\reals\to\reals$
with the property that $f$ is discontinuous at $x\in\reals$
if and only if $x\in\rationals$.
\begin{proof}
We will define a function $H$ which is continuous from the left at ever $x \in [0,1]$ but is only continuous from the right at the irrationals. To build such a function we need the folliowing scaffolding.

First let $\psi: \mathbb{N} \to \mathbb{Q} \cap [0,1]$ be a bijection enumerating the rationals in the interval $[0,1]$. Then define $B(x) = \{n : \psi(n)< x\}$ or equivalently $B(x) = \psi^{-1}([0,x))$. Finally let $\mu: P(\mathbb{N}) \to \overline{\mathbb{R}}$ be the counting measure. We additionally define a measure $\nu: P(\mathbb{N}) \to \mathbb{R}$ such that 
$$\nu(A) = \int_A 2^{-n} d \mu(n).$$
The measure $\nu$ has the additional property that $\nu \ll \mu$ and $\nu(\mathbb{N}) = \sum_{n \in \mathbb{N}} 2^{-n} \mu({n}) = 1 < \infty.$

We claim that the function $H(x) = \nu(B(x))$ has the properties of $f$ in the statement of the problem. We will first show that for
every $x \in [0,1]$ the function $H(x)$ is left continuous. Take a sequence of $x_k \to x$ from the left, we can then rearrange the sequence to be strict monotonic. It follows that if $k > m$ then $B(x_m) = \{n: \psi(n) < x_m < x_k < x\} \subset \{n: \psi(n) < x_k < x\} = B(x_k)$. By the 
finiteness of $\nu$ we have that by upward measure continuity
$$\lim_{k \to \infty} H(x_k) = \lim_{k \to \infty} \nu(B(x_k)) = \nu\left(\bigcup_{k=1}^\infty B(x_k) \right) = \nu(\left \{n: \psi(n) < x\}\right ) = H(x).$$
Note that if $ n \in \bigcup B(x_k)$  there is an $K$ so that $\psi(n) < x_k < x $ so any $n$ with $\psi(n) < x$ is in $\bigcup B(x_k).$

Next we claim that $H$ is only right continuous only when $x$ is irrational. Take a sequence of $x_k \to x$ from the right and rearrange the sequence to be  strict montonic. It follows that if $k > m$ then $B(x_m) = \{n: \psi(n) < x_m\} \supset \{n: \psi(n) < x_k < x_m\} = B(x_k)$. By finiteness of $\nu$ and downard measure continuity
$$\lim_{k \to \infty} H(x_k) = \lim_{k \to \infty} \nu(B(x_k)) = \nu\left(\bigcap_{k=1}^\infty B(x_k) \right) = \nu(\left \{n: \psi(n) < x_k\ \forall k\}\right ).$$

If $x$ is irrational then $m \in \{n: \psi(n) < x_k\ \forall k\}$ implies that $\psi(m) < x$ and if $\psi(m) < x$ then $\psi(m) < x_k$ for all $k$ so $\{n: \psi(n) < x_k\ \forall k\} = B(x)$ and $H(x_k) \to H(x)$ from the right. If $x$ is rational then $x= \psi(q)$ for some
$q \in \mathbb{N}.$ Thus $x < x_k \forall k$ implies that $\{n: \psi(n) < x_k\ \forall k\} = B(x) + \{q\} = D$. It follows that $\nu(D) = \nu(B(x)) + 2^{-q} > H(x)$. So $H(x_k) \to H(x) + 2^{-q} \neq H(x)$ from the right, and so $H$ is not right continuous at the rationals.

We have thus shown that for any $x \in [0,1] \setminus \mathbb{Q}$, any sequence $x_k \to x$ has the property $\lim H(x_k) = x$ from the left and the right, and if $x \in [0,1] \cap \mathbb{Q}$ then if $x_k \to x$, $\lim H(x_k)$ does not exist. Therefore $H$ is continuous at every irrational and discontinuous at every rational.
\end{proof}


\medskip \noindent {\bf (9.8)}\
Construct an example of a continuous strictly increasing function $f:[0,1]\to\reals$
(that is, $x<x'\Rightarrow f(x)<f(x')$)
whose derivative exists and is equal to $0$ at almost every $x\in(0,1)$.
\begin{proof}
	Let $F$be the standard cantor function. We extend the cantor function such that $H(x+n) = F(x) +n$ for all $n\in \mathbb{Z}$; that is repeat the cantor function increasing by the nearest $n$ each time. Then define $J(x) = \sum_{k=0}^\infty \dfrac{H(3^k x)}{4^k}.$ In Pugh's class we referred to this as the Devil's Ski slope as we claim that the function is monotone increasing continuous surjective but the derivative where it exists is positive on a zeroset.

	First, we show that $J(x)$ is well defined. Take any $x \in [0,1]$, then the cantor function $H(3^kx) \leq 3^k,$ but then it follows that
	\begin{equation*}
		J(x) \leq \sum \frac{3^k}{4^k} \leq \frac{1}{1 - \frac{3}{4}} = 4.
	\end{equation*}
	So the series in $J$ converges for all $x \in [0,1].$

	Next we show that $J$ is continuous. To do so, we invoke an argument about convergent series of continuous functions bounded by a constant; this is known as the weierstrass $M$ test. We claim  that if $\sum M_k$ is a convergent series of constants and $f_k$ are bounded functions on $[0,1]$ such that $\|f_k\| \leq M_k$ for all $k$ under the $\sup$ norm, then $\sum f_k$ converges uniformly and absoluteley. The proof of this assertion is as follows, consider the sequence of partial sums $F_k$ in $J$. It follows that by the triangle inequality and $\|\cdot\|$ a metric on $C^0[0,1]$ when $n > m$
	\begin{equation*} 
		\|F_m - F_n\| \leq \|F_m - F_{m-1} \| + \cdots \|F_{n+1} - F_n\| = \sum_{k=n+1}^m \|F_m\| \leq \sum_{k=n+1}^m M_k.
	\end{equation*}
	But the right hand series converges to $0$ by the convergence of $\sum M_k$. Therefore the sequence of partial sums is uniformly Cauchy, and by the completeness of the space of bounded continuous functions on $[0,1]$, $F_k$ converges uniformly to a limit.

	Now since $ \sum H(3^nx)/4^n$ is $\sum M_k$ absolutely bounded then it converges uniformly to a limit. Additionally by the partial sums, the sum of the continuous cantor functions, continuous, the function $J(x)$ must be continuous by uniform convergence. 

	The monotonicity of $J$ is determined as follows. Suppose that $x > y$ in $[0,1]$. Then 
	\begin{equation*}
		J(y) = \sum_{n=1}^\infty \frac{H(3^ny)}{4^n} \leq \sum_{n=1}^\infty \frac{H(3^nx)}{4^n} = J(x) 
	\end{equation*}
	since every $H$ is non-decreasing. The function $H(3^n x)$ scales $x$ by $3^n$ and then takes its modulo $n$ value via $F$ and adds its residue. To show that $J$ is strictly increasing we need find only one $n$ so that $H(3^nx) > H(3^ny)$ since we just showed that at least the difference of the sums is equal, and so if we find for every $x > y$ an $n$ with $H(3^nx) > H(3^ny)$ then the difference of $J(y) - J(x)$ will at least contain that term.

	Therefore we must find an $n$ such that $H(3^nx) \geq m \in \mathbb{Z}$ and $H(3^ny) \leq m.$ First $3^nx = a + F(3^nx - a) \leq a +1$ and $3^ny = b +r_y \leq F(3^ny) - b \leq b +1$ where $a,b \in \mathbb{Z}$. Does there exist an $n$ such that 
	\begin{equation*}
	x -y = \frac{a - b + r_x + r_y}{3^n} > \frac{5}{3^n} 
	\end{equation*}
	Since $x -y$ must have two rationals in between, the difference between those two rationals is eventually larger than $5/{3^n}$ since $5/3^n \to 0$ so the difference $3^nx - 3^ny > 1$ eventually for some $n$ and so there is a term in the sum such that $3^nx > 3^ny + 1$ and so $H(3^nx) > H(3^ny)$. Since we can find such an $n$ for every $y$ and $x$ in $0,1$ (without loss of generality $x > y)$, then $J(x) > J(y)$, and $J$ is monotonic.

	We claim that the set of $x$ for which $J'(x)$ exists and is positive is a zero set. Let $g(x) = 0$ if $x < 0$, $g(x) = x$ if $x \in [0,1]$ and $g(x) = 1$ if $x > 1$. First we construct a new function $I(x) = J(x) + \frac{g(x)}{1-1/4}.$ The series $g(x)\sum /4^n = x\sum 1/4^n$ converges to $\frac{x}{1-1/4}.$ If we take the partial sums before we have
	\begin{equation*}
		Q_m = \sum_{n=1}^m \frac{H(3^nx ) + x}{4^n} = F_m(x) + x\sum_{n=1}^m \frac{1}{4^n}
	\end{equation*}
	By the same $M$ test, $\|Q_m\| \leq \sum \frac{3^n + 1}{4^n} \to M$ so $Q_m$ converges absolutely  and uniformly to $I(x)$. Each $Q_m$ has the additional property that $Q_m$ is strictly increasing because the difference in $I$ between $y> x$ and $x$ is always greater than or equal to $g(y) - g(x) = y - x > 0$ when $x,y \in [0,1]$, additionally this argument applies to each constituent function $q_k = (H(3^nx) + g(x))/4^n$ ($k \leq m) $ of the partial sum $Q_m$.

	Now by Exercise 39 (proved after this proof) we have that if $q_k$  strictly increasing non negative functions on $[0,1]$ so that $I(x) = \sum q_k = \lim Q_m < \infty$ which is true by uniform and absolute convergence on all $x \in [0,1]$, then $Q'(x) = \sum q_k'(x)$ almost everywhere in $x$; that is if the derivatives exist at $x$
	\begin{equation*}
		J'(x) + \frac{g'(x)}{1-1/4} = Q'(x) = \sum_{n=1}^\infty \frac{\left(H(3^nx)\right)' + g'(x)}{4^n} =   \sum_{n=1}^\infty \frac{\left(H(3^nx)\right)' + 1}{4^n}  =  \sum_{n=1}^\infty \frac{\left(H(3^nx)\right)'}{4^n} + \frac{4}{3}
	 \end{equation*} 
	 Thus for almost every $x$, $J'(x) = \sum_{n=1}^\infty \frac{\left(H(3^nx)\right)'}{4^n}$.

	 For all a let $Z_a = \{x\ :\ \exists J'(x)\wedge J'(x) > a \wedge \exists k\ x \notin C_k\}$ where $C_k$ is the interval along which $H(3^kx)$ is constant. We claim $m^*(Z_a) = 0$. If $$x \in S_a  = \{x\ :\ \exists J'(x)\wedge J'(x) > a \wedge \forall k\ x \in C_k\} =  \{x\ :\ \exists J'(x)\wedge J'(x) > a\} \cap \bigcap_{k=1}^\infty C_k$$
	 then $x \in \bigcap_{k=1}^\infty C_k$ and $x \notin Z_a$. Additionally if $x \in Z_a$ then $x \notin \bigcap_{k=1}^\infty C_k$. Thus $Z_a \cap  \bigcap_{k=1}^\infty C_k = \emptyset$. As far as measure is concerned $[0,1] \setminus C_k$ is the cantor set which is measurable so we mean $m$ when we claim
	 \begin{equation*}
	 	m([0,1]) \leq m\left(\bigcap_{k=1}^\infty C_k \right) + m\left([0,1] \setminus \bigcap_{k=1}^\infty C_k\right) = 1
	 \end{equation*}
	 but then \begin{equation*}
	 	 m\left([0,1] \setminus \bigcap_{k=1}^\infty C_k\right) =  m\left( \bigcup_{k=1}^\infty [0,1] \setminus C_k\right) \leq \sum_{k=1}^\infty m([0,1] \setminus C_k) = 0
	 \end{equation*}
	 Therefore  $Z_a \cap  \bigcap_{k=1}^\infty C_k = \emptyset$ implies that $Z_a \subset m\left([0,1] \setminus \bigcap_{k=1}^\infty C_k\right)$ which is an $m$ null set so $Z_a$ is an $m$ null set.

	 Therefore we consider all most every $x \in \bigcap C_k$. For almost every $x$ in $\bigcap C_k$ we have
	 that $$J'(x) = \sum_{n=1}^\infty \frac{\left(H(3^nx)\right)'}{4^n} = 0$$
	 since $x \in C_k \iff (H(3^kx))' = 0$ thus for almost every $x$, $J'(x) = 0,$ and this completes the construction.


\end{proof}

\medskip \noindent {\bf (Suppliment)}\ (Folland problem 3.39)\ If $\{F_j\}$ is a sequence of nonnegitive increasing functions
on $[a,b]$ such that $F(x) = \sum_{1}^\infty F_j < \infty$ for all $x \in [a,b]$ then $F'(x) = \sum_{1}^\infty F'_j$ for almost
every $x \in [a,b]$. (It suffices to assume $F_j \in NBV$.)
\begin{proof}
	Consider the series of partial sums $S_n(x) = \sum_{k=1}^n F_k$. We have that $S_n \to F$ pointwise on $[a,b]$. Since $[a,b]$ is
	compact we thus have $S_n \to F$ uniformly. We would like to show that for almost every $x$,  $S_n(x)' = \sum_1^n F'_j(x) \to F'(x).$ 

	For every $j$ we have a unique complex Borel measure $\mu_{F_j}$ such that $\mu_{F_j}((-\infty, x]) = F_j(x)$ by Theorem 3.29. Now by theorem $3.22$ we have $$F_j'(x) = \lim_{r \to 0}\frac{\mu_{F_j}(E_r)}{m(E_R)}$$ since theorem 1.18 gives Borel regularity. The partial sum $S_n$ also has a measure $\mu_{S_n} = \mu_{\sum F_j}$ such that $ \mu_{\sum_1^n F_j}((-\infty, x]) = \sum_1^n F_j(x) = \sum_1^n \mu_{F_1}(j).$
	Thus $ \mu_{\sum F_j} = \sum \mu_{F_j}$ since we can generate the Borel $\sigma$-algebra from the half intervals on which the measures are equal. Since $S_n$ is a finite sum of $F_j \in NBV$ it is $NBV$ and so $ \mu_{\sum F_j}$ is NBV and borel regular. Finally we apply theorem 3.22 again and show that
	\begin{equation*}
		S_n'(x) = \lim_{r \to 0}\frac{\mu_{ \sum F_j}(E_r)}{m(E_R)} = \lim_{r \to 0}\frac{\sum_{j=1}^n \mu_{  F_j}(E_r)}{m(E_R)} = \sum_{j=1}^n  \lim_{r \to 0} \frac{\mu_{  F_j}(E_r)}{m(E_R)} = \sum_{j=1}^n F'_j(x). \;\;\;\;\;\;\;\;(\mathfrak{Z})
	\end{equation*}
	We claim that $S_n'(x)$ converges uniformly to a limit almost everywhere. First we have that by $S_n \to F$ montonically everhwhere, and so there is a $\psi: \mathbb{N} \to \mathbb{N}$ so that
	$0 \leq F - S_{\psi(n)} \leq 2^{-n}.$ Then $\sum_{n=1}^\infty F - S_{\psi(n)} \leq \sum_{n=1}^\infty 2^{-n} \leq 1.$  It follows that by 
	$F = S_\infty$ we have $\sum_{n=1}^\infty F - S_{\psi(n)} = \sum_{n=1}^\infty \sum_{j=\psi(n)+1}^\infty F_j \leq 1$. 

	Observe that for any convergent series $\sum d_k \to D$ of NBV functions $(\mathfrak{Z})$ applied to the partial sums $D_n = \sum_1^n d_k$
	$$\frac{\mu_{ \sum D_n}(E_r)}{m(E_R)} \leq \frac{\mu_{D}(E_r)}{m(E_R)} \underbrace{\implies}_\mathfrak{Z} \sum_{k=1}^\infty d_k'(x) \leq  D'(x) $$ 
	by Theorem 3.22 and the convergence of $D_n \to D$ montonically by $S_n \in NBV.$ Then applying the same argument
	to the following series with  terms $g_n = F - S_{\psi(n)}$
	\begin{equation*}
		\sum_{n=1}^\infty F' - S_{\psi(n)}'  \leq \left(\sum_{n=1}^\infty F - S_{\psi(n)}\right)' 
	\end{equation*}
	almost everywhere. So the series on the left hand side must converge and thus the summands must converge to $0$. Therefore $S'_{\psi(n)} \to F'$ and $S'_n$ increasing so $S'_n \to F'$. Therefore $\sum_{1}^\infty F'_j(x) = F'(x)$ almost everywhere.

\end{proof}


\end{document}\end
