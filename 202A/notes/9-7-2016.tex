%%%%%%%%%%%%%%%%%%%%%%%%%%%%%%%%%%%%%%%%%%%%%%%%%%%%%%%%%%%%%%%%%%
%%%                      Homework _                            %%%
%%%%%%%%%%%%%%%%%%%%%%%%%%%%%%%%%%%%%%%%%%%%%%%%%%%%%%%%%%%%%%%%%%

\documentclass[letter]{article}

\usepackage{lipsum}
\usepackage[pdftex]{graphicx}
\usepackage[margin=1.5in]{geometry}
\usepackage[english]{babel}
\usepackage{listings}
\usepackage{amsthm}
\usepackage{amssymb}
\usepackage{framed} 
\usepackage{amsmath}
\usepackage{titling}

\pagestyle{fancy}


\newtheorem{theorem}{Theorem}
\newtheorem{lemma}{Lemma}
\newtheorem{fact}{Fact}
\newtheorem{example}{Example}
\newtheorem{definition}{Definition}
\newtheorem{proposition}{Proposition}

\newenvironment{menumerate}{%
  \edef\backupindent{\the\parindent}%
  \enumerate%
  \setlength{\parindent}{\backupindent}%
}{\endenumerate}



\textheight=600pt %574pt
\textwidth=480pt %432pt
\oddsidemargin=15pt %18.88pt
\evensidemargin=15pt
\topmargin=10pt %14.21pt

\parskip=1pt %2pt

\def\reals{{\mathbb R}}
\def\torus{{\mathbb T}}
\def\integers{{\mathbb Z}}
\def\rationals{{\mathbb Q}}
\def\naturals{{\mathbb N}}
\def\complex{{\mathbb C}\/}
\def\distance{\operatorname{distance}\,}
\def\support{\operatorname{support}\,}
\def\dist{\operatorname{dist}\,}
\def\Span{\operatorname{span}\,}
\def\degree{\operatorname{degree}\,}
\def\kernel{\operatorname{kernel}\,}
\def\dim{\operatorname{dim}\,}
\def\codim{\operatorname{codim}}
\def\trace{\operatorname{trace\,}}
\def\dimension{\operatorname{dimension}\,}
\def\codimension{\operatorname{codimension}\,}
\def\kernel{\operatorname{Ker}}
\def\Re{\operatorname{Re\,} }
\def\Im{\operatorname{Im\,} }
\def\eps{\varepsilon}
\def\lt{L^2}
\def\bull{$\bullet$\ }
\def\det{\operatorname{det}}
\def\Det{\operatorname{Det}}
\def\diameter{\operatorname{diameter}}
\def\symdif{\,\Delta\,}
\newcommand{\norm}[1]{ \|  #1 \|}
\newcommand{\set}[1]{ \left\{ #1 \right\} }
\def\one{{\mathbf 1}}
\def\cl{\text{cl}}

\def\newbull{\medskip\noindent $\bullet$\ }
\def\nobull{\noindent$\bullet$\ }



\def\scriptf{{\mathcal F}}
\def\scriptq{{\mathcal Q}}
\def\scriptg{{\mathcal G}}
\def\scriptm{{\mathcal M}}
\def\scriptb{{\mathcal B}}
\def\scriptc{{\mathcal C}}
\def\scriptt{{\mathcal T}}
\def\scripti{{\mathcal I}}
\def\scripte{{\mathcal E}}
\def\scriptv{{\mathcal V}}
\def\scriptw{{\mathcal W}}
\def\scriptu{{\mathcal U}}
\def\scriptS{{\mathcal S}}
\def\scripta{{\mathcal A}}
\def\scriptr{{\mathcal R}}
\def\scripto{{\mathcal O}}
\def\scripth{{\mathcal H}}
\def\scriptd{{\mathcal D}}
\def\scriptl{{\mathcal L}}
\def\scriptn{{\mathcal N}}
\def\scriptp{{\mathcal P}}
\def\scriptk{{\mathcal K}}
\def\scriptP{{\mathcal P}}
\def\scriptj{{\mathcal J}}
\def\scriptz{{\mathcal Z}}
\def\scripts{{\mathcal S}}
\def\scriptx{{\mathcal X}}
\def\scripty{{\mathcal Y}}
\def\frakv{{\mathfrak V}}
\def\frakG{{\mathfrak G}}
\def\frakB{{\mathfrak B}}
\def\frakC{{\mathfrak C}}




%%%%%%%%%%%%%%%
%% DOC INFO %%%
%%%%%%%%%%%%%%%
\newcommand{\bHWN}{ }
\newcommand{\bCLASS}{MATH 202A}

\title{\bCLASS: Notes }
\author{Scribe: William Guss}
\usepackage{csquotes}

%%%%%%%%%%%%%%

\begin{document}
\maketitle
\thispagestyle{empty}
We will learn how to make a measure out of an outer measure.
\begin{definition}
	We call $\mu^*$ an outer measure if, $\mu^*: P(X) \to [0, \infty]$ such that  $\mu^*(\emptyset) = 0$, $\mu^*(A) \leq \mu^*(B)$ if $A \subset B$,
	and for all $E_n$,$ \mu^*(\bigcup_{n=1}^\infty E_n) \leq \sum_{j=1}^\infty \mu^*(E_j)$
\end{definition}
\begin{definition}
	Define some volume set function, $\rho: \scripte \subset P(X) \to [0, \infty]$
\end{definition}
\begin{definition}
	For any $A \subset X, X \in \scripte$, then the outer measure of $A$ is
	\begin{equation*}
		\mu^*(A) = \inf_{(E_i) \in \mathcal{E}, A \subset \bigcup E_i} \sum_{n=1}^\infty \rho(E_n).
	\end{equation*}
\end{definition}
\begin{lemma}
	Assume that $\scripte \subset P(x)$ and that $\emptyset, X \in \scripte$. Then let $\rho: \scripte \to [0, \infty]$, $\rho(\emptyset) = 0$ and 
	define $\mu^*$ as above. Then $\mu^*$ is an outer measure. 
\end{lemma}
\begin{proof}
	Clearly $\mu^*(\emptyset) = 0$ since the empty set covers itself. If $A \subset B$ then we get subaddativity. Let $A_n \subset X$. Let $A = \bigcup_{n=1}^\infty A_n$. Let $\epsilon > 0$. \emph{Naive}: For each $n$ there is $E_{n,i} \in \scripte$ so that $A_n \subset \bigcup_{i=1}^\infty E_{n,i}$ and \begin{equation*}
		\sum_{i=1}^\infty \rho^{*}(E_{n,i}) \leq \mu^{*}(A_n) + 2^{-n}\epsilon.
	\end{equation*}
	Now the family $E_{n,i}$ covers $A$. This is ac ountable family so $\mu^*(A) \leq \sum \sum \rho(E_{n,i}) \leq \sum_n \mu^*(A_n) + \sum_{n=1}^\infty 2^{-n}\epsilon$. The other direction was proven in the homeworks.
\end{proof}
\noindent \textbf{Remark}. The $2^{-n}\epsilon$ trick is effectively a divide and conquer.
\begin{fact}
	The outer measure of $\mathbb{Q}$ is 0.
\end{fact}
\begin{fact}
	The out measure of $[a,b]$ is $|b-a|$.
\end{fact}
\begin{definition}
	Let $\mu^*$ be an outer measure on $P(X)$. A set $A \subset X$ is called $\mu^*$-measurable and $A \in \scriptm$
	iff for every $E \subset X$
	\begin{equation*}
		\mu^*(E) = \mu^*(E \cap A) + \mu^*(E \setminus A).
	\end{equation*}
\end{definition}
\textbf{Remark}. $E = (E \cap A) \sqcup (E \setminus A)$ and so $
		\mu^*(E) \leq \mu^*(E \cap A) + \mu^*(E \setminus A)$ for any $A$.
\noindent \textbf{Remark}. A measurable set is like a knife which cuts any $E$ so that every part of $E$ which has mass is conserve; "there is no blood loss, or unintententional injuries on the edges."
\begin{theorem}
	If $\mu^*$ is an outer measure on $P(X)$ then $\scriptm$ is a $\sigma$-algebra and $\mu^*|_\scriptm = \mu$ is a complete measure.
\end{theorem}

\begin{proof} \emph{Sketchy proof sketch.} \\
	1. A set $A \in \scriptm$ if and only if $X \setminus A \in \scriptm$. Therefore $\mu^*(E) = \mu^*(E \cap A) + \mu^*(E \cap (X \setminus A))$. Therefore $\scriptm$ is closed under compliments.\\

	2. Suppose $A, B \in \scriptm$. Let $E \subset X.$ By $A$ measurable
	\begin{equation*}
		\begin{aligned}
			\mu^*(E) &\geq \mu^*(E \cap A) + \mu^*(E \setminus A) \\
			&\geq \mu^*(E \cap A \cap B) + \mu^*((E \cap A) \setminus B) + \mu^*((E \setminus A) \cap B) + \mu^*((E \setminus A) \setminus B) \\
			&=\mu^*(E \cap A \cap B) + \mu^*(E \cap (A \setminus B)) + \mu^*(E \cap (B \setminus A)) + \mu^*(E \setminus (A \cup B))
		\end{aligned}
	\end{equation*}
	Now we want to show that $\mu^*E) \geq \mu^*(E \cap (A \cup B)) + \mu^*(E \setminus (A \cup B))$. We can use that $\mu^*(E \cap (A \cup B) \leq \mu^*(E \cap ( A \cap B)) + \mu^*(E \cap ( A \setminus B)) + ...$ So $\scriptm$ is closed under unions.

	3. Assume $A_n$ measurable for $n \in \mathbb{N}$ if $A = \bigcup A_n$ then want to show $A$ measurable. Without loss of generality, assume that $A_1 \subset A_2 \subset A_3 \subset \cdots$. Set $B_1 = A_1,$ $B_n = A_n \setminus A_{n-1}$, and $\bigcup B_n = A$. So each $B_n \in \scriptm$ and they are pairwise disjoint. Let $E \subset X$....
\end{proof}


%%%%%%% Be sure to set the counter and use menumerate

\end{document}	