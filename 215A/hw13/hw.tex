\documentclass[11pt]{amsart}

\usepackage{amsmath,amsthm}
\usepackage{amssymb}
\usepackage{graphicx}
\usepackage{enumerate}
\usepackage{fullpage}
\usepackage{tikz-cd}
% \usepackage{euscript}
% \makeatletter
% \nopagenumbers
\usepackage{verbatim}
\usepackage{color}
\usepackage{hyperref}

\usepackage{fullpage,tikz,float}
%\usepackage{times} %, mathtime}

\textheight=600pt %574pt
\textwidth=480pt %432pt
\oddsidemargin=15pt %18.88pt
\evensidemargin=18.88pt
\topmargin=10pt %14.21pt

\parskip=1pt %2pt

% define theorem environments
\newtheorem{theorem}{Theorem}    %[section]
%\def\thetheorem{\unskip}
\newtheorem{proposition}[theorem]{Proposition}
%\def\theproposition{\unskip}
\newtheorem{conjecture}[theorem]{Conjecture}
\def\theconjecture{\unskip}
\newtheorem{corollary}[theorem]{Corollary}
\newtheorem{lemma}[theorem]{Lemma}
\newtheorem{sublemma}[theorem]{Sublemma}
\newtheorem{fact}[theorem]{Fact}
\newtheorem{observation}[theorem]{Observation}
%\def\thelemma{\unskip}
\theoremstyle{definition}
\newtheorem{definition}{Definition}
%\def\thedefinition{\unskip}
\newtheorem{notation}[definition]{Notation}
\newtheorem{remark}[definition]{Remark}
% \def\theremark{\unskip}
\newtheorem{question}[definition]{Question}
\newtheorem{questions}[definition]{Questions}
%\def\thequestion{\unskip}
\newtheorem{example}[definition]{Example}
%\def\theexample{\unskip}
\newtheorem{problem}[definition]{Problem}
\newtheorem{exercise}[definition]{Exercise}

\numberwithin{theorem}{section}
\numberwithin{definition}{section}
\numberwithin{equation}{section}

\def\reals{{\mathbb R}}
\def\torus{{\mathbb T}}
\def\integers{{\mathbb Z}}
\def\rationals{{\mathbb Q}}
\def\naturals{{\mathbb N}}
\def\complex{{\mathbb C}\/}
\def\distance{\operatorname{distance}\,}
\def\support{\operatorname{support}\,}
\def\dist{\operatorname{dist}\,}
\def\Span{\operatorname{span}\,}
\def\degree{\operatorname{degree}\,}
\def\kernel{\operatorname{kernel}\,}
\def\dim{\operatorname{dim}\,}
\def\codim{\operatorname{codim}}
\def\trace{\operatorname{trace\,}}
\def\dimension{\operatorname{dimension}\,}
\def\codimension{\operatorname{codimension}\,}
\def\nullspace{\scriptk}
\def\kernel{\operatorname{Ker}}
\def\p{\partial}
\def\Re{\operatorname{Re\,} }
\def\Im{\operatorname{Im\,} }
\def\ov{\overline}
\def\eps{\varepsilon}
\def\lt{L^2}
\def\curl{\operatorname{curl}}
\def\divergence{\operatorname{div}}
\newcommand{\norm}[1]{ \|  #1 \|}
\def\expect{\mathbb E}
\def\bull{$\bullet$\ }
\def\det{\operatorname{det}}
\def\Det{\operatorname{Det}}
\def\rank{\mathbf r}
\def\diameter{\operatorname{diameter}}

\def\t2{\tfrac12}

\newcommand{\abr}[1]{ \langle  #1 \rangle}

\def\newbull{\medskip\noindent $\bullet$\ }
\def\field{{\mathbb F}}
\def\cc{C_c}



% \renewcommand\forall{\ \forall\,}

% \newcommand{\Norm}[1]{ \left\|  #1 \right\| }
\newcommand{\Norm}[1]{ \Big\|  #1 \Big\| }
\newcommand{\set}[1]{ \left\{ #1 \right\} }
%\newcommand{\ifof}{\Leftrightarrow}
\def\one{{\mathbf 1}}
\newcommand{\modulo}[2]{[#1]_{#2}}

\def\bd{\operatorname{bd}\,}
\def\cl{\text{cl}}
\def\nobull{\noindent$\bullet$\ }

\def\scriptf{{\mathcal F}}
\def\scriptq{{\mathcal Q}}
\def\scriptg{{\mathcal G}}
\def\scriptm{{\mathcal M}}
\def\scriptb{{\mathcal B}}
\def\scriptc{{\mathcal C}}
\def\scriptt{{\mathcal T}}
\def\scripti{{\mathcal I}}
\def\scripte{{\mathcal E}}
\def\scriptv{{\mathcal V}}
\def\scriptw{{\mathcal W}}
\def\scriptu{{\mathcal U}}
\def\scriptS{{\mathcal S}}
\def\scripta{{\mathcal A}}
\def\scriptr{{\mathcal R}}
\def\scripto{{\mathcal O}}
\def\scripth{{\mathcal H}}
\def\scriptd{{\mathcal D}}
\def\scriptl{{\mathcal L}}
\def\scriptn{{\mathcal N}}
\def\scriptp{{\mathcal P}}
\def\scriptk{{\mathcal K}}
\def\scriptP{{\mathcal P}}
\def\scriptj{{\mathcal J}}
\def\scriptz{{\mathcal Z}}
\def\scripts{{\mathcal S}}
\def\scriptx{{\mathcal X}}
\def\scripty{{\mathcal Y}}
\def\frakv{{\mathfrak V}}
\def\frakG{{\mathfrak G}}
\def\aff{\operatorname{Aff}}
\def\frakB{{\mathfrak B}}
\def\frakC{{\mathfrak C}}

\def\symdif{\,\Delta\,}
\def\mustar{\mu^*}
\def\muplus{\mu^+}

\def\soln{\noindent {\bf Solution.}\ }


%\pagestyle{empty}
%\setlength{\parindent}{0pt}

\begin{document}

\begin{center}{\bf Math 215A --- Homework 13--- UCB, Spring 2017 --- William Guss} \\
Partners: Alekos, Chris \\
Selected Problems: 1,2,3
\end{center}

\medskip \noindent {\textbf{(13.1) } \emph{(Automated homology computations)}: Use \verb|polymake| to compute the homology of the $2$-skeleton of the $4$-cube. Furthermore use \verb|polymake| to comptue the homology of the suspension of the orientable genus $5$ surface.

\medskip \noindent \emph{Solution}. First we will compute the homology of the $2$-skeleton of the $4$-cube. Polymake convieniently provides functions for building such a construction. 
\begin{verbatim}
polytope > application 'topaz';
topaz > $cc=cube_complex(1,1,1,1); # Makes a 4-cube 
topaz > $2skel = k_skeleton($cc, 2);
\end{verbatim}
The above code produces the following TIkz visualization.
\begin{center}
	\input{myfile.tikz}
\end{center}
Finally we can compute the homology
\begin{verbatim}
topaz > print $skel->HOMOLOGY;
({} 0)
({} 0)
({} 60)
\end{verbatim}
Therefore the simplicial homology is a long exact sequence
\begin{equation*}
	\cdots \to 0\to \mathbb{Z}^{60} \to 0 \to 0
\end{equation*}
with $H_2(\text{2skel}; \mathbb{Z}) = \mathbb{Z}^{60}.$ Note that this is the simplicial $2$-skeleton, which has a $H_2$ depending on how many simplices were used to approximate the $4$-cube.

Next we compute the homology of the suspension of the orientable genus $5$ surface.
\begin{verbatim}
topaz > $cc = surface(5);
topaz > $sus = suspension($cc, 1);
topaz > print $sus->HOMOLOGY;
({} 0)
({} 0)
({} 10)
({} 1)
\end{verbatim}
Before suspension, we have the following surface
\begin{center}
	\input{sus.tikz}
\end{center}
Using topaz, we yield that the homology is simply a long exact seuence
\begin{equation*}
	0 \to \mathbb{Z} \to \mathbb{Z}^{10} \to 0 \to 0
\end{equation*}
with $H_2(\text{sus}; \mathbb{Z}) = \mathbb{Z}^{10}$ and $H_1(\text{sus}; \mathbb{Z}) = \mathbb{Z}.$ This completes the exercise.

\medskip \noindent \textbf{(13.2)} \emph{(Simplicial Approximation):} Use the simplicial approximation theorem to prove that $S^{n}$ is $(n-1)$-connected, ie. that the homotopy groups vanish below dimension $n$.

\begin{proof}
	First consider the $(n-1)$-sphere, $S^{n-1}$. Any simplicial triangulation of $S^{n-1}$ cannot contain an $n$-simplex because $S^{n-1}$ is locally homeomorphic to $\mathbb{R}^{n-1}$. Let $T^{n}$ denote some simplicial triangulation of the $n$-sphere. To show that $S^n$ is $n-1$ connected we will show that the homotopy groups vanish below dimension $n$; that is any continuous map $S^{m} \to S^n$ is homotopic to the constant map when $m < n$. 

	For any continuous map $F: |T^m| = S^m\to S^n = |T^n|$, there exists an $r$ and a simplicial approximation $f: Bd^{r}(T^m) \to T^n$ so that $|f| \simeq F$ where $Bd^r(\cdot)$ denotes the $r$th Barycentric subdivision. Since $f$ is a simplicial map, $|F|(S^m)$ is not contained in the interior of $k > m$ simplices of $T^n$. To see this, note that
	the Barycentric subdivision of $T^m$ (to any degree) does contain higher dimensional simplices by definition and additionally that $T^m$ does not contain any $k$-simplices inductively when $k > m.$ Therefore $|f|$ must miss at least one point in the interior of some $n$ simplex in $T^n$ and therefore $F \simeq |f|:S^m \to S^n \setminus \{pt\}$.

	Since $S^n \setminus \{pt\}$ is homeomorphic to $\mathbb{R}^{n-1}$, say through $h$, then by $\mathbb{R}^{n-1}$ contractible, $h \circ f \simeq c$ where $c$ is the constant map. Composing this homotopy with $h^{-1}$ we yield that $h^{-1} \circ h\circ f \simeq h^{-1} \circ c = f:S^m \to \{pt\} \subset S^n;$ that is $F \simeq f$ is nullhomotopic.

	Therefore $\pi_m(S^n) = 0$ when $m < n$ and so $S^n$ is $(n-1)$-connected. This completes the proof.
\end{proof}



\medskip \noindent {\bf (13.3) [Already Graded]}\ \emph{(Vector-fields and Euler characteristic):} Let $\scriptm$ be a compact smooth manifold with $\chi(\scriptm) \neq 0$. If $F$ is a mapping from $\scriptm$ onto the tangent bindle so that $\pi \circ F = id$ where $\pi: T\scriptm \to M$, then there exists a $x \in \scriptm$ so that $F(x) = 0.$ 

\medskip\noindent 
\begin{proof}
  Suppose that for all $x$ $F(x) \neq 0$. Then by Corollary 11.5 of Bredon, there exists a map $f: M \to M$ without fixed points and $f \simeq id$. By the Whitney embedding theorem $\scriptm$ is a compact smooth manifold and so it is embeddable in $\mathbb{R}^k$ for some $k$. Therefore it is homeomorphic to a retract of the bounding open ball of the compact image. Thus $\scriptm$ is a Euclidean neighboorhood retract and Corollary 23.5 gives that if $\scriptm$ is a compact ENR, then  $L(f) \neq 0$ implies $f$ has a fixed point.

  It remains to show that $L(f) \neq 0$. Recall that 
  \begin{equation*}
    L(f) = \sum_{i}(-1)^i \text{tr}_i(f_*)
  \end{equation*}
  where $f_*: H_i(\scriptm; \mathbb{Z}) \to H_i(\scriptm; \mathbb{Z})$ is the homomorphism of homology induced by $f$. Since $f \simeq \text{id}$ then the homotopy invariance of homology homomorphisms\footnote{See Concise Alg. Top.  Chapter 12 for a nice proof!} gives that $0 = L(f) = L(\text{id}).$ But then 
  \begin{equation*}
    L(\text{id}) = \sum_{i} \text{tr}_i(\text{id}_*) = \sum_i \text{rank}(H_i(\scriptm; \mathbb{Z}\textbf{})) = \chi(\scriptm) 
   \end{equation*}  which contradicts $\chi(\scriptm) = 0$. Therefore there is an $x$ so that $F(x) = 0$. This completes the proof.
\end{proof}


 \end{document}\end
