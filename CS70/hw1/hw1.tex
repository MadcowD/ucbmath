%%%%%%%%%%%%%%%%%%%%%%%%%%%%%%%%%%%%%%%%%%%%%%%%%%%%%%%%%%%%%%%%%%
%%%                      Homework 1                            %%%
%%%%%%%%%%%%%%%%%%%%%%%%%%%%%%%%%%%%%%%%%%%%%%%%%%%%%%%%%%%%%%%%%%

\documentclass[letter]{article}

\usepackage{lipsum}
\usepackage[pdftex]{graphicx}
\usepackage[margin=1.5in]{geometry}
\usepackage[english]{babel}
\usepackage{listings}
\usepackage{amsthm}
\usepackage{amssymb}
\usepackage{framed} 
\usepackage{amsmath}
\usepackage{titling}
\usepackage{fancyhdr}

\pagestyle{fancy}


\newtheorem{theorem}{Theorem}
\newtheorem{definition}{Definition}

\newenvironment{menumerate}{%
  \edef\backupindent{\the\parindent}%
  \enumerate%
  \setlength{\parindent}{\backupindent}%
}{\endenumerate}







%%%%%%%%%%%%%%%
%% DOC INFO %%%
%%%%%%%%%%%%%%%
\newcommand{\bHWN}{1}
\newcommand{\bCLASS}{CS 70}

\title{\bCLASS: Homework \bHWN}
\author{William Guss\\26793499\\wguss@berkeley.edu}

\fancyhead[L]{\bCLASS}
\fancyhead[CO]{Homework \bHWN}
\fancyhead[CE]{GUSS}
\fancyhead[R]{\thepage}
\fancyfoot[LR]{}
\fancyfoot[C]{}
\usepackage{csquotes}

%%%%%%%%%%%%%%

\begin{document}
\maketitle
\thispagestyle{empty}
%%%%%%% Be sure to set the counter and use menumerate
\begin{menumerate}
    \item Watsons experiment.
    \begin{theorem}
        If a person has ice cream for desert, he/she has to do the dishes after dinner.
    \end{theorem}
    \begin{proof}
        Flip Charlie and Bob.
    \end{proof}

    \item For the following answers I employed a truth table generator as a latex extension.
    This is a programmatic method of proof, but it does not detract from the argument. 
    \begin{menumerate}
        \item 
        \begin{theorem}
            $A\ \vee\ ( B \wedge C ) \equiv (A \vee B) \wedge (A \vee C)$
        \end{theorem}
        \begin{proof}
            On the left hand side we have that 
            \begin{equation*}
                \begin{array}{ccc|ccc@{}ccc@{}c}
                a&b&c&a&\lor&(&b&\land&c&)\\\hline
                1&1&1&1&\mathbf{1}&&1&1&1&\\
                1&1&0&1&\mathbf{1}&&1&0&0&\\
                1&0&1&1&\mathbf{1}&&0&0&1&\\
                1&0&0&1&\mathbf{1}&&0&0&0&\\
                0&1&1&0&\mathbf{1}&&1&1&1&\\
                0&1&0&0&\mathbf{0}&&1&0&0&\\
                0&0&1&0&\mathbf{0}&&0&0&1&\\
                0&0&0&0&\mathbf{0}&&0&0&0&
                \end{array}
            \end{equation*}
            On the right hand side we have
            \begin{equation*}
                \begin{array}{ccc|c@{}ccc@{}ccc@{}ccc@{}c}
                a&b&c&(&a&\lor&b&)&\land&(&a&\lor&c&)\\\hline
                1&1&1&&1&1&1&&\mathbf{1}&&1&1&1&\\
                1&1&0&&1&1&1&&\mathbf{1}&&1&1&0&\\
                1&0&1&&1&1&0&&\mathbf{1}&&1&1&1&\\
                1&0&0&&1&1&0&&\mathbf{1}&&1&1&0&\\
                0&1&1&&0&1&1&&\mathbf{1}&&0&1&1&\\
                0&1&0&&0&1&1&&\mathbf{0}&&0&0&0&\\
                0&0&1&&0&0&0&&\mathbf{0}&&0&1&1&\\
                0&0&0&&0&0&0&&\mathbf{0}&&0&0&0&
                \end{array}
            \end{equation*}
            Since these exhibit ident ical truth values,
            they myust therefore be the same. 
        \end{proof}
        \item 
        \begin{theorem}
          $A \wedge (B \vee C) \equiv (A \wedge B) \vee (A \wedge C).$
        \end{theorem}
        \begin{proof}
          On the left hand side it follows that,
          \begin{equation*}
              \begin{array}{ccc|ccc@{}ccc@{}c}
                a&b&c&a&\land&(&b&\lor&c&)\\\hline
                1&1&1&1&\mathbf{1}&&1&1&1&\\
                1&1&0&1&\mathbf{1}&&1&1&0&\\
                1&0&1&1&\mathbf{1}&&0&1&1&\\
                1&0&0&1&\mathbf{0}&&0&0&0&\\
                0&1&1&0&\mathbf{0}&&1&1&1&\\
                0&1&0&0&\mathbf{0}&&1&1&0&\\
                0&0&1&0&\mathbf{0}&&0&1&1&\\
                0&0&0&0&\mathbf{0}&&0&0&0&
              \end{array}
          \end{equation*}
          On the right hand side the truth table gives
          \begin{equation*}
              \begin{array}{ccc|c@{}ccc@{}ccc@{}ccc@{}c}
              a&b&c&(&a&\land&b&)&\lor&(&a&\land&c&)\\\hline
              1&1&1&&1&1&1&&\mathbf{1}&&1&1&1&\\
              1&1&0&&1&1&1&&\mathbf{1}&&1&0&0&\\
              1&0&1&&1&0&0&&\mathbf{1}&&1&1&1&\\
              1&0&0&&1&0&0&&\mathbf{0}&&1&0&0&\\
              0&1&1&&0&0&1&&\mathbf{0}&&0&0&1&\\
              0&1&0&&0&0&1&&\mathbf{0}&&0&0&0&\\
              0&0&1&&0&0&0&&\mathbf{0}&&0&0&1&\\
              0&0&0&&0&0&0&&\mathbf{0}&&0&0&0&
              \end{array}
          \end{equation*}
          There is logical equivalence and the proof is complete.
        \end{proof}

        \item
        \begin{theorem}
            $A \implies (B \wedge C) \equiv (A \implies B) \wedge (A \implies C)       $
        \end{theorem}
        \begin{proof}
            Let $Q = (B \wedge C).$ Then $A \implies Q$ if and only if
            $\neg A \vee Q.$ And so, $\neg A \vee ( B \wedge C)$
            if and only if $(\neg A \vee B) \wedge (\neg A \vee C)$ by theorem $2$ 
            All of that holds if and only if $(A \implies B) \wedge (A \implies C).$
            This completes the proof.
        \end{proof}

        \item
        \begin{theorem}
            $A \implies (B \vee C) \equiv (A \implies B) \vee (A \implies C)$ 
        \end{theorem}
        \begin{proof}
         Let $Q = (B \vee C).$ Then $A \implies Q$ if and only if
            $\neg A \vee Q.$ And so, $\neg A \vee (B \vee  C)$
            if and only if $(\neg A \vee  B) \vee (\neg A \vee  C)$ by associativity. 
            All of that holds if and only if $(A \implies B) \vee (A \implies C).$
            This completes the proof.
        \end{proof}
    \end{menumerate}
    \item Justify equivalence.
    \begin{menumerate}
        \item There exists an equivalence since the only use of $y$ is for the expressioin
        involving $Q(x,y).$ In particular the implication is equivalent to $\not P(x) \vee Q(x,\pmb{y}).$ 
        So it follows that $\exists$ can be inserted deeper within the statement.
        \item Since negation flips qualifiers we have the following logic,
        \begin{equation}
          \begin{aligned}
            &\neg \exists x \forall y (P(x) \implies \neq Q(x,y)) \\
            \iff&  \forall x \neg \forall y (P(x) \implies \neq Q(x,y)) \\
            \iff&  \forall x \exists y \neg (P(x) \implies \neq Q(x,y)) \\
            \iff&  \forall x \exists y  \neg(\neg P(x) \vee \neq Q(x,y)) \\
            \iff&  \forall x \exists y  (\neg(\neg P(x)) \wedge \neg(\neq Q(x,y))) \\
            \iff&  \forall x \exists y  ( P(x) \wedge  Q(x,y)).
          \end{aligned}  
        \end{equation}  
        Therefore, the equivalence holds.

        \item There is not an equivalence by the following argument:
          \begin{equation}
            \begin{aligned}
              & \forall x \exists y (Q(x,y) \implies P(x)) \\
              \iff &\forall x \exists y (\neg Q(x,y) \vee P(x)) \\
              \iff &\forall x \exists y \neg Q(x,y) \vee P(x) \\
              \iff &\forall x \neg \forall y Q(x,y) \vee P(x) \\
              \iff &\forall x (\neg (\forall y Q(x,y)) \vee P(x)) \\
              \iff &\forall x (\forall y Q(x,y)) \implies P(x)) \\
            \end{aligned}
          \end{equation}
          Whichn is certainly not equal to the right hand side.
    \end{menumerate}

    \item Prove or disprove!
    \begin{menumerate}
        \item 
        \begin{theorem}
            The following is true. For every $x$ there exists a $y$ such that $xy > 0$ implies $y > 0.$
        \end{theorem}
        \begin{proof}
           Fix $x.$ Then take any $y > 0.$ Clearly, $y >0,$ and so the implication is always true
           since it is equivalent to $xy \leq 0$ or $y > 0.$ This completes the proof.
        \end{proof}
        \item 
        \begin{theorem}
          The following is false. There exists a $x$ such that for all $y$, $xy < x^2.$
        \end{theorem}
        \begin{proof}
          Suppose it were true. Then consider the rectangle of side-length $x.$
          The closed and bounded set $S_y = [0,x]\times[0,y]$ must then have outer measure
          less than that of $X =[0,x]^2$ for all $x.$ Since $x \in \mathbb{R},$
          we have that $\forall y, m(S_y) < X.$ Then take the sequence $\{a_n\}_{n \in \mathbb{N}}$
          where $a_n = n$. The mesure sequence $(m(S_{a_n})$ is bounded and monotone increasing 
          by the initial supposition, so by the monotone convergence theorem, it converges.

          Since the measure sequence is bounded and $S_y$ is a closed and bounded compact set
          for all $y,$ we have that the sequence of diameters is bounded and converges $diam(S_{a_n})$.
          Furthermore the diameter of such a set is then dominated by $a_n$ by the archimedian property.
          So we have that $a_n \to a \in \mathbb{R}.$ A contradiction  to the unboundedness of $\mathbb{N}!$

          This completes the proof without loss of generality since negative rectangles make sense from
          a measure theory prospective. 
        \end{proof}
        \item 
        \begin{theorem}
          There exist a $y$ such that for all $x$, $xy \geq x^2.$
        \end{theorem}
        \begin{proof}
          Take the sequence $a_n = n.$ Then if there existed $y$ such that $ny \geq n^2$, then $y \geq n$ for all $n,$
          a contradiction to the archimedian property of $\mathbb{R}.$ QED
        \end{proof}
    \end{menumerate}

      \item Problems concerning ducks.
      \begin{menumerate}
        \item
          \begin{menumerate}
           \item $\forall x D(x) \implies  I(x).$
           \item $\forall x V(x) \implies H_{issues}(x)$
           \item $\forall x C(x) \implies \neg W(x)$
           \item $\forall x H_{issues}(x) \implies W(x)$
           \item $\forall x  I(x) \implies C(x)$
           \item $\forall x P(x) \implies V(x) $
          \end{menumerate}
        \item 
          \begin{menumerate}
           \item $\forall x  \neg I(x) \implies \neg D(x)$
           \item $\forall x \neg H_{issues}(x) \implies \neg V(x)$
           \item $\forall x  W(x) \implies \neg C(x)$
           \item $\forall x \neg W(x) \implies \neg H_{issues}(x) $
           \item $\forall x \neg C(x) \implies \neg I(x)$
           \item $\forall x \neg V(x) \implies \neg P(x) $
          \end{menumerate}
        \item We use the following argument
        \begin{equation*}
          \begin{aligned}
            P(x) &\implies V(x) \\
            &\implies H_{issues}(x) \\
            &\implies W(x) \\
            &\implies \neg C(x) \\
            &\implies \neg I(x) \\
            &\implies \neg D(x).
          \end{aligned}
        \end{equation*}
        to conclude that those who wear party hats vote; and so
        they have done their homework on the issues; and so
        they are well informed; and so they are not confused;
        and so they have read the candidates positions; and so
        they are not a Duck dynasty viewer.
      \end{menumerate}
    \item 
    \begin{menumerate}
      \item The following truth table is produced
      \begin{equation}
      \begin{array}{cccc|c@{}ccccccccccc@{}ccc@{}cccccccccc@{}ccc@{}cccccccccc@{}ccc@{}ccccccccc@{}c}
        a&b&c&d&O(a,b,c,d)\\\hline
        1&1&1&1&\mathbf{0}\\
        1&1&1&0&\mathbf{0}\\
        1&1&0&1&\mathbf{0}\\
        1&1&0&0&\mathbf{0}\\
        1&0&1&1&\mathbf{0}\\
        1&0&1&0&\mathbf{1}\\
        1&0&0&1&\mathbf{0}\\
        1&0&0&0&\mathbf{1}\\
        0&1&1&1&\mathbf{0}\\
        0&1&1&0&\mathbf{0}\\
        0&1&0&1&\mathbf{0}\\
        0&1&0&0&\mathbf{0}\\
        0&0&1&1&\mathbf{0}\\
        0&0&1&0&\mathbf{1}\\
        0&0&0&1&\mathbf{0}\\
        0&0&0&0&\mathbf{1}\\
        \end{array}
      \end{equation}
      \item Thereby giving the following Karneugh table:
      \begin{equation}
        \begin{array}{c|c|c|c|c|}
              &00&01&11&10 \\\hline
            00&1&0&0&1\\\hline
            01&0&0&0&0\\\hline
            11&0&0&0&0\\\hline
            10&1&0&0&1\\\hline
        \end{array}
      \end{equation}
      \item It is equivalent to $\neg B \wedge \neg D.$
      This follows since we have $(\neg B \wedge \neg D) \vee \neg ( A \vee C) \vee \neg (A \vee \neg C) \vee \neg (\neg A \vee C) \vee \neg (\neg A \vee \neg C).$
      And so we have cancellation.
    \end{menumerate}
    \item Proof by contrapositive
    \begin{enumerate}
        \item \begin{theorem}
            If $x,y, a \in \mathbb{Z}$ if $a$ does not divide $xy$, then 
            $a$ does not divide $x$ and $x$ does not divide $y.$
        \end{theorem}   
        \begin{proof}
          Suppose that $a | x$ or $a | y$. Then there exists a $k$ so that
          $ka = x$ or $ma = y.$ Then $xy = kay$ or $xy = max$. In either case 
          $a | xy.$ Take the contraposition and the theorem holds.
        \end{proof}
        \item See the proof of (a).
        \item Consider the case when $a =9,$ $b = 12$, $c = 30,$ clearly $9$ doesnt divide 12 and 30, but it does divide $360.$
        So the converse is not true. 
    \end{enumerate}
    \item Proof time.
    \begin{menumerate}
        \item Direct
        \begin{theorem}
            For all natural numbers n, if n is odd then $n^2 + 3n$ is even.
        \end{theorem}
        \begin{proof}
            if $n$ is odd, then $n = 2k+1,$ it follows that $n^2 = 4k^2+ 4k + 1$ and $3n = 6k + 3$,
            so $n^2 + 3n = 4k^2 + 10k + 4$ which is divisible by $2.$
        \end{proof}

        \item Direct
        \begin{theorem}
            For all natural numbers $n$, $n^2 + 7n$ is even. 
        \end{theorem}
        \begin{proof}
            If $n$ is even $n^2 + 7n= 4k^2 + 14k$, and the theorem is complete.
            If $n$ is odd then $n^2 + 7n = n^2 + 3n +4n$ which is $2m + 4n$ by the previous theorem
            and so $n^2 + 7n$ is divisible by 2.
            This completes the proof.
        \end{proof}

        \item Contraposition
        \begin{theorem}
            If $a,b \in \mathbb{R}$ and $a + b \geq 10$ then $a \geq 7$ or $b\geq 3.$
        \end{theorem}
        \begin{proof}
            Consider the contrapositive. If $a < 7$ and $b < 3$ then $a + b < 7 +3 = 10.$
            There fore $a +b \geq 10$ implies $a \geq 7$ or $b \geq 3.$
        \end{proof}

        \item Contraposition
        \begin{theorem}
        If $r \in \mathbb{Q}^c$ the $r+1 \in \mathbb{Q}^c.$
        \end{theorem}
        \begin{proof}
        Consider the contrapositive. If $r + 1 \in \mathbb{Q}$ then
        $r + 1 = \frac{a}{b}$ and $r = \frac{a}{b} -\frac{b}{b} = \frac{a-b}{b}$.
        So $r \in \mathbb{Q}.$ Therefore the contrapositive holds and the proof is complete.
        \end{proof}

        \item Counterexample
        \begin{proof}
            Take $n = 100,$ then clearly $1000 < 100*10*9 < 100!.$
        \end{proof}

        \item Contrapositive,
        \begin{theorem}
            For all natural numbers $a$ where $a^5$ is odd.
        \end{theorem}
        \begin{proof}
            Consider the contrapositive. If $a$ is even the $a = 2k,$ and $a^5 = 2^5k$ which is
            divisible by $2.$ So the contrapositive holds. This completes the proof.
        \end{proof}
    \end{menumerate}

\end{menumerate}


\end{document}