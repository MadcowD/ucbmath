%%%%%%%%%%%%%%%%%%%%%%%%%%%%%%%%%%%%%%%%%%%%%%%%%%%%%%%%%%%%%%%%%%
%%%                      Challenge 1                           %%%
%%%%%%%%%%%%%%%%%%%%%%%%%%%%%%%%%%%%%%%%%%%%%%%%%%%%%%%%%%%%%%%%%%

\documentclass[letter]{article}

\usepackage{lipsum}
\usepackage[pdftex]{graphicx}
\usepackage[margin=1.5in]{geometry}
\usepackage[english]{babel}
\usepackage{listings}
\usepackage{amsthm}
\usepackage{amssymb}
\usepackage{framed} 
\usepackage{amsmath}
\usepackage{titling}
\usepackage{fancyhdr}

\pagestyle{fancy}


\newtheorem{theorem}{Theorem}
\newtheorem{definition}{Definition}

\newenvironment{menumerate}{%
  \edef\backupindent{\the\parindent}%
  \enumerate%
  \setlength{\parindent}{\backupindent}%
}{\endenumerate}







%%%%%%%%%%%%%%%
%% DOC INFO %%%
%%%%%%%%%%%%%%%
\newcommand{\bHWN}{Explorations}
\newcommand{\bCLASS}{MATH H104}

\title{\bCLASS: Homework \bHWN}
\author{William Guss\\26793499\\wguss@berkeley.edu}

\fancyhead[L]{\bCLASS}
\fancyhead[CO]{Homework \bHWN}
\fancyhead[CE]{GUSS}
\fancyhead[R]{\thepage}
\fancyfoot[LR]{}
\fancyfoot[C]{}
\usepackage{csquotes}

%%%%%%%%%%%%%%

\begin{document}
\maketitle
\thispagestyle{empty}


%%%%%%% Be sure to set the counter and use menumerate
\setcounter{section}{1}
\section{Some Interesting Problems}

Professor Pugh, as per your suggestion.
Here are some of the difficult problems I've done (attempted).

 \begin{theorem}
	Let $S,R \subset \mathbb{R}$ be closed intervals and $\Sigma_l : C(R) \to C(S)$ be a linear operator such that $$\xi \mapsto \int_R \xi(i)w(i,j)\ di$$ then for every $\epsilon >0$ and every $\xi : R \to Q \subset S$, there exists a weight function $w(i,j)$ such that the supremum norm over $S$ 
	\begin{equation}
		\left\|K\xi -\Sigma_{l+1}\xi\right\|_{\infty} < \epsilon
	\end{equation}
\end{theorem}
\begin{proof}
	 Let $\zeta_t :C(R)\to S$ be a linear form which evaluates its arguments at $t\in R$; that is, $\zeta_t(f) = f(t)$.  Then because $\zeta_t$ is bounded on its domain, $\zeta_t\circ K = K^\star\zeta_t$ is a bounded linear functional. Then from the Riesz Representation Theorem we have that there is a unique regular Borel measure $\mu_t$ on $R$ such that 
	\begin{equation}
	\begin{aligned}
	    \left(K\xi\right)(t) = K^\star \zeta_t\left(\xi\right) &= \int_{R} \xi(s)\ d\mu_t(s), \\
	    \|\mu_t\| &= \|K^\star \zeta_t\| 
	\end{aligned}
	\end{equation}
	Then if there exists a regular Borel measure $\mu$ such that $\mu_t$ is significantly smaller that $\mu$ for all $t$, then we have that, by the Radon-Nikodim derivative, $d\mu_t(s) = K_t(s)d\mu(s)$ under the assumption that $K_t$ is $L^1$ integrable over $R$ with the measure $\mu$. Thus it follows that 
	\begin{equation}
	K\left[\xi\right](t) = \int_{R} \xi(s)K_t(s)\ d\mu(s) = \int_{R} \xi(s)K(t,s)\  d\mu(s).
	\end{equation}
	Therefore, for any bounded linear operator $K:C(X)\to C(X)$ there exists a unique $K(t,s)$ such that $K[f] = \int_X f(s)K(t,s) d\mu(s)$ . Now we show that the operation of $\Sigma_l$ can approximate any such operator. Because $K$ is of the form of $\Sigma_l$ where the only difference is the weighting function, so the proof follows.
\end{proof}

\end{document}