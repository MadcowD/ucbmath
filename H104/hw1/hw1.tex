%%%%%%%%%%%%%%%%%%%%%%%%%%%%%%%%%%%%%%%%%%%%%%%%%%%%%%%%%%%%%%%%%%
%%%                      Homework 1                            %%%
%%%%%%%%%%%%%%%%%%%%%%%%%%%%%%%%%%%%%%%%%%%%%%%%%%%%%%%%%%%%%%%%%%

\documentclass[letter]{article}

\usepackage{lipsum}
\usepackage[pdftex]{graphicx}
\usepackage[margin=1.5in]{geometry}
\usepackage[english]{babel}
\usepackage{listings}
\usepackage{amsthm}
\usepackage{amssymb}
\usepackage{framed} 
\usepackage{amsmath}
\usepackage{mathabx}
\usepackage{titling}
\usepackage{fancyhdr}

\pagestyle{fancy}


\newtheorem{theorem}{Theorem}
\newtheorem{definition}{Definition}

\newenvironment{menumerate}{%
  \edef\backupindent{\the\parindent}%
  \enumerate%
  \setlength{\parindent}{\backupindent}%
}{\endenumerate}







%%%%%%%%%%%%%%%
%% DOC INFO %%%
%%%%%%%%%%%%%%%
\newcommand{\bHWN}{1}
\newcommand{\bCLASS}{MATH H104}

\title{\bCLASS: Homework \bHWN}
\author{William Guss\\26793499\\wguss@berkeley.edu}

\fancyhead[L]{\bCLASS}
\fancyhead[CO]{Homework \bHWN}
\fancyhead[CE]{GUSS}
\fancyhead[R]{\thepage}
\fancyfoot[LR]{}
\fancyfoot[C]{}
\usepackage{csquotes}

%%%%%%%%%%%%%%

\begin{document}
\maketitle
\thispagestyle{empty}


%%%%%%% Be sure to set the counter and use menumerate
\section{Real Numbers}

\begin{menumerate}
	\setcounter{enumi}{2}
	\item Recast the following English sentences in mathematics, using correct mathematical grammar. Preserve their meaning.
		\begin{menumerate}
			\item \textit{2 is the smallest prime number.} Let $P \subset \mathbb{N}$ denote the set of prime numbers.
			Consider that $t = 2$ is clearly a member of P.
			Then for all $p \in P$, $t \leq P$.

			\item \textit{The area of any bounded plane region is bisected by some line parallel to $x$-axis.} 

			Before we claim the above we must first rigorize the notion of a bounded plane region. The following uses a notion of open and closed sets in $\mathbb{R}^2$.

			\begin{definition}
				 We say that $B_r(x_0)$ is an open ball of radius $r>0$ if and only if
				 $$B_r(x_0) = \{ x \in \mathbb{R}^2 \ |\ \|x - x_0 \| < r \}.$$
				 Furthermore $\bar{B}_r(x_0)$ is a closed ball of radius $r>0$ if and only if
				 $$\bar{B}_r(x_0) = \{ x \in \mathbb{R}^2 \ |\ \|x - x_0 \| \leq r \}.$$
			 \end{definition} 
 			 Using the above definition we now give our notion of a bounded plane reigon. 
 			 \begin{definition}
 			 	If $A$ is a subset of $\mathbb{R}^2$ we will say that $A$ is the area of a bounded plane region if and only if for every $x \in A$, there is an open or closed ball centered at $x$ which is a subset of $A.$
 			 \end{definition}
 			 Lastly, we give the notion of a parallel line to the $x$-axis
 			 \begin{definition}
 			 	We say that $L_r \subset \mathbb{R}^2$ is a line parallel to the $x$-axis at radius $r$ if and only if 
 			 		$$L_r = \{ (x,y) \in \mathbb{R}\ | \ y = r \}.$$ 
 			 \end{definition}
			 
			 Now it is simple to propose the theorem of symantic equivalence to the question.

			 \begin{theorem}
			 Let $A$ be the area of a bounded plane region in $\mathbb{R}^2$. Then, there exists some line parallel to the $x$-axis of height $r$, $L_r$, such that $L_r \cap A \neq \emptyset$ and both
			 	\begin{equation}
			 			A_L = \{ (x,y) \in A\ |\ y < r \}\ \mathrm{and}\ 
			 			A_U = \{ (x,y) \in A\ |\ y \geq r\}
	 			\end{equation}	
	 		are areas of bounded plane regions.

 			 \end{theorem}

 			 \item \textit{"All that glitters is mot gold."}  Let $G$ be the set of all object which glitter. Then let $A$ be the set of all gold objects. $A\neq G$.

		\end{menumerate} 

	\setcounter{enumi}{11}
	\item \textit{Prove the following.}
		 \begin{theorem}
			There exists no smallest positive real number.
		\end{theorem}
		\begin{proof}
		 Suppose that there exists a smallest real number, say $a \in \mathbb{R}$. Clearly $a >0$ and so is $\frac{a}{2}$. Furthermore $\frac{a}{2} < a,$  and hence we reach a contradiciton. Therefore does not exist a smallest postivie real number.
		\end{proof}
		\begin{theorem}
			There exist no smallest positive rational number. 
		\end{theorem}
		\begin{proof}
			 Suppose that there exists a smallest rational number, say $q \in \mathbb{Q}$. Clearly $q >0$ and so is $\frac{q}{2}$. Furthermore $\frac{q}{2} < q,$  and hence we reach a contradiciton. Therefore does not exist a smallest postivie rational number.
		\end{proof}

		\begin{theorem}
			Let $x \in \mathbb{R}.$ Then there does not exist a smallest real number $y$ such that $y > x$.
		\end{theorem}
		\begin{proof}
			Suppose that such a $y$ exists. Now consider $\frac{x+y}{2}= b$. Clearly $b > x$, and remarkably $b < y$. Hence $y$ is not the smallest real number such that $y > x$. This leads to a contradiction, and therefore there is no smallest $y$ satisfying the conditions.
			
		\end{proof}

	\setcounter{enumi}{21}
	\item \textit{Show the following.} 
		\begin{menumerate}
			\item Fixed points: \begin{theorem}
				The function $f: A \to A$ has a fixed point if and only if the graph of $f$ interesects the diagonal.
			\end{theorem}
			\begin{proof}
				We first show the right implication. If $f$ has a fixed point, then there is some $a \in A$ such that $f(a) = a$. Now consider the graph of $f$, 
				$$f(A) = \{(a, f(a) \in A\}.$$ 
				Since $f$ has a fixed point, $f(A)$ contains $(a,a).$ Hence the intersection of $f(A)$ with the diagonal of $A\times A$,$D,$ must contain $(a,a)$ at the least and hence is nonempty.

				On the otherhand if the graph of $f$ intersects the diagonal, then there exists some $(a,a) \in D$ such that $(a,a) \in f(A).$ Then by definition of the graph of $f$, $(a,a) = (a,f(a))$, which implies that $f(a) = a.$ This completes the proof.
			\end{proof}


			\item Intermediate fixed point
			\begin{theorem}
				Every continuous function $f: [0,1] \to [0,1]$ has at least one fixed-point.
			\end{theorem}
			\begin{proof}
				To show this we recall the intermediate value theorem. More specifically we want to find a function whose $0$ exists on $[0,1]$ which implies the theorem. Consider that $f(x) =x$ implies that $0 = f(x) - x$, so let's simply let $q(x) = f(x) - x$. By definition of the bound on the codomain, $g(0) \geq 0$ and $g(1) \leq 0$. Then application of the intermediate value theorem yields that there exists at $c \in [0,1]$ with $g(c) = 0.$ Hence, $f(a) = a$. This completes the proof.
			\end{proof}
			\item No, consider the case of some function for which $f(x) > x$ on $(0,1)$. Such a function need not attain the value $f(0) = 0, f(1) = 1$ because such values could not possiblt exist on its graph. Hence, $f(x) \neq x$ for all $x$. 
			\item No, consider the function $f(x) = x+0.5$ when $0\leq x < 0.5$, and $f(x) = x-0.5$ when $0.5 \leq x \leq 1$. This function never is equivalent to $g(x) = x.$
		\end{menumerate}

	\item \textit{Show the following.} 
		\begin{menumerate}
			\item Dyadic squares:
				\begin{theorem}
					 If $x$ and $y$ are two dyadic squares, then they are either identical, intersect along a common edge, intersect at a common vertex, or do not intersect at all.
				\end{theorem}
				\begin{proof}
					Since we must show all cases, let us consider them with respect to the general definition of a planar dyadic cube. In particular, $x,y \in \mathbb{Q}_2^2.$ Let us fix $x$ such that
					$$x = \left[\frac{p}{2^k}, \frac{p+1}{2^k}\right]^2 \mathrm{and}\ y = \left[\frac{q}{2^k}, \frac{q+1}{2^k}\right]^2$$
					for some $p, k, q \in \mathbb{Z}.$

					If $q = p$, then $y = x$ naturaly. In the case that $q > p+1$ or $q+1 < p$, we have that $x \cap y = \emptyset$. Next consider intersections along different edges. If
					 $$y = \left[\frac{p}{2^k}, \frac{p+1}{2^k}\right]\times \left[\frac{p+1}{2^k}, \frac{p+2}{2^k}\right],$$
					 then $y \cap x = [(\frac{p}{2^k}\frac{p+1}{2^k}), (\frac{p+1}{2^k},\frac{p+1}{2^k})]$. In general, 
					 $$y = \left[\frac{p+r}{2^k}, \frac{p+r+1}{2^k}\right]\times \left[\frac{p+s}{2^k}, \frac{p+s+1}{2^k}\right]$$
					 implies the following intersections.

					 If $r = 1, s = 0$, then $x \cap y = \left[(\frac{p+1}{2^k},\frac{p}{2^k}),(\frac{p+1}{2^k}, \frac{p+1}{2^k})\right]$. If $r = -1, s= 0$, then  $x \cap y = \left[(\frac{p}{2^k},\frac{p}{2^k}),(\frac{p}{2^k}, \frac{p+1}{2^k})\right].$ If $r= 0, s = 1$, then  $x \cap y = \left[(\frac{p}{2^k},\frac{p+1}{2^k}),(\frac{p+1}{2^k}, \frac{p+1}{2^k})\right]$. If $r=0,s=-1,$ then  $x \cap y = \left[(\frac{p}{2^k},\frac{p}{2^k}),(\frac{p+1}{2^k}, \frac{p}{2^k})\right]$.

					 Lastly we need to consider the vertex edge cases. If $r =1, s= 1$, then $x \cap y = \{(\frac{p+1}{2^k},\frac{p+1}{2^k})\}$. If $r =-1, s= 1$, then $x \cap y = \{(\frac{p}{2^k},\frac{p+1}{2^k})\}$. If $r =-1, s= -1$, then $x \cap y = \{(\frac{p}{2^k},\frac{p}{2^k})\}$. If $r =1, s= -1$, then $x \cap y = \{(\frac{p+1}{2^k},\frac{p}{2^k})\}$.  

					 Furthermore if $r$ and $s$ attain other values, we have those cases previously considered. Hence the proof is complete.

 				\end{proof}

 			\item For the following problem we adopt the following notation.
 				\begin{definition}
 				 We say that say that some $X \subset \mathbb{R}^n$ is a dyadic hyper-interval of partition $2^{-\gamma}$ if and only if 
 				 $$X \in \overline{\Delta_n^k} = \left\{ Y \subset \mathbb{R}^n\ |\ Y = \bigtimes_{i\in\delta_k}2^{-\gamma}\left[m_i,m_i +1\right]\right 
 				 \},$$
 				 where $\delta_k$ is the index set of dimensions in which the interval is non-empty and non-singular. Furthermore, $|\delta_k| = k$.
 				 \end{definition}
 					So now we need to operationalize this proof. If $x$ and $y$ are two dyadic hypercubes of $\mathbb{R}^n$, then they are either identical, intersect along a common hyper-edge, intersect at a common vertex, or do not intersect at all.
				\begin{theorem}
					  In other words, if $X,Y \in \overline{\Delta_n^n}$ are of the same partition, $2^{-\gamma}$, let
					   $$Y = \bigtimes_{i=1}^k2^{-\gamma}\left[m_i+r_i,m_i +1 + r_i\right],$$ 
					   where the $m_j$ are those which define $X$, and $r_j \in \mathbb{Z}.$ Then, if $|r_j| \leq 1$ for all $j$, the following two results hold. If $k = n - \sum_i|r_i| > 0$,  $X \cap Y \in \overline{\Delta_n^k}$. If $k = 0$, $X \cap Y \subset \mathbb{Q}_2^n$ with $|X\cap Y| = 1$. Otherwise if there exists some $j$ such that $|r_j| > 1$, then $X \cap Y = \emptyset.$

				\end{theorem}
				\begin{proof}
					We denote $X_j, Y_j$ as the $j^\mathrm{th}$ interval composing $X$ and $Y$. In the above definition of $Y$ we wish to explore a multitude of different $r_j$ values so as to express the theorem.

					In the simplest case, $|r_j| > 1$ for some $j$ then $$y_j = 2^{-k}[m_j+r_j,m_j+r_j+1].$$ Clearly $m_j + 1 < m_j + r$ or $m_j  > m_j +r_j + 1$, and thus $y_j \cap x_j = \emptyset$, we have that the whole cartesian product, 
						$$X\cap Y = \emptyset \times \left(\bigtimes_{i\neq j}^n x_j\cap y_j\right) = \emptyset,$$ 
						because $\emptyset \times B$ cannot form any pair $(a,b)$ as there is no $a\in\emptyset.$

					We claim that when $|r_i| \leq 1 $, $X \cap Y\in \overline{\Delta_n^k}$ for $k = n - \sum_{i=1}^n|r_i| > 0.$ Let $(n_p)$ denote the finite (possibly empty) list of indices for which $|r_j| = 1.$ In other words, for all $p$, $|r_{n_p}| = 1,$ else $|r_j| = 0.$ The intersection as aforementioned is the cartesian product of all $x_j, y_j$. Hence for $j \notin \{n_p\}$,  $x_j \cap y_j \in \overline{\Delta_n^1}$ with $\delta_1 = {j}.$ The cartesian product of all such $j$ is $X^* \cap Y^* \in \overline{\Delta_n^c}$ with $\delta_{c} = \{j \neq n_p \forall p\},$ and $c= n-|\{n_p\}| = k.$ We claim that $X \cap Y$ cannot exist in any higher dimenisonality than $X^* \cap Y^*.$ 

					Suppose $X \cap Y \in \overline{\Delta_n^d},$ with $n \geq d > c$.  This implies that there exists a $q \in \{n_p\}$ such that $x_q \cap y_q = z_q$ is non-singular and non-empty. We have that
					\begin{equation*}
						\begin{aligned}
							z_q &= 2^{-\gamma}[m_q, m_q+1]\cap  2^{-\gamma}[m_q \pm 1, m_q+1 \pm 1] \\
								&= 2^{-\gamma}\left\{m_q+\frac{1\pm1}{2}\right\}
						\end{aligned}
					\end{equation*}
					is singular. Hence we reach a contradiction and $X\cap Y \in \overline{\Delta_n^k}. $

					Lastly in the case that $|r_j| = 1$ for all $j$ if and only if $k = 0$, the intersection is a cartesian product of $n$ singular points as in $z_q.$ Thus $X\cap Y \in \mathbb{Q}_2^m$. This completes the proof.

 				\end{proof}
		\end{menumerate} 


		\setcounter{enumi}{31}
		\item \textit{Suppose that $E$ is a convex region in the plane bounded by a curve $C.$}
			\begin{menumerate}
				\item \textit{Show the following}
					\begin{theorem}
						The curve $C$ has a tangent line except at a countable number of points.
					\end{theorem}

					\begin{proof}
						By definition if $E$ is a convex region, then for any two points $x,y \in E$, all points on the line $L(x,y) = \{z \in E\ :\ tx +sy = z, t+s = 1, 0 \leq t,s\leq 1\}.$ 

						Let $a,b_t$ be two points on the curve $C$. 
					\end{proof} 
			\end{menumerate} 

		
		
\end{menumerate}


\end{document}