%%%%%%%%%%%%%%%%%%%%%%%%%%%%%%%%%%%%%%%%%%%%%%%%%%%%%%%%%%%%%%%%%%
%%%                      Homework 11_                          %%%
%%%%%%%%%%%%%%%%%%%%%%%%%%%%%%%%%%%%%%%%%%%%%%%%%%%%%%%%%%%%%%%%%%

\documentclass[letter]{article}

\usepackage{lipsum}
\usepackage[pdftex]{graphicx}
\usepackage[margin=1.5in]{geometry}
\usepackage[english]{babel}
\usepackage{listings}
\usepackage{amsthm}
\usepackage{amssymb}
\usepackage{framed} 
\usepackage{amsmath}
\usepackage{titling}
\usepackage{fancyhdr}

\pagestyle{fancy}


\newtheorem{theorem}{Theorem}
\newtheorem{definition}{Definition}

\newenvironment{menumerate}{%
  \edef\backupindent{\the\parindent}%
  \enumerate%
  \setlength{\parindent}{\backupindent}%
}{\endenumerate}







%%%%%%%%%%%%%%%
%% DOC INFO %%%
%%%%%%%%%%%%%%%
\newcommand{\bHWN}{11}
\newcommand{\bCLASS}{MATH: H104}

\title{\bCLASS: Homework \bHWN}
\author{William Guss\\26793499\\wguss@berkeley.edu}

\fancyhead[L]{\bCLASS}
\fancyhead[CO]{Homework \bHWN}
\fancyhead[CE]{GUSS}
\fancyhead[R]{\thepage}
\fancyfoot[LR]{}
\fancyfoot[C]{}
\usepackage{csquotes}

%%%%%%%%%%%%%%

\begin{document}
\maketitle
\thispagestyle{empty}


%%%%%%% 8-10, 13, 18, 22, 24.
\begin{menumerate}
	\setcounter{enumi}{7}
	\item %8
	\item %9
	\item %10

	\setcounter{enumi}{12}
	\item %13

	\setcounter{enumi}{17}
	\item %18

	\setcounter{enumi}{21}
	\item %22
	Consider the following example. 
	\begin{theorem}
		Let $f:[0,1] \to \mathbb{R}$ such that if $x \neq 0, x \mapsto x\sin\left(\frac{1}{x}\right)$ and $x\mapsto 0$ otherwise. Furthermore,
		let $(g_n)$ be a family of functions such that $g_n : [0,1] \to \mathbb{R}$ such that
		$$g_n(x) = \left\{
                \begin{array}{ll}
                  0,\;\forall x \in [0,1/n] \\
                  e^{\frac{1}{(x-1/n)^2}},\;\forall x \in (1/n,2/n)\\
                  1,\;\forall x \in [2/n,1]
                \end{array}
              \right.$$
        If $(f_n)$ is defined such that $f_n(x) = f(x)g_n(x)$, then the family $(f_n)$ is smooth, equicontinuous, with unbounded derivatives.
	\end{theorem}
	\begin{proof}
		Let $x\in (0,1], \gamma > 0.$ Then there exists an $N$ such that $2/n < x-\gamma.$
		 In this case for all $n>N$ $f_n(y) = f(y)$ for each and every $y \in (x-\gamma,x+\gamma).$
		 Then for every $\epsilon > 0,$ the continuity of $f$ gives that there is a $\delta < \gamma$ with $|f_n(y) - f_n(x)| = f(y)-f(x) < \epsilon.$
		 Take the smallest delta for which all $f_1,\dots,f_N$ are satisfied and yield that this $\delta'$ gives equicontinuity. At $x = 0$, $|f_n(y)| \leq y\sin(1/y)\leq y$ for all $y \in [0,1]$ and for all $n > N.$ So  $f_n(0)$ is equicontinuous. Then the compactness of $[0,1]$ implies uniform equicontinuity by the Arzela Ascoli theorem.
		 Clearly $f'(x)$ is unbounded as it approaches $0$ so in every case the derivatives of $f_n$ are unbounded by the product rule.
		 Smoothness comes from the fact that $f_n = 0$ in $[0,1/n]$ and the derivatives at $0$ are $0.$
	\end{proof}
\end{menumerate}	

\end{document}