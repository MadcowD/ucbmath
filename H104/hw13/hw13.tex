%%%%%%%%%%%%%%%%%%%%%%%%%%%%%%%%%%%%%%%%%%%%%%%%%%%%%%%%%%%%%%%%%%
%%%                      Homework 13                           %%%
%%%%%%%%%%%%%%%%%%%%%%%%%%%%%%%%%%%%%%%%%%%%%%%%%%%%%%%%%%%%%%%%%%

\documentclass[letter]{article}

\usepackage{lipsum}
\usepackage[pdftex]{graphicx}
\usepackage[margin=1.5in]{geometry}
\usepackage[english]{babel}
\usepackage{listings}
\usepackage{amsthm}
\usepackage{amssymb}
\usepackage{framed} 
\usepackage{amsmath}
\usepackage{titling}
\usepackage{fancyhdr}
\usepackage{algorithm,algorithmic}
\usepackage{pgfplots}
\usepackage{tikz}

\pagestyle{fancy}


\newtheorem{theorem}{Theorem}
\newtheorem{definition}{Definition}
\newtheorem{example}{Example}

\newenvironment{menumerate}{%
  \edef\backupindent{\the\parindent}%
  \enumerate%
  \setlength{\parindent}{\backupindent}%
}{\endenumerate}







%%%%%%%%%%%%%%%
%% DOC INFO %%%
%%%%%%%%%%%%%%%
\newcommand{\bHWN}{13}
\newcommand{\bCLASS}{MATH H104}

\title{\bCLASS: Homework \bHWN}
\author{William Guss\\26793499\\wguss@berkeley.edu}

\fancyhead[L]{\bCLASS}
\fancyhead[CO]{Homework \bHWN}
\fancyhead[CE]{GUSS}
\fancyhead[R]{\thepage}
\fancyfoot[LR]{}
\fancyfoot[C]{}
\usepackage{csquotes}

%%%%%%%%%%%%%%

\begin{document}
\maketitle
\thispagestyle{empty}


%%%%%%% Be sure to set the counter and use menumerate
\setcounter{section}{3}
\section{Function Spaces}
\begin{menumerate}
\setcounter{enumi}{29}
    \item Consider the following example.
    \begin{example}
        The 1-sphere, $S^1$ is a compact path-connected and nonempty 1-manifold. Now consider the continuous mapping,
         $\phi: S^1 \to S^1$ which takes a point $p \in S^1$ and adds to it's angle $\sqrt{2}.$ The function $\phi$ 
         has no fixed points.
    \end{example}
    \begin{proof}
        It is fair to represent a point locally by its angle as $S^1$ is a one manifold and therefore has an atlas of functions {bad grammar here dude!}
        $f: S^1 \to E \subset \mathbb{R}$. The assertion that $\exists p \in S^1$ such that $\phi(p) = p$ implies that there exists
        an angle $\theta$ such that $\sqrt{2} + \theta \equiv \theta  \  (\mod 2\pi).$ Suppose such a $\theta$ existed.
         Then $n\sqrt{2} + \theta = \theta + k2\pi$ implies $n2^{1/2} = k2\pi$, and so $\pi$ is an algebraic number. A contradiction!
    \end{proof}

\setcounter{enumi}{33}
\item Consider the ODE $$y' = 2\sqrt{|y|}.$$  
    \begin{theorem}
        The ODE does not have unique solutions for $x \geq 0,$ and $y(0) = 0.$
    \end{theorem}
    \begin{proof}
        Consider the solution $y_1(x) = 0.$ Clearly $y_1(0) = 0,$ and $y'(x) = 2\sqrt{|0|} = 0.$ 
        Then consider likewise4 the solution $y_2(x) = x^2.$ Observer that $y(0) = 0^2 = 0$ and
        $y'(x) = 2x = 2\sqrt{x^2} = 2x$ when $x \geq 0.$
    \end{proof}

      \begin{figure}
          \centering
          \begin{tikzpicture}
              \begin{axis}[
                axis lines=middle,
                clip=false,
                ylabel=$y(x)$,
                xlabel=$x$,
                every axis y label/.style={
                    at={(ticklabel* cs:1.02)},
                    anchor=south,
                },
                legend pos=north west,
                width=10cm,
                height=7cm,z
                ytick={-1,1}
              ]
              \addplot+[mark=none,samples=200,unbounded coords=jump, restrict x to domain=-1:5] {(x+1)^2)} node[right,pos=.57,black];
              \addplot+[mark=none,samples=200,unbounded coords=jump, restrict x to domain=-2:5] {(x+2)^2)} node[right,pos=.57,black];
              \addplot+[mark=none,samples=200,unbounded coords=jump, restrict x to domain=-3:5] {(x+3)^2)} node[right,pos=.57,black];
              \addplot+[const plot, no marks, thick] coordinates {(-4,0) (-1,0)};
              \end{axis}
          \end{tikzpicture}

        \caption{Other solutions to the ODE}
        \label{fig:test}
        \end{figure}
    In fact there are even more examples of solutions which are not unique. See figure 1 for those whose domain
    is infact in $\mathbb{R}^-.$

    This does not however contradict Picard's theorem since, the function $f(y')$ defined is not
    uniformly lipschitz continuous. 
    \begin{proof}
        Suppose that $f(t)$ where lipschitz continuous. Then in particular, there is a constant
        $L$ such that $d(fx,fy) \leq Ld(x,y)$ for all $x,y \in M$ the domain of $f.$ So take for the sake
        of contradiction $x = 0,$ and let $y$ approach $0.$ By $f$ Lipschitz, we have that 
        $$\sqrt{y} \leq Ly$$
        which is true if and only if $y/y^2 \leq L.$ Since $y \to 0$ let us take $y = 1/n.$ This asserts that,
        $n \leq L$ for all $n$ which contradicts the archimedian property of $\mathbb{R}.$ 
    \end{proof}

\item We conjecture about the following ODE.
    $$x' = x^2 \in \mathbb{R}.$$
    The solution to the above ODE is obtained through the following calculations.
    \begin{equation}
        \begin{aligned}
            \frac{dx}{dt} &= x^2 \\
            \int \frac{dx}{x^2} &= \int_{t_0}^t ds \\
            -\frac{1}{2x(t)} + c_1 &= t        
        \end{aligned}
    \end{equation}
    and so we have that $x(t) = -\frac{2}{t-c_1}.$ Where $c_1$ shifts the solution to satisfy the initial condition. 
    However consider the solution where $x(-1) = 2.$ It's clear that this solution is unbounded as $t \to 0$, and therefore 
    escape to infinity in finite time.

\item We conjecture generally about separable ODE; that is differential equations of the following form.
    \begin{equation}
        x' = f(x) \in \mathbb{R}
    \end{equation}

    \begin{theorem}
        If $f(x)$ is bounded then no solution of the ODE escape to infinity in finite time.
    \end{theorem}
    \begin{proof}
        If the ODE is bounded then there exists an $M$ such that for all $x,$ $|f(x)| \leq M.$ 
        Furthermore, we have that since the ODE is separable,
        \begin{equation*}
            \begin{aligned}
                x'(s)\frac{1}{f(x(s))} &= 1 \\
                \int_{t_0}^{t} x'(s)\frac{1}{f(x(s))}\ ds &= \int_{t_0}^{t}ds 
            \end{aligned}
        \end{equation*}
    \end{proof}

\end{menumerate}


\end{document}