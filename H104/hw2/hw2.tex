%%%%%%%%%%%%%%%%%%%%%%%%%%%%%%%%%%%%%%%%%%%%%%%%%%%%%%%%%%%%%%%%%%
%%%                      Homework 2                            %%%
%%%%%%%%%%%%%%%%%%%%%%%%%%%%%%%%%%%%%%%%%%%%%%%%%%%%%%%%%%%%%%%%%%

\documentclass[letter]{article}

\usepackage{lipsum}
\usepackage[pdftex]{graphicx}
\usepackage[margin=1.5in]{geometry}
\usepackage[english]{babel}
\usepackage{listings}
\usepackage{amsthm}
\usepackage{amssymb}
\usepackage{framed} 
\usepackage{amsmath}
\usepackage{titling}
\usepackage{fancyhdr}

\pagestyle{fancy}


\newtheorem{theorem}{Theorem}
\newtheorem{definition}{Definition}

\newenvironment{menumerate}{%
  \edef\backupindent{\the\parindent}%
  \enumerate%
  \setlength{\parindent}{\backupindent}%
}{\endenumerate}







%%%%%%%%%%%%%%%
%% DOC INFO %%%
%%%%%%%%%%%%%%%
\newcommand{\bHWN}{2}
\newcommand{\bCLASS}{Math H104}

\title{\bCLASS: Homework \bHWN}
\author{William Guss\\26793499\\wguss@berkeley.edu}

\fancyhead[L]{\bCLASS}
\fancyhead[CO]{Homework \bHWN}
\fancyhead[CE]{GUSS}
\fancyhead[R]{\thepage}
\fancyfoot[LR]{}
\fancyfoot[C]{}
\usepackage{csquotes}

%%%%%%%%%%%%%%

\begin{document}
\maketitle
\thispagestyle{empty}


%%%%%%% Be sure to set the counter and use menumerate
\section{Real Numbers}
\begin{menumerate}
	\setcounter{enumi}{35}
	\item \textit{Without using the Schroeder-Bernstein Theorem,}
		\begin{menumerate}
		 	\item \textit{Prove the following}
		 		\begin{theorem}
		 			The cardinalities of $[a,b], (a,b], (a,b)$ are equivalent.
		 			 That is, $[a,b]\sim (a,b]\sim (a,b)$.
	 			 \end{theorem}
	 			 \begin{proof}
	 			 	To show cardinal equivalence, we must find a bijection between the three sets. This argument follows from Hilberts's hotel. Consider the two sets set $A^o = (a,b), A^h = (a,b].$ Because every uncountable set has a countable subset, let $$A^o_C = \left\{a_n \in A^o\  | \ a_n = \frac{an +b}{n+1} ,\  n \in \mathbb{N} \right\}.$$
	 			 	In the same light, let $$A^h_C = \left\{a_n \in A^h\  |\ a_n = \frac{a(n-1) +b}{n} ,\   n \in \mathbb{N} \right\}.$$
	 			 	Then $f: A^o_C \to A^h_C $, such that $f(\frac{an +b}{n+1}) = \frac{a(n-1) +b}{n}$, is clearly a bijection.

	 			 	We now make a functiong $g: A^o \to   A^h$ such that for $x \in A^o_C$, $g(x) = f(x)$, otherwise $g(x) = x.$ Since $f$ is a a bijection, then $g$ is a bijection on $A^o_C$. Furthermore since $b \notin A^h \setminus A^h_C,$ we have that $g$ surjective when $x \notin A^o_C.$ Furthermore if $x \neq y,$ then clearly $g(x) = x \neq y = g(y).$ So $g$ is injective, and therefore bijective. We have shown that $A^o \sim A^h.$

	 			 	Lastly, we will prove that $A^c = [a,b]$ is bijective to $A^h.$ Redefine again the following countable subsets
	 			 	$$A^h_C = \left\{a_n \in A^g\  | \ a_n = \frac{a +bn}{n+1} ,\  n \in \mathbb{N} \right\}.$$
	 			 	In the same light, let $$A^c_C = \left\{a_n \in A^c\  |\ a_n = \frac{a +b(n-1)}{n} ,\   n \in \mathbb{N} \right\}.$$

	 			 	Now take the function $f:A^h_C \to A^c_C$ to map $(\frac{a +bn}{n+1}) \mapsto \frac{a +b(n-1)}{n}$. The function $f$ is a clear bijection. Finally let $g: A^h \to A^c$ be such that $x \in A^h_C \implies g(x) = f(x),$ otherwise $g(x) = x$. This new function is clearly a bijection in the same sense that the previous definition of $f$ was, and hence we have shown that $A^c \sim A^h$.

	 			 	Therefore by bijective composition, $A^c \sim A^h \sim A^o$, and the proof is complete. 

	 			 \end{proof}

 			 \item \textit{Prove the following.} 
 			 	\begin{theorem}
 			 	If $C$ is countable, then $\mathbb{R}\setminus C \sim \mathbb{R} \sim \mathbb{R} \cup C.$
 			 	\end{theorem}
 			 	\begin{proof}
 			 		 If $C$ is a countable subset of $\mathbb{R},$ then clearly $\mathbb{R} \cup C = \mathbb{R}$ and hence there is equivalence. In the other case, consider the following. 

 			 		To show this theorem, we first find a bijection between $\mathbb{R}$ and $\mathbb{R} \setminus C$. Because $C$ is a countable, we can take an index set $I = \{1,2,\dots\},$ which is finite if and only if $C$ is finite and contiguous such that $|I| = |C|.$ We then can define $C = \left\{a_i \right\}_{i \in I}$ for real numbers $a_i \in C,$ such that $i > j \implies a_i > a_j$.

 			 		We have that $$\mathbb{R}\setminus C = (-\infty,a_1)\cup \left[\bigcup_{i\in I \setminus \{1\}}(a_{i-1},a_i)\right]\cup (\sup C,\infty).$$ So it follows,
 			 		 $$\mathbb{R} = (-\infty,a_1]\cup \left[\bigcup_{i\in I \setminus \{1\}}[a_{i-1},a_i]\right]\cup [\sup C,\infty).$$ We will now constuct a bijection $f:\mathbb{R}\setminus C \to \mathbb{R}.$ By Theorem 1, there exists a bijective function $h_i : (a_{i-1}, a_i) \to [a_{i-1}, a_i].$ In the edge cases, we propose the following method of producing a bijection.

 			 		 Let $A_L^o = (-\infty, a_1)$. Then clearly, $(-\infty, b] \subset \mathbb{R}$ for any $b < a_1.$ Hence we can use $(-\infty,b] \cup [b,a_1) = A_L^o.$ By theorem there exists an $h_L:[b,a_1) \to [b,a_1].$ Again for the upperbound, let $A_U^o = (\sup C, \infty).$ We have that $A_U^o = (\sup C, d] \cup [d, \infty)$. Let $h_U : (\sup C, d] \to [\sup C, d]$ be the bijection producable from Theorem 1. Finally let $g_1: (-\infty,b]\to (-\infty,b]$ and $g_2: [d,\infty)\to [d,\infty)$ be the identity functions. 

 			 		 Taking all aforementioned functions and defining a new function $f$ which acts as each function along its corresponding domain, we have a bijection from $\mathbb{R}$ to $\mathbb{R}\setminus C$ when $C$ is bounded. 



 			 		 If $C$ is not bounded, we can use the fact that $C$ is listable to strongly inductively create a mapping. Let $L:\mathbb{N}\to C$ be the listing function for $C$ itself. We propose that $\mathbb{R} \sim \mathbb{R} \setminus \{L(n)\}_{n=0}^\infty$.

 			 		 In the base case we have that $\mathcal{L}_1 = \{L(0)\}$ is a countable and finite set, so by the first half of this proof, $\mathbb{R} \sim \mathbb{R} \setminus \mathcal{L}_1.$ Now suppose $\mathbb{R} \sim \mathbb{R} \setminus \bigcup_n^k \mathcal{L}_k;$ assume that removing all previous points using this method yields an injection from $\mathbb{R}$ onto $\mathbb{R} \setminus \bigcup_n^k \mathcal{L}_k$ It follows that because $\mathcal{L}_k$ is countable, $\mathcal{L}_{k+1} = \mathcal{L}_k \cup \{L(k+1)\}$ is also countable and bounded. Hence by the first half of the proof, $\mathbb{R} \sim \mathbb{R} \setminus \bigcup_n^k \mathcal{L}_k \sim \mathbb{R} \setminus \left( \bigcup_n^k \mathcal{L}_k \cup \mathcal{L}_{k+1} \right).$ Repeating this process without stopping, we get that  $\mathbb{R} \sim \mathbb{R} \setminus \{L(n)\}_{n=0}^\infty$ if and only if  $\mathbb{R} \sim \mathbb{R} \setminus C$. This completes the proof for $C$ as numbers.

					 Now if $C$ is not a set of numbers, then clearly $\mathbb{R} \setminus C = \mathbb{R}$ since $\mathbb{R}$ only contains numbers. If we wish to add a countable set of let's say dogs. Let us denote these dogs $d_j$, simply said "scrappy j", for the $j$th dog. Now take the following sequence $D_j = \{x \in \mathbb{R} \cup {d_j}\ | \ x = n + \frac{1}{j}+j, n =0 \implies x = d_j , n \in \mathbb{N}\}.$  The sequence $D^o_j = \{x \in \mathbb{R} | \ x = n + \frac{1}{j}+j, n =0 \implies x = \frac{1}{j}+j , n \in \mathbb{N}\}.$ Clearly there is a mapping which is bijective  between these two sequences as follows. The function $f_j$ maps scrappy $j$ to the first element of $D^o_j$ and then maps the second element of $D_j$ to the third element of $D^o_j$ and so on and so forth. This function is clearly bijective. Then consider a function $g_j$ which will map $x \in \mathbb{R} \setminus D_j$ to $\mathbb{R} \setminus D_j^o$ with the mapping $x \mapsto x.$ In the case that $g_j \in D_j$, $x \mapsto f_j(x).$ Since $g_j$ is bijective, we have that $\mathbb{R} \sim \mathbb{R} \cup D_j = \mathbb{R} \cup \{d_j\}.$

					  Now that we have shown for arbitrary dogs, we can add such dogs to $\mathbb{R}$ and maintain cardinal equivalence, consider the following process. Since all $D_j$ are disjoint, we may repeat this process infinitely. We first take $\mathbb{R}$ and embed $d_1$ into the reals and find an equivalence. Let us denote the first $n$ scrappies embedded in $\mathbb{R}$, $\mathcal{D}_n$. Then we can put $d_2$ into $\mathcal{L}_1$ since $D_1 \cap D_2 = \emptyset$. We call this new set $\mathcal{D}_2$, and combining $f_1,f_2$ and the identity function otherwise to find a bijection $g_2$ from $\mathbb{R}$ onto $\mathcal{D}_2.$ We can repeat this process with out stopping since the set $C$ of dogs is denumerable, and when this process finishes, we will have that $\mathbb{R} \cup C \sim \mathbb{R} \sim \mathbb{R} \sim \mathbb{R} \setminus C.$

					  This completes the proof. 
 
	 		  \end{proof}

	 		  \item \textit{Infer the thing about $\mathbb{Q}$ } 
	 		  	\begin{theorem}
	 		  		The set of irrational numbers has the same cardinality as $\mathbb{R}.$
	 		  	\end{theorem}
	 		  	\begin{proof}
	 		  	Simple! We establish that $\mathbb{Q}$ is countable. Then, it follows since $\mathbb{Q}$ is not bounded, that $\mathbb{R} \sim \mathbb{R} \setminus \mathbb{Q}$ by Theorem 2.
	 		  	\end{proof}
		 \end{menumerate} 
		 \item \textit{Prove that the plane is bijective to the line.}
		 	\begin{theorem}
		 		The real plane $\mathbb{R}^2$ has the same cardinality as the real line $\mathbb{R}.$
		 	\end{theorem}
		 	\begin{proof}
		 		To show that there is equivalent cardinality between the plane and the line, we must find a bijection. It follows from the Schroeder-Bernstein theorem, that if two injections are found, there exists a hyper-composite bijection. Let $f: \mathbb{R} \to \mathbb{R}^2$ such that $x \mapsto (0, x) \in \mathbb{R}.$ If $z\neq w,$ then $f(z) = (0,z) \neq (0,w) = f(w).$ Hence, $f$ is injective. Let $g:\mathbb{R}^2 \to \mathbb{R}$, such that for every decimal expansion (not nine-non terminating) of $(a,b) = (a_1.a_2a_{u(3)}\dots,b_1.b_2b_{u(3)}\dots),\  g\left((a,b)\right) \mapsto a_1.b_1a_2b_2\dots.$ Clearly $g$ is an injection for if $z \neq w,$ then the decimal expansion is non-equal at some index. This implies that $f(z) \neq f(w)$.

		 		Now by the the Shroeder-Bernstein theorem, we have that there exists a bijection from $\mathbb{R}^2 \to \mathbb{R}.$ Hence $\mathbb{R} \sim \mathbb{R}^2.$
		 	\end{proof}
 		\item \textit{Cardinality of power sets }
	 		\begin{menumerate}
				\item \textit{Prove the following}
				\begin{theorem}
					Let $S$ be a set and, $\mathcal{P} = \mathcal{P}(S)$ be the power set of $S.$ Furthermore let $\mathcal{F}$ be the set of functions $f: S \to \{0,1\}$. There exists a natural bijection from $\mathcal{F}$ onto $\mathcal{P}$ defined by $$f \mapsto \{s \in S:f(s) = 1\}.$$
				\end{theorem}

				\begin{proof}
					For this proof, we referr to the above transformation as $A.$ We will first show that this mapping is clearly a surjection. Given any $Q \in \mathcal{P}(S),$ consider the following function $f_Q:S \to {0,1},$ such that if $x \in Q$, $x\mapsto 1$ otherwise, $x \mapsto 0$. Clearly, $f \in \mathcal{F},$ and applying $A(f)$ yields $Q.$ Hence $A$ is surjective.

					To show that there is a bijection, we must show that if $f\neq g$ then $A(f) \neq A(g).$ In the case that $f \neq g,$ then there exists a $x \in S$ such that $f(x) \neq g(x).$ Then it follows that either $A(f)$ or $A(g)$ contains $x$, but not both (if both did, then $f(x) = g(x).$) Hence $A(f) \neq A(g).$ This concludes that $A$ is injective.

					As $A$ is both surjective and injective, $A$ must be bijective by definition. This completes the proof.
				\end{proof}
				


				\begin{theorem}
				Let $S$ be a set and, $\mathcal{P} = \mathcal{P}(S)$ be the power set of $S.$ Furthermore let $\mathcal{F}$ be the set of functions $f: S \to \{0,1\}$. The cardinality of $\mathcal{P}$ is greater than the cardinality of $S$ even when $S$ is empty or finite.
				\end{theorem}

				\begin{proof}
				We will show that theorem holds for the finite, empty, and non-finite cases.

				If $S$ is empty, it follows that $\mathcal{P}$ has cardinality $1$ since $S \subset S$, but $S$ has cardinality $0$ since it is empty.If $S$ is finite, consider that for every $s \in S, \{s\} \in \mathcal{P}(S) $ and $S \in \mathcal{P}(S)$, so the cardinality of the powerset $\mathcal{P}(S)$ is at least one more than that of $S$. 

				If $S$ is not finite we wish to show that there cannot exist a bijection from $\mathcal{P}(S) \to S$ because $\mathcal{P}(S)$ has larger cardinality. By theorem 5, it follows that if there does not exist a bijection from $\mathcal{P} \to \mathcal{F},$ then the cardinalities are clearly not equal.

				Suppose that there existed such a bijection, $B:S\to \mathcal{F}.$ Then for every $s \in S$, $B(s) = f_s: S \to \{0,1\},$ and for each $f_s \in \mathcal{F}$ there exists a unique $s \in S$ such that $B(s) = f_s.$
				Now consider $G:S\to\{0,1\}$ as follows. For every $s$, if $(Bs)(s) = f_s(s) =1, g(s) = 0,$ or if $(Bs)(s) = f_s(s) =0, g(s) = 1.$ Clearly such a $g$ exist and for each $s$, $g \neq f_s \neq B(s)$. However, we supposed that for every $h \in \mathcal{F}$ there existed a $s \in S$ such that $B(s) = h;$ so we reach a contradiction.

				Since such a bijection cannot exist, $\mathcal{P}(S) \sim \mathcal{F} \nsim S$, and in particular because there exists an injection from $S$ to $\mathcal{F}$ but not a surjection, the cardinality of $ \mathcal{F} $ is greater than $S.$ This completes the proof.
				\end{proof}
	 		\end{menumerate}

 		\item \textit{Show the following.}.
	 		\begin{menumerate}
	 		 	\item \textit{Denumerability of the algebraic numbers.} 
	 		 		\begin{theorem}
	 		 			The set $A$ of all algebraic numbers is denumerable; that is, $A \sim \mathbb{N}.$
	 		 		\end{theorem}
	 		 		\begin{proof}
	 		 			In order to show that the set of all algebraic numbers is denumerable, we must show that the set of all polynomial roots is denumerable. We will introduce the following notation. We say that $\mathcal{P}^m$ denotes the set of polynomials up to order $n$ such that if $p\in \mathcal{P}^m$,$$p(x) = \alpha \prod_{n=1}^m(a_nx + \beta_n),$$ where $a_n \in \{0,1\}$ is an activation constant. We denote the order of a $p \in \mathcal{P}^m$ with $R(p)$ such that $R(p)$ is equal to the number of non-zero activation constants in $p.$ Furthermore, we say that $\mathcal{R}^m \subset \mathbb{R}$ is the set of roots for all polynomials $p\in\mathcal{P}^m.$ This notation implies that $\mathcal{R}^\infty = A$. Lastly, observe that by definition, if $n \leq m,$ $\mathcal{P}^m \supset \mathcal{P}^n.$

	 		 			Now we wish to define an upperbound for cardinality of $\mathcal{R}^m$. Take some $p \in \mathcal{P}^m.$ Clearly this polynomial has up to $n$ roots, and these roots can be translated countably many times by changes in coefficients. Therefore, the cardinality of $\mathcal{R}^m$ is roughly less than $n^{2m+1}$ as $n$ approaches infinity. In fact it is possible to show that these roots are countable using a diagonalization argument. Let $r_{k_1,k_2,\dots,k_{2m+1}}$ be the root of the $p\in\mathcal{P}^m$ when $\alpha = k_1, a_1 = k_2,\dots \beta_m = k_{2m+1}.$ Arrange these roots in a $2m+1$ dimensional integer lattice, and perform the diagonalization procedure. It is easy to see that such a procedure list all possible roots of $\mathcal{P}^m$; that is, $\mathcal{R}^m$ is denumerable. 

	 		 			Suppose that $\mathcal{R}^m$ is listed by some function $f_m:\mathbb{N}\to \mathcal{R}^m$. We must now find a procedure such that the roots of all polynomials is denumerable. Consider the following arrangement:
	 		 			\begin{equation*}	
	 		 				\begin{array}{|c|cccccc}
	 		 					\hline
								  & n \to \infty & \\
								\hline
								 m & f_1(1) & f_1(2) & f_1(3) & \dots & f_1(n) &\dots \\
								 \downarrow  & f_2(1) & f_2(2) & f_2(3) & \dots   & f_2(n) &\dots  \\
								  \infty & f_{u(3)}(1) & f_{u(3)}(2) & f_{u(3)}(3) & \dots  & f_{u(3)}(n) & \dots   \\
								  & \vdots & \vdots & \vdots &\ddots & \vdots & \ddots \\
								  & f_m(1) & f_m(2) & f_m(3) & \dots & f_m(n) & \dots \\
								   & \vdots & \vdots & \vdots &\ddots & \vdots & \ddots \\
							\end{array}
	 		 			\end{equation*}
	 		 			We can perform the diagonalization procedure on this table without stopping and yield a listing function $L:\mathbb{N} \to \mathcal{R}^\infty = A.$ Therefore $A$ is denumerable and so the set of algebraic numbers is countable.
	 		 		\end{proof}

	 		 		\item \textit{Arbitrary countable coefficient sets}
	 		 		\begin{theorem}
	 		 			The set $A$ of all roots for polynomials whose coefficients belong to an arbitrary denumerable set $S \subset \mathbb{R}$ is denumerable; that is, $A \sim \mathbb{N}.$
	 		 		\end{theorem}

	 		 		\begin{proof}
	 		 			Simple but not obvious! Consider the proof of Theorem 7 above. Suppose the coefficients $\alpha$ and $\beta_n$ belong to $S$. Then we referr to $\mathcal{P}^m$ as the set of all $m$th order polynomial which only employ $m+1$ coefficients from $S$. Furthermore denote $\mathcal{R}^m$ as the set of all roots for the polynomials of $\mathcal{P}^m.$ 

	 		 			Clearly the upperbound for the cardinality of $\mathcal{R}^m$ is bounded again by $n^{2m+1}$ as we vary each coefficient in $S$ such that $\beta_1,\dots,\beta_m \in B=\{s_j,\dots,s_q\}, (|B| = m, B\subset S)$ where $s_i$ is the $i$th element of $S$. 
	 		 			Since the upperbound still holds, we perform the same $2m+1$ dimensional lattice diagonalization method on the $S^{m+1} \times \mathbb{N}^{m}$ lattice of roots $r_{k_1,\dots,k_{2m+1}}$ as $n$ grows. 

	 		 			This process yields similar listing functions $f_m$ as in the proof above that can be arranged in a similar diagonalization table as before. We then perform the same listing process and yields that $\mathcal{R}^\infty = A$ is denumerable. This completes the proof.
	 		 		\end{proof}

 		 		\item \textit{Trigonometric polynomials}
 		 		\begin{theorem}
 		 			The set of all roots of trigonometric polynomials $A$ is countable.
 		 		\end{theorem}
 		 		\begin{proof}
 		 		 We say that $\tau^m$ is the set of trigonometric polynomials of order $m$ such that $t \in \tau$ implies that $$t(x) = \sum_{j=1}^m sin(a_jx) + cos(b_j x)$$ where $a_j, b_j$ are arbitrary integer coefficients. For the purpose of the first half of this proof we only wish to consider the roots in the interval $[0,2\pi].$ Let $\mathcal{R}_u^m$ denote the set of roots for trigonometric polynomials of order $m$ in $[0,1]$. 

 		 		 Assuming that $a\sin(nx)$ has $n$ roots in $[0,\pi]$ we claim that the cardinality of $\mathcal{R}_u^m$ is bounded by $n^{2m}$ as $a_j,b_j$ vary in $\mathbb{Z}$ and $n \to \infty.$ To see this, let $r_{k_1,\dots,k_{2m}}$ be a root in $\mathcal{R}_u^m$ where $k_1 = a_1,\dots, k_{2m} = b_m.$ Then construct a $2m$ dimensional lattice where the $j$th axis is the coefficient $k_j$ varying in $\mathbb{Z}.$ We can easily perform the diagonalization procedure on such a lattice to yield a listing function $f_m:\mathbb{N} \to \mathcal{R}_u^m$ for the roots of $m$th order trigonometric polynomials in this basic interval $u = [0,2\pi].$

 		 		 Similar to Theorem 7. We then perform a listing procedure using the diagonalization argument on a table of $f_m(n)$ as both $m$ and $n$ vary in $\mathbb{Z}.$ This yields a listing function for the roots of all trigonmetric polynomials in the basic interval $u$, $\mathcal{L}_u:\mathbb{N} \to \mathcal{R}_u^\infty.$

 		 		 Define $u(\eta) = [-(2\pi + \eta-1), 2\pi + \eta-1], \eta \in \mathbb{N}.$ Again let us create a diagonalization table similar to theorem 7, as follows:
 		 		 \begin{equation*}	
	 		 				\begin{array}{|c|cccccc}
	 		 					\hline
								  & n \to \infty & \\
								\hline
								 \eta & \mathcal{L}_{u(1)}(1) & \mathcal{L}_{u(1)}(2) & \mathcal{L}_{u(1)}(3) & \dots & \mathcal{L}_{u(1)}(n) &\dots \\
								 \downarrow  & \mathcal{L}_{u(2)}(1) & \mathcal{L}_{u(2)}(2) & \mathcal{L}_{u(2)}(3) & \dots   & \mathcal{L}_{u(2)}(n) &\dots  \\
								  \infty & \mathcal{L}_{u(3)}(1) & \mathcal{L}_{u(3)}(2) & \mathcal{L}_{u(3)}(3) & \dots  & \mathcal{L}_{u(3)}(n) & \dots   \\
								  & \vdots & \vdots & \vdots &\ddots & \vdots & \ddots \\
								  & \mathcal{L}_{u(\eta)}(1) & \mathcal{L}_{u(\eta)}(2) & \mathcal{L}_{u(\eta)}(3) & \dots & \mathcal{L}_{u(\eta)}(n) & \dots \\
								   & \vdots & \vdots & \vdots &\ddots & \vdots & \ddots \\
							\end{array}
	 		 			\end{equation*}
 		 			Since $$\bigcup_{\eta_2=0}^\infty\left[-(\eta_2+2\pi), 2\pi + \eta_2\right] = \mathbb{R},$$diagonalizing this table leads to a function $\mathfrak{L}:\mathbb{N}\to A$ which denumerates the set of all trigonometric polynomial roots on the reals. This completes proof.
 		 		\end{proof}
 		 		\item The transcendental numbers which are the compliment of the algebraic numbers must be dense in $\mathbb{R}$ as the algebraic numbers $A \nsim \mathbb{R}.$ Hence if one were to choose some real at random, the 'probabiltity' that such a number is  transcendental is roughly $\sim\%100$. That is to say, the majority of numbers in $\mathbb{R}$ are transcendental.
	 		 \end{menumerate} 
\end{menumerate}


\section{A Taste of Topology}
\begin{menumerate}
	\item \textit{An ant walks on the floor,ceiling, and walls of a cubical room. What metric is natural for the ant's view of its world? What metric would a spider consider natural? If the ant wants to walk from a point $p$ to a point $q$, how could it determine the shortest path?} Simple! The ant exists on subspaces of three space. In fact, the ant exists on planes within three space. This means that the ant can only move from one location to another using the inherited metric of a plane subspace in $\mathbb{R}^3,$ not the one provided for $\mathbb{R}^3$ itself. To give a precise definition, if the ant wants to travel from the center of the ceiling to the floor, he must take the shortest path to the closest contiguous plane series to that of the floor. Put more simply, the ant needs to walk to the wall and then to the floor to get to his destination.

	 In this sense we can actually construct a homeomorphism from the ant's $3$-space domain, to a simple subset of $\mathbb{R}^2.$ Take all of the planes constructing the walls and the floor and knock them down such that they become six planes adjacent in a cross pattern. It is easy to see that the ant must use the standard metric in $\mathbb{R}^2$ between those points, $p,q$ in contiguously adjacent subplanes. If the planes are not adjacent, the ant must find the best possible from one plane to another and then readopt the $\mathbb{R}^2$ metric.

	 The spider's space would be endowed with the ant metric except for when the spider wishes to go from points on the ceiling, $C$, to the floor $F$, in that case, the spider would use the taxicab metric for $\mathbb{R}^3$ to reach a point directly below the point on the ceiling, on the floor, and then again assume the ant metric. This process would not work in reverse however, considering that the ant cannot ascend, unless a web has already been created (a can of worms into which we will not venture!).

	 \item If $M = \mathbb{R}^2$ is a metric space, we say that it is endowed with the taxicab metric if and only if $$d(x,y)=\|x-y\|_1=|x_1 - y_1| + |x_2 - y_2|.$$ This name arises if one considers the natural metric for a taxicab driver. Clearly the driver must stay aligned to the grid system which is a subset of $\mathbb{R}^2$. Hence if such a driver wishes to travel from point $q$ to point $p$. He must first go a long the grid in the $x$ direction until his $x$ position coincides with $p$, and then head in the $y$ direction untill his $x$ and $y$ positions are at $p$. The process requires that he travel the distances in the $x$ and $y$ directions independently. Essentially $d(p,q) = \mathrm{distance\_x\_direction} + \mathrm{distance\_y\_direction} =\|x-y\|_1=|p_1 - q_1| + |p_2 - q_2|$. Therefore it is natural to arive at our definition of the taxicab metric.

	 \setcounter{enumi}{7}
	 \item \textit{Prove the following}
	 	\begin{menumerate}
	 	 	\item \textit{Absolute convegence}
	 	 		\begin{theorem}
	 	 		 	Let $(x_n)$ be a convegent sequence in $\mathbb{R}.$ Then, the sequence of absolute values $(|x_n|)$ converges in $\mathbb{R}.$
	 	 		 \end{theorem} 
	 	 		 \begin{proof}
	 	 		 	If $(x_n)$ converges, then there exists a limit $x$ such that for every $\epsilon > 0$, there exists an $N$ such that for all $n > N$, $d(x_n,x) < \epsilon$. Since the natural metric on $\mathbb{R}$ is the absolute value, we have that $|x_n - x| < \epsilon.$ This holds for any sign of $x_n$ and $x$, so $||x_n| - |x||\leq |x_n - x| < \epsilon$ implies that $|x_n| \to |x|.$
	 	 		 \end{proof}
 	 		 \item \textit{State the converse.}
 	 		 	\begin{theorem}
 	 		 	 	If for some sequence $(x_n)$ in $\mathbb{R}$, $|x_n| \to |L|$, then $*(x_n)$ converges to some limit.
 	 		 	 \end{theorem} 
	 		 \item \textit{Disprove the the previous statement. } We show by counter example, the theorem cannot be true. Take $(a_n) = (-1)^n$. Clearly $|a_n| \to 1,$ but there exists and $\epsilon >0, $ say $\epsilon = 0.5$, such that for all $N$ there exists an $n$, (take the next odd $n$), such that $|a_n - 1| = |-1 - 1| = 2 \geq 0.5 = \epsilon.$ Hence $(a_n)$ does not converge to the limit of its absolute value sequence. The theorem cannot be true. 
	 	 \end{menumerate} 

 	 \setcounter{enumi}{11}
 	 \item \textit{Prove the following}
 	 	\begin{menumerate}
 	 		\item \textit{Bijective:} 
		 	 	\begin{theorem}
		 	 		If $(p_n)$ is a sequence, and $f: \mathbb{N} \to \mathbb{N}$  is a bijection, then the sequence $(q_k)_{k\in \mathbb{N}}$ with $q_k = p_{f(k)}$ is a rearangement. If $p_n \to L, $ then $q_n \to L$ for all such $f.$
		 	 	\end{theorem}
		 	 	\begin{proof}
		 	 		If $p_n \to L$, then for all $\epsilon > 0$ $p_n$ is a distance less than $\epsilon$ from $L$ for all but finiteley many $n$. Let $N$ be the number of those first elements of the sequence which are more than $\epsilon$ to L. Since $f$ is bijective, the set $f^{-1}(\{0,...,N\})$ is also finite, and therefore there exists and element $M \in \mathbb{N}$ such that for all $n \in f^{-1}(\{0,\dots,N\}), M > n$. Therefore, for all $m > M$, we have that $|q_m - L|< \epsilon$ if and only if $q_m \to L$.
		 	 	\end{proof}
	 	 	\item \textit{Injective}
		 	 	\begin{theorem}
		 	 		If $(p_n)$ is a sequence, and $f: \mathbb{N} \to \mathbb{N}$  is an injection, then the sequence $(q_k)_{k\in \mathbb{N}}$ with $q_k = p_{f(k)}$ is a rearangement. If $p_n \to L, $ then $q_n \to L$ for all such $f.$
		 	 	\end{theorem}
		 	 	\begin{proof}
		 	 	If $f$ is an injection the theorem will still hold. Recall that an injection implies that each element in the range has a singleton pre-image. Furthermore, each element in the co-domain has a singleton or empty pre-image. Thus if $p_n$ is convergent, then for all $\epsilon > 0$, there exists an $N$ such that for all $n > N$, $ |p_n - L | < \epsilon$. Hence $\bigcup_{j=1}^N f^{pre}(j) = P$ is finite and contains the $n$ for which $q_n$ is a distance greater than or equal to $\epsilon$ away from L. Thus take the maximal element of, $P$, say $M$. Then for all $n > M$ (such that there exists a $x$ with $f(x) = n$. Note: there must be infinitely many such $n$), \  $|q_n - L | < \epsilon$, and hence $q_n \to L.$
		 	 	\end{proof}
	 	 \item \textit{Surjective:}
	 	 	In the case that $f$ is only a surjection, then we show that not all rearrangements converge by counterexample. Take, for example, a sequence $p_n = \frac{n}{n+1}.$ This sequence clearly converges to $1$, but consider the surjective rearrangement, $f(2n) = 1, f(2n-1) = n.$ Such a map  takes even elements and maps them to $1$, and otherwise takes odd elements and maps them to half their double. It's easy to see that this function is surjective, $f(1) = 1, f(2) = 1, f(3) = 2, f(4) =1, \dots$. However, there exists an $\epsilon > 0,$ say $\epsilon = 0.5$, such that for all $N$, $n > N$, and $n= 2N$  implies that $|p_{f(n)} - 1| \geq \epsilon.$ Hence, the rearrangement does not converge. By counterexample, not all surjective rearrangements converge if the normal sequence does.

 	 	\end{menumerate}

 	 	
 	 	
\end{menumerate}

\end{document}