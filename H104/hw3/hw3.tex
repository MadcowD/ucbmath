%%%%%%%%%%%%%%%%%%%%%%%%%%%%%%%%%%%%%%%%%%%%%%%%%%%%%%%%%%%%%%%%%%
%%%                      Homework 3                            %%%
%%%%%%%%%%%%%%%%%%%%%%%%%%%%%%%%%%%%%%%%%%%%%%%%%%%%%%%%%%%%%%%%%%

\documentclass[letter]{article}

\usepackage{lipsum}
\usepackage[pdftex]{graphicx}
\usepackage[margin=1.5in]{geometry}
\usepackage[english]{babel}
\usepackage{listings}
\usepackage{amsthm}
\usepackage{amssymb}
\usepackage{framed} 
\usepackage{amsmath}
\usepackage{titling}
\usepackage{fancyhdr}

\pagestyle{fancy}


\newtheorem{theorem}{Theorem}
\newtheorem{definition}{Definition}

\newenvironment{menumerate}{%
  \edef\backupindent{\the\parindent}%
  \enumerate%
  \setlength{\parindent}{\backupindent}%
}{\endenumerate}







%%%%%%%%%%%%%%%
%% DOC INFO %%%
%%%%%%%%%%%%%%%
\newcommand{\bHWN}{3}
\newcommand{\bCLASS}{MATH H104}

\title{\bCLASS: Homework \bHWN}
\author{William Guss\\26793499\\wguss@berkeley.edu}

\fancyhead[L]{\bCLASS}
\fancyhead[CO]{Homework \bHWN}
\fancyhead[CE]{GUSS}
\fancyhead[R]{\thepage}
\fancyfoot[LR]{}
\fancyfoot[C]{}
\usepackage{csquotes}

%%%%%%%%%%%%%%

\begin{document}
\maketitle
\thispagestyle{empty}


%%%%%%% Be sure to set the counter and use menumerate
\setcounter{section}{1}
\section{A Taste of Topology}
	\begin{menumerate}
		\setcounter{enumi}{16}
		\item \textit{Isometry and homeomorphism of roman letters.} To discuss the isometry of roman letters, the typeface of these letters is very important. We will only coinisder this issue when it arises.

		Before we begin our discussion of continuous transformation between these topological spaces, it is important to mention that such transformations preserve connectedness. Therefore, there are immediately invariants which distinguish certain topological spaces from each other. Consider the topological space of $O$, this space cannot be continuously transformed to $|$ because the connectedness of the end points is not preserved. With some confidence, we can even say that the number of holes in an object is a topological invariant.

		Examination of the difference between letters $X$ and $C$ requires that we employ another topological invariant, vertices. If two lines which meet at a common vertex make a space, then this space is only homeomorphic to a space with a similar property. To see this, if the vetex is removed from this space and another space, the homeomorphism should this new space and map it to a space with four disjoint connected pieces. Thus, the number of vertices is essential to defining topological equivalence classes. 

		We now buuild these classes. Let $\tau_{h,v_3,v_4,\dots}$ denote a subpsace of $\mathbb{R}^2$ wiuth $h$ holes, $v_3$ three-vertices, $v_4$ four vertices, etc. Then
		\begin{equation}
			\begin{aligned}
				\tau_{0,0,0,\dots} &\cong C,I,J,L,M,N,S,U,V,W,Z\\
				\tau_{0,1,0,\dots} &\cong E,F,T,Y \\
				\tau_{0,2,0,\dots} &\cong H \\ 
				\tau_{0,0,1,\dots} &\cong X,K \\
				\tau_{1,0,0,\dots} &\cong D,O \\
				\tau_{1,1,0,\dots} &\cong P \\
				\tau_{1,2,0,\dots} &\cong A,R\\
				\tau_{1,0,1,\dots} &\cong Q \\
				\tau_{2,2,0,\dots} &\cong B.
			\end{aligned}
		\end{equation}
		As for isometry, depedning on the typeface, one might find an isometry between $M$ and $W$ since distances could be preserved. Furthermore, if $C =\  \subset$ then $\subset$ is isometric to $\cup = U$. Otherwise no other isometries exist.
		\item  \textit{Is $\mathbb{R}$ homeomorphic to $\mathbb{Q}?$}
			\begin{theorem}
				The set of all reals $\mathbb{R}$ is not homeomorphic to $\mathbb{Q}$.
			\end{theorem}
			\begin{proof}
				For the sake of contradiction, suppose there existed a homeomorphism betwee the two metric spaces. Then, there exists a bicontinuous bijection, say $f$, such that bijectively $\mathbb{R}\to\mathbb{Q}$. Therefore $\mathbb{R} \sim \mathbb{Q}$ which leads to a contradiction since $\mathbb{R}\nsim\mathbb{Q}.$ This completes the proof.
			\end{proof}
		\item \textit{Is $\mathbb{Q}$ homeomorphic to $\mathbb{N}?$ }  
		\begin{theorem}
			The rational numbers are not homeomorphic to the natural numbers.
		\end{theorem}
		\begin{proof}
			As from previous exercises, $\mathbb{N} \sim \mathbb{Q},$ so we must show that there cannot possibly exist a bicontinuous bijection between the two sets. For any $f: \mathbb{Q} \to \mathbb{N}$ bijective,
			take a singleton in $\mathbb{N},$ $S= \{n\}.$ Then $S$ is trivially open. The inverse of $f$ applied to $S$ must also be a singleton whose contents are a point in $\mathbb{Q}.$ Since $\mathbb{Q}$ inherits its topology from $\mathbb{R}$ it follows that $f^{-1}(S) = S'$ is not open. Therefore $f$ cannot be continuous.

			Since the above has been shown for any $f$ bijective, there cannot exist a homeomorphism between the two spaces. The proof is complete.
		\end{proof}


		\setcounter{enumi}{20}
		\item \emph{Is the plane without four points on the $x$-axis homeomorphic to the plane without four points in arbitrary configuration?} Yes. Consider the following less rigorous argument. Consider that we can draw a continuous curve from the origin through the four points of aribtrary configuration. We can find a cointinuous deformation of the $x$-axis to this line. That is we can bend the initial plane continuouskly so that the $x$-axis becomes this curve, and essentially there is a homeomorphism between these two spaces. We then can stretch (contract/expand) the line so that the points line up. So these two spaces are topologically equivalent. 
		\item \emph{Prove the following}
			\begin{theorem}
				If every subset of a metric space $M$, which closed and bounded, is compact then the metric space $M$ is complete.
			\end{theorem}
			\begin{proof}
				If $\{x_n\} \subset A^{closed,bounded,compact}$ is a cauchy sequence then for every $\epsilon/2 > 0$ there exists an $N_1$ such that for all $n,m > N_1$, $d(x_n,x_m)<\epsilon.$ Since $A$ is compact we have for some subsequence $x_{q_k},$ there exists an $N_2$ such that for every $k > N_2$, $d(x_{q_k}, x) < \epsilon/2.$ Then for every $n > \max\{N_2,N_1\} = N,$ $d(x_n,x) \leq d(x_n, x_{q_n}) + d(x_{q_n}, x) < \epsilon$. So a sequence cauchy in $A$ implies that such a sequence converges. 

				Hence, $M$ is complete, as is the proof!
			\end{proof}


		\item \textit{Prove that $(0,1)$ is an open subset of $\mathbb{R}$ but 
			not of $\mathbb{R}^2,$ when we think of $\mathbb{R}$ as the $x$-axis in $\mathbb {R}^2$.} 
			\begin{theorem}
				The open interval $(0,1)$ in $\mathbb{R}$ is not open in $\mathbb{R}^2$ when $\mathbb{R}$ is the $x$-axis.
			\end{theorem}
			\begin{proof}
				Simple! When $\mathbb{R}$ is the $x$-axis, we can denote it to be 
				\begin{equation}
					\begin{aligned}
						\mathbb{R}_x &=  \mathbb{R} \times \{0\} 
							= \left\{(x,y) \in \mathbb{R}^2\ | \ y = 0, x \in \mathbb{R} \right\}.
					\end{aligned}
				\end{equation}
				Furthermore, we denote an interval on the $x$-axis to be 
				\begin{equation}
					(a,b)_x = (a,b) \times \{0\}.
				\end{equation}
				Recall that the definiton of openness in $\mathbb{R}^2$ with its standard metric is such that $S$ is an open set in $\mathbb{R}^2$ if for any element $p\in S$ there exists an $r > 0$ such that $d_E(p,q) <r \implies q \in S.$ 

				Now take any $s \in (0,1)_x.$ Suppose that there existed an $r >0$ such that $d_E(s,q) \implies q \in (0,1)_x$. If $s = \langle s_1, 0 \rangle$, then $|s - q| =  |\langle s_1 - q_1, 0 - q_2| < r$ implies that $q \in (0,1)_x$. Take the case of $q_1 = s_1$ and $0 < s_1 < r$. Clearly $d_E(s,q) < r$, but $q = \langle s_1, q_2 \rangle$ for non-zero $q_2$. This means that $q\notin (0,1)_x$ which is a contradiction. Hence, $(0,1)_x$ is not open. The proof is complete.
			\end{proof}


		\setcounter{enumi}{25}
		\item \textit{Prove the following.}
			\begin{theorem}
			 	A set $U \subset M$ is open if and only if none of its points are limits of its complement.
			 \end{theorem} 
			 \begin{proof}
			 If $U$ is an open subset of $M$, then by theorem $4$ its complement is closed. To see this, if $p_n \to p,$ and $p_n \in U^c$ then it is clear that $p \in U^c.$ If $p \notin U^c$ then $p \in U$ and there exists $r >0,$ $d(p,q) < r \implies q \in S.$ Since $p$ is a limit of $(p_n)$ there exists a $N$ such that for all $n \geq N,$ $d(p_n,p) <r$ which implies that $p_n \in U$, a contradiction! Therefore if $U$ is open is compliment is closed and $U^c$ contains all of its limits.

			 In the case that $U^c$ contains all of its limits, then we claim that its compliment is open. By theorem 4 this holds since $U^c$ is closed and therefore its compliment $U$ is open. If $U$ is not open then there exists a $p$, for all $r >0$ there exists a $q_n$ such that, $d(p,q_n) < r=1/n$ and $q_n \notin U.$ This implies that $p$ is the limit of a sequence which is not contained in $U$. That is $q_n \in U^c$ and $q_n \to P \notin U^c$ which is a contradiction since $U^c$ contains all of its limit points. 

			 This completes the proof.
			 \end{proof}
		 \setcounter{enumi}{27}
		 \item \emph{A map $f: M \to N$ is \textbf{open} if for each open set $U\subset M$, the image set $f(U)$ is open in $N$. }
		 	\begin{menumerate}
		 		\item \emph{If $f$ is open, is it continuous?} This implication is not true since $f$ is continuous if for every set $S \subset N$ open, $f^{pre}(S)$ is open in $M$. If $f$ is bijective, then $f$ open implies that $f^{-1}$ is continuous but not the other way around.
		 		\item \emph{If $f$ is a homeomorphism, is it open?} In the case that $f$ is a homeomorphism, $f$ is bicontinuous which implies that its inverse is continuous. Hence if $S$ is an open subset of $M$ then $f^(S)\subset N$ is open. Furthermore if $T \subset N$ is open then $f^{-1}(T) \subset N$ is open. Therefore $f$ is open.
		 		\item \emph{If $f$ is an open, continuous bijection, is it a homeomorphism?}  If $f$ is open then for every $S\subset N$ open then $f(S)$ is open. This holds if and only if $f^{-1}$ is continuous. So, $f$ is a bicontinuous bijection and therefore is a homeomorphism.
		 		\item \emph{If $f: \mathbb{R} \to \mathbb{R}$ is a continuous surjection must it be open?} No, there is no logical resolution to the statement. Take $x\mapsto 0$ if $x \in (0,1)$ and $x\mapsto x -1$ if $x \geq 1$ and $x \mapsto x$ otherwise. This is clearly a continuous surjection, but is not open. 
		 		\item \emph{If $f:\mathbb{R} \to \mathbb{R}$ is a continuous, open surjection, must it be a homeomorphism?} Yes. We need to show that $f$ is an injection for the above statement to hold. We rely heavily on the idea that $f$ continuous implies that on a closed and bounded interval $f$ attains a maximum. Suppose that $f$ is not an injection, then there exist $x <y$ with $f(x) = f(y).$ If on an interval $(x,y), f((x,y))$ is constant then clearly $f((x,y)) = [f(x),f(y)]$ which is closed. If it is not the case that the function is constant then $f([x,y])$ attains a maximum in $f((x,y))$ and therefore is not closed. This is a contradiction and $f$ must be injective and therefore a homeomorphism.
		 		\item \emph{If $\ \mathbb{R}$ from (e) is replaced with the unit circle does this theorem hold?} No. Consider $f: S^1 \to S^1$ such that $f:S^1 \to \mathbb{C} \to S^1$ with
		 		 \begin{equation*}
		 		 	x \mapsto \|x\|^2\left(
		 		 	\begin{array}{c}
		 		 		\cos(2\cos^{-1}(x_1)) \\
		 		 		\sin(2\cos^{-1}(x_1))
		 		 	\end{array}\right).
	 		 	\end{equation*}
	 		 	Clearly $S^1$ is closed under $f$. Furthermore $f$ is continuous and open since $\cos^{-1}$ is continuous and open from $\mathbb{R} \to [0,2\pi]$ and thereafter $\sin$ and $\cos.$ We also claim that $f$ is surjective, this is easy since any $x \in S^1$ visibly corresponds to at least another point $x'$ with half its angle from the origin and equal distance from the origin. However, $f$ is not injective because $x$ and $-x$ both map to $f(x)=f(x')$. This completes the counter example.
	 		 	This would suggest that self-homeomorphism is a property of the topology of a space.

	 		 	 In $\mathbb{R}$ it suffices to show that open continuous surjections are slef-homeomorphisms. Is this itself a property of the line or is there something else occuring here. For which spaces does this property exist? Since $\mathcal{X} \cong \mathcal{Y}$ connectedness is preserved. In the case of the line and the circle, we could try to assume a homeomorphism, then the submap $f':[0,1] \setminus \{\frac{1}{2}\} \to S^1 \setminus \{f(\frac{1}{2})\}$ is also a homeomorphism if and only if connectedness of these two subspaces are preserved. $[0,1] \setminus \{\frac12\}$ is not connected because there cannot exist some $\gamma_l :[0,1] \to [0,1] \setminus \{\frac12\}$ such that $\gamma_l$ is continuous by contradiction to the intermediate value theorem. However $S^1 \setminus \{f(\frac12)\}$ is connected since for any point I can just travel along the side of the circle for which there is no hole and reach that point.
		 	\end{menumerate}
 	\end{menumerate}

\end{document}