%%%%%%%%%%%%%%%%%%%%%%%%%%%%%%%%%%%%%%%%%%%%%%%%%%%%%%%%%%%%%%%%%%
%%%                      Homework 6                            %%%
%%%%%%%%%%%%%%%%%%%%%%%%%%%%%%%%%%%%%%%%%%%%%%%%%%%%%%%%%%%%%%%%%%

\documentclass[letter]{article}

\usepackage{lipsum}
\usepackage[pdftex]{graphicx}
\usepackage[margin=1.5in]{geometry}
\usepackage[english]{babel}
\usepackage{listings}
\usepackage{amsthm}
\usepackage{amssymb}
\usepackage{framed} 
\usepackage{amsmath}
\usepackage{titling}
\usepackage{fancyhdr}

\pagestyle{fancy}


\newtheorem{theorem}{Theorem}
\newtheorem{definition}{Definition}

\newenvironment{menumerate}{%
  \edef\backupindent{\the\parindent}%
  \enumerate%
  \setlength{\parindent}{\backupindent}%
}{\endenumerate}







%%%%%%%%%%%%%%%
%% DOC INFO %%%
%%%%%%%%%%%%%%%
\newcommand{\bHWN}{6}
\newcommand{\bCLASS}{MATH H104}

\title{\bCLASS: Homework \bHWN}
\author{William Guss\\26793499\\wguss@berkeley.edu}

\fancyhead[L]{\bCLASS}
\fancyhead[CO]{Homework \bHWN}
\fancyhead[CE]{GUSS}
\fancyhead[R]{\thepage}
\fancyfoot[LR]{}
\fancyfoot[C]{}
\usepackage{csquotes}

%%%%%%%%%%%%%%

\begin{document}
\maketitle
\thispagestyle{empty}


%%%%%%% Be sure to set the counter and use menumerate
\setcounter{section}{1}
\section{A Taste of Topology}
\begin{menumerate}
	%%%%%%%%%%%%%%%%%%%%%%%%%% 115 %%%%%%%%%%%%%%%%%%%%%%%%%%%%%%%%%%
	\setcounter{enumi}{114} % 
	\item \emph{Rotate the unit circle by a fixed angle $\alpha$, say $R: S \to S$.}
		\begin{menumerate}
			\item \emph{Show the following.}
			\begin{theorem}
				If $\alpha/\pi$ is rational, each orbit of $R$ is a finite set.
			\end{theorem}
			\begin{proof}
				Simple! Since $d = \alpha/\pi \in \mathbb{Q}$ we have that the rotation is equivalently $d = p/q.$ Then, multiplication of $d$ by two times its reciprocal is $2.$ Under such a relation, $2\alpha\pi$ is a complete rotation from the origin of the orbit. Thereafter. That is by eventual partial rotation, we achieve the identity element of the orbit. The orbital group must be finite since only a finite rational amount of rotations were required to acheive this identity.
			\end{proof}

			\item \emph{Show the following.}
			\begin{theorem}
				If $\alpha/\pi$ is irrational, each orbit is infinite and has closure equivalent to $S^1$,
			\end{theorem}
		\end{menumerate}

	\setcounter{enumi}{125}
	%%%%%%%%%%%%%%%%%%%%%%%%%% 126 %%%%%%%%%%%%%%%%%%%%%%%%%%%%%%%%%%
	\item \emph{Prove the following.} 
	\begin{theorem}
		If $E$ is an unvountable subset of $\mathbb{R}$, then there exsists a point at which $E$ condenses.
	\end{theorem}
	\begin{proof}
		We know that $p\in E$ is a condensation point iff for every $r > 0$ $\mathbb{R}_r (p)$ contains uncountably many points of $E.$ We wish to show this by using a decimal expansion of $p.$ There exists an interval $[n,n+1)$ containing uncountably many elements of $E$.  Suppose for the sake of contradiction that there were no particulart interval in which uncountable elements of $E$ resided.Intuitively, this means that at the least there is a collection of intervals $\{I_i = [i, i+1)\}$ such that $\bigcup_i I_i \supset E$, and furthermore for each $I_i, E \cap I_i$ is countable. Since there is a rational in each of these intervals the collection of intervals is countable. Therefore the whole set $E$ must be countable, a contradiction to $E$ uncountable. 

		Let the containing interval be $E_0.$ We will use the following notation for sub intervals. Let $I^k_i$ denote the interval of the form
		 $$I^k_i = \left[ \sum_j^k \frac{\omega_j}{10^j}, \sum_j^k \frac{\omega_j}{10^j}  + \frac{i}{10^{k+1}}\right]$$
		  We want to show that for every $k$ there exists a sequence of $\omega_k$ such that $I^k_i$ is uncountable for some $i.$ Let $I^0 = E_0$ be uncountable with $\omega_0 = n$. Furthermore, if $I^k_i$ is satisfied by some $i$, let $\omega_{k+1} = i.$


		  Suppose that $E_k = I^k_i$ for a satisfying $i$ is uncountable. Then we wish to show that $E_k$ contains a subinterval where uncountably many elements of $E$ reside. Suppose for the sake of contradiction that for every $i$, 
		  $I^{k+1}_i \cap E$ is countable. Because $i$ is finite and enumerable, we have that $ \bigcup_i I^{k+1}_i \cap E$ is countable which is a contradiction to $I^k_{\omega_k} = E_k \cap E$ being uncountable. So there is a subinterval $I^{k+1}_i$ which contains uncountably many elements of $E$. Choose $\omega_{k+1} := i$. 

		  Therefore by the induction hypothesis, $E_k$ is uncountable for all $k$. Lastly we must show that there is a $p$ common to all $E_k$ and then that for every $\epsilon$ neighboorhood of $p$, there are uncountably many elements of $E$ therein. Let $$p_n = \omega_1.\omega_2\omega_3\dots\omega_n$$. For every $\epsilon > 0,$ there exists an $N = \log_{10} \epsilon ^-1$ such that for all $m,n \geq N$, where without loss of generality $n>m$,  $$|\omega_0.\omega_1\dots\omega_n - \omega_0.\omega_1\dots\omega_m| = 10^{-m}|\omega_m\dots\omega_n| \leq 10^{-M} = \epsilon. $$ Hence $p_n \to p \in E$ by the completeness of $\mathbb{R}$ and for all $r > 0$ consider $E_j$ such that $j$ is the ceiling of $\log_{10}r^-1+1$. The set $E_j \subset \mathbb{R}_r(p)$ contains uncountably many elements of $E$ as aforementioned, and so every $r$ neighboorhood of $p \in E$ contains uncountably many points, and $p$ is a cluster point.

		  This completes the proof.
	\end{proof}
	%%%%%%%%%%%%%%%%%%%%%%%%%% 127 %%%%%%%%%%%%%%%%%%%%%%%%%%%%%%%%%%
	\item
	%%%%%%%%%%%%%%%%%%%%%%%%%% 128 %%%%%%%%%%%%%%%%%%%%%%%%%%%%%%%%%%
	\item
	%%%%%%%%%%%%%%%%%%%%%%%%%% 129 %%%%%%%%%%%%%%%%%%%%%%%%%%%%%%%%%%
	\item
	%%%%%%%%%%%%%%%%%%%%%%%%%% 130 %%%%%%%%%%%%%%%%%%%%%%%%%%%%%%%%%%
	\item 
	%%%%%%%%%%%%%%%%%%%%%%%%%% 131 %%%%%%%%%%%%%%%%%%%%%%%%%%%%%%%%%%
	\item . \\
	. \\ 
	. \\ 
	 . \\
	 . \\
	 . \\


	\setcounter{enumi}{132}
	%%%%%%%%%%%%%%%%%%%%%%%%%% 133 %%%%%%%%%%%%%%%%%%%%%%%%%%%%%%%%%%
	\item 

	\setcounter{enumi}{151}
	%%%%%%%%%%%%%%%%%%%%%%%%%% 152 %%%%%%%%%%%%%%%%%%%%%%%%%%%%%%%%%
	\item \textbf{Greens theorem:}

	\emph{If $E$ is $2$-cell \\
		and $\phi$ is $1$-form, \\ 
		then over $E$'s $\partial$ \\
		the trace must transform }

		\emph{		A volume may be \\
		with $\phi$'s' diff'rential \\
		again over $E$ \\
		a sure bit equal.}
\end{menumerate}

\end{document}