%%%%%%%%%%%%%%%%%%%%%%%%%%%%%%%%%%%%%%%%%%%%%%%%%%%%%%%%%%%%%%%%%%
%%%                      Homework 9                            %%%
%%%%%%%%%%%%%%%%%%%%%%%%%%%%%%%%%%%%%%%%%%%%%%%%%%%%%%%%%%%%%%%%%%

\documentclass[letter]{article}

\usepackage{lipsum}
\usepackage[pdftex]{graphicx}
\usepackage[margin=1.5in]{geometry}
\usepackage[english]{babel}
\usepackage{listings}
\usepackage{amsthm}
\usepackage{amssymb}
\usepackage{framed} 
\usepackage{amsmath}
\usepackage{titling}
\usepackage{fancyhdr}

\pagestyle{fancy}


\newtheorem{theorem}{Theorem}
\newtheorem{definition}{Definition}

\newenvironment{menumerate}{%
  \edef\backupindent{\the\parindent}%
  \enumerate%
  \setlength{\parindent}{\backupindent}%
}{\endenumerate}







%%%%%%%%%%%%%%%
%% DOC INFO %%%
%%%%%%%%%%%%%%%
\newcommand{\bHWN}{9}
\newcommand{\bCLASS}{MATH H104}

\title{\bCLASS: Homework \bHWN}
\author{William Guss\\26793499\\wguss@berkeley.edu}

\fancyhead[L]{\bCLASS}
\fancyhead[CO]{Homework \bHWN}
\fancyhead[CE]{GUSS}
\fancyhead[R]{\thepage}
\fancyfoot[LR]{}
\fancyfoot[C]{}
\usepackage{csquotes}

%%%%%%%%%%%%%%

\begin{document}
\maketitle
\thispagestyle{empty}

%#59, 61-63, 67-70.
	\begin{menumerate}
		\setcounter{enumi}{58}
		\item %59

		\begin{theorem}
			If $\sum a_n$ converges and $a_n \geq 0,$ then show $\sum \sqrt{a_n}/n$ converges.
		\end{theorem}

		\begin{proof}
			Let $x = (\sqrt{a_n})_n$, $y = \left(\frac{1}{n}\right)_n$. Clearly $y \in \ell_1,$ and since $\sum a_n \to c,$ $a_n \to 0$ implies that $\sqrt{a_n} \to 0.$ Therefore, $x \in \ell_1.$ Since $\ell_1$ is an inner product space, the cauchy schwartz inequality gives, 
			$$0 \leq \sum_{n=1}^\infty \frac{\sqrt{a_n}}{n} = \langle x,y \rangle \leq |x||y| = \sqrt{\sum_{n=1}^\infty a_n} \sqrt{\sum_{n=1}^\infty \frac{1}{n^2}} = \sqrt{\frac{c}{6}}\pi$$
			and so the series is bounded and therefore converges.
		\end{proof}

		\setcounter{enumi}{60}
		\item Consider the following $\{a_n\} \in \ell_1.$ We say that $a_n = 1/4^n$ if $n$ odd and $a_n = 1/2^n$ otherwise. Clearly $$0 < \sum_{n\in\mathbb{N}}a_n = \sum_{n\ \text{odd}} \frac{1}{4^n} + \sum_{n\ \text{even}} \frac{1}{2^n} < \sum \frac{1}{2^n} < \sum \frac{1}{n^2} = \frac{\pi^2}{6}.$$ So the series converges. Let $\rho_N = \sup_{n>N} |a_{n+1}|/|a_n| = \sup_{n>N} 2^n = \infty$. So clearly $\rho = \lim \rho_N = \infty,$ and yet the series converges. If we were to suppose that $\lambda = \rho$ then the test would be wrong since $\lambda > 1$ implies divergence. So it must be the case that the test is inconclusive when $\rho \geq 1.$
		\item 
		\begin{theorem}
			Let $\{a_n\} \in \ell_1$ be a monotonically non-increasing sequence. Then, the series $\sum a_n$ converges if and only if the associated dyadic series 
			$$a_1 + 2a_2 +4a_4 + 8a_8 + \cdots = \sum 2^k a_{2^k}$$
			converges.
		\end{theorem}

		\begin{proof}
		Suppose that $\sum 2^na_{2^n}$ converges. Since the sequence is monotone decreasing, we have that $a_1 \geq a_1$, $2a_2 > a_2 + a_3, \dots, 2^ka_{2^k} > a_{2^k} + \dots + a_{2^{k+1}}.$ In other words the series is dominated by the dyagic series. Therefore by the comparison test, the series must converge.

		CIBERSKEVET.  
		\end{proof}

		\item \begin{theorem}
			The series $\sum 1/(k(\log k)^p)$ diverges for $p \leq 1$ and converges for $p > 1.$
		\end{theorem}
		\begin{proof}
		 We will show the theorem using the integral test. In particular, we need only show that the function $$f(x) = \frac{1}{x\log^px}$$ has convergent/divergent improper integrals for the corresponding $p$-cases. In the special case of $p = 1$ simple $\upsilon$-substituion yields that, $\int_2^\infty f(x)\ dx =  \log(\log(x))|_2^n \to \infty.$ When $p \neq 1$, we have that the indefinite integral is $$F(n) = \int_2^n \frac{1}{x\log^px}\ dx = \frac{\log^{1-p}(x)}{1-p}\Bigg|_2^n.$$ It is clear in the  case that $p>1,$ the log function is in the denominator and therefore $F(n) \to 0 + \log^p(2)/(1-p) \in \mathbb{R}.$ Otherwise, for $p < 1, F(n) \to \infty.$ Therefore, by the integral test we have that the series converges for $p >1$ and diverges otherwise.
		\end{proof}

		\setcounter{enumi}{66}
		\item
		\begin{theorem}
			Let $\{a_k\} \in \ell_1$ such that $a_k$ has the same sign for all $k.$
			The series $\sum a_k$ converges if and only if $\prod (a_k +1)$ converges.
		\end{theorem}
		\begin{proof}
			The infinite product $\prod (a_k +1)$ converges if and only if $$\log\left(\prod_{k =1}^\infty a_k +1\right) = \sum_{k=1}^\infty \log(a_k+1)$$ converges. If $\sum a_n $ converges then $a_n \to 0$, and if $\sum a_n$ does not converge then we do not grant this. By the comparison test $\lim ln(1+a_n)/a_n = 1$ if and only if $\sum a_n$ converges if and only if the infinite product converges. This completes the proof.  
		\end{proof}

		\item %68
		\item %69
		\item %70
	\end{menumerate}

\end{document}