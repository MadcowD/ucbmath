%%%%%%%%%%%%%%%%%%%%%%%%%%%%%%%%%%%%%%%%%%%%%%%%%%%%%%%%%%%%%%%%%%
%%%                      Homework 9                            %%%
%%%%%%%%%%%%%%%%%%%%%%%%%%%%%%%%%%%%%%%%%%%%%%%%%%%%%%%%%%%%%%%%%%

\documentclass[letter]{article}

\usepackage{lipsum}
\usepackage[pdftex]{graphicx}
\usepackage[margin=1.5in]{geometry}
\usepackage[english]{babel}
\usepackage{listings}
\usepackage{amsthm}
\usepackage{amssymb}
\usepackage{framed} 
\usepackage{amsmath}
\usepackage{titling}
\usepackage{fancyhdr}

\pagestyle{fancy}


\newtheorem{theorem}{Theorem}
\newtheorem{definition}{Definition}

\newenvironment{menumerate}{%
  \edef\backupindent{\the\parindent}%
  \enumerate%
  \setlength{\parindent}{\backupindent}%
}{\endenumerate}







%%%%%%%%%%%%%%%
%% DOC INFO %%%
%%%%%%%%%%%%%%%
\newcommand{\bHWN}{9}
\newcommand{\bCLASS}{MATH H104}

\title{\bCLASS: Homework \bHWN}
\author{William Guss\\26793499\\wguss@berkeley.edu}

\fancyhead[L]{\bCLASS}
\fancyhead[CO]{Homework \bHWN}
\fancyhead[CE]{GUSS}
\fancyhead[R]{\thepage}
\fancyfoot[LR]{}
\fancyfoot[C]{}
\usepackage{csquotes}

%%%%%%%%%%%%%%

\begin{document}
\maketitle
\thispagestyle{empty}

%#59, 61-63, 67-70.
	\begin{menumerate}
		\setcounter{enumi}{58}
		\item %59

		\begin{theorem}
			If $\sum a_n$ converges and $a_n \geq 0,$ then show $\sum \sqrt{a_n}/n$ converges.
		\end{theorem}

		\begin{proof}
			Let $x = (\sqrt{a_n})_n$, $y = \left(\frac{1}{n}\right)_n$. Clearly $y \in \ell_1,$ and since $\sum a_n \to c,$ $a_n \to 0$ implies that $\sqrt{a_n} \to 0.$ Therefore, $x \in \ell_1.$ Since $\ell_1$ is an inner product space, the cauchy schwartz inequality gives, 
			$$0 \leq \sum_{n=1}^\infty \frac{\sqrt{a_n}}{n} = \langle x,y \rangle \leq |x||y| = \sqrt{\sum_{n=1}^\infty a_n} \sqrt{\sum_{n=1}^\infty \frac{1}{n^2}} = \sqrt{\frac{c}{6}}\pi$$
			and so the series is bounded and therefore converges.
		\end{proof}

		\setcounter{enumi}{60}
		\item Consider the following $\{a_n\} \in \ell_1.$ We say that $a_n = 1/4^n$ if $n$ odd and $a_n = 1/2^n$ otherwise. Clearly $$0 < \sum_{n\in\mathbb{N}}a_n = \sum_{n\ \text{odd}} \frac{1}{4^n} + \sum_{n\ \text{even}} \frac{1}{2^n} < \sum \frac{1}{2^n} < \sum \frac{1}{n^2} = \frac{\pi^2}{6}.$$ So the series converges. Let $\rho_N = \sup_{n>N} |a_{n+1}|/|a_n| = \sup_{n>N} 2^n = \infty$. So clearly $\rho = \lim \rho_N = \infty,$ and yet the series converges. If we were to suppose that $\lambda = \ro$ then the test would be wrong since $\lambda > 1$ implies divergence. So it must be the case that the test is inconclusive when $\ro \geq 1.$
		\item %62
		\item %63

		\setcounter{enumi}{66}
		\item %67
		\item %68
		\item %69
		\item %70
	\end{menumerate}

\end{document}