%%%%%%%%%%%%%%%%%%%%%%%%%%%%%%%%%%%%%%%%%%%%%%%%%%%%%%%%%%%%%%%%%%
%%%                      Homework 9                            %%%
%%%%%%%%%%%%%%%%%%%%%%%%%%%%%%%%%%%%%%%%%%%%%%%%%%%%%%%%%%%%%%%%%%

\documentclass[letter]{article}

\usepackage{lipsum}
\usepackage[pdftex]{graphicx}
\usepackage[margin=1.5in]{geometry}
\usepackage[english]{babel}
\usepackage{listings}
\usepackage{amsthm}
\usepackage{amssymb}
\usepackage{framed} 
\usepackage{amsmath}
\usepackage{titling}
\usepackage{fancyhdr}

\pagestyle{fancy}


\newtheorem{theorem}{Theorem}
\newtheorem{definition}{Definition}

\newenvironment{menumerate}{%
  \edef\backupindent{\the\parindent}%
  \enumerate%
  \setlength{\parindent}{\backupindent}%
}{\endenumerate}







%%%%%%%%%%%%%%%
%% DOC INFO %%%
%%%%%%%%%%%%%%%
\newcommand{\bHWN}{9}
\newcommand{\bCLASS}{MATH H104}

\title{\bCLASS: Homework \bHWN}
\author{William Guss\\26793499\\wguss@berkeley.edu}

\fancyhead[L]{\bCLASS}
\fancyhead[CO]{Homework \bHWN}
\fancyhead[CE]{GUSS}
\fancyhead[R]{\thepage}
\fancyfoot[LR]{}
\fancyfoot[C]{}
\usepackage{csquotes}

%%%%%%%%%%%%%%

\begin{document}
\maketitle
\thispagestyle{empty}

%#59, 61-63, 67-70.
	\begin{menumerate}
		\setcounter{enumi}{58}
		\item %59

		\begin{theorem}
			If $\sum a_n$ converges and $a_n \geq 0,$ then show $\sum \sqrt{a_n}/n$ converges.
		\end{theorem}

		\begin{proof}
			Let $x = (\sqrt{a_n})_n$, $y = \left(\frac{1}{n}\right)_n$. Clearly $y \in \ell_1,$ and since $\sum a_n \to c,$ $a_n \to 0$ implies that $\sqrt{a_n} \to 0.$ Therefore, $x \in \ell_1.$ Since $\ell_1$ is an inner product space, the cauchy schwartz inequality gives, 
			$$0 \leq \sum_{n=1}^\infty \frac{\sqrt{a_n}}{n} = \langle x,y \rangle \leq |x||y| = \sqrt{\sum_{n=1}^\infty a_n} \sqrt{\sum_{n=1}^\infty \frac{1}{n^2}} = \sqrt{\frac{c}{6}}\pi$$
			and so the series is bounded and therefore converges.
		\end{proof}

		\setcounter{enumi}{60}
		\item Consider the following $\{a_n\} \in \ell_1.$ We say that $a_n = 1/4^n$ if $n$ odd and $a_n = 1/2^n$ otherwise. Clearly $$0 < \sum_{n\in\mathbb{N}}a_n = \sum_{n\ \text{odd}} \frac{1}{4^n} + \sum_{n\ \text{even}} \frac{1}{2^n} < \sum \frac{1}{2^n} < \sum \frac{1}{n^2} = \frac{\pi^2}{6}.$$ So the series converges. Let $\rho_N = \sup_{n>N} |a_{n+1}|/|a_n| = \sup_{n>N} 2^n = \infty$. So clearly $\rho = \lim \rho_N = \infty,$ and yet the series converges. If we were to suppose that $\lambda = \rho$ then the test would be wrong since $\lambda > 1$ implies divergence. So it must be the case that the test is inconclusive when $\rho \geq 1.$
		\item %62
		\item %63

		\setcounter{enumi}{66}
		\item %67
		\item Here is confirmation of the existence of an unrelation between the convergence of a series and the related infinite product,.
		\begin{menumerate}
			\item Suppose a series is defined as the infinite sum of a sequence $a_k = (-1)^k/\sqrt{k}$. We first show that such a series converges. Consider that $a_k \leq f(x) = (-1)^xx^{-0.5}$ for all $x = k.$ So we simply must show convergence of the improper integral of $f(x)$. Recall that $(-1)^x = e^{i\pi x},$ then
			$$\int_1^\infty(-1)^xx^{-0.5}\ dx \sim \int_{\mathbb{R}^+} e^{i\pi x} x^{-0.5}\ dx = \mathcal{L}\{t^{-0.5}\} = \frac{\Gamma(1/2)}{(\pi)^{\frac{1}{2}}}$$
			So at least the series converges. Consider the infinite product in terms of its partial products. Specifically we consider the partial products in pairs, $c_n*c_{n+1} = (1 +k^{-0.5} + (k+1)^{-0.5} + (k^2+k)^{-0.5})$ and so $$\prod_{k=1}^\infty (1 + a_k)=  \prod_{k=1}^\infty (1 +(2k)^{-0.5} + (2k+1)^{-0.5} + (2k(2k+1))^{-0.5})$$  
			which converges if and only if $\sum_{k=1}^\infty b_k + c_k + \frac{1}{\sqrt k}$ converges, which it does not. Therefore the infinite product can't converge.

			\item Let $b_k = e_k/k + (-1)^k/\sqrt{k}.$ Clearly $\sum b_k$ diverges since $\sum b_k = \sum e_k/k + \sum (-1)^k/\sqrt{k} \geq 0.5\sum 1/n + (1- \sqrt{2})\zeta(1/2) = \infty.$ However, by performing the same grouping of two test on the infinite product we get that the product converges if and only if $$\sum \frac{1}{n} + \frac{1}{\sqrt{n}} - \frac{1}{n\sqrt{n+1}} - \frac{1}{\sqrt{n}\sqrt{n+1}} - \frac{1}{\sqrt{n+1}}$$ converges. This is true if and only if $\sum n^{-3/2}$ converges (which it does by the integral test.) Thus, here is an example where the infinite product converges and the sum does not.
		\end{menumerate}

		\item %69
		\item %70
	\end{menumerate}

\end{document}