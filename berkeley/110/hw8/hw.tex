\documentclass[11pt]{amsart}

\usepackage{amsmath,amsthm}
\usepackage{amssymb}
\usepackage{graphicx}
\usepackage{enumerate}
\usepackage{fullpage}
\usepackage{tikz-cd}
% \usepackage{euscript}
% \makeatletter
% \nopagenumbers
\usepackage{verbatim}
\usepackage{color}
\usepackage{hyperref}

\usepackage{fullpage,tikz,float}
%\usepackage{times} %, mathtime}

\textheight=600pt %574pt
\textwidth=480pt %432pt
\oddsidemargin=15pt %18.88pt
\evensidemargin=18.88pt
\topmargin=10pt %14.21pt

\parskip=1pt %2pt

% define theorem environments
\newtheorem{theorem}{Theorem}    %[section]
%\def\thetheorem{\unskip}
\newtheorem{proposition}[theorem]{Proposition}
%\def\theproposition{\unskip}
\newtheorem{conjecture}[theorem]{Conjecture}
\def\theconjecture{\unskip}
\newtheorem{corollary}[theorem]{Corollary}
\newtheorem{lemma}[theorem]{Lemma}
\newtheorem{sublemma}[theorem]{Sublemma}
\newtheorem{fact}[theorem]{Fact}
\newtheorem{observation}[theorem]{Observation}
%\def\thelemma{\unskip}
\theoremstyle{definition}
\newtheorem{definition}{Definition}
%\def\thedefinition{\unskip}
\newtheorem{notation}[definition]{Notation}
\newtheorem{remark}[definition]{Remark}
% \def\theremark{\unskip}
\newtheorem{question}[definition]{Question}
\newtheorem{questions}[definition]{Questions}
%\def\thequestion{\unskip}
\newtheorem{example}[definition]{Example}
%\def\theexample{\unskip}
\newtheorem{problem}[definition]{Problem}
\newtheorem{exercise}[definition]{Exercise}

\numberwithin{theorem}{section}
\numberwithin{definition}{section}
\numberwithin{equation}{section}

\def\reals{{\mathbb R}}
\def\torus{{\mathbb T}}
\def\integers{{\mathbb Z}}
\def\rationals{{\mathbb Q}}
\def\naturals{{\mathbb N}}
\def\complex{{\mathbb C}\/}
\def\distance{\operatorname{distance}\,}
\def\support{\operatorname{support}\,}
\def\dist{\operatorname{dist}\,}
\def\Span{\operatorname{span}\,}
\def\degree{\operatorname{degree}\,}
\def\kernel{\operatorname{kernel}\,}
\def\dim{\operatorname{dim}\,}
\def\codim{\operatorname{codim}}
\def\trace{\operatorname{trace\,}}
\def\dimension{\operatorname{dimension}\,}
\def\codimension{\operatorname{codimension}\,}
\def\nullspace{\scriptk}
\def\kernel{\operatorname{Ker}}
\def\p{\partial}
\def\Re{\operatorname{Re\,} }
\def\Im{\operatorname{Im\,} }
\def\ov{\overline}
\def\eps{\varepsilon}
\def\lt{L^2}
\def\curl{\operatorname{curl}}
\def\divergence{\operatorname{div}}
\newcommand{\norm}[1]{ \|  #1 \|}
\def\expect{\mathbb E}
\def\bull{$\bullet$\ }
\def\det{\operatorname{det}}
\def\Det{\operatorname{Det}}
\def\rank{\mathbf r}
\def\diameter{\operatorname{diameter}}

\def\t2{\tfrac12}

\newcommand{\abr}[1]{ \langle  #1 \rangle}

\def\newbull{\medskip\noindent $\bullet$\ }
\def\field{{\mathbb F}}
\def\cc{C_c}



% \renewcommand\forall{\ \forall\,}

% \newcommand{\Norm}[1]{ \left\|  #1 \right\| }
\newcommand{\Norm}[1]{ \Big\|  #1 \Big\| }
\newcommand{\set}[1]{ \left\{ #1 \right\} }
%\newcommand{\ifof}{\Leftrightarrow}
\def\one{{\mathbf 1}}
\newcommand{\modulo}[2]{[#1]_{#2}}

\def\bd{\operatorname{bd}\,}
\def\cl{\text{cl}}
\def\nobull{\noindent$\bullet$\ }

\def\scriptf{{\mathcal F}}
\def\scriptq{{\mathcal Q}}
\def\scriptg{{\mathcal G}}
\def\scriptm{{\mathcal M}}
\def\scriptb{{\mathcal B}}
\def\scriptc{{\mathcal C}}
\def\scriptt{{\mathcal T}}
\def\scripti{{\mathcal I}}
\def\scripte{{\mathcal E}}
\def\scriptv{{\mathcal V}}
\def\scriptw{{\mathcal W}}
\def\scriptu{{\mathcal U}}
\def\scriptS{{\mathcal S}}
\def\scripta{{\mathcal A}}
\def\scriptr{{\mathcal R}}
\def\scripto{{\mathcal O}}
\def\scripth{{\mathcal H}}
\def\scriptd{{\mathcal D}}
\def\scriptl{{\mathcal L}}
\def\scriptn{{\mathcal N}}
\def\scriptp{{\mathcal P}}
\def\scriptk{{\mathcal K}}
\def\scriptP{{\mathcal P}}
\def\scriptj{{\mathcal J}}
\def\scriptz{{\mathcal Z}}
\def\scripts{{\mathcal S}}
\def\scriptx{{\mathcal X}}
\def\scripty{{\mathcal Y}}
\def\frakv{{\mathfrak V}}
\def\frakG{{\mathfrak G}}
\def\aff{\operatorname{Aff}}
\def\frakB{{\mathfrak B}}
\def\frakC{{\mathfrak C}}

\def\symdif{\,\Delta\,}
\def\mustar{\mu^*}
\def\muplus{\mu^+}

\def\soln{\noindent {\bf Solution.}\ }


%\pagestyle{empty}
%\setlength{\parindent}{0pt}

\begin{document}

\begin{center}{\bf Math 110 --- Homework 8 --- UCB, Summer 2017 --- William Guss}
\end{center}

\medskip \noindent {\textbf{(8.1) } Find an example of an inner-product space $V$ and a linear transformation $T: V \to V$ with eigen value $\lambda$ such that $\lambda^* \notin Eig(T^*).$
\begin{proof}
	Let $V = \ell_2(\mathbb{C})$ and let $T := L$ where $L$ is the left-shift operator. Then $\lambda$ with $|\lambda| < 1 $ is an eigenvalue of the $T$; that is, when $W_\lambda = span\left((\lambda^i)_{i=1}^\infty\right)$ we have that $L(w) = \lambda w$, $w_\lambda \in W.$ As in the notes $T^* = R$ where $R$ is the right shift operator, however, the right shift operator $R$ has no eigenvalues as proven on a previous homework.
\end{proof}

\medskip \noindent {\textbf{(8.2) }  Let $V$ be a finite-dimensional inner-product space, and let $T: V \to V$ be linear. Show that $\langle v | T(v) \rangle \geq 0$ for all $v \in V$ if and only if there is some $S: V\to V$ such that $T = S^*S.$

\begin{proof}
If $\langle v | T(v) \rangle \geq 0$ then $\langle v | T(v) \rangle = \langle T^*(v) | v \rangle = \overline{\langle v | T^*(v)\rangle} = \langle v | T^*(v)\rangle = \langle T(v) | v \rangle$, and therefore $T$ is self-adjoint. 

Then $T$ has a diagonalizable in some basis, that is there exist a $\scriptb$ so that $[T]_{\scriptb \to \scriptb}$ is diagonal. Since it is semi-positive definite for all $v$,
\begin{equation*}
	 0 \leq \langle v|T(v)\rangle = [v]^*_{\scriptb}[T]_{\scriptb \to \scriptb}[v]_{\scriptb} = \sum_{k=1}^d |v_\scriptb^k|^2 \lambda_k
\end{equation*}
which can only be the case if $\lambda_k \geq 0.$ Then define $S$ such that $[S]_{\scriptb \to \scriptb} = \sqrt{[T]_{\scriptb\to \scriptb}}$ where the square root is applied pointwise. Then $T = S^*S$ as the square roots are real and on the diagonal. This completes the implication.

In the other direciton if $T = S^*S$ then 
\begin{equation*}
	\langle v | T(v) \rangle = \langle v | S^*S v \rangle = \langle  S(v) | S(v) \rangle = \|S(v)\|^2 \geq 0.
\end{equation*}
This completes the proof.
\end{proof}

\medskip \noindent {\textbf{(8.3) } Define $T: C^\infty[-1,1] \to C^\infty[-1,1]$ by $T(f) = f''$. Show that $-T$ is nonnegative.

\begin{proof}
	Recall from the text that the derivative $D$ is anti-self adjoint. Furthermore, $T = D^2 = -D^*D$ by the anti-self-adjoint property. Thus $-T$ is nonnegative.
\end{proof}

\medskip \noindent {\textbf{(8.4) }  Let $R: \ell^2(\mathbb{N}) \to \ell^2(\mathbb{N})$ be the right-shift operator. Compute $|R|.$
\begin{proof}
	By definition $|R| = \sqrt{R^* R}.$ Recall that $R^* = L$ and therefore $R^*R = LR: (v_1, \dots) \mapsto L(0, v_1, \dots) \mapsto (v_1, \dots).$ Therefore $LR = Id_{\ell_2(\mathbb{N})}.$ Since the identity is idempotent, it is also its own square root. Thus $|R| =  Id_{\ell_2(\mathbb{N})}$.
\end{proof}

\medskip \noindent {\textbf{(8.5) } Let $V$ and $W$ be inner-product spaces and let $T: V \to W$ be linear. Show that $T$ is unitary iff $\|T(v)\| =\|v\|$ for all $v$.

\begin{proof}
	If $ \|T(v)\| = \|v\|$ for all $v$, then $ \langle T(v) | T(v) \rangle  = \|T(v)\|^2 = \|v\|^2 \langle v | v \rangle$ for all $v$. Therefore $T$ is unitary. The foregoing steps algebraically are logically bidirectional and thus the proof is complete.
\end{proof}

\medskip \noindent {\textbf{(8.6) } Find an exampole of a finite-dimensional inner-product space $V$, a linear-transformation $T: V \to V$, and a basis $\scriptb$ of $V$ such that $[T^*]_\scriptb \neq [T]_\scriptb^*.$
\begin{proof}
	Let $T: \mathbb{R}^2 \to \mathbb{R}^2$ have matrix representation in the standard basis $\scripte$ given by
	\begin{equation*}
		[T]_\scripte = \begin{bmatrix}
			1 & 1 \\
			0 & 0
		\end{bmatrix}.
	\end{equation*}
		Now, let $\scriptb = \left\{(1, 0), (1,1)\right\} \subset \mathbb{R}^2$ be the desired basis. We have that
		\begin{equation*}
		[T^*]_\scriptb	= \begin{bmatrix}
			\begin{bmatrix}
				1 \\
				1
			\end{bmatrix}_\scriptb 
			&
			\begin{bmatrix}
				1 \\ 1
			\end{bmatrix}_\scriptb
		\end{bmatrix} =
		\begin{bmatrix}
			0 & 0\\
			1 & 1
		\end{bmatrix}.
		\end{equation*}
		Applying computation to the basic operator we get
		\begin{equation*}
			[T]_\scriptb	= \begin{bmatrix}
			\begin{bmatrix}
				1 \\
				0
			\end{bmatrix}_\scriptb 
			&
			\begin{bmatrix}
				2 \\ 0
			\end{bmatrix}_\scriptb
		\end{bmatrix} =
		\begin{bmatrix}
			1 & 2\\
			0 & 0
		\end{bmatrix}.
		\end{equation*}
		By observation of the foregoing $[T^*]_\scriptb \neq [T]_\scriptb^*.$ This completes the proof.
\end{proof}

 \end{document}\end
