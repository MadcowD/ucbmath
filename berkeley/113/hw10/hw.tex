\documentclass[11pt]{amsart}

\usepackage{amsmath,amsthm}
\usepackage{amssymb}
%\usepackage{graphicx}
%\usepackage{enumerate}
\usepackage{fullpage}
% \usepackage{euscript}
% \makeatletter
% \nopagenumbers
\usepackage{verbatim}
\usepackage{color}
\usepackage{hyperref}
%\usepackage{times} %, mathtime}

\textheight=600pt %574pt
\textwidth=480pt %432pt
\oddsidemargin=15pt %18.88pt
\evensidemargin=18.88pt
\topmargin=10pt %14.21pt

\parskip=1pt %2pt

\def\reals{{\mathbb R}}
\def\torus{{\mathbb T}}
\def\integers{{\mathbb Z}}
\def\rationals{{\mathbb Q}}
\def\naturals{{\mathbb N}}
\def\complex{{\mathbb C}\/}
\def\distance{\operatorname{distance}\,}
\def\support{\operatorname{support}\,}
\def\dist{\operatorname{dist}\,}
\def\Span{\operatorname{span}\,}
\def\degree{\operatorname{degree}\,}
\def\kernel{\operatorname{kernel}\,}
\def\dim{\operatorname{dim}\,}
\def\codim{\operatorname{codim}}
\def\trace{\operatorname{trace\,}}
\def\dimension{\operatorname{dimension}\,}
\def\codimension{\operatorname{codimension}\,}
\def\kernel{\operatorname{Ker}}
\def\Re{\operatorname{Re\,} }
\def\Im{\operatorname{Im\,} }
\def\eps{\varepsilon}
\def\lt{L^2}
\def\bull{$\bullet$\ }
\def\det{\operatorname{det}}
\def\Det{\operatorname{Det}}
\def\diameter{\operatorname{diameter}}
\def\symdif{\,\Delta\,}
\newcommand{\norm}[1]{ \|  #1 \|}
\newcommand{\set}[1]{ \left\{ #1 \right\} }
\newcommand{\group}[2]{\left\langle #1, #2\right\rangle}
\def\one{{\mathbf 1}}
\def\cl{\text{cl}}

\def\newbull{\medskip\noindent $\bullet$\ }
\def\nobull{\noindent$\bullet$\ }



\def\scriptf{{\mathcal F}}
\def\scriptq{{\mathcal Q}}
\def\scriptg{{\mathcal G}}
\def\scriptm{{\mathcal M}}
\def\scriptb{{\mathcal B}}
\def\scriptc{{\mathcal C}}
\def\scriptt{{\mathcal T}}
\def\scripti{{\mathcal I}}
\def\scripte{{\mathcal E}}
\def\scriptv{{\mathcal V}}
\def\scriptw{{\mathcal W}}
\def\scriptu{{\mathcal U}}
\def\scriptS{{\mathcal S}}
\def\scripta{{\mathcal A}}
\def\scriptr{{\mathcal R}}
\def\scripto{{\mathcal O}}
\def\scripth{{\mathcal H}}
\def\scriptd{{\mathcal D}}
\def\scriptl{{\mathcal L}}
\def\scriptn{{\mathcal N}}
\def\scriptp{{\mathcal P}}
\def\scriptk{{\mathcal K}}
\def\scriptP{{\mathcal P}}
\def\scriptj{{\mathcal J}}
\def\scriptz{{\mathcal Z}}
\def\scripts{{\mathcal S}}
\def\scriptx{{\mathcal X}}
\def\scripty{{\mathcal Y}}
\def\frakv{{\mathfrak V}}
\def\frakG{{\mathfrak G}}
\def\frakB{{\mathfrak B}}
\def\frakC{{\mathfrak C}}

\newtheorem{corollary}{Corollary}
\newtheorem{theorem}{Theorem}
\newtheorem{lemma}{Lemma}
\newtheorem{definition}{Definition}

\def\soln{\noindent {\bf Solution.}\ }

\begin{document}

\begin{center}{\bf Math 113 --- Problem Set 10 --- William Guss} \end{center}


\bigskip


\medskip \noindent {\bf (P174. 6)}\ Compute $(-3,5)(2	,-4) \in \mathbb{Z}_4 \times \mathbb{Z}_{11}$
\begin{proof}
	The product is as follows. First $-3 \times 2 \mod 4 = -6 \mod 4 = 2.$ Then $5 \times -4 = -20 \mod 11 = -22 + 2 = 2 \mod 11.$ Thus the product is $(2,2).$
\end{proof}
\medskip \noindent {\bf (P175. 12)}\ Decide whether or not the indicated operations of addition and multiplicatioin are defined on the set and givea  ring structure. Then describe the ring, if $S = \{a +b\sqrt{2}| a, b \in \mathbb{Q}\}$ with usual addition and multiplication.
\begin{proof}
	First $(a + b \sqrt{2}) + (c+ d\sqrt{2}) = a+c + (b+d)\sqrt{2} \in S$. Then since $+$ is the usual addition on $\mathbb{R},$ it is Abelian. Furthermore $0 =0 + 0\sqrt{2} \in S$ and by the inheritied additio operation $a+b \sqrt{2} + 0 = a+ b \sqrt{2}$. Lastly there are addative inverses, let $a+b\sqrt{2} \in S$ then claim that $(-a) + (-b)\sqrt{2}$ is its inverse; namely, $a+b\sqrt{2} + (-a) + (-b)\sqrt{2} = (a- a) + (b -b)\sqrt{2} = 0 + 0\sqrt{2} = 0.$ Thus $\langle S, +\rangle$ is a commutative group.

	Next $(a + b \sqrt{2}) \times (c+ d\sqrt{2}) = ac + ad \sqrt{2} + bc \sqrt{2} + 2db =(ac + 2db) + (ac + ad)\sqrt{2} \in S$
	and thus $S$ is closed under multiplication. Note we used that $\times$ is distributed as inherited by the ring $\langle \mathbb{R}, +, \times \rangle$Furthermore $1 \in S$ and $1(a + b\sqrt{2}) = 1a + 1b\sqrt{2} = a + b \sqrt{2} \in S$ so there is a unital element. Again since $\langle \mathbb{R}, \times \rangle$ is a commutative group then $\times$ is commutative on $S$.

	Now $S$ is a commutative unital subring of $\mathbb{R}.$ Now to find multiplictive inverses, observe the following
	\begin{equation*}
		\frac{1}{a + b\sqrt{2}} = \frac{1}{a + b\sqrt{2}}\frac{\overline{a + b\sqrt{2}}}{\overline{a + b\sqrt{2}}} = \frac{\overline{a + b\sqrt{2}}}{a^2 - 2b^2} = \frac{a}{a^2 - 2b^2} - \frac{b}{a^2 - 2b^2}\sqrt{2} \in S
	\end{equation*}
	and this solution is algebraically unique since we inherit the operations of $\mathbb{R}.$ Thus $\langle S, +, \times \rangle$ is a field.

\end{proof}

\medskip \noindent {\bf (P182. 2)}\ Solve the equation $3x =2$ in the field\\
(a) $\mathbb{Z}_7$.
\begin{proof}	
	We must find $x$ so that $x \mod 7 = 2$ and $3 | x$.
	First $3\times 1 = 3 \mod 7 = 3$, then $3 \times 3 = 9 \mod 7 = 2,$ thus $x = 2$ in $\mathbb{Z}_7.$
\end{proof}
(b) $\mathbb{Z}_{23}$
\begin{proof}
	We msut find $x$ so that $x \mod 23 = 2$ and $3 | x$.
	Take $x = 16$, then $3x = 48.$ Finally $23\times 2 = 46$ so $48 \mod 46 = 2$ and $3x = 2.$ We could have found this by showing that $3y = 1 $ if $y = 3^{-1}$ and thus $3\times 8 = 24 \mod 23 = 1$ so $y = 8$.
	Then $3x \equiv 2$ is solved by $x \equiv y3x  \equiv y \times 2 = 16.$ 
\end{proof}
\medskip \noindent {\bf (P182. 3)}\ Find all solutions of the equation $x^2 + 2x + 2= 0$ in $\mathbb{Z}_6.$
\begin{proof}
	First $\mathbb{Z}_6$ is not a field since $2 \times 3 = 6 \equiv 0$ so $\mathbb{Z}_6$ is not an integral domain. We factor the polynomial however and get
	$(x+1)(x+1) = -1 \mod 6 = 5 \in \mathbb{Z}_6.$ Thus we find all $y$ so that $y^2 = 5,$ or
	equivalently we must find all $y$ so that $y^2 +1 = 0.$ We get initially that
	\begin{equation*}
		\begin{aligned}
			0 +1 = 1 \\
			1 + 1= 2\\
			4 + 1= 5  \\
			9 + 1 = 10 \equiv 4 \\
			16 + 1 = 17 \equiv 5 \\
			25 + 1 = 26 \equiv 2	
		\end{aligned}
	\end{equation*}
	Thus there are no such solutions. This problem illustrates that $\mathbb{Z}^6$ does not have a square root of $-1$. For the grader the following is an exact verification.
	\begin{equation*}
		\begin{aligned}
			0^2 + 0 + 2 = 2 \\
			1^2 + 2 + 2 = 5 \\
			2^2 + 4 + 2 = 8 + 2= 10 \equiv 4 \\
			3^2 + 6 + 2 = 9 + 8 = 17 \equiv 5 \\
			4^2 + 8 + 2 = 16 + 10 = 26 \equiv 2 \\
			5^2 + 10 + 2 = 25 + 12 = 37 \equiv 1 \\
		\end{aligned}
	\end{equation*}
\end{proof}
\medskip \noindent {\bf (P182. 14)}\ Show that the matrix $A = \begin{bmatrix}
	1 & 2 \\
	2 & 4
\end{bmatrix}$ is a divisior of zero in $M_2(\mathbb{Z})$.
\begin{proof}
	We need to show that $AB = 0$ in $M_2(\mathbb{Z}).$ Then we must solve specifically 
	\begin{equation*}
		AB = \begin{bmatrix}
	1 & 2 \\
	2 & 4
\end{bmatrix} \begin{bmatrix}
	a & b \\
	c& d 
\end{bmatrix} = 0
	\end{equation*}
	As a system of equations we get
	\begin{equation*}
		a + 2c = 0 \\
		b + 2d = 0 \\
		2a + 4c = 0 \\
		2b + 4d = 0
	\end{equation*}
	which is then just
	\begin{equation*}
		a + 2c = 0 \\
		b + 2d = 0 \\
	\end{equation*}
	So take $a = -2, c = 1$ and $b = -2, d = 1$.
	Thus
	\begin{equation*}
		\begin{bmatrix}
			1 & 2 \\
			2 & 4
		\end{bmatrix} \begin{bmatrix}
			-2 & -2 \\
			1 & 1
		\end{bmatrix} = 0
	\end{equation*}
	Next we must show that $CA = 0$. Since
	$A = (1,2) \otimes (1,2)$, $A^T = A$ and thus
	$(AB)^T = 0^T = 0 = B^TA^T=B^TA$; therefore $CA = 0$ with $C = B^T \neq 0$. Therefore $A$ is a divisor of $0$ since it is a left and right divisor of $0.$
\end{proof}
\end{document}\end
