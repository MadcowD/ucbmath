\documentclass[11pt]{amsart}

\usepackage{amsmath,amsthm}
\usepackage{amssymb}
%\usepackage{graphicx}
%\usepackage{enumerate}
\usepackage{fullpage}
% \usepackage{euscript}
% \makeatletter
% \nopagenumbers
\usepackage{verbatim}
\usepackage{color}
\usepackage{hyperref}
%\usepackage{times} %, mathtime}

\textheight=600pt %574pt
\textwidth=480pt %432pt
\oddsidemargin=15pt %18.88pt
\evensidemargin=18.88pt
\topmargin=10pt %14.21pt

\parskip=1pt %2pt

\def\reals{{\mathbb R}}
\def\torus{{\mathbb T}}
\def\integers{{\mathbb Z}}
\def\rationals{{\mathbb Q}}
\def\naturals{{\mathbb N}}
\def\complex{{\mathbb C}\/}
\def\distance{\operatorname{distance}\,}
\def\support{\operatorname{support}\,}
\def\dist{\operatorname{dist}\,}
\def\Span{\operatorname{span}\,}
\def\degree{\operatorname{degree}\,}
\def\kernel{\operatorname{kernel}\,}
\def\dim{\operatorname{dim}\,}
\def\codim{\operatorname{codim}}
\def\trace{\operatorname{trace\,}}
\def\dimension{\operatorname{dimension}\,}
\def\codimension{\operatorname{codimension}\,}
\def\kernel{\operatorname{Ker}}
\def\Re{\operatorname{Re\,} }
\def\Im{\operatorname{Im\,} }
\def\eps{\varepsilon}
\def\lt{L^2}
\def\bull{$\bullet$\ }
\def\det{\operatorname{det}}
\def\Det{\operatorname{Det}}
\def\diameter{\operatorname{diameter}}
\def\symdif{\,\Delta\,}
\newcommand{\norm}[1]{ \|  #1 \|}
\newcommand{\set}[1]{ \left\{ #1 \right\} }
\newcommand{\group}[2]{\left\langle #1, #2\right\rangle}
\def\one{{\mathbf 1}}
\def\cl{\text{cl}}

\def\newbull{\medskip\noindent $\bullet$\ }
\def\nobull{\noindent$\bullet$\ }



\def\scriptf{{\mathcal F}}
\def\scriptq{{\mathcal Q}}
\def\scriptg{{\mathcal G}}
\def\scriptm{{\mathcal M}}
\def\scriptb{{\mathcal B}}
\def\scriptc{{\mathcal C}}
\def\scriptt{{\mathcal T}}
\def\scripti{{\mathcal I}}
\def\scripte{{\mathcal E}}
\def\scriptv{{\mathcal V}}
\def\scriptw{{\mathcal W}}
\def\scriptu{{\mathcal U}}
\def\scriptS{{\mathcal S}}
\def\scripta{{\mathcal A}}
\def\scriptr{{\mathcal R}}
\def\scripto{{\mathcal O}}
\def\scripth{{\mathcal H}}
\def\scriptd{{\mathcal D}}
\def\scriptl{{\mathcal L}}
\def\scriptn{{\mathcal N}}
\def\scriptp{{\mathcal P}}
\def\scriptk{{\mathcal K}}
\def\scriptP{{\mathcal P}}
\def\scriptj{{\mathcal J}}
\def\scriptz{{\mathcal Z}}
\def\scripts{{\mathcal S}}
\def\scriptx{{\mathcal X}}
\def\scripty{{\mathcal Y}}
\def\frakv{{\mathfrak V}}
\def\frakG{{\mathfrak G}}
\def\frakB{{\mathfrak B}}
\def\frakC{{\mathfrak C}}

\newtheorem{corollary}{Corollary}
\newtheorem{theorem}{Theorem}
\newtheorem{lemma}{Lemma}
\newtheorem{definition}{Definition}

\def\soln{\noindent {\bf Solution.}\ }

\begin{document}

\begin{center}{\bf Math 113 --- Problem Set 11 --- William Guss} \end{center}


\bigskip


\medskip \noindent {\bf (P189. 1)}\ We will see later that the multiplicitive group of nonzero elements of a finite field is cyclic. Find a generator for this group for the finite field $\mathbb{Z}_7.$
\begin{proof}
	The multiplicitive group of nonzero elements of $\mathbb{Z}_7$ is $G = \langle \{1, \cdots, 6\}, \cdot_7\rangle$. If an element $a$ generates $G$ for every coprime of $6$, (there must be $6$ elements) Then $a^5$ must also generate the group. Thus the generators are $\{5\}$ and by $\{3\}$
	because
	\begin{equation*}
	\begin{aligned}
		\ [{5^n}]_{n=0}^{20} = [1, 5, 4, 6, 2, 3, 1, 5, 4, 6, 2, 3, 1, 5, 4, 6, 2, 3, 1, 5]\\
		[3^n]_{n=0}^{20} = [1, 3, 2, 6, 4, 5, 1, 3, 2, 6, 4, 5, 1, 3, 2, 6, 4, 5, 1, 3]
	\end{aligned}
	\end{equation*}	


\end{proof}
\medskip \noindent {\bf (P189. 4)}\ Using Fermat's theorem compute the remainder of $3^{47}$ when it is divided by 23.
\begin{proof}
	Although $a$ is not divisible by $23$, Fermat's theorem says that if $3$ is not divisible by $23$, then $3^{22}= 1 \mod 23$.
	Thus $3^{22\times2 +3} = 3^{22}\times 3^{22} \times 3^3 \mod 23 = 1 \times 1 \times 3 ^ 3 \mod 23.$ Computing $3^3 = 27 = 4 \mod 23$ we get $3^{47} = 4 \mod 23.$
\end{proof}
\end{document}\end
