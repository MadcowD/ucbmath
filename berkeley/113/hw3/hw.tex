\documentclass[11pt]{amsart}

\usepackage{amsmath,amsthm}
\usepackage{amssymb}
%\usepackage{graphicx}
%\usepackage{enumerate}
\usepackage{fullpage}
% \usepackage{euscript}
% \makeatletter
% \nopagenumbers
\usepackage{verbatim}
\usepackage{color}
\usepackage{hyperref}
%\usepackage{times} %, mathtime}

\textheight=600pt %574pt
\textwidth=480pt %432pt
\oddsidemargin=15pt %18.88pt
\evensidemargin=18.88pt
\topmargin=10pt %14.21pt

\parskip=1pt %2pt

\def\reals{{\mathbb R}}
\def\torus{{\mathbb T}}
\def\integers{{\mathbb Z}}
\def\rationals{{\mathbb Q}}
\def\naturals{{\mathbb N}}
\def\complex{{\mathbb C}\/}
\def\distance{\operatorname{distance}\,}
\def\support{\operatorname{support}\,}
\def\dist{\operatorname{dist}\,}
\def\Span{\operatorname{span}\,}
\def\degree{\operatorname{degree}\,}
\def\kernel{\operatorname{kernel}\,}
\def\dim{\operatorname{dim}\,}
\def\codim{\operatorname{codim}}
\def\trace{\operatorname{trace\,}}
\def\dimension{\operatorname{dimension}\,}
\def\codimension{\operatorname{codimension}\,}
\def\kernel{\operatorname{Ker}}
\def\Re{\operatorname{Re\,} }
\def\Im{\operatorname{Im\,} }
\def\eps{\varepsilon}
\def\lt{L^2}
\def\bull{$\bullet$\ }
\def\det{\operatorname{det}}
\def\Det{\operatorname{Det}}
\def\diameter{\operatorname{diameter}}
\def\symdif{\,\Delta\,}
\newcommand{\norm}[1]{ \|  #1 \|}
\newcommand{\set}[1]{ \left\{ #1 \right\} }
\def\one{{\mathbf 1}}
\def\cl{\text{cl}}

\def\newbull{\medskip\noindent $\bullet$\ }
\def\nobull{\noindent$\bullet$\ }



\def\scriptf{{\mathcal F}}
\def\scriptq{{\mathcal Q}}
\def\scriptg{{\mathcal G}}
\def\scriptm{{\mathcal M}}
\def\scriptb{{\mathcal B}}
\def\scriptc{{\mathcal C}}
\def\scriptt{{\mathcal T}}
\def\scripti{{\mathcal I}}
\def\scripte{{\mathcal E}}
\def\scriptv{{\mathcal V}}
\def\scriptw{{\mathcal W}}
\def\scriptu{{\mathcal U}}
\def\scriptS{{\mathcal S}}
\def\scripta{{\mathcal A}}
\def\scriptr{{\mathcal R}}
\def\scripto{{\mathcal O}}
\def\scripth{{\mathcal H}}
\def\scriptd{{\mathcal D}}
\def\scriptl{{\mathcal L}}
\def\scriptn{{\mathcal N}}
\def\scriptp{{\mathcal P}}
\def\scriptk{{\mathcal K}}
\def\scriptP{{\mathcal P}}
\def\scriptj{{\mathcal J}}
\def\scriptz{{\mathcal Z}}
\def\scripts{{\mathcal S}}
\def\scriptx{{\mathcal X}}
\def\scripty{{\mathcal Y}}
\def\frakv{{\mathfrak V}}
\def\frakG{{\mathfrak G}}
\def\frakB{{\mathfrak B}}
\def\frakC{{\mathfrak C}}

\newtheorem{corollary}{Corollary}
\newtheorem{theorem}{Theorem}
\newtheorem{lemma}{Lemma}
\newtheorem{definition}{Definition}

\def\soln{\noindent {\bf Solution.}\ }

\begin{document}

\begin{center}{\bf Math 113 --- Problem Set 3 --- William Guss} \end{center}


\bigskip


\medskip \noindent {\bf (P19.24)}\ \emph{Compute $20.5 +_{25} 19.3$}. 

Clearly $20.5+19.3 = 39.8$ which is equivalent to $39.8 -25  = 14.8 \mod 25$.


\medskip \noindent {\bf (P19.25)}\ \emph{Compute $\frac{1}{2} +_1 \frac{7}{8}.$}

Clearly $1/2 + 7/8 = 4/8 + 7/8 = 11/8$ which is equivalent to $11/8 -1 = 3/8 \mod 1.$




\medskip \noindent {\bf (P27.23)}\ \emph{Let $H \subset M_2 (\reals)$ consisting all matrices of the form \(\begin{bmatrix}
	a & -b \\
	b & a
\end{bmatrix}\) for $a,b \in \reals$. Is H closed under}


a) \emph{matrix addition?} Yes.


b)\emph{matrix multiplication?} Yes.

\begin{lemma}
	If $H$ is defined as above, then $H \cong \complex$.
\end{lemma}
\begin{proof}
	Define the mapping $\phi: H \to \mathbb{C}$ such that $\phi(M) = a +ib$. It is obvious that this mapping is a bijection since $a,b \in\mathbb{R}$ implies that for any $x + iy \in \mathbb{C}$ there exist $a,b \in \mathbb{R}$ so that $a + ib = x + iy$.

	Next we show that the mapping is a homomorphism under addition and multiplication. Take $Z,W \in H$. Then
	\begin{equation*}
		\begin{aligned}
			\phi(ZW) &= \phi\left(\begin{bmatrix}
				z_1w_1 -z_2w_2 & -(z_2w_1 +z_1w_2) \\
				z_2w_1 +z_1w_2 & z_1w_1 -z_2w_2
			\end{bmatrix}\right) \\ &= z_1w_1 - z_2w_2 +i(z_2w_1 + z_1w_2) \\
			&= (z_1+iz_2)(w_1 + iw_2) \\
			&= \phi(Z)\phi(W) \in \complex
		\end{aligned}
	\end{equation*}
	Furthermore we consider the addition operation
	\begin{equation*}
		\begin{aligned}
			\phi(Z+W) &= \phi\left(\begin{bmatrix}
				z_1 + w_1 & -(z_2 + w_2) \\
				z_2 + w_2 & z_1 + w_1
			\end{bmatrix}\right) \\
			 &= z_1 + w_1 + i(z_2 + w_2) \\
			&= (z_1+iz_2) + (w_1 + iw_2) \\
			&= \phi(Z) + \phi(W) \in \complex
		\end{aligned}
	\end{equation*}

	Therefore $H$ is isomorphic to $\complex$ under addition and multiplication.

\end{proof}

\noindent	\emph{Note I implicitly showed that $H$ was closed in the proof by explicitly calculating $Z +W, ZW$ and grouping the terms in the upper right hand corner of the the result. I also take for granted that $\mathbb{C}$ is closed under multiplication and addition.}
\begin{corollary}
	The set $H$ is closed under multiplication and addition.
\end{corollary}
\begin{proof}
	If $H$ were not closed under these operations then $H$ would not be isomorphic to $\mathbb{C}$.
\end{proof}



\medskip \noindent {\bf (P27.24)}\ 

a) False

b) True

c) False

d) What? lol. \textbf{False}

e) True

f) True.
\begin{proof}
	If $S = {a}, f: S \times S \to S$ then $f$ must be the mapping $(a,a) \mapsto a$. This is obvious. Then $f(a,a) = f(a,a)$ gives commutativity, and $f(a,f(a,a)) = f(a,a) = f(f(a,a),a)$ gives associativity.
\end{proof}

g) True
 
h) True

j) False

\end{document}\end
