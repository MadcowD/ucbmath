\documentclass[11pt]{amsart}

\usepackage{amsmath,amsthm}
\usepackage{amssymb}
%\usepackage{graphicx}
%\usepackage{enumerate}
\usepackage{fullpage}
% \usepackage{euscript}
% \makeatletter
% \nopagenumbers
\usepackage{verbatim}
\usepackage{color}
\usepackage{hyperref}
%\usepackage{times} %, mathtime}

\textheight=600pt %574pt
\textwidth=480pt %432pt
\oddsidemargin=15pt %18.88pt
\evensidemargin=18.88pt
\topmargin=10pt %14.21pt

\parskip=1pt %2pt

\def\reals{{\mathbb R}}
\def\torus{{\mathbb T}}
\def\integers{{\mathbb Z}}
\def\rationals{{\mathbb Q}}
\def\naturals{{\mathbb N}}
\def\complex{{\mathbb C}\/}
\def\distance{\operatorname{distance}\,}
\def\support{\operatorname{support}\,}
\def\dist{\operatorname{dist}\,}
\def\Span{\operatorname{span}\,}
\def\degree{\operatorname{degree}\,}
\def\kernel{\operatorname{kernel}\,}
\def\dim{\operatorname{dim}\,}
\def\codim{\operatorname{codim}}
\def\trace{\operatorname{trace\,}}
\def\dimension{\operatorname{dimension}\,}
\def\codimension{\operatorname{codimension}\,}
\def\kernel{\operatorname{Ker}}
\def\Re{\operatorname{Re\,} }
\def\Im{\operatorname{Im\,} }
\def\eps{\varepsilon}
\def\lt{L^2}
\def\bull{$\bullet$\ }
\def\det{\operatorname{det}}
\def\Det{\operatorname{Det}}
\def\diameter{\operatorname{diameter}}
\def\symdif{\,\Delta\,}
\newcommand{\norm}[1]{ \|  #1 \|}
\newcommand{\set}[1]{ \left\{ #1 \right\} }
\newcommand{\group}[2]{\left\langle #1, #2\right\rangle}
\def\one{{\mathbf 1}}
\def\cl{\text{cl}}

\def\newbull{\medskip\noindent $\bullet$\ }
\def\nobull{\noindent$\bullet$\ }



\def\scriptf{{\mathcal F}}
\def\scriptq{{\mathcal Q}}
\def\scriptg{{\mathcal G}}
\def\scriptm{{\mathcal M}}
\def\scriptb{{\mathcal B}}
\def\scriptc{{\mathcal C}}
\def\scriptt{{\mathcal T}}
\def\scripti{{\mathcal I}}
\def\scripte{{\mathcal E}}
\def\scriptv{{\mathcal V}}
\def\scriptw{{\mathcal W}}
\def\scriptu{{\mathcal U}}
\def\scriptS{{\mathcal S}}
\def\scripta{{\mathcal A}}
\def\scriptr{{\mathcal R}}
\def\scripto{{\mathcal O}}
\def\scripth{{\mathcal H}}
\def\scriptd{{\mathcal D}}
\def\scriptl{{\mathcal L}}
\def\scriptn{{\mathcal N}}
\def\scriptp{{\mathcal P}}
\def\scriptk{{\mathcal K}}
\def\scriptP{{\mathcal P}}
\def\scriptj{{\mathcal J}}
\def\scriptz{{\mathcal Z}}
\def\scripts{{\mathcal S}}
\def\scriptx{{\mathcal X}}
\def\scripty{{\mathcal Y}}
\def\frakv{{\mathfrak V}}
\def\frakG{{\mathfrak G}}
\def\frakB{{\mathfrak B}}
\def\frakC{{\mathfrak C}}

\newtheorem{corollary}{Corollary}
\newtheorem{theorem}{Theorem}
\newtheorem{lemma}{Lemma}
\newtheorem{definition}{Definition}

\def\soln{\noindent {\bf Solution.}\ }

\begin{document}

\begin{center}{\bf Math 113 --- Problem Set 4 --- William Guss} \end{center}


\bigskip


\medskip \noindent {\bf (P28. 37)}\ \emph{Suppose that * is an associative and commutative binary operation on a set $S.$ 
Show that $H = \set{a \in S\mathrel{}|\mathrel{} a*a = a}$ is closed under *.}
\begin{proof}
	We need show for every $(a,b) \in H \times H,$ $a*b \in H$. Clearly $a*b \in S,$ we claim that $(a*b)*(a*b) \in H$. 
	Essentially $(a*b)*(a*b) = (a*b)*(b*a)$ by commutativity. Then $(a*b)*(b*a) = a*(b*b)*a = a*b*a$ by associativity and idempotents of $b$. Finally $a*b*a = b*a*a = b*(a*a) = b*a = a*b$ by commutativity, associativity, and commutativity again. Therefore $a*b$ is idemopotent and in $H$, so $H$ is closed under $*$.
\end{proof}

\medskip \noindent {\bf (P34. 4)}\ \emph{Determine whether or not $\phi$ is an isomorphism between $\group{\integers}{+}$ and $\group{\integers}{+}$ when $\phi(n) = n+1, n \in \integers$.}\\
\textbf{Claim.} The mapping is not an isomorphism.
\begin{proof}
	We will show that despite the bijection of $\phi$, it is not a homomorphism.
	We first show that $\phi:\integers \to \integers$ is bijective. Clearly for ever $n \in \integers$, $\phi^{pre}(n) = n+ (-1) \in \integers$ so the map is surjective. Next, the map is injective since every successor of an integer is unique by Piano's axioms. Hence $\phi$ is a bijection. But consider $\phi(n +m) = (n+m) + 1 = (n+ 1) + m \neq (n+1) + (m+1) =  \phi(n) + \phi(m)$, therefore the mapping is not a homomorphism. This completes the proof of the claim.
\end{proof}

\medskip \noindent {\bf (P34. 6)}\ \emph{Determine whether or not $\phi$ is an isomorphism between $\group{\rationals}{\cdot}$ and $\group{\rationals}{\cdot}$ when $\phi(x) = x^2$ for $x \in \rationals$.}\\
\textbf{Claim}. The mapping is not an isomorphism.
\begin{proof}
	We need only show that $\phi$ is not a bijection. If $\phi$ is an isomorphism then it is in an invertible mapping. Therefore take $\phi^{-1}(2) = \sqrt{2} \notin \rationals$. This is a contradiction to the surjection of the inverse, therefore $\phi$ is not an isomorphism.
\end{proof}

\medskip \noindent {\bf (P28. 7)}\ \emph{Determine whether or not $\phi$ is an isomorphism between  $\group{\reals}{\cdot}$ and $\group{\reals}{\cdot}$ where $\phi(x) = x^3.$}\\
\textbf{Claim.} The mapping \emph{is} an isomorphism.
\begin{proof}
	We first show the bijection. Clearly if $y \neq x$ then without loss of generality $y > x$ and $\phi(y) > \phi(x)$ by the monotonicity of $x^3$, (to see this, observe that the mapping preserves sign and if $(y-x) >0$ then $(y-x)^3 > 0$), so the mapping is injective. For surjection, take $a \in \mathbb{R}$ and observe that $a^{1/3} \in \mathbb{R}$, by the completeness of $\mathbb{R}$ (take the sequence $x_1 = a$, $x_n = \frac{1}{n}\left[(n-1)x_{n-1} + \frac{a}{x_{n-1}^3}\right]$, and see its cauchyness, then reverse Newton's method to see that the $3$rd power exponentiation of the limit tends to $a$.) So the mapping is bijective.

	Now $\phi(ab) = (ab)^3 = a^3b^3 = \phi(a)\phi(b)$ and the mapping is a homormorphism by the distributive power law.

	This completes the proof. 
\end{proof}

\medskip \noindent {\bf (P45. 9)}\ \emph{Show that the group $\group{U}{\cdot}$ is not isomorphic to either $\group{\reals}{+}$ and $\group{\reals}{\cdot}.$}
\begin{proof}
	Since all sets have the same cardinality, it must only be that $\group{U}{\cdot}$ does not share structural properties with either of the real groups. It suffices to show that $\group{U}{\cdot} \not\simeq \group{\reals}{\cdot}$, since $\group{\reals}{\cdot} \simeq \group{\reals}{+}$ and isomorphisms form an equivalence relation on families of groups.

	Structurally, take any $z \in U$ so that $\theta = Arg(z)$, then $\theta/2\pi \in \mathbb{R}$ so $Arg(z^{n2\pi/\theta}) = n2\pi$ and so $z^{n2\pi/\theta} =1$  for all $n \in \mathbb{N}$. It is not the case that for every $x \in \mathbb{R}$ there is are $a, r \in \mathbb{R}$ so that $x^{\phi(r)} = a$ where $\phi(r)$ is a set so that $p,q \in \phi(r) \implies p -q = nr \wedge p,q > 0$ for some $n \in \integers$. Take $x = 2, x^{r > 0}$ increases monotonically :(.  Since this cyclicity property is not common, there could not be a homormorphism, and so the groups are not isomorphic.
\end{proof}

\medskip \noindent {\bf (P46. 13)}\ \emph{Determine if the set $S$ of $n \times n$ matrices with no zero diagonal enteries is  a group under matrix multiplication.}\\
\textbf{Claim.} The set $S$ is not a group.
\begin{proof}
	Consider the multiplication following two elements in $S$, 
	\begin{equation*}
		\begin{bmatrix}
			1 & 1\\
			1& 1
		\end{bmatrix} \begin{bmatrix}
			1 & 1 \\
			-1 & -1
		\end{bmatrix} = \begin{bmatrix}
			0 & 0 \\
			0 & 0
		\end{bmatrix}
	\end{equation*}
	So the set is not even closed under multiplication.
\end{proof}

\medskip \noindent {\bf (P46. 17)}\ \emph{Determine if the set $S$ of $n \times n$ upper triangular matrices with determinant 1 under matrix multiplication is a group.}\\
\textbf{Claim}. The set $S$ is a group.
\begin{proof}
	From linear algebra we know that a matrix is invertible if and only if it has determinant $D \neq 0$. Therefore every
	member of $S$ is a full rank matrix. We now show that $S$ is closed under matrix multiplication by the determinant laws from linear algebra. If $A,B \in S$ then $det(AB) = det(A)det(B) = 1$ so $AB$ is invertible and thereby upper triangular. It follows that $AB$ is closed under multiplication. From the chapter we know the set of invertible $n \times n$ matrices is the general linear group. Since invertibility if and only if determinant non-zero, it follows that $S$ is a sub-group, inheriting associativity of multiplication from $GL(\mathbb{R}^n)$.
\end{proof}

\medskip \noindent {\bf (P49. 37)}\ \emph{Let $G$ be a group and let $a,b c\in G$. Show that if $a*b*c = e$ then $b*c*a = e$.}
\begin{proof}
	If $a*b*c = e$ then $a = c^{-1}b^{-1}$,  $c =b^{-1}a^{-1}$ and $b = a^{-1}*c^{-1}$. Then 
	$a*b*c * {(c*b*a)}^{-1} = a*b*c*a^{-1}*b^{-1}*c^{-1}.$ Then by $a*b*c = e$ we have $ a*b*c*a^{-1}*b^{-1}*c^{-1} = c^{-1}*b^{-1}*a^{-1} *a^{-1}*b^{-1}*c^{-1} = a*b*b^{-1}*a^{-1}*b^{-1}*c^{-1} = a*a^{-1}*b^{-1}*c^{-1} = a*b*c*b^{-1}*c^{-1} = b^{-1}*c^{-1}$ so ${c*b*a}^{-1} = b^{-1}*c^{-1} = a^{-1}*b^{-1}*c^{-1}$ so $a^{-1} = a$ without loss of generality, and $b*c*a = b*c*b*c = a^{-1}*b*c = a*b*c = e.$ This completes the proof.
\end{proof} 
\end{document}\end
