\documentclass[11pt]{amsart}

\usepackage{amsmath,amsthm}
\usepackage{amssymb}
\usepackage{graphicx}
\usepackage{enumerate}
\usepackage{fullpage}
% \usepackage{euscript}
% \makeatletter
% \nopagenumbers
\usepackage{verbatim}
\usepackage{color}
\usepackage{hyperref}
\usepackage{tikz-cd}
%\usepackage{times} %, mathtime}

\textheight=600pt %574pt
\textwidth=480pt %432pt
\oddsidemargin=15pt %18.88pt
\evensidemargin=18.88pt
\topmargin=10pt %14.21pt

\parskip=1pt %2pt

% define theorem environments
\newtheorem{theorem}{Theorem}    %[section]
%\def\thetheorem{\unskip}
\newtheorem{proposition}[theorem]{Proposition}
%\def\theproposition{\unskip}
\newtheorem{conjecture}[theorem]{Conjecture}
\def\theconjecture{\unskip}
\newtheorem{corollary}[theorem]{Corollary}
\newtheorem{lemma}[theorem]{Lemma}
\newtheorem{sublemma}[theorem]{Sublemma}
\newtheorem{fact}[theorem]{Fact}
\newtheorem{observation}[theorem]{Observation}
%\def\thelemma{\unskip}
\theoremstyle{definition}
\newtheorem{definition}{Definition}
%\def\thedefinition{\unskip}
\newtheorem{notation}[definition]{Notation}
\newtheorem{remark}[definition]{Remark}
% \def\theremark{\unskip}
\newtheorem{question}[definition]{Question}
\newtheorem{questions}[definition]{Questions}
%\def\thequestion{\unskip}
\newtheorem{example}[definition]{Example}
%\def\theexample{\unskip}
\newtheorem{problem}[definition]{Problem}
\newtheorem{exercise}[definition]{Exercise}

\numberwithin{theorem}{section}
\numberwithin{definition}{section}
\numberwithin{equation}{section}

\def\reals{{\mathbb R}}
\def\torus{{\mathbb T}}
\def\integers{{\mathbb Z}}
\def\rationals{{\mathbb Q}}
\def\naturals{{\mathbb N}}
\def\complex{{\mathbb C}\/}
\def\distance{\operatorname{distance}\,}
\def\support{\operatorname{support}\,}
\def\dist{\operatorname{dist}\,}
\def\Span{\operatorname{span}\,}
\def\degree{\operatorname{degree}\,}
\def\kernel{\operatorname{kernel}\,}
\def\dim{\operatorname{dim}\,}
\def\codim{\operatorname{codim}}
\def\trace{\operatorname{trace\,}}
\def\dimension{\operatorname{dimension}\,}
\def\codimension{\operatorname{codimension}\,}
\def\nullspace{\scriptk}
\def\kernel{\operatorname{Ker}}
\def\p{\partial}
\def\Re{\operatorname{Re\,} }
\def\Im{\operatorname{Im\,} }
\def\ov{\overline}
\def\eps{\varepsilon}
\def\lt{L^2}
\def\curl{\operatorname{curl}}
\def\divergence{\operatorname{div}}
\newcommand{\norm}[1]{ \|  #1 \|}
\def\expect{\mathbb E}
\def\bull{$\bullet$\ }
\def\det{\operatorname{det}}
\def\Det{\operatorname{Det}}
\def\rank{\mathbf r}
\def\diameter{\operatorname{diameter}}

\def\t2{\tfrac12}

\newcommand{\abr}[1]{ \langle  #1 \rangle}

\def\newbull{\medskip\noindent $\bullet$\ }
\def\field{{\mathbb F}}
\def\cc{C_c}



% \renewcommand\forall{\ \forall\,}

% \newcommand{\Norm}[1]{ \left\|  #1 \right\| }
\newcommand{\Norm}[1]{ \Big\|  #1 \Big\| }
\newcommand{\set}[1]{ \left\{ #1 \right\} }
%\newcommand{\ifof}{\Leftrightarrow}
\def\one{{\mathbf 1}}
\newcommand{\modulo}[2]{[#1]_{#2}}

\def\bd{\operatorname{bd}\,}
\def\cl{\text{cl}}
\def\nobull{\noindent$\bullet$\ }

\def\scriptf{{\mathcal F}}
\def\scriptq{{\mathcal Q}}
\def\scriptg{{\mathcal G}}
\def\scriptm{{\mathcal M}}
\def\scriptb{{\mathcal B}}
\def\scriptc{{\mathcal C}}
\def\scriptt{{\mathcal T}}
\def\scripti{{\mathcal I}}
\def\scripte{{\mathcal E}}
\def\scriptv{{\mathcal V}}
\def\scriptw{{\mathcal W}}
\def\scriptu{{\mathcal U}}
\def\scriptS{{\mathcal S}}
\def\scripta{{\mathcal A}}
\def\scriptr{{\mathcal R}}
\def\scripto{{\mathcal O}}
\def\scripth{{\mathcal H}}
\def\scriptd{{\mathcal D}}
\def\scriptl{{\mathcal L}}
\def\scriptn{{\mathcal N}}
\def\scriptp{{\mathcal P}}
\def\scriptk{{\mathcal K}}
\def\scriptP{{\mathcal P}}
\def\scriptj{{\mathcal J}}
\def\scriptz{{\mathcal Z}}
\def\scripts{{\mathcal S}}
\def\scriptx{{\mathcal X}}
\def\scripty{{\mathcal Y}}
\def\frakv{{\mathfrak V}}
\def\frakG{{\mathfrak G}}
\def\aff{\operatorname{Aff}}
\def\frakB{{\mathfrak B}}
\def\frakC{{\mathfrak C}}

\def\symdif{\,\Delta\,}
\def\mustar{\mu^*}
\def\muplus{\mu^+}

\def\soln{\noindent {\bf Solution.}\ }


%\pagestyle{empty}
%\setlength{\parindent}{0pt}

\begin{document}

\begin{center}{\bf Math 185 --- UCB, Fall 2016 -- Smirnov}
\\
{\bf Problem Set 2 due Thursday September 22 - William Guss}
\end{center}
\medskip \noindent {\bf (18.5)}\ Show that the function $f(z) = \left(\frac{z}{\overline{z}}\right)^2$
has the value $1$ at all nonzero points on the real and imaginary axes, where $z = (x,0)$ and $z = (0,y)$ but that it has the value $-1$ at all non zero points on the line $y=z, z= (x,x)$. Thus show that the limit of $f(z)$ as $z \to 0$ does not exist.
\begin{proof}
  Consider that for all $z \neq 0$ we have that $z/\overline{z} = z^2/|z|^2.$ Therefore $f(z) = z^4/|z|^4$. Now take
  the limit along the imaginary axes and get $f(x^1_n) = (x^1_ni)^4/(x^1_n)^4.$ Using $i^2 = -1$ we get $f(x^1_n) = (x^1_n)^4/(x^1_n)^4 = 1 \to 1$ as $x^1_n \to 0$. Additionally take the real sequence $f(x) = (x + 0i)^4/x^4 = 1$ as $x\to 0$. Finally take $f(x + xi) = (x +xi)^4/(2x^2)^2 = (x^4 + 4x^4i -6x^4 -4x^4i +x^4)/4x^4 = -4x^4/4x^4 = -1 \to -1$ as $x \to 0$. So since all sequences of $z \to 0$ do not converge to the same limit, the function is not ocntinuous at $z = 0.$  
\end{proof}

\medskip \noindent {\bf (18.9)}\ Show that 
\begin{equation*}
  \lim_{z \to z_0} f(z) g(z) = 0
\end{equation*}
if $\lim_{z \to z_0} f(z) = 0$.
\begin{proof}
  If $\lim_{z \to z_0} f(z) = 0$ then for every $\epsilon > 0$ $|f(z)| < \epsilon$ as $d(z, z_0) < \delta$ for some $\delta$ and the distance function $d(z, z_0).$ By convention we define $0 \cdot \infty = 0$. Now within a compact $\delta$-ball, $B_\delta(z_0)$ open around $z_0$ and any $z$ in that ball, $|f(z)g(z)| \leq |f(z)|\sup_{y \in B_{\delta}(z_0)} |g(z)| \leq |f(z)|\times\infty$ However as $\delta \to 0$, $|f(z)| \to 0$ and $|f(z)|\times \infty \to 0$ by our convention, so $|f(z)g(z)| \to 0$ and so $\lim_{z \to z_0} f(z)g(z) = 0$.
\end{proof}

\medskip \noindent {\bf (18.9)}\ Use theorem in Sec. 17 to show the convergence of the following limits.\\

\noindent (a) $\lim_{z \to \infty} \frac{4z^2}{(z-1)^2}.$
\begin{proof}
  We let $f(z)$ be the function of the limit, and then show that $f(1/z) \to w_0$ as $z \to 0$ implies $f(z) \to w_0$ as $z \to \infty.$ Clearly $f(1/z)$ is given by
  \begin{equation*}
    \begin{aligned}
      \frac{4(1/z)^2}{((1/z)-1)^2} = \frac{4}{z^2((1/z)^2 -2/z +1)} = \frac{4}{1 -2z +z^2} = \frac{4}{1+z(z-2)}
    \end{aligned}
  \end{equation*}
  And as $z \to 0$ we have $f(1/z) \to 4/1 = 4$ using that $z^2$ and $z$ are continuous functions and complex multiplication is continuous. 
\end{proof}

\noindent (b) $\lim_{z \to 1} \frac{1}{(z-1)^3} = \infty$.
\begin{proof}
  We let $f(z)$ be the function of the limit, and then show that $\lim_{z \to 1} \frac{1}{f(z)} = 0.$ Clearly $\frac{1}{f(z)} = (z-1)^3$. From the book, all polynomial functions are continuous for all $\mathbb{R}$ so $\lim_{z \to 1}(z-1)^3 = ((1)-1)^3 = 0^3 = 0$ and by the Sec 17 theorem the limit in (b) converges to $\infty.$ \emph{Yay single point compactifications!}
\end{proof}


\noindent (c) $\lim_{z \to \infty} \frac{z^2 + 1}{z-1} = \infty$.
\begin{proof}
  We let $f(z)$ be the function of the limit. First observe that $(z-i)(z+i) = z^2 - z + -iz +iz -i^2 = z^2 + 1$. We must now show that $\lim_{z \to \infty} \frac{z-1}{(z-i)(z+i)} = 0$ which requires that $\lim_{z \to 0}\frac{1/z -1}{(1/z-i)(1/z + i)} =0$. Clearly $1/(f(1/z)) = \frac{\overline{z}/|z|^2 - 1}{\overline{z}^2/|z|^4 + 1}.$ Applying the complex conjugate method again we get
  \begin{equation*}
  \begin{aligned}
    \frac{1}{f(1/z)} = \frac{(\overline{z}^2/|z|^4 + 1)(\overline{z}/|z|^2 - 1)}{(\overline{z}^2/|z|^4 + 1)\overline{(\overline{z}^2/|z|^4 + 1)}}   &= \frac{(z^2/|z|^4 + 1)(\overline{z}/|z|^2 - 1)}{(\overline{z}^2/|z|^4 + 1)(z^2/|z|^4 + 1)} \\
    &= \frac{(z^2/|z|^4 + 1)(\overline{z}/|z|^2 - 1)}{z^2\overline{z}^2/|z|^8 + \overline{z}^2/|z|^4 + z^2/|z|^4 + 1} \\
     &= \frac{z(1/|z|^4 - z/|z|^4) + \overline{z}/|z|^2 -1}{\overline{z}^2/|z|^4 + z^2/|z|^4 + 2} \\
  \end{aligned}
  \end{equation*}
  Taking the absolute value of the expression it is immediate that $|1/f(1/z)| \leq \frac{|z - 1 + 1 -1|}{|1 + 1 + 3|} \to 0$ as $|z| \to 0$ so the infinite limit holds. Another way to see this is that $|(1/z-1)|/|1/z^2 + 1| \leq C|1/z|/|1/z^2| \leq |z| \to 0$. Then follow application of Sec 17 Theorem twice and get the limit in (c)
\end{proof}

\medskip \noindent {\bf (18.11)}\ With the aid of the theorem in Sec 17. show that when \begin{equation*}
  T(z) = \frac{az + b}{cz + d},
\end{equation*}\\
\noindent (a) $\lim_{z\to \infty} T(z) = \infty$ if $c = 0$
\begin{proof}
  First we show that $\lim_{z \to \infty} 1/T(z) = 0$ iff $\lim_{z \to 0} 1/T(1/z)) = 0$ iff
  \begin{equation*}
    \frac{d}{az + b} \to 0,\ z \to\infty \iff \frac{d}{a/z + b} \to 0,\  z \to 0
  \end{equation*}
  Consider the magnitude $|1/(a/z +b)| \leq |d|/|a/z + b|$. Clearly $ab \neq 0$ so $ |d|/|a/z + b| \leq d(1/b)/|a/zb + 1| \leq |d(1/b)|/|a/zb| \leq |dz/b|/|a| \to 0$ as $z \to 0$, so the first assertion is proved by folling the if ($\Leftarrow$) logic.
\end{proof}
\noindent (b) $\lim_{z\to \infty} T(z) = a/c$ and $\lim_{z\to d/c} T(z) = \infty$ if $c \neq 0$
\begin{proof}
  If $c \neq 0$ we first show that $\lim_{z \to \infty} T(z) = a/c $ iff $\lim_{z \to 0} T(1/z) = a/c.$ It follows 
  \begin{equation*}
    \frac{(a\overline{z}/|z|^2 +B)cz/|z|^2 +d}{|c \overline{z}/|z|^2 + d|^2} \sim \frac{ac \overline{z}/|z|^4}{c^2| \overline{z}/|z|^2|^2} \sim \frac{a}{c} \to \frac{a}{c}.
  \end{equation*}
  Now for the second assertion, we will show that $\lim_{z \to d/c} 1/T(z) = 0$  which holds if and only if the second assertion does. Using 
  \begin{equation*}
    \lim_{z \to d/c} 1/T(z) = \lim_{z \to d/c} \frac{cz + d}{az + b} = \lim_{z \to d/c} f(z)g(z)
  \end{equation*}
  where $f(z) = cz +d$ and $1/g(z) = az+b$ and a previous proven theorem in the homework, we need show that $f(z) \to 0$ as $z \to d/c$. This is clear since $c(d/c) -d = d- d = 0$ so $fg \to 0$ so the limit goes to $0$ so the inverse of the limit goes to infinity so the assertion is proved.
\end{proof}

\medskip \noindent {\bf (20.4)}\ Suppose that $f(z_0) = g(z_0) = 0$ and that $f'(z_0)$ and $g'(z_0)$ exist, wehre $g'(z_0) \neq 0$ then show that \begin{equation*}
  \lim_{z \to z_0} \frac{f(z)}{g(z)} = \frac{f'(z_0)}{g'(z_0)}.
\end{equation*}
\begin{proof}
  Using the definition of the derivative we have that
\begin{equation*}
  \frac{f'(z_0)}{g'(z_0)} = \frac{\lim x \to z_0 \frac{f(x - z_0) - f(z_0)}{x - z_0} }{\lim y \to z_0 \frac{g(y - z_0) - g(z_0)}{y - z_0}}
\end{equation*}
Since $x,y$ are any arbitrary sequence (by the existence of $f', g'$) take any sequence $z \to z_0$ then
\begin{equation*}
   \frac{f'(z_0)}{g'(z_0)} = \frac{\lim z \to z_0 \frac{f(z - z_0) - f(z_0)}{z - z_0} }{\lim z \to z_0 \frac{g(z - z_0) - g(z_0)}{z - z_0}} =\lim_{z\to z_0} \frac{(f(z - z_0) - f(z_0))(z - z_0)}{(g(z - z_0) - g(z_0))(z - z_0)}
\end{equation*}
it follows that \begin{equation*}
  \frac{f'(z_0)}{g'(z_0)} = \lim_{z\to z_0} \frac{f(z - z_0) - f(z_0)}{g(z - z_0) - g(z_0)} = \lim_{z\to z_0} \frac{f(z - z_0) - 0}{g(z - z_0) - 0}
\end{equation*}
by the hypothesis $f(z_0) = g(z_0) = 0$ and so $f'(z_0)/g'(z_0)$ is the limit of the fraction!
\end{proof}

\medskip \noindent {\bf (20.8)}\ Show that $f'(z)$ does not exist at any point $z$ when \\
\noindent (a) $f(z) = Re(z)$
\begin{proof}
  Observe that $f(z) = \frac{z + \overline{z}}{2}$ and so $D_{\overline{z}} f \neq 0$ clearly and so Cauchy Riemann equations do not hold at any point $z$ and so $f$ is not differentiable.
\end{proof}
\noindent (b) $f(z) = Im(z) = \frac{iz + \overline{iz}}{2} = Im(z),$ but this is dependent on $\overline{z}$ so the Cauchy Riemann equations are satisfied nowhere and $f$ is nowhere differentiable. 

\medskip \noindent {\bf (24.1)} Use the theorem in Section 21 to show that $f'(z)$ does not exist at any point if \\
\noindent (c) $f(z) = 2x +ixy^2$.
\begin{proof}
  If $f'$  exists the cauchy riemann equations are satisfied; that is $2 = 2yx$ and $0 = y^2$, so $2 = 0$ if the cauchy riemann equations hold, this is a contradiction. Therefore the derivative lives no where.
\end{proof}
\noindent(d) $f(z) = e^xe^{-iy}$.
\begin{proof}
  Equivalently we have that $f(z) = e^{x -iy} = e^{\overline{z}}.$ Therefore $\partial_{\overline{z}} f(z) = e^{\overline{z}} \neq 0$! So the Cauchy-Riemann equations could not hold at any $z$ and the function is nowhere differentiable.
\end{proof}
\medskip \noindent {\bf (24.3)} From results obtained in $21$ and $23$ determine where $f'(z)$ exists and find its value when \\(a) $f(z) = 1/z$.
\begin{proof}
  Using the power rules for differentiation we have that $f'(z) = -z^{-2}$ iff $f$ is diferentiable. To show differentiability we recall that $f(z) = \frac{\overline{z}}{|z|^2} = \frac{x -iy}{x^2 + y^2}$. So the real component of the derivative is consistent iff $\frac{(x^2+y^2) -2x^2}{(x^2+y^2)^2} = \frac{-(x^2+y^2) +y(2y)}{(x^2+y^2)^2}$ wgucg follows since $2y^2 -y^2 -x^2 = x^2 +y^2 - 2x^2$. For the second component of the derivative we have Cauchy riemman conistency since $\frac{-2yx}{(x^2 + y^2)^2} = -\frac{2xy}{(x^2 + y^2)^2}$. So the function is differentiable every where except for $z = 0$.
\end{proof}
\noindent (b) $f(z) = x^2 + iy^2$.
\begin{proof}
  We can actually calculate the derivative using the Cauchy-Riemann equations; that is by the isomorphism between $Df \in E \subset \mathbb{R}^2 \otimes \mathbb{R}^2$ and $f' \in \mathbb{C}$, we use the following derivation to calculaute $f'$. First $2x =2y \implies x=y$ and $0 = -0$ so it must be that $x = y$, lest the derivative not exist. Therefore we have $f'(z) = 2x + 0i= 2y -0i$
\end{proof}

\medskip \noindent {\bf (24.7)} (a) With the aid of the polar form $(6)$, derive the alternative form $f'(z_0) = -\frac{i}{z_0}(u_\theta + iv_\theta)$.
\begin{proof}
  From the section we know that $v_\theta = ru_r$ and $u_\theta = -rv_r$. Therefore $f'(z_0) = e^{-i\theta}(u_r + iv_r) = e^{-i\theta}(v_\theta/r - iu_\theta/r) = e^{-i\theta}/(ri)(u_\theta + i v_\theta) = \frac{-i}{r}e^{-i\theta}(u_\theta + i v_\theta).$ Next $z_0 = re^{i\theta}$ so $1/z_0 = 1/re^{-i\theta}$ and we have the theorem
  \begin{equation*}
    f'(z_0) = -\frac{i}{z_0}(u_\theta + iv_\theta).
  \end{equation*}
  This completes the proof.
\end{proof}

\noindent (b) Derive the derivative of $f(z) = 1/z$ using the above formula.
\begin{proof}
We use the expression and find that $f(z) = 1/z = 1/re^{-i\theta} = 1/r(\cos \theta - i \sin \theta)$ .
Then $f'(z) = -i/z(-\sin \theta + i\cos \theta)= -1/z(\cos \theta - i\sin \theta) = -1/z^2.$
\end{proof}

\medskip \noindent {\bf (26.1)} Apply the main theorem of Section 23 to verify that each of these functions is entire. \\
\noindent (a) $f(z) = e^{-y}\sin x - ie^{-y} \cos x$.
\begin{proof}
  C.R gives $(LHS)\;\;e^{-y}\cos x = e^{-y} \cos(x)\;\; (RHS)$ and $-e^{-y}\sin x = -(-\sin x e^{-y})$ and so the functions are analytic since the partial derivatives are continuous on $\complex$.
\end{proof}
\noindent (d) $f(z) = (z^2 -2)/z $ 
\begin{proof}
  We show that the partial derivative of $f(z)$ w.r.t the conjugate of $z$ is always $0$; that is since $f(z)= (z^2 -2)\times 1/z$, $f(z)$ is the product of two analytic functions, again analytic on the largest open covering contained in the intersections of their domains.
\end{proof}

For  27.4, 27.5, 27.6, the sketches are attached!

\end{document}\end

