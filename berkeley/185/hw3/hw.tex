\documentclass[11pt]{amsart}

\usepackage{amsmath,amsthm}
\usepackage{amssymb}
\usepackage{graphicx}
\usepackage{enumerate}
\usepackage{fullpage}
% \usepackage{euscript}
% \makeatletter
% \nopagenumbers
\usepackage{verbatim}
\usepackage{color}
\usepackage{hyperref}
%\usepackage{times} %, mathtime}

\textheight=600pt %574pt
\textwidth=480pt %432pt
\oddsidemargin=15pt %18.88pt
\evensidemargin=18.88pt
\topmargin=10pt %14.21pt

\parskip=1pt %2pt

% define theorem environments
\newtheorem{theorem}{Theorem}    %[section]
%\def\thetheorem{\unskip}
\newtheorem{proposition}[theorem]{Proposition}
%\def\theproposition{\unskip}
\newtheorem{conjecture}[theorem]{Conjecture}
\def\theconjecture{\unskip}
\newtheorem{corollary}[theorem]{Corollary}
\newtheorem{lemma}[theorem]{Lemma}
\newtheorem{sublemma}[theorem]{Sublemma}
\newtheorem{fact}[theorem]{Fact}
\newtheorem{observation}[theorem]{Observation}
%\def\thelemma{\unskip}
\theoremstyle{definition}
\newtheorem{definition}{Definition}
%\def\thedefinition{\unskip}
\newtheorem{notation}[definition]{Notation}
\newtheorem{remark}[definition]{Remark}
% \def\theremark{\unskip}
\newtheorem{question}[definition]{Question}
\newtheorem{questions}[definition]{Questions}
%\def\thequestion{\unskip}
\newtheorem{example}[definition]{Example}
%\def\theexample{\unskip}
\newtheorem{problem}[definition]{Problem}
\newtheorem{exercise}[definition]{Exercise}

\numberwithin{theorem}{section}
\numberwithin{definition}{section}
\numberwithin{equation}{section}

\def\reals{{\mathbb R}}
\def\torus{{\mathbb T}}
\def\integers{{\mathbb Z}}
\def\rationals{{\mathbb Q}}
\def\naturals{{\mathbb N}}
\def\complex{{\mathbb C}\/}
\def\distance{\operatorname{distance}\,}
\def\support{\operatorname{support}\,}
\def\dist{\operatorname{dist}\,}
\def\Span{\operatorname{span}\,}
\def\degree{\operatorname{degree}\,}
\def\kernel{\operatorname{kernel}\,}
\def\dim{\operatorname{dim}\,}
\def\codim{\operatorname{codim}}
\def\trace{\operatorname{trace\,}}
\def\dimension{\operatorname{dimension}\,}
\def\codimension{\operatorname{codimension}\,}
\def\nullspace{\scriptk}
\def\kernel{\operatorname{Ker}}
\def\p{\partial}
\def\Re{\operatorname{Re\,} }
\def\Im{\operatorname{Im\,} }
\def\ov{\overline}
\def\eps{\varepsilon}
\def\lt{L^2}
\def\curl{\operatorname{curl}}
\def\divergence{\operatorname{div}}
\newcommand{\norm}[1]{ \|  #1 \|}
\def\expect{\mathbb E}
\def\bull{$\bullet$\ }
\def\det{\operatorname{det}}
\def\Det{\operatorname{Det}}
\def\rank{\mathbf r}
\def\diameter{\operatorname{diameter}}

\def\t2{\tfrac12}

\newcommand{\abr}[1]{ \langle  #1 \rangle}

\def\newbull{\medskip\noindent $\bullet$\ }
\def\field{{\mathbb F}}
\def\cc{C_c}



% \renewcommand\forall{\ \forall\,}

% \newcommand{\Norm}[1]{ \left\|  #1 \right\| }
\newcommand{\Norm}[1]{ \Big\|  #1 \Big\| }
\newcommand{\set}[1]{ \left\{ #1 \right\} }
%\newcommand{\ifof}{\Leftrightarrow}
\def\one{{\mathbf 1}}
\newcommand{\modulo}[2]{[#1]_{#2}}

\def\bd{\operatorname{bd}\,}
\def\cl{\text{cl}}
\def\nobull{\noindent$\bullet$\ }

\def\scriptf{{\mathcal F}}
\def\scriptq{{\mathcal Q}}
\def\scriptg{{\mathcal G}}
\def\scriptm{{\mathcal M}}
\def\scriptb{{\mathcal B}}
\def\scriptc{{\mathcal C}}
\def\scriptt{{\mathcal T}}
\def\scripti{{\mathcal I}}
\def\scripte{{\mathcal E}}
\def\scriptv{{\mathcal V}}
\def\scriptw{{\mathcal W}}
\def\scriptu{{\mathcal U}}
\def\scriptS{{\mathcal S}}
\def\scripta{{\mathcal A}}
\def\scriptr{{\mathcal R}}
\def\scripto{{\mathcal O}}
\def\scripth{{\mathcal H}}
\def\scriptd{{\mathcal D}}
\def\scriptl{{\mathcal L}}
\def\scriptn{{\mathcal N}}
\def\scriptp{{\mathcal P}}
\def\scriptk{{\mathcal K}}
\def\scriptP{{\mathcal P}}
\def\scriptj{{\mathcal J}}
\def\scriptz{{\mathcal Z}}
\def\scripts{{\mathcal S}}
\def\scriptx{{\mathcal X}}
\def\scripty{{\mathcal Y}}
\def\frakv{{\mathfrak V}}
\def\frakG{{\mathfrak G}}
\def\aff{\operatorname{Aff}}
\def\frakB{{\mathfrak B}}
\def\frakC{{\mathfrak C}}

\def\suchthat{\mathrel{}:\mathrel{}}
\def\symdif{\,\Delta\,}
\def\mustar{\mu^*}
\def\muplus{\mu^+}

\def\soln{\noindent {\bf Solution.}\ }


%\pagestyle{empty}
%\setlength{\parindent}{0pt}

\begin{document}

\begin{center}{\bf Math 185 --- UCB, Fall 2016 --- William Guss}
\\
{\bf Problem Set 3, due October 4th}
\end{center}

\medskip \noindent {\bf (30.7)}\ Prove that $|exp(-2z)| < 1$ if and only if $Re(z) > 0$.
\begin{proof}
	Let $z = x +iy$. It follows that $|exp(-2z)| < 1$ if and only if $|e^{-2x}e^{-i2y}| = |e^{-2x}||e^{-i2y}| < 1$
	if and only if $|e^{-i2y}| < \frac{1}{e^{-2x}} = e^{2x}.$ The LHS is equivalent to $|\cos({-2y}) +i\sin(-2y)|$ and for every $y |e^{-i2y}| = 1$ so $1 < e^{2x}$ if and only if $ 0 < 2x.$
\end{proof}


\medskip \noindent {\bf (30.11)}\ Describe the behaviour of $e^z$ as $x \to -\infty$ and $y \to \infty$.
\begin{proof}
	In the first case $e^z = e^xe^{iy}$ gives that $|e^z| \to 0$ since $e^{x} \to 0$ as $x \to - \infty.$ Therefore the function $e^z \to 0$ in the limit w.r.t $x$.

	In the second case, we fix $x$ and observe that $e^z = e^xe^{iy}$ parameterizes a circle of radius $e^x$  by angle w.r.t $y$. Therefore, increasing $y$ only results in movement along the disk in the counter clockwise direction. Therefore $e^z$ does not converge since $y \mod 2\pi$ is an equivalence class of infiniteley many elements $y + n2\pi, n \in \mathbb{N}$. 

	However if the limits are achieved simultaneously then $|e^z| \to 0$ implies that $e^z \to 0$ regardless of the angle of approach\footnote{Imagine a marble spiruling down a funnel.}.
\end{proof}


\medskip \noindent {\bf (30.12)}\  Write $Re(e^{1/z})$ in terms of $x$ and $y$. Why is this function harmonic on every domain that does not contain the origin.
\begin{proof}
	Again let $z = x + iy$. Then
		$e^{1/z} = e^{z^{-1}}$. First $1/z = \overline{z}/|z|^2.$ Therefore
		 $e^{z^{-1}} = exp(x/|z|^2)(\cos(-y/|z|^2) +i\sin(-y/|z|^2)$
		 So the real part is 
		 \begin{equation*}
		 	Re(f) = exp\left(\frac{x}{|z|^2}\right) \cos\left(\frac{y}{|z|^2}\right)
		 \end{equation*}
		 Then $Re(f)_{xx}$ is given by
		 	\begin{align}
			\frac{\partial^2}{\partial x^2} exp\left(\frac{x}{|z|^2}\right) \cos\left(\frac{y}{|z|^2}\right) &= \frac{\partial}{\partial x} \left[\left(\cos\left(\frac{y}{|z|^2}\right)-\sin\left(\frac{y}{|z|^2}\right)\right)exp\left(\frac{x}{|z|^2}\right) \frac{\partial}{\partial x}\frac{x}{|z|^2}\right] \\
			\frac{\partial^2}{\partial y^2} exp\left(\frac{x}{|z|^2}\right) \cos\left(\frac{y}{|z|^2}\right) &= \frac{\partial}{\partial y} \left[\left(\cos\left(\frac{y}{|z|^2}\right)-\sin\left(\frac{y}{|z|^2}\right)\right)exp\left(\frac{x}{|z|^2}\right) \frac{\partial}{\partial y} \frac{y}{|z|^2}\right] 	
		 	\end{align}
		 The first equation gives a product rule with a symetric derivative on $(\partial/\partial x) x/|z|^2$ and $(\partial/\partial y)y/|z|^2$  and the antisymmetry on differentiation of trigonometric functions gives that the second derivatives in $x$ and $y$ are equal, so $Re(f)$ is harmonic as long as $x \neq 0$ and $y \neq 0$ since $1/|z|$ is not defined.
\end{proof}


\medskip \noindent {\bf (33.3)}\ Show $Log(i^3) \neq 3 Log(i).$
\begin{proof}
	Recall that $Log(z) = \log |i^3| + i (\Theta + 2n\pi)$ where $n = 0$ and $\Theta = Arg(i^3).$ Computation gives $i^3 = -i$ so $Arg(i^3) = Arg(-i) = -\pi/2.$ Therefore $\log |-i| = \log 1 = 0$ Therefore $Log(i^3) = -i\pi/2$. However, $Log(i) = 0 + i\Theta = 0 + i\pi/2 \neq -3i\pi/2.$
\end{proof}


\medskip \noindent {\bf (33.4)}\ Show that $log(i^2) \neq 2 log(i)$ when the branch
\begin{equation*}
	\log z = \log r + i \theta\ \ \ \ \left(r > 0,\ \frac{3\pi}{4} < \theta < \frac{11\pi}{4}\right).
\end{equation*}
\begin{proof}
	In this case $\log i, \log i^2$ has real part $0$. Then for the imaginary part. $i^2 = -1$ so  $\Theta = -\pi \equiv \pi \in \scriptb_\Theta $ so $\log(i^2) = \pi$ and then $Arg(i) = \pi/2 \equiv 5\pi/2.$ And so $5\pi/2 \neq 4\pi/2$ so the logarithims are not equal.
\end{proof}

\medskip \noindent {\bf (33.7)}\ Show that a branch (Sec. 33)
\begin{equation*}
	\log z = \log r + i \theta \;\;\;\;\;\;\;(r ? 0, \alpha < \theta < \alpha + 2\pi)
\end{equation*}
of the logarithmic function can be written $$\log z = 1/2 \log (x^2 + y^2) + i \tan^{-1}\left(\frac{x}{y}\right)$$
in rectangular coordinates. Then, using the theorem in Sec 23, show that a the given branch is analytic in its domain of definition and that $\frac{d}{dz} \log z = \frac{1}{z}.$
\begin{proof}
	For the first assertion let $z = x + iy$ then $\tan^{-1}\left(\frac{x}{y}\right) = arg(z)$ on the partiular branch,
	by the definition of $\tan \theta = o/a$ for a triangle where $o$ is the height and $a$ is the base/ Therefore 
	$\tan^{-1}(x/y) + n2\pi \in [a, a+2\pi]$. Next $r = |z|.$ So for a real number $|z|,$ it follows that $\ln (|z|^2)^{1/2} = 1/2 \ln |z|^2 = \ln(x^2 + y^2).$

	To show analycity, we compute the partial derivatives as follows. First $u = 1/2 \ln(x^2 + y^2)$ so $u_x = \frac{x}{(x^2 + y^2)}.$ Recalling elementary calculus we have that $v_y = \frac{1}{x(1 + y^2/x^2)} = \frac{x}{x^2(1 + y^2)} = \frac{x}{x^2 + y^2}$. Therefore $u_x = u_y$, and the partial derivatives are continuous in the domain (as long as $x=y \neq 0$).
	Next $v_x = -\frac{y}{x^2(1 + y^2/x^2)} = \frac{-y}{x^2 + y^2}, $ and $u_y = \frac{y^2}{x^2 + y^2}$ so the Cauchy riemann equations are sovled and $\log$ is analytic on its branch.
\end{proof}

\medskip \noindent {\bf (33.12)}\ Show that 
\begin{equation*}
	Re[\log(z-1)] = \frac{1}{2}\ln[(x-1)^2 + y^2]\;\;\;\;\;\;(z\neq 1).
\end{equation*}
\begin{proof}
If $z = x + iy$ then as long as $z -1 \neq 0$ then $x = y \neq 0$ and application of the previous formula (33.7) proven gives $Re[\log(z-1)] = \frac{1}{2}\ln[(x-1)^2 + y^2].$ When $z \neq 1$ this function is the real part of the analytic function in (33.7) and so it is a harmonic function which satisfies Laplace's equation as in Theorem 27.
	
\end{proof}

\medskip \noindent {\bf (34.1)}\ Show that for any two nonzero complex numbers $z_1$ and $z_2$
\begin{equation*}
	Log(z_1z_2) = Log(z_1) + Log(z_2) + 2N\pi i
\end{equation*}
where $N$ has one of the values $0, \pm 1$.
\begin{proof}
	The principle logarithm is defined on the branch of angles by the principle argument. 
	Application of the definition gives
	$\log(z_1z_2) = \ln|z_1z_2| + i arg(z_1z_2) = 
	\ln|z_1||z_2| + iarg(z_1z_2) = \ln|z_1|+\ln|z_2| + i(arg(z_1) + arg(z_2)) = \log(z_1) + \log(z_2)$.
	Now on the principle branch, the real part is stable, but the imaginary part of the log product must be reduced modulo $2\pi.$ Since at most $\pi <\theta_1 + \theta_2 3\pi$ the principle modulo reduces the sum by $2\pi.$ The reverse holds for the lower bound, increasing by $2\pi$ so as to fit the $2\pi$ modulo range of the principle argument.

	Therefore $N = \pm 1$, and if the sume of the principle angles is in the principle branch $N=0$. Thus
	\begin{equation*}
		Log(z_1z_2) = Log(z_1) + Log(z_2) + 2N\pi i
	\end{equation*}
	and this completes the proof.
\end{proof}

\medskip \noindent {\bf (34.5)}\ Let $z$ denote any nonzero complex number, writeen $z = re^{i\Theta}$, $(-\pi < \Theta < \pi),$ and let $n$ denote any fixed positive integer $(n = 1, 2, \dots).$ Show that all of the values of $\log(z^{1/n})$ are given by the equation
\begin{equation*}
	\log(z^{1/n}) = \frac{1}{n} \ln r + i \frac{\Theta + 2(pn +k)\pi}{n}
\end{equation*}
\begin{proof}
First recall from a previous exercise if $z = x+y$
\begin{equation*}
		\log z = 1/2 \ln (x^2 + y^2) + i (\Theta \mod \scriptb_m)
	\end{equation*}	
	If $w = z^{1/n}$ then $w = |z|^{1/n}e^{\frac{i\Theta}{n}}.$ Then $|w|^2 = (|z|^{1/n})^2 = \sqrt[n]{|z|^2}$ so $Re(\log w) = \frac{1}{2n}\ln(x^2 + y^2) = \frac{1}{n}\ln r.$ Since there are $n$ solutions to $w$ we get that the principle argument of each forms a set $Arg w = \{\Theta/n +2k\pi/n\ |\ k \in \mathbb{Z}_n\}$ using eulers formula on the polar form of $z$. Clearly 
	$-\pi < \Theta/n +2k\pi/n\ < \pi$ so now each branch of the logarithm will be identified by moving the principle by $2\pi$. Thus 
	$arg w = \{\Theta/n +2k\pi/n + \rho2\pi\ |\ k \in \mathbb{Z}_n, \rho \in \mathbb{Z}\}$ gives
	\begin{equation*}
		\log(z^{1/n}) = \frac{1}{n} \ln r + i \frac{\Theta + 2(pn +k)\pi}{n},\;\;\;\;\;\;\;\;;\;\;\;\ p\in \mathbb{Z}, k \in \mathbb{Z}_n
	\end{equation*}
\end{proof}
\medskip \noindent {\bf (38.2)}\ (a) With the aid of expression (4), Sec 37. show that 
\begin{equation*}
	exp(iz_1)exp(iz_2) = \cos z_1 \cos z_2 - \sin z_1 \sin z_2 + i(\sin z_1 \cos z_2 + \cos z_1 \sin z_2)
\end{equation*}
\begin{proof}
	Recall that $e^{iz} = \cos z + i \sin z.$ Then $e^{iz_1}e^{iz_2} = (\cos z_1 + i\sin z_1)(\cos z_2 + i \sin z_2)$. Expanding this relation we get $e^{iz_1}e^{iz_2} = \cos z_1 \cos z_2 + i^2 \sin z_1 \sin z_2 + i(\cos z_1 \sin z_2 + \cos z_2 \sin z_1)$ which gives the statement of the exercise.
\end{proof}
(b)  Using the results in part (a) and the fact that
\begin{equation*}
	\sin(z_1 + z_2) = \frac{1}{2i}\left[e^{i(z_1 + z_2)} - e^{-i(z_1 + z_2)}\right] = \frac{1}{2i}\left(e^{iz_1}e^{iz_2} - e^{-iz_1}e^{-iz_2}\right)
\end{equation*}
to obtain the identity \begin{equation*}
	\sin(z + z_2) = \sin z \cos z_2 + \cos z \sin z_2.
\end{equation*}
\begin{proof}
	Observe the following algebra.
	\begin{equation*}
		\begin{aligned}
			\sin(z + z_2) &= \frac{1}{2i}\left(e^{iz}e^{iz_2} - e^{-iz}e^{-iz_2}\right) \\
			&= \frac{1}{2i}(\cos z \cos z_2 - \sin z \sin z_2 + i(\sin z \cos z_2 + \cos z \sin z_2) 
				\\&- \left(\cos (-z) \cos (-z_2) - \sin (-z) \sin (-z_2) + i(\sin (-z) \cos (-z_2) + \cos (-z) \sin (-z_2))\right)) \\
			&= \frac{1}{2i}\cos z \cos z_2 - \sin z \sin z_2 + i(\sin z \cos z_2 + \cos z \sin z_2 
				\\&- \left(\cos (z) \cos (z_2) - \sin (z) \sin (z_2) - i(\sin (z) \cos (z_2) + \cos (z) \sin (z_2))\right)) \\
			&= \frac{2i}{2i}\left(\sin (z) \cos (z_2) + \cos (z) \sin (z_2)\right) \\
			&= \sin (z) \cos (z_2) + \cos (z) \sin (z_2)
		\end{aligned}
	\end{equation*}
\end{proof}
\medskip \noindent {\bf (38.3)}\ Show that $\cos(z_1 + z_2) = \cos z_2 \cos z_2 - \sin z_1 \sin z_2.$
\begin{proof}
	From the previous exercise we differentiate the expression \begin{equation*}
		d/dz \sin(z + z_2) = \cos(z + z_2) = d/dz \sin(z)\cos(z_2) + \cos(z)+\sin(z_2)
	\end{equation*}
	So it follows immediately that $\cos(z_1 + z_2) = \cos z_2 \cos z_2 - \sin z_1 \sin z_2.$
\end{proof}
\medskip \noindent {\bf (39.2)}\ Prove that $\sinh 2z = 2 \sinh z \cosh z$.
\begin{proof}
	We do the following algebra
	\begin{equation*}
		\sinh 2z = \frac{(e^{z} - e^{-z})}{2} = \frac{(e^ze^z + e^ze^{-z} - e^ze^{-z} e^{-z}e^{-z})}{2} = 2\frac{(e^{z} - e^{-z})}{2}\frac{(e^{z} + e^{-z})}{2}
	\end{equation*}
	Observe the right side is $\sinh 2z = 2\cosh z \sinh z$ and the algebra is if and only if.
\end{proof}
\medskip \noindent {\bf (40.3)}\ Solve $\cos z = \sqrt{2} $ for $z$.
\begin{proof}
	Using the formula, we have that 
	\begin{equation*}
		\cos^{-1}(z) = -i \log\left[z + i(1 - z^2)^{1/2}\right].
	\end{equation*}
	Then $\cos^{-1}(\sqrt{2})$, it follows that $-i \log\left[\sqrt{2} + i(1 - 2)^{1/2}\right].$ Then we have all solutions are $-i \log\left[\sqrt{2} \pm i\right]. $
	Therefore $\cos^{-1} = -i \left[\frac{1}{2} \ln 3 \pm i (\tan\left(\frac{1}{\sqrt{2}}\right) + 2k\pi)\right] =  \pm\tan\left(\frac{1}{\sqrt{2}}\right) + 2k\pi -i \frac{\ln 3}{2}$.
\end{proof}


\end{document}\end
