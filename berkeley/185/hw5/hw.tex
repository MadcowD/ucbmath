\documentclass[11pt]{amsart}

\usepackage{amsmath,amsthm}
\usepackage{amssymb}
\usepackage{graphicx}
\usepackage{enumerate}
\usepackage{fullpage}
% \usepackage{euscript}
% \makeatletter
% \nopagenumbers
\usepackage{verbatim}
\usepackage{color}
\usepackage{hyperref}
%\usepackage{times} %, mathtime}

\textheight=600pt %574pt
\textwidth=480pt %432pt
\oddsidemargin=15pt %18.88pt
\evensidemargin=18.88pt
\topmargin=10pt %14.21pt

\parskip=1pt %2pt

% define theorem environments
\newtheorem{theorem}{Theorem}    %[section]
%\def\thetheorem{\unskip}
\newtheorem{proposition}[theorem]{Proposition}
%\def\theproposition{\unskip}
\newtheorem{conjecture}[theorem]{Conjecture}
\def\theconjecture{\unskip}
\newtheorem{corollary}[theorem]{Corollary}
\newtheorem{lemma}[theorem]{Lemma}
\newtheorem{sublemma}[theorem]{Sublemma}
\newtheorem{fact}[theorem]{Fact}
\newtheorem{observation}[theorem]{Observation}
%\def\thelemma{\unskip}
\theoremstyle{definition}
\newtheorem{definition}{Definition}
%\def\thedefinition{\unskip}
\newtheorem{notation}[definition]{Notation}
\newtheorem{remark}[definition]{Remark}
% \def\theremark{\unskip}
\newtheorem{question}[definition]{Question}
\newtheorem{questions}[definition]{Questions}
%\def\thequestion{\unskip}
\newtheorem{example}[definition]{Example}
%\def\theexample{\unskip}
\newtheorem{problem}[definition]{Problem}
\newtheorem{exercise}[definition]{Exercise}

\numberwithin{theorem}{section}
\numberwithin{definition}{section}
\numberwithin{equation}{section}

\def\reals{{\mathbb R}}
\def\torus{{\mathbb T}}
\def\integers{{\mathbb Z}}
\def\rationals{{\mathbb Q}}
\def\naturals{{\mathbb N}}
\def\complex{{\mathbb C}\/}
\def\distance{\operatorname{distance}\,}
\def\support{\operatorname{support}\,}
\def\dist{\operatorname{dist}\,}
\def\Span{\operatorname{span}\,}
\def\degree{\operatorname{degree}\,}
\def\kernel{\operatorname{kernel}\,}
\def\dim{\operatorname{dim}\,}
\def\codim{\operatorname{codim}}
\def\trace{\operatorname{trace\,}}
\def\dimension{\operatorname{dimension}\,}
\def\codimension{\operatorname{codimension}\,}
\def\nullspace{\scriptk}
\def\kernel{\operatorname{Ker}}
\def\p{\partial}
\def\Re{\operatorname{Re\,} }
\def\Im{\operatorname{Im\,} }
\def\ov{\overline}
\def\eps{\varepsilon}
\def\lt{L^2}
\def\curl{\operatorname{curl}}
\def\divergence{\operatorname{div}}
\newcommand{\norm}[1]{ \|  #1 \|}
\def\expect{\mathbb E}
\def\bull{$\bullet$\ }
\def\det{\operatorname{det}}
\def\Det{\operatorname{Det}}
\def\rank{\mathbf r}
\def\diameter{\operatorname{diameter}}

\def\t2{\tfrac12}

\newcommand{\abr}[1]{ \langle  #1 \rangle}

\def\newbull{\medskip\noindent $\bullet$\ }
\def\field{{\mathbb F}}
\def\cc{C_c}

\newenvironment{solution}
  {\begin{proof}[Solution]}
  {\end{proof}}



% \renewcommand\forall{\ \forall\,}

% \newcommand{\Norm}[1]{ \left\|  #1 \right\| }
\newcommand{\Norm}[1]{ \Big\|  #1 \Big\| }
\newcommand{\set}[1]{ \left\{ #1 \right\} }
%\newcommand{\ifof}{\Leftrightarrow}
\def\one{{\mathbf 1}}
\newcommand{\modulo}[2]{[#1]_{#2}}

\def\bd{\operatorname{bd}\,}
\def\cl{\text{cl}}
\def\nobull{\noindent$\bullet$\ }

\def\scriptf{{\mathcal F}}
\def\scriptq{{\mathcal Q}}
\def\scriptg{{\mathcal G}}
\def\scriptm{{\mathcal M}}
\def\scriptb{{\mathcal B}}
\def\scriptc{{\mathcal C}}
\def\scriptt{{\mathcal T}}
\def\scripti{{\mathcal I}}
\def\scripte{{\mathcal E}}
\def\scriptv{{\mathcal V}}
\def\scriptw{{\mathcal W}}
\def\scriptu{{\mathcal U}}
\def\scriptS{{\mathcal S}}
\def\scripta{{\mathcal A}}
\def\scriptr{{\mathcal R}}
\def\scripto{{\mathcal O}}
\def\scripth{{\mathcal H}}
\def\scriptd{{\mathcal D}}
\def\scriptl{{\mathcal L}}
\def\scriptn{{\mathcal N}}
\def\scriptp{{\mathcal P}}
\def\scriptk{{\mathcal K}}
\def\scriptP{{\mathcal P}}
\def\scriptj{{\mathcal J}}
\def\scriptz{{\mathcal Z}}
\def\scripts{{\mathcal S}}
\def\scriptx{{\mathcal X}}
\def\scripty{{\mathcal Y}}
\def\frakv{{\mathfrak V}}
\def\frakG{{\mathfrak G}}
\def\aff{\operatorname{Aff}}
\def\frakB{{\mathfrak B}}
\def\frakC{{\mathfrak C}}

\def\suchthat{\mathrel{}:\mathrel{}}
\def\symdif{\,\Delta\,}
\def\mustar{\mu^*}
\def\muplus{\mu^+}

\def\soln{\noindent {\bf Solution.}\ }


%\pagestyle{empty}
%\setlength{\parindent}{0pt}

\begin{document}

\begin{center}{\bf Math 185 --- UCB, Fall 2016 --- William Guss}
\\
{\bf Problem Set 5, due October 18th}
\end{center}

\medskip \noindent {\bf (49.1)}\ Use an antiderivative to show that for every contour $C$ extending from a point
$z_1$ to a point $z_2$,
\begin{equation*}
	\int_Cz^n\ dz = \frac{1}{n+1}(z^{n+1}_2 - z^{n+1})
\end{equation*}
\begin{proof}
	From previous analysis we have that multiplication on the complex plane is a continuous mapping, so the map $f(z) = z^n$
	is a continuous map. Therefore it is integrable. We claim that $F(z) = \frac{z^{n+1}}{n+1}$ is the antiderivative of $f$
	and then apply Theorem 1 of section 48 to show that $\int_C f\ dz = F(z_2) - F(z_1).$ 
	Take $z \in \mathbb{C}$. Then if $F(z)(n+1) = z^{n+1}$ we have that
	\begin{equation*}
		F'(z) = \frac{n+1}{n+1} z^n = f(z) 
	\end{equation*}
	using the differentiation rules established in earlier chapters. Therefore $f$ has an antiderivative, nameley $F$, and
	for any $1$-cell such that $C(0) = z_1, C(1) = z_2$ we have that $f\ dz(C) = F(z_2) - F(z_1) = \frac{1}{n+1}\left[z^{n+1}_2 - z^{n+1}_1\right].$
\end{proof}

\medskip \noindent {\bf (49.5)}\ Show that 
\begin{equation*}
	\int_{-1}^1 z^i\ dz = \frac{1+e^{-\pi}}{2}(1-i)
\end{equation*}
where the integrand denotes the principle branch
\begin{equation*}
	z^i = \exp(i\ Log*(z))\;\;\;\;(|z|>0, -\pi < Arg(z) < \pi).
\end{equation*}
and the path $C$ lies above the real axis.
\begin{proof}
	As per the books suggestion we shift the domain of the branch cut on $log$ so that the particular chart of the manifold
	over which we integrate has the fully connected property and the contour is properly defined. Now the antiderivative of $z^i$ on such a branch cut is
	\begin{equation*}
		F(z) = \frac{1}{i+1}e^{(i+1)\log(z)} \implies F'(z) = \frac{d}{dz}\frac{1}{i+1}e^{i \log(z)}z = \frac{1}{i+1}\left[e^{i \log(z)} + \frac{iz}{z} e^{i\log(z)}\right] = \frac{i+1}{i+1}e^{i \log(z)}
	\end{equation*}
	Therefore by Theorem 1 of section 48 we have that
	\begin{equation*}
		\int_C\ f\ dz = \frac{1}{1+i} \left[e^{(1+i)\log(1)} - e^{(1+i)\log(-1)}\right] = \frac{e^{(1+i)(0+0i)} -e^{(1+i)(0 + \pi i)}}{1+i} = \frac{1 - e^{\pi i - \pi}}{1 + i}
	\end{equation*}
	Thus the integral is always
	\begin{equation*}
		\int_C\ f\ dz = \frac{1+e^{-\pi}}{2}(1-i)
	\end{equation*}
\end{proof}

\medskip \noindent {\bf (53.1)}\ Apply the Cauchy-Gorsat theorem to show that 
\begin{equation*}
	\int_C f(z)\ dz = 0
\end{equation*}
when the contour $C$ is the unit circle $|z| = 1$, in either direction, and when \\
(a) $f(z) = \dfrac{z^2}{z+3}$
\begin{proof}
	Consider that $f$ is not analytic at $z = -3$ which lies outside of the unit circle and so the contour
	is over a fully connected domain with no holes in its interior, and the $1$-form is $0$ on any $1$-cell which is a closed loop with countably many intersections
	by the Cauchy-Goursat theorem.
\end{proof}
\noindent (b) $f(z) = ze^{-z}$
\begin{proof}
	Again, in the domain of $f$ within the unit ball in $\mathbb{C}$ there are no singularities and the multiplication
	of two analytic functions is an analytic function on the interesection of their analytic domains. Thus,
	for any $1$-cell which is closed the one form $f\ dz(C) = 0$ by the Cauchy-Goursat theorem.
\end{proof}
\noindent (c) $f(z) = \dfrac{1}{z^2 + 2z + 2}$
\begin{proof}
	Observe using the quadractic formula we have that \begin{equation*}
		z = \frac{-2 \pm \sqrt{4 - 8}}{2} = -1 \pm i \implies z^2 + 2z +2 = 0. 
	\end{equation*}
	Thus 
	\begin{equation*}
		f(z) = \frac{1}{(z+1-i)(z+1+i)}
	\end{equation*}
	has singularities lying outside of the unit circle and so on $1$-cells which are closed in the unit ball the one form $f\ dz(C) = 0$ by the Cauchy-Goursat theorem.
\end{proof}


\medskip \noindent {\bf (53.2)}\ Let $C_1$ denote the postiveley oriented boundary of the square whose sides lie along the lines $x = \pm 1$, $y = \pm 1$ and $C_2$ be the  postiveley oriented cirlce $|z| = 4$. With the aid of the corollary in Sec 53. point out why
\begin{equation*}
	\int_{C_1} f(z)\ dz = \int_{C_2} f(z)\ dz
\end{equation*}
when \\
(a) $f(z) = \dfrac{1}{3z^2 + 1}$
\begin{solution}
The function $f$ is analytic everywhere except for $z^2 = -1/3$, and all points there satisfying lie in the interior region of $C_1$. We can homotop $C_1$ to $C_2$ such that the interiors are endowed with the same topology and the image of the homotopy is a domain on which $f$ is analytic. Therefore the evaluations of the $1$-forms are equivalent.
\end{solution}
\noindent (b) $f(z) = \dfrac{z+2}{\sin{(z/2)}}$
\begin{solution}
	By a theorem of Section 38, the only zeroes of the complex trigonometric functions are on the real line, spaced in integral differences of multiples of $\pi$. In particular $\sin(z) = 0$ iff $z = n\pi$, and $n$ an integer. Therefore within the interior of $C_1$ $f$ is not analytic only at $0$. Since $\sin(z/2) = 0$ iff $z = 2n\pi$ and $|2n\pi| > |4|$ for all $n \neq 0$ in the integers, there are no zeroes besides that particular one aforementioend in the interior of $C_2$. Then by the previous homotopy principle the evaluations of the $1$-forms are equivalent.
\end{solution}
\noindent (c) $f(z) = \dfrac{1}{1-e^z}$.
\begin{solution}
 	The only $z$ for which $e^z = 1$ is $z = 0$ by definition and therefore $f(z)$ is only not analytric on $z = 0$ so both $C_2$ and $C_1$ have only one (the same) singularity on their interior  and by the homotopy principle  the evaluations of the $1$-forms are equivalent.
 \end{solution} 

\medskip \noindent {\bf (53.3)}\ If $C_0$ denotes a positiveley oriented circle $|z - z_0| = R$, then
\begin{equation*}
	\int_{C_0}(z- z_0)^{n-1}\ dz = \begin{cases}
		0, &\text{when } n = \pm1, \pm2, \dots, \\
		2pi &\text{when } n = 0 
	\end{cases}
\end{equation*}
accoridng to Exercise $13$ Section 46. Use that result and the homotopy principle to show that if $C$ is the boundary
of the rectangle $0 \leq x \leq 3$, $0 \leq y \leq 2$ described in the positive sense
\begin{equation*}
	\int_C (z -2 -i)^{n-1}\ dz =  \begin{cases}
		0, &\text{when } n = \pm1, \pm2, \dots, \\
		2pi &\text{when } n = 0 
		\end{cases}
\end{equation*}
\begin{proof}
	First we show that the integrand is analytic except for $z_0 = 2 + i$. By the closure of $C^{\omega}(\mathbb{C}, \mathbb{C})$ under multiplication $(z - z_0)^{m}$ is analytic everywhere for all $m \geq 0$. It follows that the inverse of such a function is analytic everywhere except for at the zeroes, which are namely $\{z_0\}$. Lastly we can write any function of the form of the integrand as either  $(z - z_0)^{m}$ or its inverse, and so the integrand is analytic every where except at $\{z_0\}$.

	Now since $z_0$ is in the interior of the $C$ ($0 \leq Re(z_0) = 2 \leq 3$ and $0 \leq Im(z_0) = 1 \leq 2$), and no other 
	singularities exist, we may homotop $C$ to a ball of any positive radius around $z_0$ and apply the homotopy principle to yield the conclusion of the theorem; that is, since the evaluation of the any $1$-cell, $\phi,$ in the homotopy class of the circle with respect to the stated $1$-form gives 
	\begin{equation*}
		\int_\phi (z -2 -i)^{n-1}\ dz =  \begin{cases}
		0, &\text{when } n = \pm1, \pm2, \dots, \\
		2pi &\text{when } n = 0 
		\end{cases}
	\end{equation*}
	the theorem holds as $\phi \simeq C$ via homotopy restricted to the analytical domain of the integrand.
\end{proof}


\medskip \noindent {\bf (53.4)}\ Show that
\begin{equation*}
	\int_0^\infty e^{-x^2}\cos 2bx\ dx = \frac{\sqrt{\pi}}{2}e^{-b^2}\;\;\;\;(b>0)
\end{equation*}
\begin{proof}
	First consider the rectangular contour $C(a,b) := C$ which visits $-a, a, a+bi, -a+bi$ in a positive orientation. We
	evaluate $C$ on the $1$-form $e^{-z^2}\ dz$, first computing the horizontal legs $C_H$
	\begin{equation*}
		\begin{aligned}
		\int_{C_H} f\ dz&= \int_{0}^1 f(C_{H_1}(t)) C_{H_1}'(t) \ dt -  \int_{0}^1 f(C_{H_2}(t)) C_{H_2}'(t)\ dt,\\
		&= 2\int_{0}^a e^{-t^2}\ dt  -  \int_{0}^1 f'(C_{H_2}(z)) C_{H_2}'(t)\ dt,\\
		&= 2\int_{0}^a e^{-t^2}\ dt - \int_{-a}^a e^{-(t^2  +tbi -b^2)}\ dt,\\
		&= 2\int_{0}^a e^{-t^2}\ dt - 2e^{b^2}\int_{0}^a e^{-t^2}\cos(2bt)\ dt.
		\end{aligned}
	\end{equation*}
	Second computing the veritcal legs we get
	\begin{equation*}
		\begin{aligned}
			\int_{C_V} f\ dz &= \int_{0}^1 f(C_{V_1}(t)) C_{V_1}'(t) \ dt -  \int_{0}^1 f(C_{V_2}(t)) C_{V_2}'(t)\ dt,\\
			&= i\int_0^b e^{-(a+ti)^2}\ dt - i\int_0^b e^{-(ti-a)^2}\ dt, \\
			&=  ie^{-a^2}\int_0^b e^{t^2}e^{-2ati}\ dt - ie^{-a^2}\int_0^b e^{t^2}e^{2ati}\ dt.
		\end{aligned}
	\end{equation*}
	By the analycity of $f$ we have that the closed loop integral is $0$ thus
	\begin{equation*}
			0 = 2\int_{0}^a e^{-t^2}\ dt - 2e^{b^2}\int_{0}^a e^{-t^2}\cos(2bt)\ dt + ie^{-a^2}\int_0^b e^{t^2}e^{-2ati}\ dt - ie^{-a^2}\int_0^b e^{t^2}e^{2ati}\ dt.
	\end{equation*}
	and so \begin{equation*}
		\int_{0}^a e^{-t^2}\cos(2bt)\ dt = e^{-b^2}\int_{0}^a e^{-t^2}\ dt + i e^{-a^2 -b^2} \left(\int_0^b e^{t^2}e^{-2ati}\ dt - \int_0^b e^{t^2}e^{2ati}\ dt \right).
	\end{equation*}
	Using that $-i\times i \sin(\gamma)$ gives $\sin(\gamma)$ for any $\gamma$ in $\mathbb{C}$, we get (upon rearrangement)
	\begin{equation*}
	\int_{0}^a e^{-t^2}\cos{2bt}\ dt = e^{-b^2}\int_{0}^a e^{-t^2}\ dt + e^{-(a^2 +b^2)} \left(\int_0^b e^{t^2}\sin{2at}\ dt \right).
	\end{equation*}

	Granting that
	\begin{equation*}
		\int_0^\infty e^{-x^2}\ dx = \frac{\sqrt{\pi}}{2}
	\end{equation*}
	and that 
	\begin{equation*}
	\left|\int_{0}^a e^{-t^2}\cos{2bt}\ dt\right| \leq \int_0^\infty e^{-t^2}\ dt 
	\end{equation*}
	we get that when $a \to \infty$ the right hand side of the third to last equation tends to $0$ (as it is absoluteley bounded) save for the term 
	\begin{equation*}
	\int_{0}^a e^{-t^2}\cos{2bt}\ dt = e^{-b^2}\int_{0}^\infty e^{-t^2}\ dt = \frac{\sqrt{\pi}}{2}e^{-b^2}.
	\end{equation*}
	This completes the proof.

\end{proof}

\medskip \noindent {\bf (53.5)}\ Let the contours be given from the book, $C_1$, $C_2,$ and $C_3$. Show that if $C = C_1 + C_2$
and $f$ is entire then
\begin{equation*}
	\int_{C_1} f(z)\ dz = \int_{C_3} f(z)\ dz\;\;\text{and}\;\;\;\int_{C_2} f(z)\ dz = - \int_{C_3} f(z)\ dz.
\end{equation*}
and finally that $\int_C\ f(z)\ dz = 0$.
\begin{proof}
	First consider $C_1$ and $C_3$. If $f$ is entire, by the image of $C_2$ diffeomorphic to $[0,1]$ we know (theorem of previous of chapter) that $\int_{C_1} f\ dz = f(1) - f(0).$ Using the same argument on $C_3$ we get $ \int_{C_3} f\ dz = f(1) - f(0).$ so the evaulations are equal. The key step here is that the image of $C_1$ is diffeomorphic to $I$ since it is defined exactly by a function of $f$ and not a relation. If this were not the case, such an argument may not be possible since contours may just be smooth mappings, and not all smooth mappings are homotopically equivalent. 

	Now we know that $0 = \int_{C_1+C_3} f\ dz = \int_{C_1} f\ dz + \int_{C_3} f\ dz = 0$ by the Cauchy-Gorsaut theorm. By the linearity of differential forms with respect to end points, it follows that
	\begin{equation*}
		\int_C f\ dz = \int_{C_1} f\ dz + \int_{C_2} f\ dz =\int_{C_3} f\ dz + \int_{C_2} f\ dz = 0
	\end{equation*}
	This completes the proof.
\end{proof}


\medskip \noindent {\bf (53.6)}\ Let $C$ denote the postiveley oriented boundary of the half disk $0 \leq r \leq 1$, $0 \leq \theta \leq \pi$
and let $f(z)$ be a continuous function defined on that half disk by writing $f(0) = 0$ and
using the branch
\begin{equation*}
	f(z) = \sqrt{r} e^{i\theta/2}\;\;\;\;\left(r > 0, -\frac{\pi}{2} < \theta < \frac{3\pi}{2}\right)
\end{equation*}
of the multiple-valued function $z^{1/2}.$ Show that $\int_C f(z)\ dz = 0$ by evaluating serparateley the integrals of $f(z)$ over the semicircle
and the two radii which make up $C$.
\begin{proof}
	First at $\theta = 0, \pi$ we evaluate the $1$-form. We get $\omega(C_1), \omega(C_2)$ as follows
	\begin{equation*}
		\omega(C_1) = e^{i0/2}\int_0^1 \sqrt{r} C_1'(r)\ dr = \int_0^1 \sqrt{r}\ dr =\frac{2}{3}
	\end{equation*}
	\begin{equation*}
		\omega(C_2) = e^{i\pi/2}\int_0^1 \sqrt{r} e^{i\pi}\ dr = e^{i3\pi/2}\int_0^1 \sqrt{r}\ dr =\frac{2}{3} e^{i3\pi/2}
	\end{equation*}
	For the third curve we parameterize in terms of $\theta$, that is we compute $\omega(C_3)$ as
	\begin{equation*}
		\omega(C_3) = \int_0^\pi e^{i\theta/2}e^{i\theta}\ d\theta = \int_0^\pi e^{i3\theta/2}\ d\theta = \frac{2}{3i}\left(e^{i3\pi/2}-1 \right)
	\end{equation*}
	and so \begin{equation*}
		\omega(C) = \omega(C_1) + \omega(C_2) - \omega(C_3) = \frac{2}{3}\left(1 + i\left(1 - e^{i3\pi/2}\right) -  e^{i3\pi/2}\right) = \frac{2}{3}\left(1 - e^{i3\pi/2}\right)(1+i)
	\end{equation*}
	This completes the proof.
\end{proof}

 \noindent Why does the Cauchy-Goursat theorem not apply here? 
 \begin{solution}
 Since the function $f$ is certainly not analytic at $0$, one could not apply the Cauchy Goursat theorem whose conditions imply its conclusion only on an analytic domain of $f$.
 \end{solution}

\medskip \noindent {\bf (53.7)}\ Show that if $C$ is a postively oriented simple closed contour, then the area of the region enclosed by $C$ can be written
\begin{equation*}
	\frac{1}{2i}\int_{C}\overline{z}\ dz.
\end{equation*}
\begin{proof}
	Consider the above integral in the language of differential forms on $\mathbb{R}^2$. The function $f = \overline{\cdot}: \mathbb{C} \to \mathbb{C}$ such that $f(x,y) = (x, -y) = x -iy$, and this mapping is smooth with respect to its pratial derivatives. Therefore $f dz = (f_1 + if_2) dz = (f_1 + if_2)(dx + idy) = (f_1+if_2)dx +(if_1  -f_2)dy = \omega$ is a differential $1$-form and we can apply Stokes Formula for a General $1$-cell as follows: if $C$ is as above and  there exists $M$ a $2$-cell such that $\partial M = C$
	\begin{equation*}
		\int_{\partial M} \omega = \int_M\ d\omega
	\end{equation*}
	where $d$ is the exterior derivitive. In this case 
	\begin{equation*}
		\begin{aligned}
				d(\omega) = d(f_1 + if_2) \wedge dx + d(if_1  -f_2) \wedge dy &=
				 \left[ \frac{\partial(f_1 + if_2)}{\partial x} dx + \frac{\partial (f_1+ if_2)}{\partial y} dy \right] \wedge dx +  \left[\cdots\right] \wedge dy \\
				&= \frac{\partial f_1+if_2}{\partial y} dy \wedge dx +  i\frac{\partial (if_1-f_2)}{\partial x} dx \wedge dy \\
				 &=\left[ \frac{\partial (if_1-f_2)}{\partial x} - \frac{\partial f_1+if_2}{\partial y} \right] dx \wedge dy  \\
				 &= \left[\left(-\frac{\partial f_1}{\partial y}  -\frac{\partial f_2}{\partial x}\right) + i\left(\frac{\partial f_1}{\partial x}  -\frac{\partial f_2}{\partial y}\right)\right] dx \wedge dy \\
				 &= \left[0 + i(1 -(-1))\right] dx \wedge dy\\
				 &= 2i dx \wedge dy.
		\end{aligned}
	\end{equation*}
	Therefore $\frac{1}{2i} \int_M d\omega = \int_M dx \wedge dy$, so 	it follows that
	\begin{equation*}
		\int_M dx \wedge dy = \frac{1}{2i} \int_{\partial M} \omega = \int_C \overline{z}\ dz.
	\end{equation*}
	Since $M$ is the interior (if it exists) of $C$ then the proof is complete.
\end{proof}


\end{document}\end
