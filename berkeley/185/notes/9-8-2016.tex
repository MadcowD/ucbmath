%%%%%%%%%%%%%%%%%%%%%%%%%%%%%%%%%%%%%%%%%%%%%%%%%%%%%%%%%%%%%%%%%%
%%%                      Homework _                            %%%
%%%%%%%%%%%%%%%%%%%%%%%%%%%%%%%%%%%%%%%%%%%%%%%%%%%%%%%%%%%%%%%%%%

\documentclass[letter]{article}

\usepackage{lipsum}
\usepackage[pdftex]{graphicx}
\usepackage[margin=1.5in]{geometry}
\usepackage[english]{babel}
\usepackage{listings}
\usepackage{amsthm}
\usepackage{amssymb}
\usepackage{framed} 
\usepackage{amsmath}
\usepackage{titling}

\usepackage{fullpage}

\newtheorem{theorem}{Theorem}
\newtheorem{lemma}{Lemma}
\newtheorem{fact}{Fact}
\newtheorem{example}{Example}
\newtheorem{definition}{Definition}
\newtheorem{proposition}{Proposition}

\newenvironment{menumerate}{%
  \edef\backupindent{\the\parindent}%
  \enumerate%
  \setlength{\parindent}{\backupindent}%
}{\endenumerate}



\textheight=600pt %574pt
\textwidth=480pt %432pt
\oddsidemargin=15pt %18.88pt
\evensidemargin=18.88pt
\topmargin=10pt %14.21pt

\parskip=1pt %2pt

\def\reals{{\mathbb R}}
\def\torus{{\mathbb T}}
\def\integers{{\mathbb Z}}
\def\rationals{{\mathbb Q}}
\def\naturals{{\mathbb N}}
\def\complex{{\mathbb C}\/}
\def\distance{\operatorname{distance}\,}
\def\support{\operatorname{support}\,}
\def\dist{\operatorname{dist}\,}
\def\Span{\operatorname{span}\,}
\def\degree{\operatorname{degree}\,}
\def\kernel{\operatorname{kernel}\,}
\def\dim{\operatorname{dim}\,}
\def\codim{\operatorname{codim}}
\def\trace{\operatorname{trace\,}}
\def\dimension{\operatorname{dimension}\,}
\def\codimension{\operatorname{codimension}\,}
\def\kernel{\operatorname{Ker}}
\def\Re{\operatorname{Re\,} }
\def\Im{\operatorname{Im\,} }
\def\eps{\varepsilon}
\def\lt{L^2}
\def\bull{$\bullet$\ }
\def\det{\operatorname{det}}
\def\Det{\operatorname{Det}}
\def\diameter{\operatorname{diameter}}
\def\symdif{\,\Delta\,}
\newcommand{\norm}[1]{ \|  #1 \|}
\newcommand{\set}[1]{ \left\{ #1 \right\} }
\def\one{{\mathbf 1}}
\def\cl{\text{cl}}

\def\newbull{\medskip\noindent $\bullet$\ }
\def\nobull{\noindent$\bullet$\ }



\def\scriptf{{\mathcal F}}
\def\scriptq{{\mathcal Q}}
\def\scriptg{{\mathcal G}}
\def\scriptm{{\mathcal M}}
\def\scriptb{{\mathcal B}}
\def\scriptc{{\mathcal C}}
\def\scriptt{{\mathcal T}}
\def\scripti{{\mathcal I}}
\def\scripte{{\mathcal E}}
\def\scriptv{{\mathcal V}}
\def\scriptw{{\mathcal W}}
\def\scriptu{{\mathcal U}}
\def\scriptS{{\mathcal S}}
\def\scripta{{\mathcal A}}
\def\scriptr{{\mathcal R}}
\def\scripto{{\mathcal O}}
\def\scripth{{\mathcal H}}
\def\scriptd{{\mathcal D}}
\def\scriptl{{\mathcal L}}
\def\scriptn{{\mathcal N}}
\def\scriptp{{\mathcal P}}
\def\scriptk{{\mathcal K}}
\def\scriptP{{\mathcal P}}
\def\scriptj{{\mathcal J}}
\def\scriptz{{\mathcal Z}}
\def\scripts{{\mathcal S}}
\def\scriptx{{\mathcal X}}
\def\scripty{{\mathcal Y}}
\def\frakv{{\mathfrak V}}
\def\frakG{{\mathfrak G}}
\def\frakB{{\mathfrak B}}
\def\frakC{{\mathfrak C}}




%%%%%%%%%%%%%%%
%% DOC INFO %%%
%%%%%%%%%%%%%%%
\newcommand{\bHWN}{ }
\newcommand{\bCLASS}{MATH 185}

\title{\bCLASS: Notes }
\author{Scribe: William Guss}
\usepackage{csquotes}

%%%%%%%%%%%%%%

\begin{document}
\maketitle
\thispagestyle{empty}
\begin{fact}
	A function $f: \mathbb{C} \to \mathbb{C}$ is differrentiable at $z = (x_0, y_0)$ implies that
	\begin{equation*}
		u_x = u_y,\;\;\;u_y=-v_x
	\end{equation*}
	and we denote such a relation the Cauchy-Riemann equations.
\end{fact}
\begin{fact}
	Let $f = u +iv$ defined  in $\epsilon$-neighborhood of $z_0$. If first order derivitives of $u,v$ exist in the neighborhood and 
	the partial derivatives are continuous at $z_0$ and satisfy the CR equations, then $f$ is differentiable at $z_0$.
\end{fact}
\begin{example}
	Let $f(z) = \frac{\overline{z}^2}{z}$. The function $f$ is not differentiable at $z = 0$.
\end{example}
\begin{proof}
	Let $z = re^{-i\theta}$ and $f$ is continuous. Then we attempt the derivative.
	\begin{equation*}
		f'(0) = \lim_{\Delta z \to 0} \frac{f(\Delta z)-f(0)}{\Delta z} = \lim_{\Delta z \to 0} \frac{\overline{\Delta z}^2}{z}
	\end{equation*}
	which does not exist as we approach from different directions.
\end{proof}

\begin{definition}
	Let $D \subset \complex$ be an open set. A function $f: \complex \to \complex$ is analytic in $D$ if and only if $f'(z)$ exists for all $z \in D$.
\end{definition}
\noindent \textbf{Remark.} This is interesting since inharently analycity of a function in $\mathbb{R}$ is defined as a functions decomposability into an infinite power series, and therefore a functions infinite differentiability. In $\complex$ however we need only that a function is first order differentiable on an open domain.
\begin{fact}
	If $f: \complex \to \complex$, $f$ does not depend on $\overline{z}$, that is; $\frac{\partial f}{\partial \overline{z}} = 0$ if and only if $f$ satisfies the cauchy Riemann equations. 
\end{fact}
\begin{example}
	Let $f: \complex \to \complex$ so that $x +iy \mapsto \sin x \cosh y + i \cos x \sinh y$. It follows that $f$ is analytic.
\end{example}
\begin{proof}
This gives that $u_x = \cos x \cosh y$, $v_y = \cos x \cosh y$ so $u_x = v_y$. Finally $v_x = -\sin x \sinh y = v_y$. These derivatives are also continuous so by Fact 2, the derivatives exist.
\end{proof}
\noindent \textbf{Remark} The hyperbolic trigonometric functions are very similar to the standard trigonometric functions.
	\begin{equation*}
		\begin{aligned}
			\sinh(x) = \frac{e^x - e^{-x}}{2}& 	&\sin(x) = \frac{e^{ix} - e^{-ix}}{2i}\\
			\cosh(x) = \frac{e^x + e^{-x}}{2}& 	&\sin(x) = \frac{e^{ix} + e^{-ix}}{2}\\
		\end{aligned}
	\end{equation*}
	We could apply this fact to the previous example and we would yield 
	\begin{equation*}
		f(z) = -\frac{i}{2}e^{ix -y} + \frac{i}{2} e^{-ix + y}
	\end{equation*}
	which gives $D_{\overline{z}}  = 0$ as an exercise to the reader.
\begin{definition}
	A function $u(x,y)$ is called harmonic in $D \subset \mathbb{R}^2$ if it satisfies Laplace's equation
	\begin{equation*}
		\frac{\partial^2 u}{\partial x^2} + \frac{\partial^2 u}{\partial y^2}
	\end{equation*}
\end{definition}
\begin{theorem}
	If $f(z) = u(x,y) +i(x,y)$ is analytic on $D \subset \complex$ then 
	$u(x,y)$ and $v(x,y)$ are harmonic functions.
\end{theorem}
\begin{proof}
	Since $f$ is analytic it follows that $u_x = v_y$, and $u_y = -v_x$. Additionally  $u_xx = v_yx$ and $u_yy = -v_xy$ so $u_xx + u_yy = 0$. The same holds for $v$.
\end{proof}
\noindent \textbf{Remark}. We haven't actually proven that if $f$ is analytic (also referred to as \emph{holomorphic}) then it is infinitely differentiable. This fact is not obvious, since $f$ holomorphic if and only if it is once differentiable on its domain and its partial derivatives are continuous. There is an intrinsic property of the complex numbers which lets us make the following statement, roughly
\begin{equation*}
	f'(z)\; \;\; \mathrm{ exists} \implies f^{(n)}(z)\; \;\; \mathrm{ exists }\;\;\;\forall n.
\end{equation*}
This should be surprising. The reader would gain insight to prove the following fact.
\begin{lemma}
	 For every analyitic $f: \complex \to \complex$, $f'(iz) = if'(z);$ that is differentiation preserves--or \emph{commutes with}--rotation.  	
\end{lemma}
\noindent Then using the above lemma prove that analytic $f$ are infinitely differentiable**. \emph{Two stars for difficulty.}
\begin{fact}
	If $f$ analytic, the level curves of $u(x,y) = c = v(x,y)$ interesct and at each intersection they are orthogonal.
\end{fact}
%%%%%%% Be sure to set the counter and use menumerate

\end{document}	