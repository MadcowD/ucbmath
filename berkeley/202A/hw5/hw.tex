\documentclass[11pt]{amsart}

\usepackage{amsmath,amsthm}
\usepackage{amssymb}
\usepackage{graphicx}
\usepackage{enumerate}
\usepackage{fullpage}
% \usepackage{euscript}
% \makeatletter
% \nopagenumbers
\usepackage{verbatim}
\usepackage{color}
\usepackage{hyperref}
\usepackage{tikz-cd}
%\usepackage{times} %, mathtime}

\textheight=600pt %574pt
\textwidth=480pt %432pt
\oddsidemargin=15pt %18.88pt
\evensidemargin=18.88pt
\topmargin=10pt %14.21pt

\parskip=1pt %2pt

% define theorem environments
\newtheorem{theorem}{Theorem}    %[section]
%\def\thetheorem{\unskip}
\newtheorem{proposition}[theorem]{Proposition}
%\def\theproposition{\unskip}
\newtheorem{conjecture}[theorem]{Conjecture}
\def\theconjecture{\unskip}
\newtheorem{corollary}[theorem]{Corollary}
\newtheorem{lemma}[theorem]{Lemma}
\newtheorem{sublemma}[theorem]{Sublemma}
\newtheorem{fact}[theorem]{Fact}
\newtheorem{observation}[theorem]{Observation}
%\def\thelemma{\unskip}
\theoremstyle{definition}
\newtheorem{definition}{Definition}
%\def\thedefinition{\unskip}
\newtheorem{notation}[definition]{Notation}
\newtheorem{remark}[definition]{Remark}
% \def\theremark{\unskip}
\newtheorem{question}[definition]{Question}
\newtheorem{questions}[definition]{Questions}
%\def\thequestion{\unskip}
\newtheorem{example}[definition]{Example}
%\def\theexample{\unskip}
\newtheorem{problem}[definition]{Problem}
\newtheorem{exercise}[definition]{Exercise}

\numberwithin{theorem}{section}
\numberwithin{definition}{section}
\numberwithin{equation}{section}

\def\reals{{\mathbb R}}
\def\torus{{\mathbb T}}
\def\integers{{\mathbb Z}}
\def\rationals{{\mathbb Q}}
\def\naturals{{\mathbb N}}
\def\complex{{\mathbb C}\/}
\def\distance{\operatorname{distance}\,}
\def\support{\operatorname{support}\,}
\def\dist{\operatorname{dist}\,}
\def\Span{\operatorname{span}\,}
\def\degree{\operatorname{degree}\,}
\def\kernel{\operatorname{kernel}\,}
\def\dim{\operatorname{dim}\,}
\def\codim{\operatorname{codim}}
\def\trace{\operatorname{trace\,}}
\def\dimension{\operatorname{dimension}\,}
\def\codimension{\operatorname{codimension}\,}
\def\nullspace{\scriptk}
\def\kernel{\operatorname{Ker}}
\def\p{\partial}
\def\Re{\operatorname{Re\,} }
\def\Im{\operatorname{Im\,} }
\def\ov{\overline}
\def\eps{\varepsilon}
\def\lt{L^2}
\def\curl{\operatorname{curl}}
\def\divergence{\operatorname{div}}
\newcommand{\norm}[1]{ \|  #1 \|}
\def\expect{\mathbb E}
\def\bull{$\bullet$\ }
\def\det{\operatorname{det}}
\def\Det{\operatorname{Det}}
\def\rank{\mathbf r}
\def\diameter{\operatorname{diameter}}

\def\t2{\tfrac12}

\newcommand{\abr}[1]{ \langle  #1 \rangle}

\def\newbull{\medskip\noindent $\bullet$\ }
\def\field{{\mathbb F}}
\def\cc{C_c}



% \renewcommand\forall{\ \forall\,}

% \newcommand{\Norm}[1]{ \left\|  #1 \right\| }
\newcommand{\Norm}[1]{ \Big\|  #1 \Big\| }
\newcommand{\set}[1]{ \left\{ #1 \right\} }
%\newcommand{\ifof}{\Leftrightarrow}
\def\one{{\mathbf 1}}
\newcommand{\modulo}[2]{[#1]_{#2}}

\def\bd{\operatorname{bd}\,}
\def\cl{\text{cl}}
\def\nobull{\noindent$\bullet$\ }

\def\scriptf{{\mathcal F}}
\def\scriptq{{\mathcal Q}}
\def\scriptg{{\mathcal G}}
\def\scriptm{{\mathcal M}}
\def\scriptb{{\mathcal B}}
\def\scriptc{{\mathcal C}}
\def\scriptt{{\mathcal T}}
\def\scripti{{\mathcal I}}
\def\scripte{{\mathcal E}}
\def\scriptv{{\mathcal V}}
\def\scriptw{{\mathcal W}}
\def\scriptu{{\mathcal U}}
\def\scriptS{{\mathcal S}}
\def\scripta{{\mathcal A}}
\def\scriptr{{\mathcal R}}
\def\scripto{{\mathcal O}}
\def\scripth{{\mathcal H}}
\def\scriptd{{\mathcal D}}
\def\scriptl{{\mathcal L}}
\def\scriptn{{\mathcal N}}
\def\scriptp{{\mathcal P}}
\def\scriptk{{\mathcal K}}
\def\scriptP{{\mathcal P}}
\def\scriptj{{\mathcal J}}
\def\scriptz{{\mathcal Z}}
\def\scripts{{\mathcal S}}
\def\scriptx{{\mathcal X}}
\def\scripty{{\mathcal Y}}
\def\frakv{{\mathfrak V}}
\def\frakG{{\mathfrak G}}
\def\aff{\operatorname{Aff}}
\def\frakB{{\mathfrak B}}
\def\frakC{{\mathfrak C}}

\def\symdif{\,\Delta\,}
\def\mustar{\mu^*}
\def\muplus{\mu^+}

\def\soln{\noindent {\bf Solution.}\ }


%\pagestyle{empty}
%\setlength{\parindent}{0pt}

\begin{document}

\begin{center}{\bf Math 202A--- UCB, Fall 2016 --- M.~Christ}
\\
{\bf Problem Set 5, due Wednesday September 28 - William Guss}
\end{center}
\medskip \noindent {\bf (5.1)}\ Let $f_n \in L^{+}$, $f_n \to f$ pointwise, and $\int f = \lim_{n \to \infty} \int f_n < \infty$. Then $\lim_{n \to \infty} \int_A f_n = \int_A f$ for every $A \in \scriptm$.
\begin{proof}
  First we show that if $g \in L^{+}$ then $\chi_{A} g \in L^{+}$ if $A$ is measurable and also that $\int_{A} g = \int \chi_{A}g$. First $\chi_A$ is measurable by $A$ measurable and the product of pointwise functions is measurable. 
  Indeed $\chi_A g \leq g$ so through the suprememum of simple functions there approximating $ \int \chi_A g \leq \int g$, and $\chi_A g \in L^+$. Lastly $$\int \chi_A g = \sup_{\phi\; simple\; \phi \leq g}\int \phi \chi_A = \sup_{\phi\; simple\; \phi \leq g} \int_{A} \phi = \int_A g$$
  by proposition 2.13.

  Now observe that $\chi_A f_n \to \chi_A f$ pointwise, since on $A$, $f_n \to f$ by our hypothesis and off $A$, $0 \to 0$. Now for every $\epsilon$ 2.13 also gives $|\chi_A f - \chi_A f_n| \leq |f - f_n| $ for every $n > N$ implies that $\left| \int_A f_n - \int_A f \right | \leq \left|\int f - \int f_n\right| < \epsilon  $. This completes the proof.

\end{proof}
\textbf{Notation.} I wil use $\phi$ to denote a simple function, in the proceeding examples. \\
\medskip \noindent {\bf (5.2)}\ Let $(X, \scriptm, \mu)$ be a measure space and let $f \in L^+$. It follows that $\lambda(E) = \int_E f\ d\mu$ defines a measure on $\scriptm$ and if $g \in L^+$, $\int g\; d\lambda = \int fg\ d\mu$.
\begin{proof}
  First recall that for any $\phi$ simple the map $E \mapsto \int_E \phi\ d\mu$ is a measure on $\scriptm.$ Then recall that since $f \in L^+$, $\int f\ d\mu = \sup_{\phi \leq f} \int \phi\ d\mu.$ Take a sequence of increasing $\phi$ so that $\lim \int \phi_n\ d\mu = \int f\ d\mu.$ Then by our previous exercise $\int_E \phi_n\ d\mu \to \int_E\ fd\mu$. It follows that the measures $\lambda_n(E) = \int_E \phi_n\ d\mu \to \lambda(E) = \int_E \phi\ d\mu$.

  We claim that $\lambda$ is a measure. First $\lambda(\emptyset) = 0$ since $\lambda_n(\emptyset) = 0$ for all $n$. So $\lambda$ is nonnegative. Furthermore $\lambda_n(E) \leq \lambda_n(F)$ if $E \subset F$ for every $n$ so $\lambda(E) \leq \lambda(F)$. Finally for $(E_k)$ pairwise disjoint family of $\scriptm$
  \begin{equation*}
     \lim_{n\to\infty} \lim_{m \to \infty} \sum_{k=1}^m \lambda_n\left(E_k\right) = \sup_{n}\sup_m \sum_{k=1}^m \lambda_n\left(E_k\right)
  \end{equation*}
  since $\lambda_n$ is nondecreasing and nonnegative. Therefore $$\sup_{n}\sup_m \sum_{k=1}^m \lambda_n\left(E_k\right) = \sup_{m}\sup_n \lambda_n\left(\bigcup_{k=1}^m E_k\right)  = \lim_{m\to\infty} \lim_{n \to \infty} \lambda_n\left(\bigcup_{k=1}^m E_k\right),$$ 
  again applying $\lambda_n$ non decreasing and $\lambda_n$ countably addative. This gives
  \begin{equation*}
      \lambda\left(\bigcup_{k=1}^\infty E_k\right) = \sum_{k=1}^\infty \lambda\left(E_k\right)
  \end{equation*}
  and so $\lambda$ is a measure. Now applying the definition of simple functions we get \begin{equation*}
    \int f\ d\lambda = \sup_{\phi \leq g} \sum_y y \lambda({J_y}) = \sup_{\phi \leq g} \sum_y y \int_{J_y} g\  d\mu = \sup_{\phi\leq g} \int \sum_y\chi_{J_y}g\ d\mu = \sup_{\phi \leq g} \int \phi g\ d\mu
  \end{equation*}
  and thus $\int f\ d\lambda = \int fg\ d\mu.$

\end{proof}

\medskip \noindent {\bf (5.3)}\ Let $f_n$, $g_n$, $f, g \in L^1$. Suppose that $f_n \to f$  a.e. and $g_n \to g$ a.e. Suppose that $|f_n| \leq g_n$, and that $\int g_n \to \int g$. Show that $\int f_n \to \int f$. 
\begin{proof}
  The function $f$ is measurable and in $L^1.$ Without loss of generality we assume that $f_n, g_n, f, g$ are real valued since the proof is the same for real and imaginary parts. Now define the following sequence $\mathfrak{g}_n = \sup_{k} g_k$ Then $|f_n| \leq \mathfrak{g}_n$ and $\mathfrak{g}_n+f_n \geq 0$ a.e. and $g_n - f_n \geq 0$ a.e. 

   Thus by Fatou's lemma,
  \begin{equation*}
    \begin{aligned}
      \int g+ \int f \leq \int \liminf \mathfrak{g}_n + f_n \leq \lim \inf \left( \int \mathfrak{g}_n + \int f_n\right) &= \int g + \lim \inf \int f_n, \\
       \int g- \int f \leq \int \liminf \mathfrak{g}_n - f_n \leq \lim \inf \left( \int \mathfrak{g}_n - \int f_n\right) &= \int g - \lim \sup \int f_n, 
    \end{aligned}
  \end{equation*}
  Therefore $\lim\inf \int f_n \geq \int f \geq \lim \sup \int f_n$ and the result follows.  
\end{proof}

\medskip \noindent {\bf (5.4)}\ (Folland problem 2.22) Let X = $\mathbb{N}$, let $\scriptm = P(\mathbb{N})$, and let $\mu$ be counting measure on $\mathbb{N}$Interpret Fatou's Lemma, the MCT, and the DCT for this measure space $(X, \scriptm, \mu)$ as statements about infinite series. 

In this case, $\int f\ d\mu = \sum f(n)\ \mu({n}) = \sum f(n).$ Then Fatou's lemma says that $\sum \lim \inf f_k(n) \leq \lim \inf \sum f_k(n)$. Let us consider $f_k$ so that $f_k(n) = a_n$ if $n \leq k$ and $0$ after, so that $\sum \lim \inf f(k) = \sum \lim_j \inf_{k\geq j} f_k(n) = \sum a_n \leq \lim_j \inf_{k \geq j} \sum f_k(n)$. Using $\sum f_k(n) = \sum_{n=1}^k a_n$ we have $\sum a_n \leq \lim_j \inf_{k\geq j} \sum_{n=1}^k a_n = \lim_j \sum_{n=1}^j a_n =\sum a_n.$ So for functions of this form the equality holds. Now for any sequence $b_n$ we have that $\sum \lim \inf b_n \leq \lim \inf \sum b_n$ which says that if $\lim b_n \to a > 0$ then $a \mu(X) \leq \sum b_n$.
Therefore if $\sum b_n \neq\infty$ then $a = 0$ so that $b_n \to 0$.

\textbf{Fatou's lemmma gives us the Convergence $\implies $ the limit of the series tends to $0$.}

\medskip \noindent {\bf (5.5)}\ Suppose $f_n \in L^1(\mu)$ and $f_n \to f$ uniformly.


(a) Show that if $\mu(X) < \infty$ then $\int f_n \to \int f.$
\begin{proof}
  If $f_n \to f$ uniformly then for every $\epsilon$ there is a $N$ so that for all $n>N$  $\sup_x d(f_n,f) < \epsilon$. Suppose that $f_n$ and $f$ are both finite on the domain, since for the infinite points, convergence could only be uniform ifd $f_n(x) = f(x) = \infty$ for all $n$ unless $\epsilon = \infty.$ WE can handle that case since $\int f = \infty$ implies that there is a set with postive measure along which $f$ is infinite and so the same goes for all $f_n$ and thus $\int f_n = \infty \to \int f = \infty.$


  For the finite case, then $f_n$ is bounded by $M = \sup_n \sup_x \sup\{|f_n|, |f|\}$ for every $x$ on the domain by uniformity and so $\mu(X)M = \int M < \infty$ so by the dominated convergence theorm $\int f_n \to \int f.$
\end{proof}

\end{document}\end

