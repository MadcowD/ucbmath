%%%%%%%%%%%%%%%%%%%%%%%%%%%%%%%%%%%%%%%%%%%%%%%%%%%%%%%%%%%%%%%%%%
%%%                      Homework _                            %%%
%%%%%%%%%%%%%%%%%%%%%%%%%%%%%%%%%%%%%%%%%%%%%%%%%%%%%%%%%%%%%%%%%%

\documentclass[letter]{article}

\usepackage{lipsum}
\usepackage[pdftex]{graphicx}
\usepackage[margin=1.5in]{geometry}
\usepackage[english]{babel}
\usepackage{listings}
\usepackage{amsthm}
\usepackage{amssymb}
\usepackage{framed} 
\usepackage{amsmath}
\usepackage{titling}



\newtheorem{theorem}{Theorem}
\newtheorem{lemma}{Lemma}
\newtheorem{fact}{Fact}
\newtheorem{example}{Example}
\newtheorem{definition}{Definition}
\newtheorem{proposition}{Proposition}

\newenvironment{menumerate}{%
  \edef\backupindent{\the\parindent}%
  \enumerate%
  \setlength{\parindent}{\backupindent}%
}{\endenumerate}



\textheight=600pt %574pt
\textwidth=480pt %432pt
\oddsidemargin=15pt %18.88pt
\evensidemargin=15pt
\topmargin=10pt %14.21pt

\parskip=1pt %2pt

\def\reals{{\mathbb R}}
\def\torus{{\mathbb T}}
\def\integers{{\mathbb Z}}
\def\rationals{{\mathbb Q}}
\def\naturals{{\mathbb N}}
\def\complex{{\mathbb C}\/}
\def\distance{\operatorname{distance}\,}
\def\support{\operatorname{support}\,}
\def\dist{\operatorname{dist}\,}
\def\Span{\operatorname{span}\,}
\def\degree{\operatorname{degree}\,}
\def\kernel{\operatorname{kernel}\,}
\def\dim{\operatorname{dim}\,}
\def\codim{\operatorname{codim}}
\def\trace{\operatorname{trace\,}}
\def\dimension{\operatorname{dimension}\,}
\def\codimension{\operatorname{codimension}\,}
\def\kernel{\operatorname{Ker}}
\def\Re{\operatorname{Re\,} }
\def\Im{\operatorname{Im\,} }
\def\eps{\varepsilon}
\def\lt{L^2}
\def\bull{$\bullet$\ }
\def\det{\operatorname{det}}
\def\Det{\operatorname{Det}}
\def\diameter{\operatorname{diameter}}
\def\symdif{\,\Delta\,}
\newcommand{\norm}[1]{ \|  #1 \|}
\newcommand{\set}[1]{ \left\{ #1 \right\} }
\def\one{{\mathbf 1}}
\def\cl{\text{cl}}

\def\newbull{\medskip\noindent $\bullet$\ }
\def\nobull{\noindent$\bullet$\ }



\def\scriptf{{\mathcal F}}
\def\scriptq{{\mathcal Q}}
\def\scriptg{{\mathcal G}}
\def\scriptm{{\mathcal M}}
\def\scriptb{{\mathcal B}}
\def\scriptc{{\mathcal C}}
\def\scriptt{{\mathcal T}}
\def\scripti{{\mathcal I}}
\def\scripte{{\mathcal E}}
\def\scriptv{{\mathcal V}}
\def\scriptw{{\mathcal W}}
\def\scriptu{{\mathcal U}}
\def\scriptS{{\mathcal S}}
\def\scripta{{\mathcal A}}
\def\scriptr{{\mathcal R}}
\def\scripto{{\mathcal O}}
\def\scripth{{\mathcal H}}
\def\scriptd{{\mathcal D}}
\def\scriptl{{\mathcal L}}
\def\scriptn{{\mathcal N}}
\def\scriptp{{\mathcal P}}
\def\scriptk{{\mathcal K}}
\def\scriptP{{\mathcal P}}
\def\scriptj{{\mathcal J}}
\def\scriptz{{\mathcal Z}}
\def\scripts{{\mathcal S}}
\def\scriptx{{\mathcal X}}
\def\scripty{{\mathcal Y}}
\def\frakv{{\mathfrak V}}
\def\frakG{{\mathfrak G}}
\def\frakB{{\mathfrak B}}
\def\frakC{{\mathfrak C}}




%%%%%%%%%%%%%%%
%% DOC INFO %%%
%%%%%%%%%%%%%%%
\newcommand{\bHWN}{ }
\newcommand{\bCLASS}{MATH 202A}

\title{\bCLASS: Notes }
\author{Scribe: William Guss}
\usepackage{csquotes}

%%%%%%%%%%%%%%

\begin{document}
\maketitle
\begin{definition}
	A topology on $X$ is $\tau \subset P(X)$ that contains $\phi, X$ and
	is closed with respect to arbitrary unions and finite intersection; 
	that is closed under an intersection of finitely many sets in $\tau.$
\end{definition}

\begin{example}\label{good}
	Take any metric space $(X ,\rho)$ and let $\tau$ be the collection of all $\rho$-open subsets of $X$, 
	then $\rho$ is a topology. 
\end{example}
 Example \ref{good} is a very good example. We'll generalize this example by removing the notion of distance but keeping the notion of infintesmally close.

 \begin{example}[Naive]\label{stupid}
 Take any $X$ and let $\tau =\{\phi, X\}$, this is a topology. Take any $X$ and let $\tau = P(X).$
 \end{example}
Example \ref{stupid} is stupid.

\begin{definition}
	A set $E \subset X$ is closed if $X \setminus E$ is open.
\end{definition}
\begin{definition}
	The interior of a set $E \subset X$ is the largest open set $O \in \tau$ contained in $E$;
	that is $int(E) = \{x \in E\ :\ \exists V^{open}\ : x \in V \subset E\}$.
\end{definition}
\begin{definition}
	The closure of $E$ is the smallest closed set containing $E$; that is,
	$cl(E) = X \setminus \int(X \setminus E).$
\end{definition}
If $E_\alpha$ is closed then $\cap E_\alpha $ is closed by DeMorgans law.
\begin{definition}
	The boundary of $E$, $\partial E = \cl(E) \setminus \int(E).$
\end{definition}
\begin{example}
	Let $X = \mathbb{R}^2$ with the usual metric. Let $E = \{x = (x_1, x_2): x_1 \geq 0 \text{ and } |x| \leq 1 \text{ or } x_1 < 0 \text{ and } |x| < 1 \}$. Then $\partial E = S^1, int(E) = B^{open}_1(0), cl(B) = B_1(0).$
\end{example}
\begin{definition}
	A set $E \subset X$ is dense if the closure of $E = X.$
\end{definition}
\begin{definition}
	A point $x \in X$ is an accumulation point of $E \subset X$ if any openset $V \ni x$
	intersects $E \setminus \{x\}.$ We denote $acc(E)$ as the set of all accumulation points of $E$.
\end{definition}
\begin{example}
 If $E = \mathbb{Z} \subset \mathbb{R}$, then there are no accumulation points at all; hence, $acc(\mathbb{Z}) = \emptyset.$
\end{example}
\begin{proposition}
	For any $E \subset X$, $cl(E) = E \cup acc(E).$
\end{proposition}
\begin{proof}
	We would like to show that $X \setminus (E \cup acc(E))$ is open.
	Suppose that $y \notin E,$ and $y \notin acc(E)$. There is an openset which
	contains why such that $V \cap E \setminus \{y\} = \emptyset.$ Therefore $V \cap E = \emptyset = \emptyset$.
	The union of all such $V$ is the compliment of $E \cup acc(E).$ Thus is open. Therefore $E \cup acc(E)$ is closed.

	Consider any closed set $A$ containig $E$. Claim $A \supset acc(E).$ If $y \in acc(E)$ and $y \notin A$,
	then $y \in X \setminus A = V^{open}.$ Since $V^{open} \cap E = \emptyset$ because $V = X \setminus A \subset X \setminus E.$ Therefore $y \notin acc(E).$ And thats a contradiction. :)

	On the one hand, $E \cup acc(E)$ is closed  and contains $E$ and is contained in any other closed
	set which contains $E$ and thus it is the smallest closed set containg $E$ or equivalently it is the closure of $E.$
\end{proof}

The following are examples of topological spaces that are not metric spaces.
\begin{example}[Simple]
	Take $X = \{a \neq b\}.$ Take the topology to be $\tau = \{\emptyset, X, \{a\}\}.$ It is obviously closed under intersections and unions. We claim that this topology is not \emph{compatible} with any metric space structure.
\end{example}
\begin{proof}
	If $X$ is a metric space, then $\rho(a,b) = r$. Then $B(a, r/2) \cap B(b, r/2) = \emptyset$. any open set in $\tau$ which contains $b$ is $X$ itself and must contain $A$ and so must intersect the openset containing $a$. BUT THIS CANNOT BE! 
\end{proof}
So toplogy is more general than metric spaces.
\begin{example}[Zariski Toplogy]
Let $X = \mathbb{C}^n$, then $V \subset \mathbb{C}^N$ is open if $V$ is the union of finite intersections of sets of form $\{z\ :\ P(z)\ \neq 0 \}$ where $P$ is any polynomial. For example when $n =1$, there are finiteley many $0s$ and so for any finite collection of points we can construct a polynomial.
\end{example}
\emph{Disatisfied.}
\begin{example}
	Set $X$ of all functions $f: \mathbb{R} \to \{0,1\}$. Consider all sets $V \subset X$ of this form:
	Choose $S \subset \mathbb{R}$ to be a finite set. (Mark finiteley many points on the axis.) For each element of the set
	choose $t_s$ to be a $0$ or a $1.$ Then $V =\{ f: \mathbb{R} \to \{0,1\}:f(s) = t_s \forall s \in S\}.$
	The topology is the set of all sets that can be written as unions of sets $V$ of this type.
\end{example}
\begin{definition}
	If $\scripte \subset P(X)$, the toplogy generated bu $\scripte$ is the smallest
	topology which containes $\scripte$. That is if $E \in \scripte$ then $E \in \tau$. This is the same
	as the collection of all unions of finite intersections of elements of $\scripte$.
\end{definition}
\begin{proposition}
	Any intersection of two (induct for finite) ... is equal to a union of finite intersections of elements of $\scripte.$ Equivalently, for any index set $A$ and associated finite indexsets $A_\alpha$
	\begin{equation*}
	\left(\bigcup_{a\in A} \bigcap_{j \in A_\alpha} V_{\alpha, j} \in \scripte\right) \cap \left( \bigcup{\beta \in B}) \bigcap_{k \in B_\beta} W_{\beta, k} \in \scripte\right) = \bigcup_{(\alpha, \beta) \in A \times B} \left(\bigcap_{j\in A_\alpha, k \in B_\beta} V_{j,\alpha} \cap W_{k, \beta}\right).
	\end{equation*}
	DANK.
\end{proposition}

%%%%%%% Be sure to set the counter and use menumerate

\end{document}	