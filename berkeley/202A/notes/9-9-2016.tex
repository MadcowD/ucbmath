

\documentclass[letter]{article}

\usepackage{lipsum}
\usepackage[pdftex]{graphicx}
\usepackage[margin=1.5in]{geometry}
\usepackage[english]{babel}
\usepackage{listings}
\usepackage{amsthm}
\usepackage{amssymb}
\usepackage{framed} 
\usepackage{amsmath}
\usepackage{titling}




\newtheorem{theorem}{Theorem}
\newtheorem{lemma}{Lemma}
\newtheorem{fact}{Fact}
\newtheorem{example}{Example}
\newtheorem{definition}{Definition}
\newtheorem{proposition}{Proposition}

\newenvironment{menumerate}{%
  \edef\backupindent{\the\parindent}%
  \enumerate%
  \setlength{\parindent}{\backupindent}%
}{\endenumerate}



\textheight=600pt %574pt
\textwidth=480pt %432pt
\oddsidemargin=15pt %18.88pt
\evensidemargin=15pt
\topmargin=10pt %14.21pt

\parskip=1pt %2pt

\def\reals{{\mathbb R}}
\def\torus{{\mathbb T}}
\def\integers{{\mathbb Z}}
\def\rationals{{\mathbb Q}}
\def\naturals{{\mathbb N}}
\def\complex{{\mathbb C}\/}
\def\distance{\operatorname{distance}\,}
\def\support{\operatorname{support}\,}
\def\dist{\operatorname{dist}\,}
\def\Span{\operatorname{span}\,}
\def\degree{\operatorname{degree}\,}
\def\kernel{\operatorname{kernel}\,}
\def\dim{\operatorname{dim}\,}
\def\codim{\operatorname{codim}}
\def\trace{\operatorname{trace\,}}
\def\dimension{\operatorname{dimension}\,}
\def\codimension{\operatorname{codimension}\,}
\def\kernel{\operatorname{Ker}}
\def\Re{\operatorname{Re\,} }
\def\Im{\operatorname{Im\,} }
\def\eps{\varepsilon}
\def\lt{L^2}
\def\bull{$\bullet$\ }
\def\det{\operatorname{det}}
\def\Det{\operatorname{Det}}
\def\diameter{\operatorname{diameter}}
\def\symdif{\,\Delta\,}
\newcommand{\norm}[1]{ \|  #1 \|}
\newcommand{\set}[1]{ \left\{ #1 \right\} }
\def\one{{\mathbf 1}}
\def\cl{\text{cl}}

\def\newbull{\medskip\noindent $\bullet$\ }
\def\nobull{\noindent$\bullet$\ }



\def\scriptf{{\mathcal F}}
\def\scriptq{{\mathcal Q}}
\def\scriptg{{\mathcal G}}
\def\scriptm{{\mathcal M}}
\def\scriptb{{\mathcal B}}
\def\scriptc{{\mathcal C}}
\def\scriptt{{\mathcal T}}
\def\scripti{{\mathcal I}}
\def\scripte{{\mathcal E}}
\def\scriptv{{\mathcal V}}
\def\scriptw{{\mathcal W}}
\def\scriptu{{\mathcal U}}
\def\scriptS{{\mathcal S}}
\def\scripta{{\mathcal A}}
\def\scriptr{{\mathcal R}}
\def\scripto{{\mathcal O}}
\def\scripth{{\mathcal H}}
\def\scriptd{{\mathcal D}}
\def\scriptl{{\mathcal L}}
\def\scriptn{{\mathcal N}}
\def\scriptp{{\mathcal P}}
\def\scriptk{{\mathcal K}}
\def\scriptP{{\mathcal P}}
\def\scriptj{{\mathcal J}}
\def\scriptz{{\mathcal Z}}
\def\scripts{{\mathcal S}}
\def\scriptx{{\mathcal X}}
\def\scripty{{\mathcal Y}}
\def\frakv{{\mathfrak V}}
\def\frakG{{\mathfrak G}}
\def\frakB{{\mathfrak B}}
\def\frakC{{\mathfrak C}}




%%%%%%%%%%%%%%%
%% DOC INFO %%%
%%%%%%%%%%%%%%%
\newcommand{\bHWN}{ }
\newcommand{\bCLASS}{MATH 202A}

\title{\bCLASS: Notes }
\author{Scribe: William Guss}
\usepackage{csquotes}

%%%%%%%%%%%%%%

\begin{document}
\maketitle
\thispagestyle{empty}
\noindent \textbf{Remark.} In mathematics, the key discovery is usually a definition!

\begin{definition}
	Let $\mu^*: \scriptp(X) \to [0,\infty]$ be an outer measure.
\end{definition}

\begin{definition}
	$A \subset X$ is $\mu^*$-measurable, if for all $E \subset X,$
	\begin{equation*}
		\mu^*(E) = \mu^*(E \cap A) + \mu^*(E \setminus A)
	\end{equation*}
\end{definition}

\begin{theorem}
	$\scriptm$ is a $\sigma$-algebra and $\mu^*|_\scriptm$ is a measure on $\scriptm$.
\end{theorem}

\noindent \textbf{Remark.} You could easily make a $\sigma$-algebra on $X$ by letting $\scriptm = \set{\emptyset, X}, \mu(\emptyset) = 0$ and $\mu(X) = \mu^*(X)$.
\begin{proof}
	Last time we prove that $\scriptm$ was an algebra using Venn-Diagrams. We also proved if $A, B \in \scriptm$ and $A \cap B = \emptyset$ then $\mu^*(A \cup B) = \mu^*(A) + \mu^*(B).$

	Now suppose that $A_n \in \scriptm$ is Cartheodory measurable for all $n \in \mathbb{N}$. We wish to show $A = \bigcup_n A_n \in \scriptm$. Define $B_1 = A_1$, and $B_n = A_n \setminus \bigcup_{k=1}^{n-1}.$ We have that $B_k \in \scriptm$ and they are pairwise disjoint. So
	$A = \bigsqcup_k B_k$. Now consider
	\begin{equation*}
		\begin{aligned}
			\mu^*(E) &\geq \mu^*\left(E \cap \left(\bigcup_{k=1}^n B_k\right)\right) + \mu^*\left((E \setminus \bigcup_{k=1}^n B_k)\right) \\
			&\geq 	\mu^*\left(E \cap \left(\bigcup_{k=1}^n B_k\right)\right) + \mu^*(E \setminus A).
		\end{aligned}
	\end{equation*}
	Recall that  $E \subset X$ and $A,B \in \scriptm$ and $A \cap B = \emptyset$ then $$\mu^*(A \cup B) \cap E) = \mu^*(A \cap E) + \mu^*(B \cap E).$$
	So for every $n$
	\begin{equation*}
		\mu^*(E) \geq 
		\sum_{k=1}^n \mu^*(E \cap B_k) + \mu^*(E \setminus A).
	\end{equation*} 
	The series is bounded and therefore as $n \to \infty$
	\begin{equation*}
		\mu^*(E) \geq 
		\sum_{k=1}^\infty \mu^*(E \cap B_k) + \mu^*(E \setminus A).
	\end{equation*}
	With $E \cap A \subset \bigcup_{k} E \cap B_k$ so $\mu^*(E \cap A) \leq \sum_{k=1}^\infty \mu^*(E \cap B_k)$ by countable subaddativity. Therefore 
	\begin{equation*}
		\mu^*(E) \geq \mu^*(E \cap A) + \mu^*(E \setminus A)
	\end{equation*}
	and so $A$ is measurable; that is $A \in \scriptm$. Therefore $\scriptm$ is closed under countable unions.

	Now we want to show that $\mu^*|_\scriptm$. However this is implicit, by applying the above argument to $E = A$; that is, if $A = \bigcup_k B_k, B_k \in \scriptm$ and $B_k$ pairwise disjoint
	\begin{equation*}
		\mu^*(A) = \sum_{k=1}^\infty \mu^*(A \cap B_k) + \mu^*(E \setminus A) = \sum_{k=1}^\infty \mu^*( B_k) + \mu^*(\emptyset).
	\end{equation*}
	This completes the proof.
\end{proof}
\noindent \textbf{Remark.} The measure $\mu := \mu^*|_\scriptm$ is a complete measure.
\begin{proof}
	Suppose $A \in \scriptm$ and $\mu^*(A) = 0$ and $B \subset A$ not necisarrily measurable. Certainly $\mu^*(B) \leq \mu^*(A) = 0.$ The set $E \setminus B = E \setminus A \cup A \setminus B$ so $\mu^*(E \setminus B) \leq \mu^*( E \setminus A) + \mu^*( \cup A \setminus B)$ so we conclude 
	$\mu^*(E \setminus B) = \mu^*(E \setminus A)$. For any set $E$
	\begin{equation*}
		\mu^*(E) \geq \mu^*(E \cap A) + \mu^*(E \setminus A) \geq \mu^*(E \cap B) + \mu^*(E \setminus B)
	\end{equation*}
	by $A$ measurable! Therefore $B$ is measurable!
\end{proof}

\begin{definition}
	A pre measure $\mu_0$ on an algebra $\scripta$ is a $\mu_0: \scripta \to [0, \infty]$ such that $\mu_0(\emptyset) = 0$ and $\mu_0(A \cup B) = \mu_0(A) + \mu_0(B)$ if $A,B \in \scripta$ disjoint. Additionally
	\begin{equation*}
		\mu_0\left(\bigcup_n^\infty A_n\right) = \sum_{n}^\infty
	\end{equation*}
	iff $A_n \in \scripta$ and pairwise disjoint.
\end{definition}
\begin{theorem}
	Let $\mu_0$ be a premeasure on an algebra $\scripta$. Define $\mu^*$ from $\mu_0, \scripte = \scripta$, and use the same construction as before. Let $\scriptm, \mu = \mu^*|_\scriptm$ be the construction from Carthedory's theorem. Then every $\scripta \subset \scriptm$ and $\mu$ and $ \mu_0$ agree on $\scripta$.

	(Thm 3) In addition if $\mu_0$ is $\sigma$-finite then the extension of $\mu_0$ to $\scriptm$.
\end{theorem}
\begin{proof}
		Consider $\set{E_j \in \scriptm}$ so that $\bigcup E_j \supset E$ and $E_j \in \scripte = \scripta$. First
		\begin{equation*}
			\begin{aligned}
				\sum_j \mu_0(E_j) &= \sum_j \mu_0(E_j \cap A) + \mu_0(E_j \setminus A) \\
				&= \sum_j \mu_0(E_j \cap A) + \sum_j \mu_0(E_j \setminus A)
 			\end{aligned}
		\end{equation*}
		Furthermore $\set{E_J \cap A: j \in \mathbb{N}}$ covers $E \cap A$ and $\set{E_j \setminus A}$ covers $E \cap A$. Therefore 
		\begin{equation*}
			\sum_j \mu_0(E_j \cap A) + \sum_j \mu_0(E_j \setminus A) \geq \mu^*(E \cap A) + \mu^*(E \setminus A) 
		\end{equation*}
		Therefore for all covers of $E$ by $E_j \in \scripta$ it follows that
		\begin{equation*}
			\inf_{(E_J) \subset \scriptm}\mu_0\left(\sum_j \mu_0 (E_j)\right) = \mu^*(E) \geq 	\mu^*(E \cap A) + \mu^*(E \setminus A) 
		\end{equation*}


		Suppose that $\nu$ is a measure on $\scriptm$ and $\nu \equiv \mu_0$ on $\scripta$. We claim that $\nu(E) \leq \mu(E)$ for all $E \in \scriptm$. Consider any countable cover of $E$ by $E_j \in \scripta$. So $$\nu(E) \leq \sum)j \nu(E_j) = \sum_j \mu_0(E_j).$$ Take the $\inf$ over all such covers and 
		\begin{equation*}
			\nu(E) \leq \sum_j \nu(E_j) = \sum_j \mu(E_j).
		\end{equation*}
\end{proof}


%%%%%%% Be sure to set the counter and use menumerate

\end{document}	