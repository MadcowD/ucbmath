\documentclass[11pt]{amsart}

\usepackage{amsmath,amsthm}
\usepackage{amssymb}
\usepackage{graphicx}
\usepackage{enumerate}
\usepackage{fullpage}
 \usepackage{euscript}
% \makeatletter
% \nopagenumbers
\usepackage{verbatim}
\usepackage{color}
\usepackage{hyperref}

\usepackage{fullpage,tikz,float}
\usepackage{tikz-cd}
%\usepackage{times} %, mathtime}

\textheight=600pt %574pt
\textwidth=480pt %432pt
\oddsidemargin=15pt %18.88pt
\evensidemargin=18.88pt
\topmargin=10pt %14.21pt

\parskip=1pt %2pt

% define theorem environments
\newtheorem{theorem}{Theorem}    %[section]
%\def\thetheorem{\unskip}
\newtheorem{proposition}[theorem]{Proposition}
%\def\theproposition{\unskip}
\newtheorem{conjecture}[theorem]{Conjecture}
\def\theconjecture{\unskip}
\newtheorem{corollary}[theorem]{Corollary}
\newtheorem{lemma}[theorem]{Lemma}
\newtheorem{sublemma}[theorem]{Sublemma}
\newtheorem{fact}[theorem]{Fact}
\newtheorem{observation}[theorem]{Observation}
%\def\thelemma{\unskip}
\theoremstyle{definition}
\newtheorem{definition}{Definition}
%\def\thedefinition{\unskip}
\newtheorem{notation}[definition]{Notation}
\newtheorem{remark}[definition]{Remark}
% \def\theremark{\unskip}
\newtheorem{question}[definition]{Question}
\newtheorem{questions}[definition]{Questions}
%\def\thequestion{\unskip}
\newtheorem{example}[definition]{Example}
%\def\theexample{\unskip}
\newtheorem{problem}[definition]{Problem}
\newtheorem{exercise}[definition]{Exercise}


\def\reals{{\mathbb R}}
\def\torus{{\mathbb T}}
\def\integers{{\mathbb Z}}
\def\rationals{{\mathbb Q}}
\def\naturals{{\mathbb N}}
\def\complex{{\mathbb C}\/}
\def\distance{\operatorname{distance}\,}
\def\support{\operatorname{support}\,}
\def\dist{\operatorname{dist}\,}
\def\Span{\operatorname{span}\,}
\def\degree{\operatorname{degree}\,}
\def\kernel{\operatorname{kernel}\,}
\def\dim{\operatorname{dim}\,}
\def\codim{\operatorname{codim}}
\def\trace{\operatorname{trace\,}}
\def\dimension{\operatorname{dimension}\,}
\def\codimension{\operatorname{codimension}\,}
\def\nullspace{\scriptk}
\def\kernel{\operatorname{Ker}}
\def\p{\partial}
\def\Re{\operatorname{Re\,} }
\def\Im{\operatorname{Im\,} }
\def\ov{\overline}
\def\eps{\varepsilon}
\def\lt{L^2}
\def\curl{\operatorname{curl}}
\def\divergence{\operatorname{div}}
\newcommand{\norm}[1]{ \|  #1 \|}
\def\expect{\mathbb E}
\def\bull{$\bullet$\ }
\def\det{\operatorname{det}}
\def\Det{\operatorname{Det}}
\def\rank{\mathbf r}
\def\diameter{\operatorname{diameter}}

\def\t2{\tfrac12}

\newcommand{\abr}[1]{ \langle  #1 \rangle}

\def\newbull{\medskip\noindent $\bullet$\ }
\def\field{{\mathbb F}}
\def\cc{C_c}



% \renewcommand\forall{\ \forall\,}

% \newcommand{\Norm}[1]{ \left\|  #1 \right\| }
\newcommand{\Norm}[1]{ \Big\|  #1 \Big\| }
\newcommand{\set}[1]{ \left\{ #1 \right\} }
%\newcommand{\ifof}{\Leftrightarrow}
\def\one{{\mathbf 1}}
\newcommand{\modulo}[2]{[#1]_{#2}}

\def\bd{\operatorname{bd}\,}
\def\cl{\text{cl}}
\def\nobull{\noindent$\bullet$\ }

\def\scriptf{{\mathcal F}}
\def\scriptq{{\mathcal Q}}
\def\scriptg{{\mathcal G}}
\def\scriptm{{\mathcal M}}
\def\scriptb{{\mathcal B}}
\def\scriptc{{\mathcal C}}
\def\scriptt{{\mathcal T}}
\def\scripti{{\mathcal I}}
\def\scripte{{\mathcal E}}
\def\scriptv{{\mathcal V}}
\def\scriptw{{\mathcal W}}
\def\scriptu{{\mathcal U}}
\def\scriptS{{\mathcal S}}
\def\scripta{{\mathcal A}}
\def\scriptr{{\mathcal R}}
\def\scripto{{\mathcal O}}
\def\scripth{{\mathcal H}}
\def\scriptd{{\mathcal D}}
\def\scriptl{{\mathcal L}}
\def\scriptn{{\mathcal N}}
\def\scriptp{{\mathcal P}}
\def\scriptk{{\mathcal K}}
\def\scriptP{{\mathcal P}}
\def\scriptj{{\mathcal J}}
\def\scriptz{{\mathcal Z}}
\def\scripts{{\mathcal S}}
\def\scriptx{{\mathcal X}}
\def\scripty{{\mathcal Y}}
\def\frakv{{\mathfrak V}}
\def\frakG{{\mathfrak G}}
\def\aff{\operatorname{Aff}}
\def\frakB{{\mathfrak B}}
\def\frakC{{\mathfrak C}}

\def\symdif{\,\Delta\,}
\def\mustar{\mu^*}
\def\muplus{\mu^+}

\def\soln{\noindent {\bf Solution.}\ }


%\pagestyle{empty}
%\setlength{\parindent}{0pt}

\begin{document}

\begin{center}{\bf CS 202B --- UCB, Spring 2017 --- Hammond ---- Scribe: William Guss}
\\
{\bf Lecture Notes}
\end{center}

\textbf{Slices and measure through integration.}
Let  $(X, \scripta, \mu)$ and $(Y, \scriptb, \nu)$ be two measure spaces. For $E \in \scripta \times \scriptb$ we set $h(x) = \nu(S_x(E))$ and $k(y) = \mu(t_y(E)).$

\begin{proposition}
	For every $E \in \scripta \times \scriptb$, $h$ is $\scripta$-measurable and $k$ is $\scriptb$-measurable. Furthermore
	\begin{equation*}
		\int_X h(x)\ d\mu(x) = \int_Y k(y)\ d\nu(y)
 	\end{equation*}
\end{proposition}
\begin{proof}
	Suppose that $\mu$ and $\nu$ are finite. Let $\scriptc$ be the collection of elements of $\scripta\times \scriptb$ such that the proposition holds.

	\textbf{Aim.} We wish to show two things, that $\scriptc$ contains $\scriptc_0$ (the generating algebra for $\scripta \times \scriptb$) and that $\scriptc$ is a monotone class. Since $\scripta \times \scriptb = \sigma(\scriptc_0)$ we will learn from this aim that $\scriptc = \scripta \times \scriptb$ by the montone class lemma.

	\textbf{Sketch.} Let's first approach the first aim, that $\scriptc_0$ satisfies the the proposition. 


	If $E = A \times B$ with $A \in \scripta$ and $B \in \scriptb$. We need to check that $E \in \scriptc$. In this case $h(x) = \xi_B(x)\nu(B)$ and $k(y) = \xi_A(y) \nu(A).$ The indicator is certainly $\scripta$ measurable because $h^{-1}([a, \infty)) = A \in \scripta$ if $a \leq \nu(B)$ and $h^{-1}(\dots) = \emptyset$ otherwise. The same approach might be checked for $k$ as well. Finally $\int_X \chi_A(x)\nu(B)\ d\mu(x) = \mu(A)\nu(B) = \int_Y \xi_B(y)\mu(A)\ d\nu(y)$ by the definition of integration on indicator functions. 

	Now we should check for disjoint unions. Take $E = \bigcup_{i=1}^n E_i$ where $E_i$ are measurable rectangles and $E_i \cap E_j = \emptyset$ when $i \neq j$. First $S_x(E) = \bigcup S_x(E_i)$ since the $E_i$ are disjoint. Now we can verify that
	\begin{equation*}
		h(x) = \nu(S_x(E)) = \nu\left(\bigcup_{i=1}^n E_i\right) = \sum_{i=1}^n \nu(S_x(E_i)) 
	 \end{equation*} 
	 by finite additivity. Then we use that the sum of finitely many measurable functions is measurable to yield that $h$ is measurable for any element $E \in C_0.$ To verify the second claim we just use linearity:
	 \begin{equation*}
	 	\int_X h(x)\ d\mu(x) = \int_X \sum_{i=1}^n \nu(S_x(E_i))\ d\mu(x) = \sum_{i=1}^n \mu(A_i)\nu(B_i) = \int_Y \sum_{i=1}^n \mu(T_y(E_i))\ d\nu(y) = \int_Y k(y)\ d\nu(y).
	 \end{equation*}
	 Thus $\scriptc \supset \scriptc_0.$

	 Now we turn to the second aim of the proof, showing that $\scriptc$ is a monotone class.

	 First consider an increasing sequence $E_n \in \scriptc$ with $E_n \subset E_{n+1}$ and of course $E = \bigcup_{i=1}^\infty E_i$. We want to check that $E \in \scriptc$. Set $h_n(x) = \nu(S_x(E_n))$ and $k_n(y) = \mu(t_y(E_n)).$ We know that $h_n$ increases to $h$ and $k_n$ increases to $k$ monotonically. Additionally by $E_n \in \scriptc$ we know $h_n, k_n$ are both measurable in $\scripta$ and $\scriptb$ respectively. We now apply the monotone convergence theorem and so $h, k$ are measurable and the limit of integration thereof converges to integration of the limit. This covers equality, property two of the proposition.

	 On the otherhand, a decreaisng sequence $E_n \in \scriptc$ withb $E_n \supset E_{n+1}$ and $E = \bigcap_{i=1}^\infty E_i.$ In this case we will apply the dominated convergence theorem with $h_1, k_1$ clearly dominating their respective sequences. Using the finiteness of of $\nu, \mu$ the $h_1, k_1$ are bounded by $\mu(X), \nu(Y)$ respectively. the conclusion of the dominated convergence theorem shows the proposition.


	 Therefore $\scriptc$ is a monotone class and this completes the proof.
\end{proof}
\end{document}\end
	