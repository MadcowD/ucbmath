\documentclass[11pt]{amsart}

\usepackage{amsmath,amsthm}
\usepackage{amssymb}
\usepackage{graphicx}
\usepackage{enumerate}
\usepackage{fullpage}
% \usepackage{euscript}
% \makeatletter
% \nopagenumbers
\usepackage{verbatim}
\usepackage{color}
\usepackage{hyperref}

\usepackage{fullpage,tikz,float}
%\usepackage{times} %, mathtime}

\textheight=600pt %574pt
\textwidth=480pt %432pt
\oddsidemargin=15pt %18.88pt
\evensidemargin=18.88pt
\topmargin=10pt %14.21pt

\parskip=1pt %2pt

% define theorem environments
\newtheorem{theorem}{Theorem}    %[section]
%\def\thetheorem{\unskip}
\newtheorem{proposition}[theorem]{Proposition}
%\def\theproposition{\unskip}
\newtheorem{conjecture}[theorem]{Conjecture}
\def\theconjecture{\unskip}
\newtheorem{corollary}[theorem]{Corollary}
\newtheorem{lemma}[theorem]{Lemma}
\newtheorem{sublemma}[theorem]{Sublemma}
\newtheorem{fact}[theorem]{Fact}
\newtheorem{observation}[theorem]{Observation}
%\def\thelemma{\unskip}
\theoremstyle{definition}
\newtheorem{definition}{Definition}
%\def\thedefinition{\unskip}
\newtheorem{notation}[definition]{Notation}
\newtheorem{remark}[definition]{Remark}
% \def\theremark{\unskip}
\newtheorem{question}[definition]{Question}
\newtheorem{questions}[definition]{Questions}
%\def\thequestion{\unskip}
\newtheorem{example}[definition]{Example}
%\def\theexample{\unskip}
\newtheorem{problem}[definition]{Problem}
\newtheorem{exercise}[definition]{Exercise}

\numberwithin{theorem}{section}
\numberwithin{definition}{section}
\numberwithin{equation}{section}

\def\reals{{\mathbb R}}
\def\torus{{\mathbb T}}
\def\integers{{\mathbb Z}}
\def\rationals{{\mathbb Q}}
\def\naturals{{\mathbb N}}
\def\complex{{\mathbb C}\/}
\def\distance{\operatorname{distance}\,}
\def\support{\operatorname{support}\,}
\def\dist{\operatorname{dist}\,}
\def\Span{\operatorname{span}\,}
\def\degree{\operatorname{degree}\,}
\def\kernel{\operatorname{kernel}\,}
\def\dim{\operatorname{dim}\,}
\def\codim{\operatorname{codim}}
\def\trace{\operatorname{trace\,}}
\def\dimension{\operatorname{dimension}\,}
\def\codimension{\operatorname{codimension}\,}
\def\nullspace{\scriptk}
\def\kernel{\operatorname{Ker}}
\def\p{\partial}
\def\Re{\operatorname{Re\,} }
\def\Im{\operatorname{Im\,} }
\def\ov{\overline}
\def\eps{\varepsilon}
\def\lt{L^2}
\def\curl{\operatorname{curl}}
\def\divergence{\operatorname{div}}
\newcommand{\norm}[1]{ \|  #1 \|}
\def\expect{\mathbb E}
\def\bull{$\bullet$\ }
\def\det{\operatorname{det}}
\def\Det{\operatorname{Det}}
\def\rank{\mathbf r}
\def\diameter{\operatorname{diameter}}

\def\t2{\tfrac12}

\newcommand{\abr}[1]{ \langle  #1 \rangle}

\def\newbull{\medskip\noindent $\bullet$\ }
\def\field{{\mathbb F}}
\def\cc{C_c}



% \renewcommand\forall{\ \forall\,}

% \newcommand{\Norm}[1]{ \left\|  #1 \right\| }
\newcommand{\Norm}[1]{ \Big\|  #1 \Big\| }
\newcommand{\set}[1]{ \left\{ #1 \right\} }
%\newcommand{\ifof}{\Leftrightarrow}
\def\one{{\mathbf 1}}
\newcommand{\modulo}[2]{[#1]_{#2}}

\def\bd{\operatorname{bd}\,}
\def\cl{\text{cl}}
\def\nobull{\noindent$\bullet$\ }

\def\scriptf{{\mathcal F}}
\def\scriptq{{\mathcal Q}}
\def\scriptg{{\mathcal G}}
\def\scriptm{{\mathcal M}}
\def\scriptb{{\mathcal B}}
\def\scriptc{{\mathcal C}}
\def\scriptt{{\mathcal T}}
\def\scripti{{\mathcal I}}
\def\scripte{{\mathcal E}}
\def\scriptv{{\mathcal V}}
\def\scriptw{{\mathcal W}}
\def\scriptu{{\mathcal U}}
\def\scriptS{{\mathcal S}}
\def\scripta{{\mathcal A}}
\def\scriptr{{\mathcal R}}
\def\scripto{{\mathcal O}}
\def\scripth{{\mathcal H}}
\def\scriptd{{\mathcal D}}
\def\scriptl{{\mathcal L}}
\def\scriptn{{\mathcal N}}
\def\scriptp{{\mathcal P}}
\def\scriptk{{\mathcal K}}
\def\scriptP{{\mathcal P}}
\def\scriptj{{\mathcal J}}
\def\scriptz{{\mathcal Z}}
\def\scripts{{\mathcal S}}
\def\scriptx{{\mathcal X}}
\def\scripty{{\mathcal Y}}
\def\frakv{{\mathfrak V}}
\def\frakG{{\mathfrak G}}
\def\aff{\operatorname{Aff}}
\def\frakB{{\mathfrak B}}
\def\frakC{{\mathfrak C}}

\def\symdif{\,\Delta\,}
\def\mustar{\mu^*}
\def\muplus{\mu^+}

\def\soln{\noindent {\bf Solution.}\ }


%\pagestyle{empty}
%\setlength{\parindent}{0pt}

\begin{document}

\begin{center}{\bf Math 215A --- UCB, Spring 2017 --- William Guss} \\
Partners: Alekos, Chris \\
Selected Problems: 1,2 (Depending on that which wasn't submitted by Alekos or Chris.)
\end{center}


\medskip \noindent {\bf (1.1)}\ (Interesting Examples)\:
\begin{itemize}
 	\item \emph{Write down interesting topological spaces that are not Hausdorff.} \\

 	\noindent Recall the definition of a Hausdorff space.
 	\begin{definition}
 		Let $X$ be a topological space endowed with a topology $\tau$. If for every $x,y \in X$ there are open neighborhoods $U \ni x$ and $V \ni y$ so that $U \cap V = \emptyset$, then $X$ is called Hausdorff. 	
 	\end{definition}
 	\noindent With this definition in mind we will propose the following examples of a topological space $(X, \tau)$ that is not Hausdorff. \\


 	\noindent \emph{Example 1.} Let $X = \{x,y\}$ and $\tau = \{\{x\}, \{x,y\}, \emptyset\}.$ First $(X, \tau)$ is a topological space as $\tau$ contains $X, \emptyset$ and is closed under union and finite intersection. However $x \neq y$ and $\{x\} \ni x$ and $\{x,y\} \ni y$ are not disjoint and there is no $U \in \tau$ so that $U \ni y$ and not $U \ni x$. Therefore $(X, \tau)$ is not Hausdorff. \\

 	\noindent \emph{Example 2. } Let $x,y \in \mathbb{R}$ be $\sim$-equivalent ($x \sim y$) if $x -y \in \mathbb{Q}.$ Then let $X = (\mathbb{R}/\sim).$ We claim that the induced quotient topology is not Hausdorff.

 	Take $x = [0], y = [e]$.  Then let $\tilde U, \tilde V$ be open neighborhoods of $x$ and $y$ respectively. We have that $\tilde U = \pi(U)$ and $\tilde V = \pi(V)$ where $U,V$ are open in $\mathbb{R}$ with the standard topology. 

 	From real analysis we know that every open $V$ in $\mathbb{R}$ must contain a rational number as $V$ is composed of open intervals. Therefore $r \in \mathbb{Q}$ so that $r \in V$. Thus $\pi(r) = [0] \in \tilde V$ and thus $\tilde U \cap \tilde V \neq \emptyset$ 

 	Since this is true for every pair of open neighborhoods $\tilde U, \tilde V$ the space could not be Hausdorff\footnote{I like this example.}. 


 	\item \emph{Write down continuous bijections that aren't homeomorphisms.}

 		 \noindent \textbf{Remark.} If $f: X \to Y$ is continuous and $Y$ is  Hausdorff then $f$ is open. Therefore we can only presume to have non-Hausdorff space as an example. \\


 		 \noindent \emph{Example 1.} Let $(X, \tau)$ be given from Example 1.  Then let $Y = X$ and $\tau_{disc}$ be the discrete topology generated by $f: Y \to X$ be identity map. Then clearly the preimage of opens, say $\{x, y\} \subset \tau$ is open $\{x,y\} \subset \tau_{disc}$, but the map is not open for $\{y\} \subset \tau_{desc}$ is not a subset of $\tau$.


 		 I think the easiest way to generate these examples is to take a non-Hausdorff space and build an automorphism. \\

 		 \noindent \emph{Example 2.} Let $X = \mathbb{R}$ with the discrete topology, and let $Y = \mathbb{R}$ wkith the standard open-ball topology. Then let $f: X \to Y$ be an identity map, clearly the preimage of opens is open since every set is open in $X$ with the discrete topology, but take the closed interval $[6,13]$ and its image is not open although under the discrete topology it is open. Therefore $f$ continous bijection and it is not a homeomorphism.  

 \end{itemize} 


\medskip \noindent {\bf (1.2)}\ (Normality)\:Prove or disprove the followikng for a topological space $(X, \tau).$
\begin{itemize}
	\item \emph{X Hausdorff and completely regular implies $X$ normal.} \\

	\noindent We will disprove the above statment by proviong the following lemma and providing a counter example. 

	\begin{lemma}
		If $(X, \tau)$ is completely regular than it is Hasudorff. 
	\end{lemma}
	\begin{proof}
		First observe that if $X$ is completely regular then it is Urhysohn; that is, for any points $x, y \in X$ we have that there is a continuous function $f: X \to [0,1]$ where $[0,1]$ is endowed with the subspace topology and $f(x) = 0, f(y) = 1.$

		Now let $x,y \in X$ be given and not identical. Take an Urhysohn function $f$ as above and then observe that $W = [0, 0.5)$ and $Z = (0.5, 1]$ are open subsets of $[0,1]$ in the inherited topology. Furthermore  $Z \cap W = \emptyset$ implies that $f^{-1}(Z \cap W) = f^{-1}(Z) \cap f^{-1}(W) = \emptyset.$ Furthermore $ f^{-1}(Z) = V \ni y$, $ f^{-1}(W) = U \ni x$. 

		Therefore $X$ is Hausdorff.
	\end{proof}

	\emph{Counter Example.} We will provide a counter example using the topology $X$ from the first example. The space is trivially normal since the only  non-empty closed set is $\{y\}$ and so it is normal, but the space is not Hausdorff and so it is not completely regular. Therefore the assertion is false ;).

	\item \emph{X Hausdorff, completely regular and Lindelof implies X normal.} \\


	\begin{proof}
		Let $A,B \subset X$ be two disjoint closed sets in $X.$ Then for every $x \in A$ let $U_x$ be an openset containing $x$ disjoint from $B$ and let $V_y$ be the converse for every $y \in B$.	

		Now $\{U_x\}_{x \in A} = \scriptu$ and $\{V_y\}_{y \in B} = \scriptv$. Then by the Linedelof property there exists a countable subcover indexed by $\{x_i\}_{i \in \mathbb{N}}$ and $\{y_j\}_{j \in \mathbb{N}}$  for $\scriptu$ and $\scriptv$ respectively.

		Observe the following disjoitness properties:
		\begin{equation*}
			\begin{aligned}
				U_{x_1} \setminus \overline{V_{y_1}} \cap V_{y_1} \setminus \overline{U_{x_1}} &=\;\; \emptyset \\
				U_{x_2} \setminus (\overline{V_{y_1}} \cup \overline{V_{y_2}}) \cap V_{y_2} \setminus (\overline{U_{x_1}} \cup \overline{U_{x_2}}) &= \;\;\emptyset\\
				\vdots \;\;\;\;\;\;\;\;\;\;\;\;\;\;\;\;\;\;\;\;\;\;\;\;\;\;\;\;\vdots\;\;\;\;\;\;\;\;\;&= \;\;\vdots \\
				\bigcup_{i=1}^\infty \left[U_{x_i} \setminus \bigcup_{j \leq i} \overline{V_{y_j}}\right] \cap \bigcup_{i=1}^\infty \left[V_{y_i} \setminus \bigcup_{j \leq i} \overline{U_{x_j}}\right] &=\;\; \emptyset
			\end{aligned}
		\end{equation*}

		Furthermore $\bigcup_{i=1}^\infty [U_{x_i} \setminus \bigcup_{j \leq i} \overline{V_{y_j}}]$ is a an open set since we subtract the finite union of closed sets from an opensets; that is we subtract a closed set from an open set and thus the resultant is open. Additionally $\bigcup_{i=1}^\infty \left[U_{x_i} \setminus \bigcup_{j \leq i} \overline{V_{y_j}}\right] \supset A$ and $ \bigcup_{i=1}^\infty \left[V_{y_i} \setminus \bigcup_{j \leq i} \overline{U_{x_j}}\right] \supset B$. Therefore there are two disjoint open sets containing both $A$ and $B$ respectively. Since $A,B$ were arbitrary $X$ is normal. This completes the proof.
	\end{proof}



\end{itemize}
\end{document}\end
