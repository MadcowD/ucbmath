\documentclass[11pt]{amsart}

\usepackage{amsmath,amsthm}
\usepackage{amssymb}
\usepackage{graphicx}
\usepackage{enumerate}
\usepackage{fullpage}
\usepackage{tikz-cd}
% \usepackage{euscript}
% \makeatletter
% \nopagenumbers
\usepackage{verbatim}
\usepackage{color}
\usepackage{hyperref}

\usepackage{fullpage,tikz,float}
%\usepackage{times} %, mathtime}

\textheight=600pt %574pt
\textwidth=480pt %432pt
\oddsidemargin=15pt %18.88pt
\evensidemargin=18.88pt
\topmargin=10pt %14.21pt

\parskip=1pt %2pt

% define theorem environments
\newtheorem{theorem}{Theorem}    %[section]
%\def\thetheorem{\unskip}
\newtheorem{proposition}[theorem]{Proposition}
%\def\theproposition{\unskip}
\newtheorem{conjecture}[theorem]{Conjecture}
\def\theconjecture{\unskip}
\newtheorem{corollary}[theorem]{Corollary}
\newtheorem{lemma}[theorem]{Lemma}
\newtheorem{sublemma}[theorem]{Sublemma}
\newtheorem{fact}[theorem]{Fact}
\newtheorem{observation}[theorem]{Observation}
%\def\thelemma{\unskip}
\theoremstyle{definition}
\newtheorem{definition}{Definition}
%\def\thedefinition{\unskip}
\newtheorem{notation}[definition]{Notation}
\newtheorem{remark}[definition]{Remark}
% \def\theremark{\unskip}
\newtheorem{question}[definition]{Question}
\newtheorem{questions}[definition]{Questions}
%\def\thequestion{\unskip}
\newtheorem{example}[definition]{Example}
%\def\theexample{\unskip}
\newtheorem{problem}[definition]{Problem}
\newtheorem{exercise}[definition]{Exercise}

\numberwithin{theorem}{section}
\numberwithin{definition}{section}
\numberwithin{equation}{section}

\def\reals{{\mathbb R}}
\def\torus{{\mathbb T}}
\def\integers{{\mathbb Z}}
\def\rationals{{\mathbb Q}}
\def\naturals{{\mathbb N}}
\def\complex{{\mathbb C}\/}
\def\distance{\operatorname{distance}\,}
\def\support{\operatorname{support}\,}
\def\dist{\operatorname{dist}\,}
\def\Span{\operatorname{span}\,}
\def\degree{\operatorname{degree}\,}
\def\kernel{\operatorname{kernel}\,}
\def\dim{\operatorname{dim}\,}
\def\codim{\operatorname{codim}}
\def\trace{\operatorname{trace\,}}
\def\dimension{\operatorname{dimension}\,}
\def\codimension{\operatorname{codimension}\,}
\def\nullspace{\scriptk}
\def\kernel{\operatorname{Ker}}
\def\p{\partial}
\def\Re{\operatorname{Re\,} }
\def\Im{\operatorname{Im\,} }
\def\ov{\overline}
\def\eps{\varepsilon}
\def\lt{L^2}
\def\curl{\operatorname{curl}}
\def\divergence{\operatorname{div}}
\newcommand{\norm}[1]{ \|  #1 \|}
\def\expect{\mathbb E}
\def\bull{$\bullet$\ }
\def\det{\operatorname{det}}
\def\Det{\operatorname{Det}}
\def\rank{\mathbf r}
\def\diameter{\operatorname{diameter}}

\def\t2{\tfrac12}

\newcommand{\abr}[1]{ \langle  #1 \rangle}

\def\newbull{\medskip\noindent $\bullet$\ }
\def\field{{\mathbb F}}
\def\cc{C_c}



% \renewcommand\forall{\ \forall\,}

% \newcommand{\Norm}[1]{ \left\|  #1 \right\| }
\newcommand{\Norm}[1]{ \Big\|  #1 \Big\| }
\newcommand{\set}[1]{ \left\{ #1 \right\} }
%\newcommand{\ifof}{\Leftrightarrow}
\def\one{{\mathbf 1}}
\newcommand{\modulo}[2]{[#1]_{#2}}

\def\bd{\operatorname{bd}\,}
\def\cl{\text{cl}}
\def\nobull{\noindent$\bullet$\ }

\def\scriptf{{\mathcal F}}
\def\scriptq{{\mathcal Q}}
\def\scriptg{{\mathcal G}}
\def\scriptm{{\mathcal M}}
\def\scriptb{{\mathcal B}}
\def\scriptc{{\mathcal C}}
\def\scriptt{{\mathcal T}}
\def\scripti{{\mathcal I}}
\def\scripte{{\mathcal E}}
\def\scriptv{{\mathcal V}}
\def\scriptw{{\mathcal W}}
\def\scriptu{{\mathcal U}}
\def\scriptS{{\mathcal S}}
\def\scripta{{\mathcal A}}
\def\scriptr{{\mathcal R}}
\def\scripto{{\mathcal O}}
\def\scripth{{\mathcal H}}
\def\scriptd{{\mathcal D}}
\def\scriptl{{\mathcal L}}
\def\scriptn{{\mathcal N}}
\def\scriptp{{\mathcal P}}
\def\scriptk{{\mathcal K}}
\def\scriptP{{\mathcal P}}
\def\scriptj{{\mathcal J}}
\def\scriptz{{\mathcal Z}}
\def\scripts{{\mathcal S}}
\def\scriptx{{\mathcal X}}
\def\scripty{{\mathcal Y}}
\def\frakv{{\mathfrak V}}
\def\frakG{{\mathfrak G}}
\def\aff{\operatorname{Aff}}
\def\frakB{{\mathfrak B}}
\def\frakC{{\mathfrak C}}

\def\symdif{\,\Delta\,}
\def\mustar{\mu^*}
\def\muplus{\mu^+}

\def\soln{\noindent {\bf Solution.}\ }


%\pagestyle{empty}
%\setlength{\parindent}{0pt}

\begin{document}

\begin{center}{\bf Math 215A --- UCB, Spring 2017 --- William Guss} \\
Partners: Alekos, Chris \\
Selected Problems: 1 
\end{center}


\medskip \noindent {\bf (11.1)}\ \emph{(Chain Homotopy)} Find degree-wise free chain complexes $A,B$ and a chain map $f: A\to B$, not chain homotopic to the zero map, but such that the induced homomorphism $f_*: H_*(A) \to H_*(B)$ is zero.\\

\noindent \textbf{\underline{Solution}.} We will first describe a suitable general structure for candidate chain complexes and maps, and then provide an example. Let $G$ be any free abelian group (or R-module), then  let $A,B$ be chain complexes diagramatically follows
\begin{equation*}
	\begin{tikzcd}
		A: 0\arrow{d} \arrow{r} & G \arrow{r}{\partial}\arrow{d}{\text{id}}  &  G \arrow{r}\arrow{d} & 0\arrow{d} \\
		B: 0 \arrow{r}& G \arrow{r}& 0 \arrow{r}& 0
	\end{tikzcd}
\end{equation*}
where the chain map $f$ is defined by the downward arrows. 
\begin{lemma}
	If $Im(\partial) \varsubsetneq G$, and for any homomorphism $\gamma: G \to G$, we have that $\gamma \circ \partial = \partial \circ \gamma$ then $f \not \simeq 0$.
\end{lemma}
\begin{proof}
	Suppose that $f$ were chain-nullhomotopic. Then there exists $\gamma$ and $\psi$ so that $\text{id} - 0 = 0 \circ \psi + \gamma \circ \partial.$ By our hypothesis, $\gamma \circ  \partial = \partial \circ \gamma$ and therefore $\text{id} = \partial \circ \gamma$. But then this contradicts $Im(\partial \circ \gamma)  \subset Im(\partial) \varsubsetneq G = Im(\text{id}).$ Therefore there cannot exist such $\gamma$ and $f \not\simeq 0.$ 
\end{proof}

\begin{lemma}
	The induced homomorphism of homologies, $f_*$ is the zero map when $Ker(\partial) = 0$.   
\end{lemma}
\begin{proof}
Application of the homology functor yields automatically the following diagram
\begin{equation*}
	\begin{tikzcd}
		A: 0\arrow{d} \arrow{r} & H_1(A) \arrow{r}{H_*(\partial)}\arrow{d}{H_*(\text{id})}  &  H_0(A) \arrow{r}\arrow{d} & 0\arrow{d} \\
		B: 0 \arrow{r}& H_1(B) \arrow{r}& 0 \arrow{r}& 0
	\end{tikzcd}
\end{equation*}
Then $H_1(B) = G$ since the kernel of thre constant boundary map from $G$ into $0$ is $G$ and the image of the constant inclusion map from $0$ into $G$ is $\{0\}$. Furthermore, since $Ker(\partial) =0$ we yield that $H_1(A) = 0.$ Therefore $f_*$ is the zero map at every degree.
\end{proof}

With this in mind, finding chain complexes satisfying the statement of the problem is reduced to finding a group $G$ and a homomorphism $\partial$ with the properties of both lemmas. Take $G =\mathbb{Z} $ and $\partial$ to be any (non-trivial) multiplicitive  operator.
\end{document}\end
