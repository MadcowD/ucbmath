\documentclass[11pt]{amsart}

\usepackage{amsmath,amsthm}
\usepackage{amssymb}
\usepackage{graphicx}
\usepackage{enumerate}
\usepackage{fullpage}
\usepackage{tikz-cd}
% \usepackage{euscript}
% \makeatletter
% \nopagenumbers
\usepackage{verbatim}
\usepackage{color}
\usepackage{hyperref}

\usepackage{fullpage,tikz,float}
%\usepackage{times} %, mathtime}

\textheight=600pt %574pt
\textwidth=480pt %432pt
\oddsidemargin=15pt %18.88pt
\evensidemargin=18.88pt
\topmargin=10pt %14.21pt

\parskip=1pt %2pt

% define theorem environments
\newtheorem{theorem}{Theorem}    %[section]
%\def\thetheorem{\unskip}
\newtheorem{proposition}[theorem]{Proposition}
%\def\theproposition{\unskip}
\newtheorem{conjecture}[theorem]{Conjecture}
\def\theconjecture{\unskip}
\newtheorem{corollary}[theorem]{Corollary}
\newtheorem{lemma}[theorem]{Lemma}
\newtheorem{sublemma}[theorem]{Sublemma}
\newtheorem{fact}[theorem]{Fact}
\newtheorem{observation}[theorem]{Observation}
%\def\thelemma{\unskip}
\theoremstyle{definition}
\newtheorem{definition}{Definition}
%\def\thedefinition{\unskip}
\newtheorem{notation}[definition]{Notation}
\newtheorem{remark}[definition]{Remark}
% \def\theremark{\unskip}
\newtheorem{question}[definition]{Question}
\newtheorem{questions}[definition]{Questions}
%\def\thequestion{\unskip}
\newtheorem{example}[definition]{Example}
%\def\theexample{\unskip}
\newtheorem{problem}[definition]{Problem}
\newtheorem{exercise}[definition]{Exercise}

\numberwithin{theorem}{section}
\numberwithin{definition}{section}
\numberwithin{equation}{section}

\def\reals{{\mathbb R}}
\def\torus{{\mathbb T}}
\def\integers{{\mathbb Z}}
\def\rationals{{\mathbb Q}}
\def\naturals{{\mathbb N}}
\def\complex{{\mathbb C}\/}
\def\distance{\operatorname{distance}\,}
\def\support{\operatorname{support}\,}
\def\dist{\operatorname{dist}\,}
\def\Span{\operatorname{span}\,}
\def\degree{\operatorname{degree}\,}
\def\kernel{\operatorname{kernel}\,}
\def\dim{\operatorname{dim}\,}
\def\codim{\operatorname{codim}}
\def\trace{\operatorname{trace\,}}
\def\dimension{\operatorname{dimension}\,}
\def\codimension{\operatorname{codimension}\,}
\def\nullspace{\scriptk}
\def\kernel{\operatorname{Ker}}
\def\p{\partial}
\def\Re{\operatorname{Re\,} }
\def\Im{\operatorname{Im\,} }
\def\ov{\overline}
\def\eps{\varepsilon}
\def\lt{L^2}
\def\curl{\operatorname{curl}}
\def\divergence{\operatorname{div}}
\newcommand{\norm}[1]{ \|  #1 \|}
\def\expect{\mathbb E}
\def\bull{$\bullet$\ }
\def\det{\operatorname{det}}
\def\Det{\operatorname{Det}}
\def\rank{\mathbf r}
\def\diameter{\operatorname{diameter}}

\def\t2{\tfrac12}

\newcommand{\abr}[1]{ \langle  #1 \rangle}

\def\newbull{\medskip\noindent $\bullet$\ }
\def\field{{\mathbb F}}
\def\cc{C_c}



% \renewcommand\forall{\ \forall\,}

% \newcommand{\Norm}[1]{ \left\|  #1 \right\| }
\newcommand{\Norm}[1]{ \Big\|  #1 \Big\| }
\newcommand{\set}[1]{ \left\{ #1 \right\} }
%\newcommand{\ifof}{\Leftrightarrow}
\def\one{{\mathbf 1}}
\newcommand{\modulo}[2]{[#1]_{#2}}

\def\bd{\operatorname{bd}\,}
\def\cl{\text{cl}}
\def\nobull{\noindent$\bullet$\ }

\def\scriptf{{\mathcal F}}
\def\scriptq{{\mathcal Q}}
\def\scriptg{{\mathcal G}}
\def\scriptm{{\mathcal M}}
\def\scriptb{{\mathcal B}}
\def\scriptc{{\mathcal C}}
\def\scriptt{{\mathcal T}}
\def\scripti{{\mathcal I}}
\def\scripte{{\mathcal E}}
\def\scriptv{{\mathcal V}}
\def\scriptw{{\mathcal W}}
\def\scriptu{{\mathcal U}}
\def\scriptS{{\mathcal S}}
\def\scripta{{\mathcal A}}
\def\scriptr{{\mathcal R}}
\def\scripto{{\mathcal O}}
\def\scripth{{\mathcal H}}
\def\scriptd{{\mathcal D}}
\def\scriptl{{\mathcal L}}
\def\scriptn{{\mathcal N}}
\def\scriptp{{\mathcal P}}
\def\scriptk{{\mathcal K}}
\def\scriptP{{\mathcal P}}
\def\scriptj{{\mathcal J}}
\def\scriptz{{\mathcal Z}}
\def\scripts{{\mathcal S}}
\def\scriptx{{\mathcal X}}
\def\scripty{{\mathcal Y}}
\def\frakv{{\mathfrak V}}
\def\frakG{{\mathfrak G}}
\def\aff{\operatorname{Aff}}
\def\frakB{{\mathfrak B}}
\def\frakC{{\mathfrak C}}

\def\symdif{\,\Delta\,}
\def\mustar{\mu^*}
\def\muplus{\mu^+}

\def\soln{\noindent {\bf Solution.}\ }


%\pagestyle{empty}
%\setlength{\parindent}{0pt}

\begin{document}

\begin{center}{\bf Math 215A --- UCB, Spring 2017 --- William Guss} \\
Partners: Alekos, Chris \\
Selected Problems: 3 (Depending on that which wasn't submitted by Alekos or Chris.)
\end{center}


\medskip \noindent {\bf (7.3a)}\ \emph{(Categorical kernels and cokernels):} \ Let $\scriptc$ be a category with an object $0\in \scriptc$ that is initial and terminal. A \emph{categorical kernel} of $g \in \scriptc(B,C)$ is a pullback
\begin{equation*}
	\begin{tikzcd}
		K \arrow{r}{} \arrow{d}{k} & 0 \arrow{d}{} \\
		B \arrow{r}{g} & C
	\end{tikzcd}
\end{equation*}
Show that $k$ is a monomorphism.
\begin{proof}
	Suppose that $J \in \scriptc$ and $\xi = k \circ f_1 = k \circ f_2$ for $f_i \in \scriptc(J, K)$. 
	Now let $\gamma: K \to 0$ be the unique terminal map for $K$ and let $\xi$ be the unqiue terminal map for $J.$ Then $\gamma \circ f_2, \gamma \circ f_1 \in C(J, 0) \implies \gamma \circ f_2 = \gamma \circ f_1 := \zeta$.
	There fore the following two diagrams commute, summerizing the situation.
	\begin{equation*}
			\begin{tikzcd}
			J \arrow[r,shift left, "f_1"] 
			\arrow[r, swap, "f_2"] 
			\arrow[rd, swap,"\xi"] & K \arrow[d, "k"] \\
			& B
		\end{tikzcd}
		\begin{tikzcd}
			& 0\\ 
			J \arrow[r, "f_1"] 
			\arrow[r,  shift right, swap, "f_2"] 
			\arrow[ru,"\zeta"] & K \arrow[u]
		\end{tikzcd}
	\end{equation*}

	By the universality of the pullback there is a unique $u: J \to K$ which is the mediating map for the categorical pullback $(J, \xi, \zeta)$, and so because $f_1, f_2$ are mediating maps, $f_1 = f_2.$ Therefore the following diagram commutes
	\begin{equation*}
		 \begin{tikzcd}
		 	J \arrow[rrd, bend left, "\zeta"] \arrow[rd, "f_1"] \arrow[rd, swap, "f_2"] \arrow[rdd,swap, bend right,  "\xi"] \\
		 	& K \arrow[r] \arrow[d, "k"] & O \arrow[d] \\
		 	& B \arrow[r, "g"] & C
		 \end{tikzcd}
	 \end{equation*} 
	 Therefore  $k$ is a monomorphism.
\end{proof}
\medskip \noindent {\bf (7.3b)}\ Given a commutative diagram
\begin{equation*}
	\begin{tikzcd}
		A' \arrow{r}{f'}\arrow{d}{\alpha} & B'  \arrow{r}{g'} \arrow{d}{\beta} &C' \arrow{d}{\gamma} \\
		A \arrow{r}{f} &B \arrow{r}{g} & C
	\end{tikzcd}
\end{equation*}
where the left square is a pullback, show that 
\begin{itemize}
	\item If $f$ is a monomorphism then so if $f'$.
	\item If $f$ is a kernel of $g$ then $f'$ is a kernel of $g'.$
\end{itemize}
\begin{proof}
	We will first show that if $f$ is a monomorphism then $f'$ is a monomorphism. Take some $J \in \scriptc$ and maps $p_1, p_2$ so that $f' \circ p_1 = f' \circ p_2 := \zeta$. Then using the commutativity of the diagram above, 
	\begin{equation*}
		\begin{aligned}
			f \circ (\alpha \circ p_1) &= \beta \circ f' \circ p_1 \\
			&= \beta \circ \zeta \\
			&=  \beta \circ f' \circ p_2 \\
			&= f \circ (\alpha \circ p_2 ).
		\end{aligned}
	\end{equation*}
	Since $f$ is a monomorphism, $\alpha \circ p_1 = \alpha  \circ p_2 = \xi$. Therefore the following diagram commutes, and using the universal property of pullbacks, $p_1, p_2$ are the same unique mediation map between the pullback in the first diagram and the new pullback $(J, \xi, \zeta)$.
	\begin{equation*}
	\begin{tikzcd}
		 	J \arrow[rrd, bend left, "\zeta"] \arrow[rd, "p_1"] \arrow[rd, swap, "p_2"] \arrow[rdd,swap, bend right,  "\xi"] \\
		 	& A' \arrow[r, "f'"] \arrow[d, "k"] & B' \arrow[d, "\beta"] \\
		 	& A \arrow[r, "f"] & B
		 \end{tikzcd}
	\end{equation*}
	Therefore $p_1 = p_2$ and $f'$ is a monomorphism.
\end{proof}

\textbf{Dual Definitions}: A \emph{categorical cokernel} of $g \in \scriptc(B,C)$ is a pushforward so that the diagram commutes
\begin{equation*}
	\begin{tikzcd}
		K  & 0 \arrow{l}{} \\
		B \arrow{u}{k}  & \arrow{l}{g}C \arrow{u}{}
	\end{tikzcd}
\end{equation*}
We say that $f: X \to Y$ is an \emph{epimorphism} if and only if $T \in \scriptc$ and $f_1, f_2: Y \to T$ so that
$f_1 \circ f = f_2 \circ f$ implies that $f_2 = f_1$. Furthermore
Given a commutative diagram
\begin{equation*}
	\begin{tikzcd}
		A'  &\arrow{l}{f'} B'   & C'  \arrow{l}{g'} \\
		A \arrow{u}{\alpha} &B \arrow{u}{\beta} \arrow{l}{f}& C \arrow{u}{\gamma}  \arrow{l}{g} 
	\end{tikzcd}
\end{equation*}
where the left square is a pushforward, show that if $f$ is an epimorphism then so is $f'$. If $f$ is a cokernel of $g$ then $f'$ is a cokernel of $g'$
\end{document}\end
