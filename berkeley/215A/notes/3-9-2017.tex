\documentclass[11pt]{amsart}

\usepackage{amsmath,amsthm}
\usepackage{amssymb}
\usepackage{graphicx}
\usepackage{enumerate}
\usepackage{fullpage}
 \usepackage{euscript}
 \usepackage{todonotes}
% \makeatletter
% \nopagenumbers
\usepackage{verbatim}
\usepackage{color}
\usepackage{hyperref}

\usepackage{fullpage,tikz,float}
\usepackage{tikz-cd}
%\usepackage{times} %, mathtime}

\textheight=600pt %574pt
\textwidth=480pt %432pt
\oddsidemargin=15pt %18.88pt
\evensidemargin=18.88pt
\topmargin=10pt %14.21pt

\parskip=1pt %2pt

% define theorem environments
\newtheorem{theorem}{Theorem}    %[section]
%\def\thetheorem{\unskip}
\newtheorem{proposition}[theorem]{Proposition}
%\def\theproposition{\unskip}
\newtheorem{conjecture}[theorem]{Conjecture}
\def\theconjecture{\unskip}
\newtheorem{corollary}[theorem]{Corollary}
\newtheorem{lemma}[theorem]{Lemma}
\newtheorem{sublemma}[theorem]{Sublemma}
\newtheorem{fact}[theorem]{Fact}
\newtheorem{observation}[theorem]{Observation}
%\def\thelemma{\unskip}
\theoremstyle{definition}
\newtheorem{definition}{Definition}
%\def\thedefinition{\unskip}
\newtheorem{notation}[definition]{Notation}
\newtheorem{remark}[definition]{Remark}
% \def\theremark{\unskip}
\newtheorem{question}[definition]{Question}
\newtheorem{questions}[definition]{Questions}
%\def\thequestion{\unskip}
\newtheorem{example}[definition]{Example}
%\def\theexample{\unskip}
\newtheorem{problem}[definition]{Problem}
\newtheorem{exercise}[definition]{Exercise}


\def\reals{{\mathbb R}}
\def\torus{{\mathbb T}}
\def\integers{{\mathbb Z}}
\def\rationals{{\mathbb Q}}
\def\naturals{{\mathbb N}}
\def\complex{{\mathbb C}\/}
\def\sphere{{\pmb S}}
\def\projspace{{ \mathbb{K}\mathbb{P} }}
\def\distance{\operatorname{distance}\,}
\def\support{\operatorname{support}\,}
\def\dist{\operatorname{dist}\,}
\def\Span{\operatorname{span}\,}
\def\degree{\operatorname{degree}\,}
\def\kernel{\operatorname{kernel}\,}
\def\dim{\operatorname{dim}\,}
\def\codim{\operatorname{codim}}
\def\trace{\operatorname{trace\,}}
\def\dimension{\operatorname{dimension}\,}
\def\codimension{\operatorname{codimension}\,}
\def\nullspace{\scriptk}
\def\kernel{\operatorname{Ker}}
\def\p{\partial}
\def\Re{\operatorname{Re\,} }
\def\Im{\operatorname{Im\,} }
\def\ov{\overline}
\def\eps{\varepsilon}
\def\lt{L^2}
\def\curl{\operatorname{curl}}
\def\divergence{\operatorname{div}}
\newcommand{\norm}[1]{ \|  #1 \|}
\def\expect{\mathbb E}
\def\bull{$\bullet$\ }
\def\det{\operatorname{det}}
\def\Det{\operatorname{Det}}
\def\rank{\mathbf r}
\def\diameter{\operatorname{diameter}}

\def\t2{\tfrac12}

\newcommand{\abr}[1]{ \langle  #1 \rangle}

\def\newbull{\medskip\noindent $\bullet$\ }
\def\field{{\mathbb F}}
\def\cc{C_c}



% \renewcommand\forall{\ \forall\,}

% \newcommand{\Norm}[1]{ \left\|  #1 \right\| }
\newcommand{\Norm}[1]{ \Big\|  #1 \Big\| }
\newcommand{\set}[1]{ \left\{ #1 \right\} }
%\newcommand{\ifof}{\Leftrightarrow}
\def\one{{\mathbf 1}}
\newcommand{\modulo}[2]{[#1]_{#2}}

\def\bd{\operatorname{bd}\,}
\def\cl{\text{cl}}
\def\nobull{\noindent$\bullet$\ }

\def\scriptf{{\mathcal F}}
\def\scriptq{{\mathcal Q}}
\def\scriptg{{\mathcal G}}
\def\scriptm{{\mathcal M}}
\def\scriptb{{\mathcal B}}
\def\scriptc{{\mathcal C}}
\def\scriptt{{\mathcal T}}
\def\scripti{{\mathcal I}}
\def\scripte{{\mathcal E}}
\def\scriptv{{\mathcal V}}
\def\scriptw{{\mathcal W}}
\def\scriptu{{\mathcal U}}
\def\scriptS{{\mathcal S}}
\def\scripta{{\mathcal A}}
\def\scriptr{{\mathcal R}}
\def\scripto{{\mathcal O}}
\def\scripth{{\mathcal H}}
\def\scriptd{{\mathcal D}}
\def\scriptl{{\mathcal L}}
\def\scriptn{{\mathcal N}}
\def\scriptp{{\mathcal P}}
\def\scriptk{{\mathcal K}}
\def\scriptP{{\mathcal P}}
\def\scriptj{{\mathcal J}}
\def\scriptz{{\mathcal Z}}
\def\scripts{{\mathcal S}}
\def\scriptx{{\mathcal X}}
\def\scripty{{\mathcal Y}}
\def\frakv{{\mathfrak V}}
\def\frakG{{\mathfrak G}}
\def\aff{\operatorname{Aff}}
\def\frakB{{\mathfrak B}}
\def\frakC{{\mathfrak C}}

\def\symdif{\,\Delta\,}
\def\mustar{\mu^*}
\def\muplus{\mu^+}

\def\soln{\noindent {\bf Solution.}\ }


%\pagestyle{empty}
%\setlength{\parindent}{0pt}

\begin{document}

\begin{center}{\bf Math 215A --- 3-9-2017  --- William Guss} \\ {\bf Lecture
Notes} \end{center}
We will first study an application. We want to show that $\mathbb{R}^n \not\simeq \mathbb{R}^m \iff m \neq n$ homeomorphic. WE can't use functors to algebraic categories because there is a homotoipy equivalence of these spaces to the constant map. So we try something different: let the one point compactification of these spaces be $S^n \simeq S^m.$ The homotopy groups are as follows
\begin{equation*}
	\pi_i(S^n) = \mathbb{Z} \simeq H_i(S^n) i = n
\end{equation*}
When $i < n$ we have $\pi_i(S^n) = 0$ we use Sard's theorem. Mayer-Victoris.

We define the homotopoy groups as the category of maps  $[(S^i, s_0), (X, x_0)]_*$ where the morphisms are in $hTop_*$.
We use the sphere because we get groups and abelian groups! Why is the sphere good? For $i \geq 1$ $(S^i, s_0)$ is a cogroup object in $hTop_*$ via $(S^i, s_0) \to (S^i, s_0) \coprod (S^i, s_0) = (S^i \wedge S^i, s_0).$ \emph{What from category theory  says that $M$ is a co group object if there is a morphism $M \to (M \coprod M).$}



\section{Lecture: Pushouts, Pullbacks} 

See notes on line.

\section{Lecture: Picturing Homology Classes}

\section{Lecture: $H_*$ as Abelianization}

$H_n$ is a functor from topological spaces to abelian groups. We will break this functor into six steps, the first three of which are called Homotopy theory, and the last three of which are basically commuttative discrete algebra. We call the first transformation homotopy theory because it preserves the homotopy type.
\begin{equation*}
\begin{tikzcd}
	H_n: \text{Top} \arrow{r}{\Delta_.} & \text{ssSet} \arrow{r}{Free} & \text{ssGr} \arrow{r}{\text{Abel}} & \text{ssAb} \arrow{r}{\sim} & \text{Chain} \arrow{r}{H_n} & \text{Ab}.
\end{tikzcd}
\end{equation*}

To break this down we have the following seperate steps
\begin{itemize}
	\item Chain complex is a sequence of abelian groups so that 
\begin{equation*}
\begin{tikzcd}
	A_* :\;\arrow{r}{\partial}&A_{n+1} \arrow{r}{\partial} & A_n \arrow{r}{\partial} & A_{n+1} \arrow{r}{\partial} & \;
\end{tikzcd}
\end{equation*}
and $\partial_n \circ \partial{n+1} = 0$ for all $n.$
\item If $\scriptc$ is a category then $ss\scriptc = Fun(\Delta_{inj}, \scriptc)$, where $[n] \in \Delta_{inj}$ has objects $\mathbb{N}_0$ and $\Delta_{inj}([m], [n])$ are the morphisms which preserve order between $[m] =\{0, \dots, m\}$ and $\{0, \dots, n\}$.  

\item To go from ssAb to Chain we just need to take the alternatiung sum of the maps generated which is exactly the border map.
\end{itemize}


\end{document}\end