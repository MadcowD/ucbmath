\documentclass[11pt]{amsart}

\usepackage{amsmath,amsthm}
\usepackage{amssymb}
\usepackage{graphicx}
\usepackage{enumerate}
\usepackage{fullpage}
 \usepackage{euscript}
% \makeatletter
% \nopagenumbers
\usepackage{verbatim}
\usepackage{color}
\usepackage{hyperref}

\usepackage{fullpage,tikz,float}
\usepackage{tikz-cd}
%\usepackage{times} %, mathtime}

\textheight=600pt %574pt
\textwidth=480pt %432pt
\oddsidemargin=15pt %18.88pt
\evensidemargin=18.88pt
\topmargin=10pt %14.21pt

\parskip=1pt %2pt

% define theorem environments
\newtheorem{theorem}{Theorem}    %[section]
%\def\thetheorem{\unskip}
\newtheorem{proposition}[theorem]{Proposition}
%\def\theproposition{\unskip}
\newtheorem{conjecture}[theorem]{Conjecture}
\def\theconjecture{\unskip}
\newtheorem{corollary}[theorem]{Corollary}
\newtheorem{lemma}[theorem]{Lemma}
\newtheorem{sublemma}[theorem]{Sublemma}
\newtheorem{fact}[theorem]{Fact}
\newtheorem{observation}[theorem]{Observation}
%\def\thelemma{\unskip}
\theoremstyle{definition}
\newtheorem{definition}{Definition}
%\def\thedefinition{\unskip}
\newtheorem{notation}[definition]{Notation}
\newtheorem{remark}[definition]{Remark}
% \def\theremark{\unskip}
\newtheorem{question}[definition]{Question}
\newtheorem{questions}[definition]{Questions}
%\def\thequestion{\unskip}
\newtheorem{example}[definition]{Example}
%\def\theexample{\unskip}
\newtheorem{problem}[definition]{Problem}
\newtheorem{exercise}[definition]{Exercise}

\numberwithin{theorem}{section}
\numberwithin{definition}{section}
\numberwithin{equation}{section}

\def\reals{{\mathbb R}}
\def\torus{{\mathbb T}}
\def\integers{{\mathbb Z}}
\def\rationals{{\mathbb Q}}
\def\naturals{{\mathbb N}}
\def\complex{{\mathbb C}\/}
\def\distance{\operatorname{distance}\,}
\def\support{\operatorname{support}\,}
\def\dist{\operatorname{dist}\,}
\def\Span{\operatorname{span}\,}
\def\degree{\operatorname{degree}\,}
\def\kernel{\operatorname{kernel}\,}
\def\dim{\operatorname{dim}\,}
\def\codim{\operatorname{codim}}
\def\trace{\operatorname{trace\,}}
\def\dimension{\operatorname{dimension}\,}
\def\codimension{\operatorname{codimension}\,}
\def\nullspace{\scriptk}
\def\kernel{\operatorname{Ker}}
\def\p{\partial}
\def\Re{\operatorname{Re\,} }
\def\Im{\operatorname{Im\,} }
\def\ov{\overline}
\def\eps{\varepsilon}
\def\lt{L^2}
\def\curl{\operatorname{curl}}
\def\divergence{\operatorname{div}}
\newcommand{\norm}[1]{ \|  #1 \|}
\def\expect{\mathbb E}
\def\bull{$\bullet$\ }
\def\det{\operatorname{det}}
\def\Det{\operatorname{Det}}
\def\rank{\mathbf r}
\def\diameter{\operatorname{diameter}}

\def\t2{\tfrac12}

\newcommand{\abr}[1]{ \langle  #1 \rangle}

\def\newbull{\medskip\noindent $\bullet$\ }
\def\field{{\mathbb F}}
\def\cc{C_c}



% \renewcommand\forall{\ \forall\,}

% \newcommand{\Norm}[1]{ \left\|  #1 \right\| }
\newcommand{\Norm}[1]{ \Big\|  #1 \Big\| }
\newcommand{\set}[1]{ \left\{ #1 \right\} }
%\newcommand{\ifof}{\Leftrightarrow}
\def\one{{\mathbf 1}}
\newcommand{\modulo}[2]{[#1]_{#2}}

\def\bd{\operatorname{bd}\,}
\def\cl{\text{cl}}
\def\nobull{\noindent$\bullet$\ }

\def\scriptf{{\mathcal F}}
\def\scriptq{{\mathcal Q}}
\def\scriptg{{\mathcal G}}
\def\scriptm{{\mathcal M}}
\def\scriptb{{\mathcal B}}
\def\scriptc{{\mathcal C}}
\def\scriptt{{\mathcal T}}
\def\scripti{{\mathcal I}}
\def\scripte{{\mathcal E}}
\def\scriptv{{\mathcal V}}
\def\scriptw{{\mathcal W}}
\def\scriptu{{\mathcal U}}
\def\scriptS{{\mathcal S}}
\def\scripta{{\mathcal A}}
\def\scriptr{{\mathcal R}}
\def\scripto{{\mathcal O}}
\def\scripth{{\mathcal H}}
\def\scriptd{{\mathcal D}}
\def\scriptl{{\mathcal L}}
\def\scriptn{{\mathcal N}}
\def\scriptp{{\mathcal P}}
\def\scriptk{{\mathcal K}}
\def\scriptP{{\mathcal P}}
\def\scriptj{{\mathcal J}}
\def\scriptz{{\mathcal Z}}
\def\scripts{{\mathcal S}}
\def\scriptx{{\mathcal X}}
\def\scripty{{\mathcal Y}}
\def\frakv{{\mathfrak V}}
\def\frakG{{\mathfrak G}}
\def\aff{\operatorname{Aff}}
\def\frakB{{\mathfrak B}}
\def\frakC{{\mathfrak C}}

\def\symdif{\,\Delta\,}
\def\mustar{\mu^*}
\def\muplus{\mu^+}

\def\soln{\noindent {\bf Solution.}\ }


%\pagestyle{empty}
%\setlength{\parindent}{0pt}

\begin{document}

\begin{center}{\bf CS 294 Deep RL --- UCB, Spring 2017 --- S. Levine ---- Scribe: William Guss}
\\
{\bf Lecture Notes}
\end{center}
\medskip \noindent \textbf{Course Info.}

rll.berkeley.edu
piazza berkeley cs294-112

\medskip \noindent \textbf{What you should know.}
Assignments will require trianing neural networks. You should know a standard automatic differentiation tool.

\medskip \noindent \textbf{What we'll cover.} (Full syllabus on course website.)
\begin{itemize}
	\item From supervised learning to decision making.
	\item Basic reinforcmeent learning: Q-learniong and policy gradients.--- Should help get you the basics!
	\item Advanced model learning and prediction, distillation, reward learning.
	\item Advanced deep RL: trust region policy gradients, actor critic methods, exploration
	\item Open problem, research talks, invited lectures.
\end{itemize}

\textbf{Assignments}
\begin{itemize}
\item Homework 1: Imitation learning (control via supervised learning)
\item Homework 2: Basic (shallow) RL
\item Homework 3: Deep Q-Learning
\item Homework 4: Deep policy gradients.
\item Final project: Research-level project of your choice (form a group of up to 2-3 students, you're welcome to start early!)
\end{itemize}

Grading for this class is: 40\% homework (10\% each), 50\% project, 10\% participation. \emph{Pretty nice!}.

\textbf{How do we build intelligent machines?} --- Sergey Levine
\begin{itemize}
	\item Imagine you have to build an intelligent machine, where do you start? You might start by looking at a persons brain and break it into different functions. Then you would some how integrate all of these modules, ie. translate these modules into seperate pieces of code! \emph{Sergey:} This is an 'early' perspective on intelligence. We'd like something more, this is just too complicated.
	\item \textbf{Instead: } Learning as the basis of intelligence.
	\begin{itemize}
		\item Some things we can all do (eg. walking), but most intelligent behaviours are emergent as a result of our 'nurture.'
		\item Some things we can only learn (eg. driving) a car.
		\item We can learn a huge variety of things including very difficult things.
		\item Therefore our learning mechanisms are likely powerful enough to do everything we associate with intelligence. (We may need to bootstrap a few things...)
	\end{itemize}
	\item \textbf{A single algorithm?}
	\begin{itemize}
		\item An algorithm for each 'module'? 
		\item What must this single algorithm do? It needs to be able to interpret rich sensory inputs (I really don't like this idea---in general, there's got to be more!)
		\item Perception is really important here.
	\end{itemize}
	\item \textbf{Why deep reinforcement learning?}
	\begin{itemize}
		\item Deep = can process complex sensory input (and complicated functions in general)
		\item Reinforcement learning = can choose complex actions.
	\end{itemize}
\end{itemize}
\end{document}\end
	