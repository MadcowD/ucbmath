%%%%%%%%%%%%%%%%%%%%%%%%%%%%%%%%%%%%%%%%%%%%%%%%%%%%%%%%%%%%%%%%%%
%%%                      Homework 11_                          %%%
%%%%%%%%%%%%%%%%%%%%%%%%%%%%%%%%%%%%%%%%%%%%%%%%%%%%%%%%%%%%%%%%%%

\documentclass[letter]{article}

\usepackage{lipsum}
\usepackage[pdftex]{graphicx}
\usepackage[margin=1.5in]{geometry}
\usepackage[english]{babel}
\usepackage{listings}
\usepackage{amsthm}
\usepackage{amssymb}
\usepackage{framed} 
\usepackage{amsmath}
\usepackage{titling}
\usepackage{fancyhdr}

\pagestyle{fancy}


\newtheorem{theorem}{Theorem}
\newtheorem{lemma}{Lemma}
\newtheorem{definition}{Definition}

\newenvironment{menumerate}{%
  \edef\backupindent{\the\parindent}%
  \enumerate%
  \setlength{\parindent}{\backupindent}%
}{\endenumerate}







%%%%%%%%%%%%%%%
%% DOC INFO %%%
%%%%%%%%%%%%%%%
\newcommand{\bHWN}{11}
\newcommand{\bCLASS}{MATH: H104}

\title{\bCLASS: Homework \bHWN}
\author{William Guss\\26793499\\wguss@berkeley.edu}

\fancyhead[L]{\bCLASS}
\fancyhead[CO]{Homework \bHWN}
\fancyhead[CE]{GUSS}
\fancyhead[R]{\thepage}
\fancyfoot[LR]{}
\fancyfoot[C]{}
\usepackage{csquotes}

%%%%%%%%%%%%%%

\begin{document}
\maketitle
\thispagestyle{empty}


%%%%%%% 8-10, 13, 18, 22, 24.
\begin{menumerate}
	\setcounter{enumi}{7}
	\item We show the following equicontinuity result.
	\begin{theorem} Let the family of functions $F = (f_n)$ be defined as follows
		$$f_n(x)=\cos(n+x)+\log\left(1+\frac{1}{\sqrt{n+2} }\sin^2(n^nx)\right).$$
		The family $F$ is equicontinuous.
	\end{theorem}
	\begin{proof}
		Observe that the sum of two families of equicontinuous functions indexed by an integer, at that integer, is equicontinuous. Then
		we need prove that $\cos(n+x)$ is equicontinuous as is the other part of the sum. Since $\cos(x)$ is uniformly continuous, so is the
		translative sum $\cos(n+x)$ uniformly equicontinuous.

		Let $x\in \mathbb{R}$ and $\epsilon > 0$. If we take $n$ very large then $\log\left(1+\frac{1}{\sqrt{n+2} }\sin^2(n^nx)\right)$ tends towards $0.$
		In fact by the continuity of log, we take $n$ large enough so that $\log(1) + \epsilon = \epsilon.$ Therefore we have that
		for any $y \in \mathbb{R} ,$
		$$\left|\log\left(1+\frac{1}{\sqrt{n+1}}\sin^2(n^nx)\right)-\log\left(1+\frac{1}{\sqrt{n+1}}\sin^2(n^ny)\right)\right|<2\epsilon$$.
		Then examine all of the $n$ that were not large enough to yield the above result. There are finiteley many of them, so they are equicontinuous.
		Hence both elements in the sum are equicontinuous and we are done.

	\end{proof}
	
	\item Prove the following theorem.
	\begin{theorem}
	If $f:\mathbb{R} \to \mathbb{R}$ is continuous and $f(nx)$ forms an equicontinuous family, then
	$f$ is uniformly continuous.
	\end{theorem}
	\begin{proof}
		If $f(nx) = f_n$ defines an equicontinuous family, then for every $\epsilon > 0$ there exists
		a $\delta(\epsilon)$ such that for every $n$ and for every $x,y$ such that $d(x,y) < \delta$
		it is the case that $|f(nx) - f(ny)| < \epsilon.$ Therefore taking $n = 1$, we yield that for every
		$\epsilon > 0$ there exists a $\delta$ such that for every $x$
		$$\sup_{y\in{(x-\delta,x+\delta)}}|f(x)-f(y)| < \epsilon.$$
		Thus $f$ is uniformly continuous.
	\end{proof}
	\item Consider the following example.
	\begin{theorem}
		Let $(f_n)$ be a family of functions which take $(0,1) \to \mathbb{R}$ defined as,
		$$f_n(x) = \frac{1}{x+\frac{1}{n}}.$$
		Then the family is uniformly continuous, pointwise equicontinuous, but not uniformly equicontinuous.
	\end{theorem}
	\begin{proof}
		Consider the extension of $f_n$ to $[0,1]$ on its domain. Then since it is bounded on this new compact
		extension, it is uniformly continuous by the Heine-Cantor theorem. It follows that if $f$ is uniformly continuous on some set
		then for any subset, $f$ is also uniformly continuos, so take $(0,1).$ 

		As for pointwise equicontinuity, for every $\epsilon > 0$ take any $x \in (0,1).$ Then we must show that there is a $\delta$ such that
		if $|x -t| < \delta$ and $n \in \mathbb{N}$ then $\frac{1}{x+\frac{1}{n}} - \frac{1}{t+\frac{1}{n}} < \epsilon.$ Simply summing both fractions yields that
		$$f(x)-f(y) = \frac{n^2(y-x)}{n^2xy+n(x+y)+1} < \frac{n^2(y-x)}{n^2xy} < \frac{y-x}{xy} < \epsilon,$$
		under proper assertions for $\delta.$

		Clearly this family is not uniformly equicontinuous since it converges uniformly to $1/x.$

	\end{proof}

	\setcounter{enumi}{12}
	\item The following argument holds.
	\begin{lemma}
 Let $\{f_n\}$ be an equicontinuous and pointwise bounded sequence of
functions $\mathbb{R} \to \mathbb{R}$. If $S \subset \mathbb{R}$ is compact, then $f$ is uniformly equicontinuous
on $S$.
	\end{lemma}
	\begin{proof}
		Take any $\epsilon > 0.$ Then bu equicontinuity, for every $z$, there is a $\delta(z)$ such that,
		$|z-y| < \delta(z)$ imples tahat $|f_n(z)-f_n(y)| < \epsilon/2.$ By the compactness of $S$, the unit disks
		$D_{\delta(z)}(z)$ form an open cover for $S$. Using the number lemma take $\delta$ such that $E \subset S$
		with $diam E < \delta$ is contained within an open set from the cover. Then if $|x-y| < \delta$ the set
		$\{x,y\}$ has diameter less than $\delta$ and so $|f_n(x) - f_n(z) < \epsilon/2$. Using the triangle inequality
		the lemma follows.
	\end{proof}


	\begin{theorem}
		Suppose that $(f_n)$ is a sequence of function $\mathbb{R} \to \mathbb{R}$ and for each compact subset $K\subset \mathbb{R}$
		the restricted sequence $(f_n|K)$ is pointwise bounded and pointwise equicontinuous. Then $(f_n)$ converges uniformly.
	\end{theorem}
	\begin{proof}
		By the previous lemma, the restriction of $(f_n)$ to $K$ compact implies that $(f_n|K)$ uniformly equicontinous over $K.$
		Thus by the AA theorem we have that $(f_n|K)$ has a uniformly convergent subsequence which converges to $g|K.$ Well then 
		it is easy to see that if $K$ are increasingly large extended compact close intervals in $\mathbb{R}$ then $g|K_j \to g|\mathbb{R}$
		is the uniform limit of some subsequence of $(f_n).$
	\end{proof}

	\setcounter{enumi}{17}
	\item %18

	The correct generalization is as follows.
	\begin{theorem}
		Let $\{f_n\}$ be a pointwise equicontinuous sequence of functions $\Omega \to \mathbb{R}$ with
		$\Omega$ an open subset of $\mathbb{R}$. Then if $\{f_n\}$ is pointwise bounded over a dense subset of $\Omega$
		then there is a subsequence which converges uniformly on compact subsets of $\Omega$ to a continuous function $f.$
	\end{theorem}

	\setcounter{enumi}{21}
	\item %22
	Consider the following example. 
	\begin{theorem}
		Let $f:[0,1] \to \mathbb{R}$ such that if $x \neq 0, x \mapsto x\sin\left(\frac{1}{x}\right)$ and $x\mapsto 0$ otherwise. Furthermore,
		let $(g_n)$ be a family of functions such that $g_n : [0,1] \to \mathbb{R}$ such that
		$$g_n(x) = \left\{
                \begin{array}{ll}
                  0,\;\forall x \in [0,1/n] \\
                  e^{\frac{1}{(x-1/n)^2}},\;\forall x \in (1/n,2/n)\\
                  1,\;\forall x \in [2/n,1]
                \end{array}
              \right.$$
        If $(f_n)$ is defined such that $f_n(x) = f(x)g_n(x)$, then the family $(f_n)$ is smooth, equicontinuous, with unbounded derivatives.
	\end{theorem}
	\begin{proof}
		Let $x\in (0,1], \gamma > 0.$ Then there exists an $N$ such that $2/n < x-\gamma.$
		 In this case for all $n>N$ $f_n(y) = f(y)$ for each and every $y \in (x-\gamma,x+\gamma).$
		 Then for every $\epsilon > 0,$ the continuity of $f$ gives that there is a $\delta < \gamma$ with $|f_n(y) - f_n(x)| = f(y)-f(x) < \epsilon.$
		 Take the smallest delta for which all $f_1,\dots,f_N$ are satisfied and yield that this $\delta'$ gives equicontinuity. At $x = 0$, $|f_n(y)| \leq y\sin(1/y)\leq y$ for all $y \in [0,1]$ and for all $n > N.$ So  $f_n(0)$ is equicontinuous. Then the compactness of $[0,1]$ implies uniform equicontinuity by the Arzela Ascoli theorem.
		 Clearly $f'(x)$ is unbounded as it approaches $0$ so in every case the derivatives of $f_n$ are unbounded by the product rule.
		 Smoothness comes from the fact that $f_n = 0$ in $[0,1/n]$ and the derivatives at $0$ are $0.$
	\end{proof}
\end{menumerate}	

\end{document}