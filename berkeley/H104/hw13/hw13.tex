%%%%%%%%%%%%%%%%%%%%%%%%%%%%%%%%%%%%%%%%%%%%%%%%%%%%%%%%%%%%%%%%%%
%%%                      Homework 13                           %%%
%%%%%%%%%%%%%%%%%%%%%%%%%%%%%%%%%%%%%%%%%%%%%%%%%%%%%%%%%%%%%%%%%%

\documentclass[letter]{article}

\usepackage{lipsum}
\usepackage[pdftex]{graphicx}
\usepackage[margin=1.5in]{geometry}
\usepackage[english]{babel}
\usepackage{listings}
\usepackage{amsthm}
\usepackage{amssymb}
\usepackage{framed} 
\usepackage{amsmath}
\usepackage{titling}
\usepackage{fancyhdr}
\usepackage{algorithm,algorithmic}
\usepackage{pgfplots}
\usepackage{tikz}
\usepackage{mathabx}

\pagestyle{fancy}


\newtheorem{theorem}{Theorem}
\newtheorem{definition}{Definition}
\newtheorem{example}{Example}
\newtheorem{lemma}{Lemma}
\newenvironment{menumerate}{%
  \edef\backupindent{\the\parindent}%
  \enumerate%
  \setlength{\parindent}{\backupindent}%
}{\endenumerate}







%%%%%%%%%%%%%%%
%% DOC INFO %%%
%%%%%%%%%%%%%%%
\newcommand{\bHWN}{13}
\newcommand{\bCLASS}{MATH H104}

\title{\bCLASS: Homework \bHWN}
\author{William Guss\\26793499\\wguss@berkeley.edu}

\fancyhead[L]{\bCLASS}
\fancyhead[CO]{Homework \bHWN}
\fancyhead[CE]{GUSS}
\fancyhead[R]{\thepage}
\fancyfoot[LR]{}
\fancyfoot[C]{}
\usepackage{csquotes}

%%%%%%%%%%%%%%

\begin{document}
\maketitle
\thispagestyle{empty}


%%%%%%% Be sure to set the counter and use menumerate
\setcounter{section}{3}
\section{Function Spaces}
\begin{menumerate}
\setcounter{enumi}{29}
    \item Consider the following example.
    \begin{example}
        The 1-sphere, $S^1$ is a compact path-connected and nonempty 1-manifold. Now consider the continuous mapping,
         $\phi: S^1 \to S^1$ which takes a point $p \in S^1$ and adds to it's angle $\sqrt{2}.$ The function $\phi$ 
         has no fixed points.
    \end{example}
    \begin{proof}
        It is fair to represent a point locally by its angle as $S^1$ is a one manifold and therefore has an atlas of functions {bad grammar here dude!}
        $f: S^1 \to E \subset \mathbb{R}$. The assertion that $\exists p \in S^1$ such that $\phi(p) = p$ implies that there exists
        an angle $\theta$ such that $\sqrt{2} + \theta \equiv \theta  \  (\mod 2\pi).$ Suppose such a $\theta$ existed.
         Then $n\sqrt{2} + \theta = \theta + k2\pi$ implies $n2^{1/2} = k2\pi$, and so $\pi$ is an algebraic number. A contradiction!
    \end{proof}

\setcounter{enumi}{33}
\item Consider the ODE $$y' = 2\sqrt{|y|}.$$  
    \begin{theorem}
        The ODE does not have unique solutions for $x \geq 0,$ and $y(0) = 0.$
    \end{theorem}
    \begin{proof}
        Consider the solution $y_1(x) = 0.$ Clearly $y_1(0) = 0,$ and $y'(x) = 2\sqrt{|0|} = 0.$ 
        Then consider likewise4 the solution $y_2(x) = x^2.$ Observer that $y(0) = 0^2 = 0$ and
        $y'(x) = 2x = 2\sqrt{x^2} = 2x$ when $x \geq 0.$
    \end{proof}

      \begin{figure}
          \centering
          \begin{tikzpicture}
              \begin{axis}[
                axis lines=middle,
                clip=false,
                ylabel=$y(x)$,
                xlabel=$x$,
                every axis y label/.style={
                    at={(ticklabel* cs:1.02)},
                    anchor=south,
                },
                legend pos=north west,
                width=10cm,
                height=7cm,z
                ytick={-1,1}
              ]
              \addplot+[mark=none,samples=200,unbounded coords=jump, restrict x to domain=-1:5] {(x+1)^2)} node[right,pos=.57,black];
              \addplot+[mark=none,samples=200,unbounded coords=jump, restrict x to domain=-2:5] {(x+2)^2)} node[right,pos=.57,black];
              \addplot+[mark=none,samples=200,unbounded coords=jump, restrict x to domain=-3:5] {(x+3)^2)} node[right,pos=.57,black];
              \addplot+[const plot, no marks, thick] coordinates {(-4,0) (-1,0)};
              \end{axis}
          \end{tikzpicture}

        \caption{Other solutions to the ODE}
        \label{fig:test}
        \end{figure}
    In fact there are even more examples of solutions which are not unique. See figure 1 for those whose domain
    is infact in $\mathbb{R}^-.$

    This does not however contradict Picard's theorem since, the function $f(y')$ defined is not
    uniformly lipschitz continuous. 
    \begin{proof}
        Suppose that $f(t)$ where lipschitz continuous. Then in particular, there is a constant
        $L$ such that $d(fx,fy) \leq Ld(x,y)$ for all $x,y \in M$ the domain of $f.$ So take for the sake
        of contradiction $x = 0,$ and let $y$ approach $0.$ By $f$ Lipschitz, we have that 
        $$\sqrt{y} \leq Ly$$
        which is true if and only if $y/y^2 \leq L.$ Since $y \to 0$ let us take $y = 1/n.$ This asserts that,
        $n \leq L$ for all $n$ which contradicts the archimedian property of $\mathbb{R}.$ 
    \end{proof}

\item We conjecture about the following ODE.
    $$x' = x^2 \in \mathbb{R}.$$
    The solution to the above ODE is obtained through the following calculations.
    \begin{equation}
        \begin{aligned}
            \frac{dx}{dt} &= x^2 \\
            \int \frac{dx}{x^2} &= \int_{t_0}^t ds \\
            -\frac{1}{2x(t)} + c_1 &= t        
        \end{aligned}
    \end{equation}
    and so we have that $x(t) = -\frac{2}{t-c_1}.$ Where $c_1$ shifts the solution to satisfy the initial condition. 
    However consider the solution where $x(-1) = 2.$ It's clear that this solution is unbounded as $t \to 0$, and therefore 
    escape to infinity in finite time.

\item We conjecture generally about separable ODE; that is differential equations of the following form.
    \begin{equation}
        D_s(x) = \mathcal{F}(x)
    \end{equation}
        where $x: \mathbb{R} \to \mathbb{R}^m$ is differentiable.
    \begin{theorem}
        Suppose $$\mathcal{F}: C^0(\mathbb{R}, \mathbb{R}^m) \to C^0(\mathbb{R}, \mathbb{R}^m).$$ is bounded;
        that is, there exists an $M$ such that for all $\omega_2, \hat 0 \in C^0(\mathbb{R}, \mathbb{R}^m),$
        \begin{equation}
            d_{C^0}\left(\hat0,\mathcal{F}[\omega_2]\right) \leq M
        \end{equation}
        Then if $x \in C^0(\mathbb{R}, \mathbb{R}^m)$ is a solution to (2), it does not escape to
        infinity in finite time.
    \end{theorem}
    \begin{proof}
        Let $Q_t = [t_0, t]$ be the finite time in which the differential form is evaluated.
        Then since the fundamental theorem of calculus yields
        \begin{equation}
                \int_{Q_t} D_s[x](s)\ ds = x(t) - x(t_0) = \int_{Q_t} \mathcal{F}[x](s)\ ds\\
        \end{equation}
        and so we need only show that the right hand side does not escape to infinity in finite time.
        It follows that since $\mathcal{F}$ is a bounded mapping, 
        \begin{equation}
            \begin{aligned}
                \left |\int_{Q_t} \mathcal{F}[x](s)\  ds \right|_2
                     &\leq \sup_{C^0} \left| \int_{Q_t} \mathcal{F}[\omega](s)\  ds \right|_2\\
                &\leq \sup_{\omega \in C^0} \sup_{z \in Q_t} |\mathcal{F}[\omega](z)|_2 m(Q_t) \\
                &\leq \sup_{\omega\in C^0} \sup_{z \in \mathbb{R}} |\mathcal{F}[\omega](z)|_2 m(Q_t)\\
                &\leq Mm(Q_t) 
            \end{aligned}
        \end{equation}
    \end{proof}

    \begin{theorem}
        Suppose that $\mathcal{F}$ from above satisfies the lipschitz condition. Then any solutions
        do not explode in finite time.
    \end{theorem}
    \begin{proof}
        Recall that Picard's theorem asserts that $\mathcal{F}$ lipschitz implies that there is a locally
        unique solution to (2). Such a solution is a continuous mapping from $(a,b) \to \mathbb{R}^m$
        which can be made to map the whole domain $\mathbb{R}.$ Since this mapping is defined for the whole path connected
        set $\mathbb{R},$ it does not blow up in finite time.
    \end{proof}

    \begin{lemma}
        Suppose that $f: X \to Y$ is a mapping between two metric spaces satisfying the Lipschitz condition.
        Then it is uniformly continuous.
    \end{lemma}
    \begin{proof}
        Let $\epsilon > 0$, and $x,y \in X.$
        Take $K$ to be the lipschitz constant for $f$.
        If $\delta = \epsilon/K$ and $d_X(x,y) < \delta$,
        then clearly $Kd_X(x,y) < \epsilon.$
        Using lipschitz, we know that $$d_Y(fx,fy) \leq K d_X(x,y) < \epsilon,$$
        so $f$ is uniformly continuous.
    \end{proof}

    \begin{theorem}
        If $\mathcal{F}$ is a uniformly continuous mapping then the solution
        to (2) does not explode in finite time.
    \end{theorem}
    \begin{proof}
        By the preceding lemma $\mathcal{F}$ satisfies the lipschitz condition
        and therefore by a previous theorem does not explode in finite time.
    \end{proof}

\setcounter{enumi}{38}
\item We give an alternative proof for completion.
    \begin{theorem}
    Every metric space can be completed.
    \end{theorem}
    \begin{proof}
        Let $M$ be a metric space with distance $d.$ Fix a point $p\in M$
        and for each $q \in M$ define a function $f_q(x) = d(p,x) - d(p,x).$
        Clearly $d(q,x) - d(p,x)$ is bounded since it is never more than $d(q,p).$
        It is continuous since arithmetic in $\mathbb{R}$ is continuous.

        Now consider the banach space $F = C_b^0(M, \mathbb{R})$. We have that
        $\phi: x \to f_x$ is an isometry from $M$ into $F$ since,
        \begin{equation}
            d(f_a,f_b) = \sup_{x\in M}|d(a,x) - d(p,x) - d(b,x) + d(p,x)| = d(a,b).
        \end{equation}

        Since the $M$ is dense in its closure and $\phi$ is an isometry,
        it follows that $\phi(M)$ is dense in the closure of $M$ in $F$.
        Since $cl_F(M)$ is a closed subset of the complete metric space $C_b(M)$
        it follows that $cl_F(M)$ is complete and therefore is the completion
        of the metric space $M.$
    \end{proof}


\setcounter{enumi}{40}
\item For the purpose of this example, consider the 
following theorem of Keshner.
    \begin{theorem}
        For any set $D \subset \mathbb{R}^n$ there is a separately continuous
        function $f: \mathbb{R}^n \mathbb{R}$ with $D(f) = D$ if and only if
        $D$ is an $F_\sigma$ set and every orthogonal projection of $D$ onto
        a coordinate hyperplane has first category image.
    \end{theorem}
    By Keshner's theorem it would seem that there are limits
    to just how discontinuous a function which is separately continuous can be.
    Therefore I will describe a function which is only discontinuous at a point
    (ie. not continuous as per the questions exact wording.)
    \begin{example}
        Let $f:[0,1] \to [0,1]$ be defined such that 
        \begin{equation}
            (x,y) \mapsto \frac{(x-0.5)(y-0.5)}{(x-0.5)^2 + (y-0.5)^2}
        \end{equation}
        when $(x,y) \neq 0$ and $(x,y) \mapsto 0$ at $(x,y) = 0.$
    \end{example}
    The partial derivatives of $f$ exist at $(0.5,0.5)$ but the function
        is clearly discontinuous as per \emph{K-Ciesielski et al}. 

\setcounter{enumi}{42}
    \item The joke is that there ODE on a greecian urn. Nice one!

\end{menumerate}
\end{document}